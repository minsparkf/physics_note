%======================================================================
%  Chapter : ベクトル解析
%  説明    : 電磁気学を記述する上で必要なベクトル解析のまとめ
%======================================================================

%======================================================================
%  Section
%======================================================================
    \section{ベクトル解析}
    \subsection{ベクトル変数(あるいは,変数ベクトル)}
    ベクトルには,スカラーにおける語彙「変数」に対応する,一般的呼称がない.
    ないと不便なので,このノートでは \textbf{ベクトル変数} という言い方を導入する.
    もしかしたら,\textbf{変数ベクトル} と書くこともあるかもしれない.

    細かいことを言うと,ベクトル変数は,
    成分の一部あるは全部が変数であるようなベクトルであり,
    次に説明するベクトル関数である
        \footnote{
            定義が論理的に循環してしまっているが,意図は伝わるはず.
            循環しないような記述も可能だが,理論構築が目的ではない
            ため,深く突っ込まないでおこう.
        }.

    変数をベクトル変数と区別する意味で,\textbf{スカラー変数} と
    書くこともある.

    \subsection{ベクトル関数}
    ベクトルが絡む関数のことを総称して \textbf{ベクトル関数} という.
    また,ベクトル関数と区別するために,
    今まで考えてきたベクトルが絡まないような関数を,
    \textbf{スカラー関数} と表現する場合がある
        \footnote{
            細かいことを言うと,スカラーは1次元ベクトルだから,
            スカラー関数もベクトル関数である.
        }.

    考えれる例をいくつか上げておこう.特にこれらを区別してよぶ必要は
    ないので,名称を与えることはしない
        \footnote{
            記述の際には,どんな形の
            ベクトル関数について議論している
            かが明確にわかるようにする.
        }.

    例えば,スカラーの独立変数 $t$ に対して,
    1つの定ベクトルが定まる関数が考えられる.これを
        \begin{align}
            \ba (t)
        \end{align}
    と表す.
    関数記号 $\ba$ を太字にした意図は,
    ベクトルが定まる(値域がベクトルである)ことを明示するためである.
    また,$(t)$ という表記は,$t$ が独立変数であることを示すものである
        \footnote{
            多変数になる場合,$\ba(t,\,s)$ と書かれることになる
            ($t$ と $s$ はスカラーである).
            このとき,$(t,\,s)$ という記述がベクトルを成分表示と同じで,
            紛らわしいかもしれない.しかし,文脈により容易に区別できる
            とし,特に書き分けることはしない.この記述の前に関数を
            表現する文字があれば,それらは独立変数である.
        }.

    別の例を上げると,ベクトル変数を独立変数にもつ関数が考えられる.
    数式で表そうとすると,
        \begin{align}
            \ba (\br)
        \end{align}
    のようになる.$\br$ はベクトル変数である.

    上記2つの混合して,スカラー変数 $t$ とベクトル変数 $\br$ から
        \begin{align}
            \ba (t,\,\br)
        \end{align}
    という関数を作ってもいい.

    ベクトル変数を独立変数として,スカラーが定まる(値域がスカラーである)
    関数もあり得る.記号化すれば,
        \begin{align}
            a (\br)
        \end{align}
    となるだろう.関数記号 $a$ を細字にした意図は,スカラーが定まることを
    明示するためである.

    もちろん,スカラー変数 $t$ とベクトル変数 $\br$ をもち,
    スカラーが定まる関数も考えられる.
        \begin{align}
            a (t,\,\br)
        \end{align}

    定ベクトルもベクトル関数の一部として考える.
    明示的な独立変数はないが,入力にかかわらず常に一定値をとるような
    関数として捉える.スカラー関数の場合と同じように考える.

    独立変数が1つのベクトル関数($\ba(t)$)を,\textbf{1変数ベクトル関数} という.
    独立変数が2つ以上のベクトル関数を総称して,\textbf{多変数ベクトル関数} という.
    ベクトル変数をもつベクトル関数($\ba(\br)$ など)は多変数ベクトル関数として考える.

    ひと目で見やすいように,表にしておこう(\Table\ref{table:f4unit}).
        \begin{table}[htb]
          \centering
          \caption{ベクトル関数の種類}
          \begin{tabular}{|l|c|c|l|}                                        \hline
            関数記号      & 独立変数   & 値域     & 例                      \\ \hline  \hline
            $\ba$         & なし       & ベクトル & 定ベクトル              \\ \hline
            $\ba(t)$      & $t$        & ベクトル & ある1点の風向の時間推移 \\ \hline
            $\ba(\br)$    & $\br$      & ベクトル & ある時刻の風向分布      \\ \hline
            $\ba(t, \br)$ & $t$,$\br$ & ベクトル & 風向分布の時間推移      \\ \hline
            $a(\br)$      & $\br$      & スカラー & 風力分布                \\ \hline
            $a(t, \br)$   & $t$,$\br$ & スカラー & 風力分布の時間推移      \\ \hline
          \end{tabular}
          \label{table:f4unit}
        \end{table}

    \subsection{ベクトル関数の微積分}
    \begin{mycomment}
        スカラー関数での微積分を,ベクトル関数へ拡張する.
        ベクトル関数の微積分も,基本的にはスカラー関数と同じように
        計算可能である.
    \end{mycomment}

    \subsubsection{極限}
    ベクトル関数の極限はスカラー関数の場合と同じように定義できる.

    \subsubsection{導関数}





