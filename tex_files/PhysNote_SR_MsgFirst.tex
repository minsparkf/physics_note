%===================================================================================================
%  Chapter : 電磁気学の不満な点について
%  説明    : アインシュタインの論文に基づいて,ローレンツ力と電磁誘導に関する説明で,
%            相対性が成り立っていないように見受けられることを指摘する
%===================================================================================================
%   %==========================================================================
%   %  Section
%   %==========================================================================
    \section{導入}
    特殊相対性理論は,アインシュタインが1905年に書いた
    “Zur Elektrodynamik bewegter K\"{o}rper”
    (動いている物体の電気力学)によって,始まった
        \footnote{
            いや,アインシュタインによって,
            「物理学的に完全な形に整えられた」
            と言った方が正確だろう.特殊相対性理論
            と同等の議論は,アインシュタイン以前にも
            盛んに行われていたからだ.言い換えれば,アインシュタインの他にも,
            特殊相対性理論の内容と同じ理論的枠組に迫った人がいる,ということ
            にもなる.
        }.
    特殊相対性理論は,光の速度の不可思議な性質
        \footnote{
                これは \textbf{光速度不変の原理} とよばれる,光のもつ
                性質の1つである.光の速さは,1つの慣性系において,
                運動している物体から光を出そうが,静止している物体から
                光を出そうが,
                両者の光の速度は一定の値($c=3\times10^{8}$[m/s])をとる,
                というものである.
        }
    をスマートに解決する理論である,と言いたいところだが,そうではない.
    アインシュタインは,光速度不変の原理を認めることで,
    より多くの物理現象を説明できる理論を組み立
    てることができると,主張する.その理論が,
    特殊相対性理論である.この章では,特殊相対
    性理論を確認する.特殊相対性理論は,アインシュタイン
    の1905年の最初の論文で,ほぼ完成している.
    有名な,時間の遅れという現象,棒の収縮といったことは
    この論文に全て書かれている
        \footnote{
            エネルギーと質量の関係式 $E=mc^{2}$ は,
            この論文には書かれていない.
        }.
    岩波文庫から,内山龍雄訳の原論文があるので,これを参照しながら,
    特殊相対性理論を勉強していこう.教科書は別のものを使う.

%   %==========================================================================
%   %  Section
%   %==========================================================================
    \section{ローレンツ力}
        ローレンツ力について,復習しよう.ある空間に,電場 $\bE$ と,磁束密度 $\bB$ が
        生じていることが分かっているとしよう.
        このとき,この空間に,電気量 $q$ をもった点電荷を,初速度 $\bv$ を与えて
        放す.すると,この点電荷は,空間からローレンツ力 $\bF$ を受けることになる.
        このローレンツ力 $\bF$ は,
            \begin{align}
                \bF = q(\bE + \bv \times \bB)
            \end{align}
        と書き表される.

%   %==========================================================================
%   %  Section
%   %==========================================================================
    \section{電磁誘導}
        ファラデーの電磁誘導の法則によると,磁束密度 $\bB$ の時間変化が
        その周りに回転する電場 $\bE$ を作り出す.式で書けば,以下の通り.
            \begin{align}
                \drot \bE = - \frac{\rd \bB}{\rd t}.
            \end{align}


%   %==========================================================================
%   %  Section
%   %==========================================================================
    \section{ローレンツ力と電磁誘導}
    原論文
        \footnote{
            原論文とはいっても,もちろん,日本語訳されたものを参照している.
            私にドイツ語が読めるわけがない.英語もよく読めないのに.

            参考図書のリスト\cite{bib:refbook_rel_1}を参照.
        }
    の冒頭で,アインシュタインは次のような,
    電磁気学における矛盾点を指摘している.
    それは,ローレンツ力と電磁誘導に関するものであ
    る.

    まず,磁石
    と金属棒を用意する.はじめに,磁石を固定し,
    金属棒を磁石の近くで,くっつけることのない
    よう,揺らしてみよう.このとき,金属棒内の
    電子に「ローレンツ力」が働く.

    今度は,逆に,金属棒を固定し,磁石を金属棒の近くで,
    棒をくっつけることなく,振ってみよう.このと
    き,磁石の振動によって,金属棒周囲の磁場が変
    動する.この磁場の変動は「電磁誘導」により,
    その周囲に電場を作り出す.この電場によって,
    金属棒に起電力が生じる.

    以上の2つの状況は,ともに電子の運動を引き起こす原因を
    説明するものである.そして,それらはともに,
    金属棒と磁石の“相対的な振動”によって,金属棒に
    起電力が生じるというものである.
    相対的な位置の変化が問題になるにもかかわらず,
    起電力発生の原因の説明が,視点を金属棒にするか,
    あるいは,磁石にするかによって,異なる.
    つまり,物理学的に,同じ状況であるのだけど,
    その現象の説明方法が異なっていまうのだ.
    これでは,納得のできないだろう.電磁気学に
    何らかの不備があると,感じてしまうこともあろう.

    そこでアインシュタインは,一歩後ろに引いて落ち着いて
    考察をする.そして,光速不変の原理と特殊相対性原理を
    基礎にし,「特殊相対性理論」を確立させる.
    この理論は,上のようなおかしな説明をせずに,
    起電力の発生を説明できる.そして,それだけにとどまらず,
    ニュートン力学を,より一般性の高い理論になるように,修正を加える.

