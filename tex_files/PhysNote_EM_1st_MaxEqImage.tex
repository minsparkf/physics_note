%   %==========================================================================
%   %  Section
%   %==========================================================================
    \section{4つの基本法則}\label{sec:4fundlaw}
    \begin{mycomment}
        ここでは,真空中の電磁気現象を想定する.言い換えれば,
        物質内のおける電場や磁束密度については想定していない.
        物質が存在する場合の理論は複雑になり,最初に学習する
        際には,理論の骨格を捉えにくいものである.理論の筋道を
        明確に捉えるために,真空中であるという制約を与える.
    \end{mycomment}

%       %======================================================================
%       %  SubSection
%       %======================================================================
        \subsection{はじめに}\label{subseq:4fundlaw_Hajimeni}
        電磁気的現象は4つの基本法則によって説明できる.ニュートン力学
        で言うところの,ニュートンの運動の3法則に対応する部分である.ついでに,
        それに対する方程式も記述しておこう.
            \begin{enumerate}
                \item 電場に対するガウス
                    \footnote{
                        Johann Carl Friedrich Gauss(Gau\ss)(1777--1855, ドイツ)
                        :ドイツの数学者,物理学者.整数や代数についての研究,曲
                        面論などに代表される幾何学の研究が有名である.近代数学の
                        大部分にその業績があり,19世紀最大の数学者とも言われる.
                        また,cgs単位系の「ガウス[G]」は磁気の単位として使われ
                        ている.そのほかにも,「ガウス記号(整数論)」,「ガウス
                        平面(複素関数論)」,「ガウス分布(誤差論)」など彼の名
                        がつけられた概念は多い.
                    }
                    の法則
                    \begin{align}
                        \ddiv \bE = \frac{1}{\varepsilon_{0}}\rho.
                    \end{align}
                \item 磁束密度に対するガウスの法則
                    \begin{align}
                        \ddiv \bB = 0.
                    \end{align}
                \item アンペール
                    =マクスウェル
                    \footnote{
                        James Clerk Maxwell(1831--1879, イギリス):古典電磁気学
                        の理論体系を築いた.この理論より電磁波の予言を行い,ヘル
                        ツ(Heinrich Rudolf Hertz, 1857--1894, ドイツ)らによって
                        実験的に実証された.

                        電磁気学的な自然現象は,4つの法則を基本法則とすることで,
                        説明のつく現象であると提唱する
                        (1865年;A dynamical theory of the electromagnetic field).
                        その後,マクスウェルは,1873年に,電磁気学を体系的に纏めた教科書を
                        出版し,電磁気学を確立させた
                        (1873年;A treatse on electricity and magnetism).
                        この教科書は,電磁気学におけるプリンキピアであると言われるようだ
                        (褒め言葉であると同時に,難解であるという意味も込められているらしい).

                        また,統計物理学の基本的な考え方である,気体分子運動論にかんする
                        研究も行なっている.「マクスウェル分布(マクスウェル--ボルツマン分布)」
                        としても,その名前を残している.これは今日の統計物理学の基礎をなすもの
                        である.

                        また,マクスウェルはキャベンディッシュ(\ref{sec:EM_ObjPreMsg}節の脚注を参照)
                        の仕事を世の中に紹介している.そして,当時新しく設けられた,
                        「キャベンディッシュ研究所」の実験物理学の教授(初代所長)として働いている.
                        なおこの研究所は,上に紹介したキャベンディッシュ(Henry Cavendish)の子孫
                        に当たる,デヴォンシャー第7大公爵が出資して作られた,
                        ケンブリッジ大学の実験物理学施設である.

                        片仮名表記される場合,「マックスウェル」とかかれることもある.

                        (参考1)太田 浩一,『マクスウェルの渦 アインシュタインの時計 現代物理学の源流』,
                        東京大学出版会

                        (参考2)William H.Cropper,『物理学天才外伝』, 講談社(ブルーバックス)
                    }
                    の法則
                    \begin{align}
                        \drot \bB = \mu_{0}\bi + \varepsilon_{0}\mu_{0}\frac{\rd \bE}{\rd t}.
                    \end{align}
                \item ファラデー
                    \footnote{
                        Michael Faraday(1791--1867, イギリス):電磁誘導の発見,
                        電気力線,磁力線の提唱(電磁気現象の近接作用の考え)など,
                        電磁気学に大きい貢献をする.また,化学者としての活躍も有名
                        であり,電気分解の法則の発見がその例である.コンデンサの容
                        量の単位「ファラド[F]」や,ファラデー定数などは,彼の名に
                        ちなんだものである.
                    }
                    の電磁誘導の法則
                    \begin{align}
                        \drot \bE = - \frac{\rd \bB}{\rd t}.
                    \end{align}
            \end{enumerate}

        以下では,これら4つの法則の内容をざっくりと見ていこう.その実験的根拠等の
        細かいことは,次章以降で考えることにする.ここでは,電磁気学は4つの基
        本法則から構成されているのだということを理解してもらいたい.
        各法則が意味する現象の詳細は,後で,それぞれ章を立てて説明する.

        もう一度注意しておこう.先のコメントの部分にも書いたが,以降で想定するのは,
        すべて,真空中で起こる電磁気現象である.物質を含む場合については,また別に
        議論することとにしたい.


%       %======================================================================
%       %  SubSection
%       %======================================================================
        \subsection{電場に対するガウスの法則}
        電場に対するガウスの法則は,電場の発生と消滅に関する法則である.内容は次
        の通りである.
            \\
            \begin{itembox}[l]{\textbf{電場に対するガウスの法則}}
                電場の\textbf{発生源}は\textbf{正に帯電した電荷}である.また,電
                場の\textbf{消滅源}は\textbf{負に帯電した電荷}である.そして,電
                場の発生と消滅は電荷においてのみ起こり,これ以外では起こらない.
            \end{itembox}
            \\

        この法則の意味するところは,電場の発生と消滅の原因は電荷にあることの主張
        である.正に帯電したから電荷から発生した電場は,負に帯電した電荷に吸収(
        消滅)するのである.そして,電場の発生,消滅は電荷以外では起こりえない.
               \begin{figure}[hbt]
                    \begin{tabular}{cc}
                        \begin{minipage}{0.5\hsize}
                            \begin{center}
                                \includegraphicsdouble{GaussLowImage_01.pdf}

                                (A) 正電荷より電場発生
                            \end{center}
                        \end{minipage}
                        \begin{minipage}{0.5\hsize}
                            \begin{center}
                                \includegraphicsdouble{GaussLowImage_02.pdf}

                                (B) 負電荷で電場消滅
                            \end{center}
                        \end{minipage}
                    \end{tabular}
                                \caption{電場に対するガウスの法則}
                                \label{fig:GaussLowImage}
                \end{figure}


%       %======================================================================
%       %  SubSection
%       %======================================================================
        \subsection{磁束密度に対するガウスの法則}
        磁束密度に対するガウスの法則は,磁束密度の発生と消滅に関する法則である.
        内容は次の通りである.
            \\
            \begin{itembox}[l]{\textbf{磁束密度に対するガウスの法則}}
                磁束密度の\textbf{発生源},\textbf{消滅源}は存在しない.
            \end{itembox}
            \\

        磁束密度がある点から生じて,他の点に吸収されるようなことは起こりえない
        ことを,この法則は主張する.つまり,電場の発生源とはる電荷に対応するよう
        な,言わば磁荷は存在しないことを意味する.
                \begin{figure}[hbt]
                    \begin{center}
                        \includegraphicsdefault{GaussLowBImage_01.pdf}
                        \caption{磁束密度に対するガウスの法則}
                        \label{fig:GaussLowBImage_01}
                    \end{center}
                \end{figure}

%       %======================================================================
%       %  SubSection
%       %======================================================================
        \subsection{ファラデーの電磁誘導の法則}
        ファラデーの電磁誘導の法則は,磁束密度の時間的な変化と電場の関係に関する
        法則である.内容は次の通りである.
           \\
           \begin{itembox}[l]{\textbf{ファラデーの電磁誘導の法則}}
               磁束密度が時間的に変化すると,その周囲には回転する電場が発生する.
           \end{itembox}
           \\

        磁束密度とは磁界のことだから,磁界の変化が回転する電場を発生させることに
        なる.具体的な例で考える.磁石は磁界を発生させていることは,中学生なら
        ば誰でもでも知っている.とすれば,磁界が変化する状況を作るには,磁石を手
        で持って振ればよい.ファラデーの電磁誘導の法則の法則は,この手で振ってい
        る磁石の周りに,電場が生じていると主張しているのである.
                \begin{figure}[hbt]
                    \begin{center}
                        \includegraphicsdefault{FaradayLowImage_01.pdf}
                        \caption{ファラデーの電磁誘導の法則}
                        \label{fig:FaradayLowImage_01}
                    \end{center}
                \end{figure}


%       %======================================================================
%       %  SubSection
%       %======================================================================
        \subsection{アンペール$=$マクスウェルの法則}
        アンペール$=$マクスウェルの法則は,電場の時間的な変化と磁束密度に関する法
        則である.内容は次の通りである.
            \\
            \begin{itembox}[l]{\textbf{アンペール$=$マクスウェルの法則}}
                電流の周りには,この電流を取り囲むように回転する磁束密度が生じる.ま
                たこれに加えて電場の状態が時間的に変化すると,その周囲には回転す
                る磁束密度が発生する.
            \end{itembox}
            \\

        電流が流れていると,その周りには磁界が発生しているということは,おそらく
        中学で習うはずだ.この法則では,それに加えて,時間的に変化する電場も,回
        転する磁束密度を作ることを主張している.
                \begin{figure}[hbt]
                    \begin{tabular}{cc}
                        \begin{minipage}{0.5\hsize}
                            \begin{center}
                                \includegraphicsdouble{AMLaw_Image00.pdf}

                                (A) 電流による磁場
                            \end{center}
                        \end{minipage}
                        \begin{minipage}{0.5\hsize}
                            \begin{center}
                                \includegraphicsdouble{AMLaw_Image01.pdf}

                                (B) 電場の時間変化による磁場
                            \end{center}
                        \end{minipage}
                    \end{tabular}
                        \caption{電流と時間変化する電場は,その周囲に回転する磁場を生じる}
                        \label{fig:AMLaw_Image00}
                \end{figure}

%   %==========================================================================
%   %  Section
%   %==========================================================================
    \section{マクスウェル方程式を見てみよう}
        \begin{mycomment}
            この節では,マクスウェル方程式を実際に見てみよう.
            だたし,その式をイメージとして捉え,意味すること
            を重視する.数式の本来の性質等は,各法則毎に詳しく
            調べることにし,ここでは数式の形に慣れることが目的
            である.徐々にマクスウェル方程式に馴染んで行こう.
        \end{mycomment}

%       %======================================================================
%       %  SubSection
%       %======================================================================
        \subsection{「マクスウェル方程式」とは}
        \begin{mycomment}
            まずは,「マクスウェル方程式」と言われる式について説明する.
            実は,マクスウェル方程式と言われてはいるが,マクスウェルがその
            方程式を発見したのではない.勘違いを起こしてしまいがちだが,
            マクスウェルが発見したという意味での,方程式は存在しないのである.
            ではどういうことかというと,言うなれば,
               “マクスウェルが提唱した,電磁気の公理的な4つの連立方程式”
            がマクスウェル方程式と呼ばれるものである.
        \end{mycomment}

%           %==================================================================
%           %  SubsubSection
%           %==================================================================
            \subsubsection{マクスウェルが電磁気学を確立する}
            電磁気的な自然現象には,冬場に発生する静電気や,より身近なものとして,
            磁石がある.携帯電話や無線LANも電磁気現象(電磁波)を利用した装置である.
            電磁気的な現象は,一見すると,その発生機構が複雑であるように感じてしまう.
            たしかに,マクスウェルにより指摘される以前では,電磁気現象は複雑であると
            感じることだったろう.しかし,今では,電磁気現象は,たった4つの法則を認め
            るだけで,説明ができることが分かっている.これを指摘した人物
            こそマクスウェルである.

            マクスウェルの提唱した4つの法則は,それ以前に先人
                \footnote{
                    例えば,アンペール,クーロンなどがいる.
                }
            により発見されていた自然現象をピックアップしたものであり,つまり,
            マクスウェル自身が4つの法則を発見したわけではない.マクスウェルの偉大なところ
            は,それまで煩雑としていた電磁気現象に関する実験結果(実験法則)を,体系化した
            ことにある.それまでには,色々な電磁気学の実験が行われて,その実験結果も
            多様にあったはずである.マクスウェルは,この煩雑な実験結果は,4つの自然現象
            を受け入れることで,すべてが上手く(数学的に)説明できることを示唆した.

%           %==================================================================
%           %  SubsubSection
%           %==================================================================
            \subsubsection{数式で表現してこそ,基本法則と言える}
            マクスウェルが基本法則として取り上げた4つの実験結果については,定性的には,
            上で紹介しと通りである(\ref{sec:4fundlaw}節参照).しかし,これらの法則は,
            数式により表現してこそ,意味を成す
                \footnote{
                    物理現象を説明するには,数式によりそれを表現すべきだ.
                    数式で表されて,初めて,詳細な推論や論理的思考が可能になるからだ.
                    数式を用いずに,言葉で表現しても論理的推論が不可能というわけではないが,
                    考えにくいし,論理もわかりにくくなる.数式で表現できれば,論理的
                    推論もしやすくなるし,論旨もわかりやすくなる.
                    不慣れな数式を扱うことが億劫かもしれなが,ここは少々なれるまで我慢
                    してもらいたい.物理学は数式を扱う学問なので,数式に慣れることは大事だ.
                    4つの法則で電磁気学現象を説明できるといったのは,各法則を方程式
                    で表したときに,電磁気現象が4つの方程式から数学的演算により,
                    論理的必然性をもった結果として導かれることをいう.
                }.
            そこで,次に,4つの基本法則を表す数式を紹介する.ただし,その数式に深入り
            することはせず,そのイメージを捉えることを第一の目的としたい.より詳しいことは,
            次章以降で考える
                \footnote{
                    実は,以下に説明する4つの方程式の理解こそが,初めて電磁気学を学ぶ際の
                    目標なのである.つまり,ここでは,その最終目標を先回りして紹介すること
                    になる.これは,学習の目標を明確にすることを考えてのことである.目標が
                    明確になれば,学習の際の不安(何をしているのかが分からないなど)も,少
                    しは解消されることと思う.
                }.

%           %==================================================================
%           %  SubsubSection
%           %==================================================================
            \subsubsection{マクスウェル方程式の2種類の表現}
            マクスウェル方程式は4つであるが,実は,その表現方法が2種類があり,
            微分を使って表現したものが \textbf{微分形},
            積分を使って表現したものが \textbf{積分形} と言われる.ここでは,
            この2種類の表現方法を紹介する.微分形と積分形の違いは,視点の違いである.
            現実に起こっている現象を肌で感じる場合には,積分形の方程式を用いる.
            そのため,工学の分野では,積分形を使うことのほうが多いと思う.これに対し,
            微分形で記述されたものは,現象を局所的に見た場合に使われる.微分形は
            理論的考察を行う場合に使うことが多い.もちろん,積分形で表されようが,
            微分形で表されようが,その式は全く等価である(当たり前のはずだが,念のために).


%       %======================================================================
%       %  SubSection
%       %======================================================================
        \subsection{約束:独立変数の記述の省略}
                マクスウェル方程式は,位置 $\br$ と時間 $t$ の4つの独立変数を
                含む関数の間に成り立つ式であるが,この独立変数をいちいち明記
                していたら,式が煩雑になり見難くなっていしまう.そこで,
                以下のように,独立変数を省略して記述する.
                すなわち,
                電場 $\bE$,磁束密度 $\bB$,電流密度 $\bi$,電荷密度 $\rho$ であり,
                その意味は次式の通り.
                    \begin{align*}
                        \bE  &= \bE(\br,\,t) \\
                        \bB  &= \bB(\br,\,t) \\
                        \bi  &= \bi(\br,\,t) \\
                        \rho &= \rho(\br,\,t)
                    \end{align*}
                また,$\varepsilon_{0}$,$\mu_{0}$ は,それぞれ,真空中の誘電率,透磁率と言われる,
                スカラー量で,単なる定数である(位置と時間の関数ではない).

                ついでに,他の記号も説明する.
                任意の閉曲面を $S$ で表現する.また,$S$ で囲まれた内部の領域全体を $\Omega_{S}$ と
                書く.閉曲面の微小部分は $\df S$ であり,
                $\df S$ の単位法線ベクトルは $\bn:=\bn(\br,\,t)$ で表現する.
                また,$l$ は任意の閉経路(ループしている経路,輪状の経路のこと)である.そして,$S_{l}$ というのは,
                閉経路 $l$ を縁とする開曲面を表す.そして,閉経路 $l$ の単位接線ベクトルを $\bt:=\bt(\br,\,t)$ と
                表現する.

                くどいかもしれないが,演算に関する記述方法の説明もしておこう.
                $\sint_{S} X\df S$ は,関数 $X$ を,$S$ に対して面積分を行うことを意味する.また,
                $\vint_{\Omega_{S}} X \df V$ は,関数 $X$ を,$S$ の内側の全領域 $\Omega_{S}$ に
                対して体積分を行うことを意味する.

%       %======================================================================
%       %  SubSection
%       %======================================================================
        \subsection{マクスウェル方程式(微分形)}
        \begin{mycomment}
            上では,現実に起こっている法則のイメージを説明したが,
            ここではもう一歩先に進んで数式を眺めてみよう.この節では,
            微分形のマクスウェル方程式を確認する.積分形も,後で確認する.
            何度も言うが,
            数式そのものを理解することが目的ではなく,数式のイメージを
            持つことが目的である.数式の詳細は後で述べる.ここではとにかく,
            求めるべきマクスウェル方程式がどのようなものかを,感覚的に
            把握してもらいたい.
        \end{mycomment}
%           %==================================================================
%           %  SubsubSection
%           %==================================================================
            \subsubsection{電場に対するガウスの法則の式:微分形}
            \begin{mysmallsec}{数式}
                電場に対するガウスの法則の,微分形の式は次式の通り.
                \begin{align}
                    \ddiv \bE = \frac{1}{\varepsilon_{0}}\rho.
                \end{align}
            \end{mysmallsec}

            \begin{mysmallsec}{法則のイメージ}
                電場の発生あるいは消滅は,電荷の存在する場所で生じる.また,言い換えれば,
                ある場所において,電場が発生あるいは消滅していることと,その場所に電荷が
                存在することは,同じ意味をなす.

                ベクトル $\bE$ は電場で,その前に書かれている $\ddiv$ は,
                divergence(発散) を意味する.なので,左辺 $\ddiv \bE$ は電場の発散を表す.
                右辺の$\rho$ は電荷密度
                    \footnote{
                        1/$\varepsilon_{0}$ がかかっているが,単なる比例定数である.
                        これは単位系としてSI単位を採用していることによって現れた定数
                        であり,式の表す物理的イメージにはあまり関係がないと思って良い.
                    }
                であり,電荷密度の存在が表されている.
                電場の発散と電荷密度が等号で結ぶことによって意味されることは,
                「電荷密度が存在すれば,その周囲には電場の発散が生じる」ということである.
                更にその逆もいうことができて,「電場の発散の原因は電荷密度である」ということも
                できる.
                電場の発生あるいは消滅は,電荷の存在しない場所では,絶対に発生しない.
            \end{mysmallsec}

                \begin{figure}[hbt]
                    \begin{center}
                        \includegraphicsdefault{GaussLowEImage_03.pdf}
                        \caption{電場は電荷より生じる}
                        \label{fig:GaussLowEImage_03}
                    \end{center}
                \end{figure}

%           %==================================================================
%           %  SubsubSection
%           %==================================================================
            \subsubsection{磁束密度に対するガウスの法則の式:微分形}
            \begin{mysmallsec}{数式}
                磁束密度に対するガウスの法則の,微分形の式は次式の通り.
                \begin{align}
                    \ddiv \bB = 0.
                \end{align}
            \end{mysmallsec}

            \begin{mysmallsec}{法則のイメージ}
                この式は,磁束密度はどこからも湧き出しがないことを表現している.
                見方を変えれば,ある部分で湧き出す磁束密度の量と,その部分で消滅する
                磁束密度の量が等しいから,正味として湧き出しがないとみなされる.

                つまり,磁束密度の発生場所を特定することはできないということである.
                こういう言い方すると,“磁束密度は存在して,その発生原因は電流にあることを,
                先に説明している.つまり,発生している場所を示すことができるではないか”と
                いう疑問を持たれてしまうかもしれない
                    \footnote{
                        少なくとも,私はそう思った.
                    }.
                しかし,これは言葉の意味の捉え方(あるいは記述の仕方)の問題であり,
                磁束密度の存在場所が特定できないことを言っているのではない.この法則は
                ,あくまでも,発生源を特定することができないということであり,発生して
                いる場所を示せないということではない.現に,磁石の周囲には磁束密度が存在
                していることは,すでに知っていることである.たしかに,磁石から磁束密度が
                湧き出していると考えても,間違いではないが,この法則の言っている「湧き出し」とは
                意味がことなる.先にも書いたとおり,ここで言う「湧き出し」とは,磁束密度の
                生じる量と消滅する量の,“正味の湧き出し”ということである.この正味の湧き出しが
                0であるというとは,この世界のどの場所を見ても磁束密度の正味の湧き出しがないという
                ことを意味しているのである.この「正味の」という部分に,注意すべきだ.
            \end{mysmallsec}

                \begin{figure}[hbt]
                    \begin{center}
                        \includegraphicsdefault{GaussLowBImage_02.pdf}
                        \caption{磁束密度の湧き出しはない}
                        \label{fig:GaussLowBImage_02}
                    \end{center}
                \end{figure}


%           %==================================================================
%           %  SubsubSection
%           %==================================================================
            \subsubsection{アンペール$=$マクスウェルの法則の式:微分形}
            \begin{mysmallsec}{数式}
                アンペール$=$マクスウェルの法則の式の,微分形の式は次式の通り.
                \begin{align}
                    \drot \bB = \mu_{0}\bi + \varepsilon_{0}\mu_{0}\frac{\rd \bE}{\rd t}.
                \end{align}
            \end{mysmallsec}

            \begin{mysmallsec}{法則のイメージ}
                右辺に項が2つあるが,これらはそれぞれ,意味することが異なる.
                右辺の第一項は,電流が磁場を作るということを主張するものである.
                これは,アンペールの法則とよばれる
                    \footnote{
                        アンペールの法則とは,上式の第二項が常に0である場合のことである.
                        すなわち,次式がアンペールの法則を表す式である.
                            \begin{align}
                                \drot \bB = \mu_{0}\bi.
                            \end{align}
                        アンペールの法則の意味するところは,式を見れば明らかである.
                        口うるさく意味を説明すれば,
                        「回転している磁束密度が存在するということは,その内側の領域に
                        電流が生じていることを意味する」ということだ.
                    }.
                そして,第二項が意味するのが,
                \textbf{変位電流} という概念である.詳細は後述するが,ここでは,
                おおよそのイメージとして,電場の時間変化がその周囲に磁束密度を
                生じさせる,と解釈して欲しい
                    \footnote{
                        式をそのまま言葉にしただけである.その真意の程は後の記述を
                        参照.
                    }.

                なぜ,「変位電流」を導入する必要があるのかというと,単なるアンペールの法則が
                電荷保存則に矛盾してしまうからである.つまり,アンペールの法則と電荷保存則の
                どちらかが,間違っている(不完全である)可能性があるということである
                    \footnote{
                        あるいは,どちらとも間違いなのかもしれない.しかしこの可能性は,
                        あとに記述する,マクスウェルの修正によってなくなる.
                    }.
                マクスウェルによる回答は,アンペールの法則が不完全である,ということだった.
                そして,マクスウェルはアンペールの法則を完全な形にすべく,「変位電流」という
                新しい概念を考案し,アンペールの法則にそれを組み込むことで,電荷保存則との
                矛盾を解消したのである.

                あとに分かることだが,この変位電流は,ファラデーが発見する電磁誘導の法則
                に対をなす現象である,と見ることもできる.この変位電流と電磁誘導とにより,
                電磁波という現象が起こるのである.電磁波についても,後ほど考えることにしたい
                    \footnote{
                        マクスウェル方程式の偉大さの一つは,電磁波の存在を予言したことである.
                    }.

                もう一度改めて,この法則の内容を確認しておこう.電流はその周囲に磁束密度を
                発生させる.さらに,それに加えて,電場の時間変化が起きた際にも,その電場の
                変化にともなって,その周囲に,磁束密度が生じるのである.
            \end{mysmallsec}

                \begin{figure}[hbt]
                    \begin{tabular}{cc}
                        \begin{minipage}{0.5\hsize}
                            \begin{center}
                                \includegraphicsdouble{dennryu_to_jisokumitudo.pdf}

                                (A) 電流による磁場
                            \end{center}
                        \end{minipage}
                        \begin{minipage}{0.5\hsize}
                            \begin{center}
                                \includegraphicsdouble{hennnidenryu_to_jisokumitudo.pdf}

                                (B) 変位電流による磁場
                            \end{center}
                        \end{minipage}
                    \end{tabular}
                        \caption{電流,変位電流と磁束密度の関係}
                        \label{fig:I_i_B}
                \end{figure}


%           %==================================================================
%           %  SubsubSection
%           %==================================================================
            \subsubsection{ファラデーの電磁誘導の法則の式:微分形}
            \begin{mysmallsec}{数式}
                ファラデーの電磁誘導の法則の式の,微分形の式は次式の通り.
                \begin{align}
                    \drot \bE = - \frac{\rd \bB}{\rd t}.
                \end{align}
            \end{mysmallsec}

            \begin{mysmallsec}{法則のイメージ}
                左辺は回転する電場を表している.右辺は負の符号がついて入るが,
                磁束密度の時間変化が記述されている.渦を巻くように電場が生じている
                ならば,その周囲に磁束密度の時間変化が起こっているということを示している.
                また,言い方を変えれば,磁束密度が時間変化するとき,その周囲に渦を巻くようにして
                電場が生じるということでもある.磁束密度の時間変化とは,現実的に言えば,例えば
                磁石を左右に振った場合のことである.このとき左右にふった磁石の周りには,その磁石を
                取り囲むように,渦電場が生じるのである.
            \end{mysmallsec}

%       %======================================================================
%       %  SubSection
%       %======================================================================
        \subsection{マクスウェル方程式(積分形)}
%           %==================================================================
%           %  SubsubSection
%           %==================================================================
            \subsubsection{電場に対するガウスの法則の式:積分形}
            \begin{mysmallsec}{数式}
                電場に対するガウスの法則の,積分形の式は次式の通り.
                \begin{align}
                    \sint_{S} \bE \cdot \bn \df S = \frac{1}{\varepsilon_{0}} \vint_{\Omega_{S}} \rho \df V.
                \end{align}
            \end{mysmallsec}

            \begin{mysmallsec}{法則のイメージ}
                この式の言っていることは,微分形のそれと同じだが,次のような
                イメージを連想させる.すなわち,ある領域 $S$ から電場が湧き出ている
                ならば,その内部領域 $\Omega_{S}$ に電荷が存在しする.そして,このとき湧き出す電場の量
                は,$\Omega_{S}$ 内に存在するすべての電荷の電気量の総和
                を $\varepsilon_{0}$(真空中の誘電率;詳細は後述する)で割った値に等しい.

                要するに,ある領域から電場が生じているのであれば,その領域には
                必ず電荷が存在するということである.
            \end{mysmallsec}

%           %==================================================================
%           %  SubsubSection
%           %==================================================================
            \subsubsection{磁束密度に対するガウスの法則の式:積分形}
            \begin{mysmallsec}{数式}
                磁束密度に対するガウスの法則の,積分形の式は次式の通り.
                \begin{align}
                    \sint_{S} \bB \cdot \bn \df S = 0.
                \end{align}
            \end{mysmallsec}

            \begin{mysmallsec}{法則のイメージ}
                この式の意味は,微分形のそれと全く同様である.

                この表現のほうが,イメージしやすいかもしれない.任意の領域 $S$ を
                設定しても,そこから生じる正味の磁束密度の量は0であることを,
                表現している.つまり,磁束密度の湧き出し場所が,$S$ 内部ではない
                ということである.といっても,湧き出しはどこか別の場所($S$ の外側)に
                あるということではない.$S$ は任意に設定できる閉曲面であるから,
                $S$ をどんな風にとっても,磁束密度の発生源は $S$ の内部ではないという
                ことになる.つまり,この世界の至る場所で,磁束密度の湧き出し量が
                正味0であるということだ.

                電場に対するガウスの法則は,電場の発生原因を電荷に押し付けいているのに対し,
                磁束密度に対するこの法則は,いわば「磁荷」が存在しないということを意味する.
                磁束密度に対して,なぜ「磁荷」がないのだろうかといった疑問も強いことと思う.
                実際,物理学者の中でもその存在を信じ,探している人もいるらしい.しかし,
                このノートでは,この現象を実験事実として認め,深入りは避けることとしたい
                    \footnote{
                        実は,特殊相対性理論を学習すると,磁束密度は,光速不変の原理と
                        特殊相対性理論から導かれるローレンツ力と,クーロンの法則から導
                        出することができてしまう.電場の発生原因は電荷にあり,電荷はク
                        ーロンの法則に従った動きをする.他方,物体が光の速さで運動する
                        さいには,ローレンツ変換に従う.とどの詰まりは,電荷が光速で運
                        動する際に,静止している観測者には,その周囲に磁束密度が分布し
                        ているようにみえてしまうのである.これは,ローレンツ力に関連す
                        る.先に,ローレンツ力は,観測者と電荷との相対速度 $\bv$ が関
                        係していることを見た.この相対速度こそが,磁束密度の存在を支え
                        ているのである.磁束密度は,観測者と電荷の相対的な関係により,
                        生じるものであると,考えることもできるのである.このことを定量
                        的に考えることは,相対論を学習した後で行うことにしよう.
                    }.
            \end{mysmallsec}

%           %==================================================================
%           %  SubsubSection
%           %==================================================================
            \subsubsection{アンペール$=$マクスウェルの法則の式:積分形}
            \begin{mysmallsec}{数式}
                アンペール$=$マクスウェルの法則の式の,積分形の式は次式の通り.
                \begin{align}
                    \oint_{l} \bB \cdot \bt \df l
                    = \sint_{S_{l}} \left( \bi + \varepsilon_{0}\frac{\rd \bE}{\rd t} \right) \cdot \bn \df S_{l}.
                \end{align}
            \end{mysmallsec}

            \begin{mysmallsec}{法則のイメージ}
                この式の意味するところは,ある閉曲線 $l$ の内側を観察したとき,
                磁束密度が生じているのであれば,$l$ を境界とする曲面 $S_{l}$ を
                電流が貫いているか,もしくは電場の時間変化が生じているということ
                である.電場の時間変化と電流は,その周囲に磁束密度を発生させるのである.

                この積分形の式の左辺により,生じる磁束密度の総量が計算される.右辺は,
                閉曲面 $S_{l}$ に生じている全電流と電場の時間変化の総量が計算される.
                つまり,全電流と電場の時間変化の総量を計算することで,生じる磁束密度の総量
                がわかるのである.
            \end{mysmallsec}

%           %==================================================================
%           %  SubsubSection
%           %==================================================================
            \subsubsection{ファラデーの電磁誘導の法則の式:積分形}
            \begin{mysmallsec}{数式}
                ファラデーの電磁誘導の法則の式の,積分形の式は次式の通り.
                \begin{align}
                    \oint_{l} \bE \cdot \bt \df l
                    = -\frac{\rd}{\rd t} \sint_{S_{l}} \bB \cdot \bn \df S_{l}.
                \end{align}
            \end{mysmallsec}

            \begin{mysmallsec}{法則のイメージ}
                磁束密度の時間変化は,その周囲に回転する電場を発生させることを,
                この式は意味している.

                微分形の式は,生じているか否かを判定するものであるが,
                この積分形の式は,生じる電場の総量が計算できる.
                閉ループ $l$ に導線を重ねあわせて導線の輪を作り,
                その導線の輪の内側において磁束密度を変化させると,
                導線に起電力が生じる.この起電力の大きさは,どの程度磁束密度を
                変化させたかによって決まる.その計算式が,積分形の電磁誘導の法則の
                式である.
            \end{mysmallsec}
