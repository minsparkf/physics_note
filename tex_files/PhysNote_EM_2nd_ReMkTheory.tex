%===================================================================================================
%  Chapter : 電磁気学の再構築
%  説明    : これまでは,マクスウェル方程式を導くように電磁気学を学習してきた.しかし,現在の電磁気学
%            の体系はマクスウェル方程式を基礎にした記述である.そこで,今までの電磁気学のイメージを取
%            り払うよう,注意を呼びかける.
%   %==========================================================================
%   %  Section
%   %==========================================================================
    \section{クーロンの法則}
        クーロンの法則は,電気の性質を示す最も基礎的な法則である.
        電気の性質をもつ物体を,\textbf{電荷} とよぶ
            \footnote{
                「電荷」と書いたとき,その物体の他の性質(大きさ,重さ等)は
                無視して,電気的性質のみに着目する.
            }.
        \begin{myshadebox}{クーロンの法則}
            クローンの法則が示す電気的性質とは,以下の通り.
            \begin{itemize}
                \item   電気には,"正(Plus)"と"負(Minus)"の2種類が存在する
                \item   2つの電荷が異種(正と負)であれば,各々の電荷
                      は互いに引き合う向きに力を受ける
                \item   2つの電荷が同種(正と正/負と負)であれば,各々の電荷
                      は互いに反発し合う向きに力を受ける
                \item   2つの電荷が受ける力の大きさは等しく,力の向きは互いに逆向きである
                \item   2つの電荷が受ける力の大きさは,2つの電荷の電気量の積に比例する
                \item   2つの電荷が受ける力の大きさは,2つの電荷の距離の2乗に反比例する
            \end{itemize}
        \end{myshadebox}
        \begin{myshadebox}{クーロン力}
            ある空間に2つの電荷 $q_{1}$,$q_{2}$ が,それぞれ
            位置 $\br_{1}$,$\br_{2}$ に
            存在するとき,この2つの電荷には,次式で
            表されるような力が作用する.この力のこと
            を \textbf{クーロン力} という.

            電荷 $q_{1}$ に対しては,
               \begin{align}
                   \bF_{12} =
                       \frac{1}{4\pi\varepsilon_{0}} \frac{q_{1}q_{2}}{r^{2}}
                           \frac{\br_{1} - \br_{2}}{|\br_{1} - \br_{2}|}
               \end{align}
            という力が働く.

            ここに,$\varepsilon_{0}$ は真空の誘電率である.
        \end{myshadebox}

        \begin{figure}[hbt]
            \begin{tabular}{cc}
                \begin{minipage}{0.5\hsize}
                    \begin{center}
                        \includegraphicsdouble{coulombs_low1.pdf}

                        (A)
                    \end{center}
                \end{minipage}
                \begin{minipage}{0.5\hsize}
                    \begin{center}
                        \includegraphicsdouble{coulombs_low2.pdf}

                        (B)
                    \end{center}
                \end{minipage}
            \end{tabular}
                        \caption{クーロン力((図\ref{fig:coulombs_low}) 再揚)}
        \end{figure}


%   %==========================================================================
%   %  Section
%   %==========================================================================
    \section{ローレンツ変換}

%   %==========================================================================
%   %  Section
%   %==========================================================================
    \section{電場と磁束密度の定義}

%   %==========================================================================
%   %  Section
%   %==========================================================================
    \section{マクスウェル方程式}
%   %==========================================================================
%   %  SubSection
%   %==========================================================================
        \subsection{ファラデーの電磁誘導の法則}

%   %==========================================================================
%   %  SubSection
%   %==========================================================================
        \subsection{アンペール$=$マクスウェルの法則}

%   %==========================================================================
%   %  SubSection
%   %==========================================================================
        \subsection{電場に対するガウスの法則}

%   %==========================================================================
%   %  SubSection
%   %==========================================================================
        \subsection{磁束密度に対するガウスの法則}

%   %==========================================================================
%   %  Section
%   %==========================================================================
    \section{電荷保存の法則}
    \begin{mysmallsec}{マクスウェル方程式は,その内部に電荷保存則を含んでいる}
        マクスウェルの法則(マクスウェル方程式)は,その内部に,電荷保存則を含んでいる.
        つまり,マクスウェル方程式から,電荷保存則を導けるのである.
        このノートでは前に,電荷保存の法則を仮定して マクスウェル方程式を
        導いたわけだが,今回はマクスウェル方程式を仮定して,電荷保存の法則を導出する
            \footnote{
                電荷保存則を仮定して,マクスウェル方程式を見つけ出したのだから,
                その内部に電荷保存則を含んでいることは,当然である.
            }.
        今日では,マクスウェル方程式を仮定して電磁気現象を説明するという立場があたりまえになっている.
        このようなことから,電荷保存の法則は法則ではなく,
        定理(法則から導かれる現象)として捉えるべきだ.
    \end{mysmallsec}

    \begin{mysmallsec}{導出}
        アンペール$=$マクスウェルの法則の両辺に,$\ddiv$ をとると
        \footnote{
            電流や変位電流の発散が電荷密度の移動であると表現したいので,
            このような操作をする.
        },
        \begin{align*}
            \ddiv\mu_{0}\left( \bi + \varepsilon_{0}\frac{\rd \bE}{\rd t} \right)
            &=\ddiv\drot\bB \\
            \Leftrightarrow
                \mu_{0}\left(\ddiv\bi+ \varepsilon_{0}\frac{\rd (\ddiv\bE)}{\rd t} \right)
            &=\ddiv\drot\bB
        \end{align*}
        である.ここで, $\ddiv$ は空間に関する作用であり, $\rd/\rd t$ は時間に対する作用であるから,
        両者は可換である.
        これに加えて,電場に対するガウスの法則 $\ddiv\bE=\rho/\varepsilon_{0}$ から,
        \begin{align*}
                \mu_{0}\left(\ddiv\bi+\frac{\rd }{\rd t}\rho\right)=\mathrm{div\,rot}\bB
        \end{align*}
        である.最後に,ベクトル解析の恒等式;任意の $\bM$ に対して,$\mathrm{div\,rot}\bM:= 0$ で
        あることを考慮すれば,
        \begin{align}
            \ddiv\bi+ \frac{\rd }{\rd t} \rho =0
        \end{align}
        を得る.これは電荷保存の法則を表現する式である.
    \end{mysmallsec}


%   %==========================================================================
%   %  Section
%   %==========================================================================
    \section{マクスウェル方程式の解}
    \begin{mysmallsec}{変数の個数と,方程式の個数が一致しない}
        マクスウェル方程式をなす4つの方程式は,電場と磁束密度を変数とする方程式である.
        電場と磁束密度はそれぞれ3つの方向成分をもっている.従って,マクスウェル方程式
        によって決定されるべき未知関数は,$3\times2$ で6個である.
        具体的には,$E_{x}(\br,\,t)$,$E_{y}(\br,\,t)$,$E_{z}(\br,\,t)$,
        $B_{x}(\br,\,t)$,$B_{y}(\br,\,t)$,$B_{z}(\br,\,t)$ である.

        ところで,マクスウェル方程式を見てみると,これらは成分で見れば8この方程式で
        あると言える.具体的にいえば,電磁誘導の法則で3つ,アンペール$=$マクスウェルの法則で3つ,
        電場に対するガウスの法則で1つ,磁束密度に対するガウスの法則で1つの
        合計 $3+3+1+1$ で8個となる
            \footnote{
                電磁誘導の法則やアンペールの法則は,ベクトルで記述されている方程式で
                あるから,その方程式には3つの方向成分($x$ 成分,$y$ 成分,$z$ 成分)
                をもっている.従って,法則の各々に,未知関数が3ずつ含まれているので
                ある.また,2つのガウスの法則はスカラー方程式であるので,未知関数は
                各々の法則で1つである.
            }.

        つまり,\textbf{方程式の個数の方が変数の個数よりも
        2つ多いということになり,矛盾なく解が求まるかどうか}といった
        疑問が生じるのである.結論から先にいえば,矛盾なく解が求まるのである.
        というのも,電場に対するガウスの法則が電場の初期条件を与え,
        磁束密度に対するガウスの法則が磁束密度の初期条件を与える式と
        考えられるのである.
    \end{mysmallsec}

    \begin{mysmallsec}{数式により,確認}
        以上のことを数式で表現すれば次のようになる.

        まず,ファラデーの電磁誘導の法則の両辺に $\ddiv$ をとる.
        \begin{align}
            \mathrm{div\,rot\,} \bE
            &=\mathrm{div\,}\left( -\frac{\rd \bB}{\rd t}\right) \notag \\
            \Leftrightarrow \quad
            \mathrm{div\,rot\,} \bE
            &=-\frac{\rd \left(\mathrm{div\,}\bB\right)}{\rd t}
        \end{align}

        さて,ベクトル解析の公式から,任意のベクトルを $\bC$ として,
        $\mathrm{div\,rot\,} \bC:=0$ が成立しているので,上式の右辺は
        恒等的に0となって,結局,
        \begin{align}
            -\frac{\rd \left(\mathrm{div\,}\bB\right)}{\rd t} =0
        \end{align}
        となる.磁束密度の初期条件として,磁束密度に対するガウスの法則を使えば,
        電磁場が変動してもファラデーの法則は自動的に成り立つのである.

        また電場についても,アンペール$=$マクスウェルの法則の両辺に $\ddiv$ をとれば,
        \begin{align}
            \mathrm{div\,}\left(\mathrm{rot\,}\bB\right)
            &=
            \mathrm{div\,}\mu_{0}
            \left(\bi+
                \varepsilon_{0}\frac{\rd \bE}{\rd t}
            \right) \notag \\
            \Leftrightarrow \quad
            \mathrm{div\,rot\,}\bB
            &=
            \mu_{0}\mathrm{div\,}\bi+
            \varepsilon_{0}\mu_{0}\frac{\rd (\mathrm{div\,}\bE)}{\rd t}
        \end{align}
        である.

        さて,ベクトル解析の公式から,任意のベクトルを $\bC$ として,
        $\mathrm{div\,rot\,} \bC:=0$ が成立しているので,上式の右辺は
        恒等的に0となって,
        \begin{align}
            \mathrm{div\,}\bi+
            \varepsilon_{0}\frac{\rd (\mathrm{div\,}\bE)}{\rd t}
            &=
            0
        \end{align}
        と書ける.ここで,
        電荷保存の法則より,$\ddiv\bi=-\rd/\rd t\rho$ を上式考慮すると,
        \begin{align}\label{eq:Gauss_first}
            \frac{\rd }{\rd t}\rho
            -\varepsilon_{0}\frac{\rd (\mathrm{div\,}\bE)}{\rd t}
            &=
            0 \notag \\
            \Leftrightarrow \quad
            \frac{\rd }{\rd t}
            \left(\rho
                -\varepsilon_{0}\mathrm{div\,}\bE
            \right)
            &=
            0
        \end{align}
        を得る.電場に対するガウスの法則を電場の初期条件として考えるならば,
        アンペール$=$マクスウェルの法則は,電場がこの後時間変化しても,成立する.

        つまり,上式(\ref{eq:Gauss_first})を時間積分して,
            \begin{equation*}
                \rho - \varepsilon_{0} \ddiv \bE = C\,,\quad \mbox{(Cは積分定数)}
            \end{equation*}
        だが,初期条件がガウスの法則によって書かれるものであるとするなら,定数 $C=0$ であり,
            \begin{equation*}
                \rho - \varepsilon_{0} \ddiv \bE =0
            \end{equation*}
            \begin{equation*}
                 \ddiv \bE =\frac{\rho}{\varepsilon_{0}}
            \end{equation*}
        さて以上によって,マクスウェル方程式を矛盾することなく解くことができるということが
        示された.
    \end{mysmallsec}
