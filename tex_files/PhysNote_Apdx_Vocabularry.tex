%===================================================================================================
%  Chapter : 語彙メモ
%  説明    : 本や辞典などで知った語彙をノートする
%===================================================================================================

%   %==========================================================================
%   %  Section
%   %==========================================================================
        \section{故事成語}
            \subsection{\ruby{華胥}{かしょ}の\ruby{国}{くに}}
            理想的な世の中のこと.また,心地よい夢の正解のこと.列子から.
                
            \subsection{\ruby{胡蝶}{こちょう}の\ruby{夢}{ゆめ}}
            夢と現実の違いは,実ははっきりとしないということ.また,人生のはかないことのたとえ.荘子から.
                
            \subsection{\ruby{上善}{じょうぜん}は\ruby{水}{みず}のごとし}
            最高の善を,水の性質にたとえたことば.老子から.

            \subsection{\ruby{驥足}{きそく}を\ruby{伸}{の}ばす}
            優秀な才能を存分に発揮することのたとえ.また,自由気ままに行動すること.三国志・蜀書-龐統伝から.

            \subsection{\ruby{驥尾}{きび}に\ruby{付}{ふ}す}
            それほど才能がない者でも,才能があるものについて行けば,何かをやりとげることができることのたとえ.史記-伯夷伝から.

            \subsection{\ruby{木}{き}に\ruby{縁}{よ}りて\ruby{魚}{さかな}を\ruby{求}{もと}む}
            やり方をまちがえると,何も得られないことのたとえ.的外れで,愚かな行為のたとえ.孟子から.

            \subsection{\ruby{人間}{じんかん}\ruby{到}{いた}る\ruby{所}{ところ}に\ruby{青山}{せいざん}\ruby{有}{あ}り}
            骨を埋めるところはどこにでもある.大望を実現するためには,故郷にこだわらず,広い世間に出て活動すべきである,ということ.

            \subsection{\ruby{過}{す}ぎたるは\ruby{及}{およ}ばざるがごとし}
            度が過ぎたものは,足りないものと同様によくない.ものごとには程よさが大切ということ.やりすぎはよくない.足りないのもよくない.ちょうどよいのがいい.論語から.