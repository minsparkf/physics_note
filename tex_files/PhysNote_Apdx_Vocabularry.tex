%===================================================================================================
%  Chapter : 語彙メモ
%  説明    : 本や辞典などで知った語彙をノートする
%===================================================================================================

%   %==========================================================================
%   %  Section
%   %==========================================================================
        \section{故事成語}
            \subsection{\ruby{華胥}{かしょ}の\ruby{国}{くに}}
            理想的な世の中のこと.また,心地よい夢の正解のこと.列子から.
                
            \subsection{\ruby{胡蝶}{こちょう}の\ruby{夢}{ゆめ}}
            夢と現実の違いは,実ははっきりとしないということ.また,人生のはかないことのたとえ.荘子から.
                
            \subsection{\ruby{上善}{じょうぜん}は\ruby{水}{みず}のごとし}
            最高の善を,水の性質にたとえたことば.老子から.

            \subsection{\ruby{驥足}{きそく}を\ruby{伸}{の}ばす}
            優秀な才能を存分に発揮することのたとえ.また,自由気ままに行動すること.三国志・蜀書-龐統伝から.

            \subsection{\ruby{驥尾}{きび}に\ruby{付}{ふ}す}
            それほど才能がない者でも,才能があるものについて行けば,何かをやりとげることができることのたとえ.史記-伯夷伝から.

            \subsection{\ruby{木}{き}に\ruby{縁}{よ}りて\ruby{魚}{さかな}を\ruby{求}{もと}む}
            やり方をまちがえると,何も得られないことのたとえ.的外れで,愚かな行為のたとえ.孟子から.

            \subsection{\ruby{人間}{じんかん}\ruby{到}{いた}る\ruby{所}{ところ}に\ruby{青山}{せいざん}\ruby{有}{あ}り}
            骨を埋めるところはどこにでもある.大望を実現するためには,故郷にこだわらず,広い世間に出て活動すべきである,ということ.

            \subsection{\ruby{過}{す}ぎたるは\ruby{及}{およ}ばざるがごとし}
            度が過ぎたものは,足りないものと同様によくない.ものごとには程よさが大切ということ.やりすぎはよくない.足りないのもよくない.ちょうどよいのがいい.論語から.

        \section{メモ}
        \subsection{世界平和評議会(1949年--)}
        世界平和評議会(World Peace Council)は、ポーランド(ワルシャワ)で設立された組織。
        冷戦当時に東側諸国(社会主義国)政府の主導で設立された。日本からは日本共産党系の日本平和委員会が加入している。
        
        \subsection{平和擁護世界大会(1950年)}
        1949年4月,パリで平和擁護世界大会(World Congress of Partisans of Peace)の第1回大会が開かれた.
        冷戦が激化し始める中,フランス政府が東側諸国代表の入国を拒否したため,東側諸国の1国であるチェコスロバキアのプラハでも同時開催された.
        
        1950年’11月16日-22日),ポーランド(ワルシャワで)第2回平和擁護世界大会が開かれた.81か国からの参加者があった(2065人).
        冷戦下において,西側諸国
        \footnote{西側諸国:アメリカ/日本/西ドイツなど}
        と対峙するの東側諸国
        \footnote{東側諸国:ソビエト連邦/ポーランド/東ドイツなど}
        の強い影響を受ける団体であることから,日本と西ドイツの再軍備を非難された.
        一方で,東ドイツやポーランドの再軍備については触れられていなかった.
        また,「世界平和評議会」の設置を決定し,核兵器使用禁止を求める「ストックホルム・アピール」を宣言し,その後5億の署名を集めた.

        \subsection{ストックホルムアピール(1950年)}
        平和擁護世界大会で核兵器の禁止が求められた.さらに核兵器を使用した者を戦争犯罪人とみなすと表明した.
        5億人の賛同署名がある.この訴えは\textbf{ストックホルムアピール}いわれる.
  
        \subsection{ラッセル・アインシュタイン宣言(1955-1957年)}
        ラッセルとアインシュタインの連名により核兵器と戦争の廃絶・禁止を求められた宣言を
        \textbf{ラッセル・アインシュタイン宣言}という.
        パグウォッシュ会議
        \footnote{
            科学と世界の諸問題に関するパグウォッシュ会議(Pugwash Conferences on Science and World Affairs).
        }
        (1957年)の契機となった.
    
 
