%******************************************************************************
%  Newcommand
%******************************************************************************
%================================================
% * 章・節などの表示変更
%================================================
\renewcommand{\thesection}{\thechapter.\arabic{section}}
\renewcommand{\thesubsection}{\thesection.\arabic{subsection}}
\renewcommand{\tablename}{表\;}
\renewcommand{\figurename}{図\;}

\renewcommand{\bibname}{参考図書・教科書等}

%\renewcommand{\prechaptername}{第}
%\renewcommand{\postchaptername}{章}

\renewcommand{\labelenumi}{(\arabic{enumi})\;}
\renewcommand{\labelenumii}{(\alph{enumii})\;}
\renewcommand{\labelenumiii}{(\roman{enumiii})\;}
 %% - 一般の脚注のスタイル変更
%\renewcommand\thefootnote{$\mbox{}^{\clubsuit}$\kern1pt\arabic{footnote})\,} % パターン1: 三つ葉
\renewcommand\thefootnote{$\mbox{}^{\blacktriangleright}$\kern1pt\arabic{footnote})\,} % パターン2: 黒三角形
\renewcommand{\footnoterule}{%
  \vspace{1pt}                      % 線から上の幅
  \noindent\rule{0.25\textwidth}{0.5pt}   % 線の長さ,太さ
  \vspace{2pt}                     % 線から下の幅
}

\renewcommand{\textreferencemark}{\textreferencemark\,\,}
\newcommand{\komemark}{\textreferencemark}
 %% - 行間の調整
%\renewcommand{\baselinestretch}{1.05}%普通のx倍の行間間隔

%================================================
% * 演算子などの省略記述用
%================================================
% * ベクトル解析用の演算子
\newcommand{\drot}{\mathrm{rot}\,}
\newcommand{\ddiv}{\mathrm{div}\,}
\newcommand{\dgrad}{\mathrm{grad}\,}
% * ゼロベクトル
\newcommand{\bzero}{\textbf{0}}
% * 面積分
\newcommand{\sint}{\int\mspace{-11mu}\int}
% * 体積積分
\newcommand{\vint}{\int\mspace{-11mu}\int\mspace{-11mu}\int}
\newcommand{\OmegaS}{\Omega_{S}}
% * 微分記号(立った"d")
\newcommand{\df}{\,\mathrm{d}}
% * 偏微分記号
\newcommand{\rd}{\partial}
% * ダランべルシアン
\newcommand{\Dal}{\Box\,}
% * ラプラシアン
\newcommand{\Lap}{\triangle}
% * 自然対数の底
\newcommand{\e}{\mathrm{e}}
%\newcommand{\exp}{\mathrm{exp}}
% * 量子論的ハミルトニアン記号:$\qH$
\newcommand{\qH}{\mathcal{H}\,}
% * ラグランジアン密度
\newcommand{\qL}{\mathcal{L}\,}
% * 量子論的運動量演算子:$\dpx , \dpy , \dpz$
\newcommand{\dpx}{-i\hbar \frac{\partial}{\partial x}\,}
\newcommand{\dpy}{-i\hbar \frac{\partial}{\partial y}\,}
\newcommand{\dpz}{-i\hbar \frac{\partial}{\partial y}\,}
% * 量子論的エネルギー演算子:$\dE$
\newcommand{\dE}{i\hbar \frac{\partial}{\partial t}\,}
% * 量子論的ハミルトニアン演算子:$\dH$
\newcommand{\dH}{-i\hbar \nabla \,}
% * その他
\newcommand{\Shrps}{$\sharp\;$} % シャープ記号#
\newcommand{\Fig}{Fig\,\,} % 参照図を明示するときに使用する:ex.) 図\ref{xxx}
\newcommand{\Table}{図\,\,}
\newcommand{\circC}{${}^{\circ}$C}

%================================================
% * 標準的な画像のサイズ
%================================================
% 4:3 Default Size
\newcommand{\includegraphicsdefault}[1]{\includegraphics[keepaspectratio, width=4.00cm, height=3.00cm, clip]{#1}}
% 4:3 Middle Size
\newcommand{\includegraphicsmiddle}[1]{\includegraphics[keepaspectratio, width=6.00cm, height=4.5cm, clip]{#1}}
% 4:3 Large Size
\newcommand{\includegraphicslarge}[1]{\includegraphics[keepaspectratio, width=7.2cm, height=4.98cm, clip]{#1}}
% 2:1 Default Size
\newcommand{\includegraphicshalfmid}[1]{\includegraphics[keepaspectratio, width=6cm, height=3cm, clip]{#1}}
% 1:1 Squeare Default Size
\newcommand{\includegraphicssqmid}[1]{\includegraphics[keepaspectratio, width=4.00cm, height=4.00cm, clip]{#1}}
% 1:1 Squeare Default Size
\newcommand{\includegraphicssqmlrg}[1]{\includegraphics[keepaspectratio, width=7.2cm, height=7.2cm, clip]{#1}}

% Two Graphics Size
\newcommand{\includegraphicsdouble}[1]{\includegraphics[keepaspectratio, width=3.2cm, height=2.4cm, clip]{#1}}


%================================================
% * アルファベットの太字(ベクトルとか)
%================================================
\newcommand{\ba}{\boldsymbol{a}}
\newcommand{\bb}{\boldsymbol{b}}
\newcommand{\bc}{\boldsymbol{c}}
\newcommand{\bd}{\boldsymbol{d}}
\newcommand{\be}{\boldsymbol{e}}
\newcommand{\bg}{\boldsymbol{g}}
\newcommand{\bldf}{\boldsymbol{f}}
\newcommand{\bj}{\boldsymbol{j}}
\newcommand{\bp}{\boldsymbol{p}}
\newcommand{\br}{\boldsymbol{r}}
\newcommand{\bv}{\boldsymbol{v}}
\newcommand{\bx}{\boldsymbol{x}}
\newcommand{\by}{\boldsymbol{y}}
\newcommand{\bz}{\boldsymbol{z}}
\newcommand{\bn}{\boldsymbol{n}}
\newcommand{\bE}{\boldsymbol{E}}
\newcommand{\bF}{\boldsymbol{F}}
\newcommand{\bC}{\boldsymbol{C}}
\newcommand{\bB}{\boldsymbol{B}}
\newcommand{\bA}{\boldsymbol{A}}
\newcommand{\bi}{\boldsymbol{i}}
\newcommand{\bt}{\boldsymbol{t}}
\newcommand{\bk}{\boldsymbol{k}}
\newcommand{\bl}{\boldsymbol{l}}
\newcommand{\bS}{\boldsymbol{S}}
\newcommand{\bI}{\boldsymbol{I}}
\newcommand{\bH}{\boldsymbol{H}}
\newcommand{\bP}{\boldsymbol{P}}
\newcommand{\bD}{\boldsymbol{D}}
\newcommand{\bX}{\boldsymbol{X}}
\newcommand{\bY}{\boldsymbol{Y}}
\newcommand{\bL}{\boldsymbol{L}}
\newcommand{\bM}{\boldsymbol{M}}
\newcommand{\bN}{\boldsymbol{N}}
\newcommand{\bo}{\boldsymbol{o}}
\newcommand{\bU}{\boldsymbol{U}}
\newcommand{\bV}{\boldsymbol{V}}
\newcommand{\bW}{\boldsymbol{W}}
\newcommand{\bZ}{\boldsymbol{Z}}
\newcommand{\setN}{\mathbb{N}}
\newcommand{\setC}{\mathbb{C}}
\newcommand{\setZ}{\mathbb{Z}}
\newcommand{\setR}{\mathbb{R}}
\newcommand{\setA}{\mathbb{A}}
\newcommand{\setB}{\mathbb{B}}
\newcommand{\setD}{\mathbb{D}}
\newcommand{\setE}{\mathbb{E}}
\newcommand{\setF}{\mathbb{F}}
\newcommand{\bldi}[1]{\textit{\textbf{{#1}}}}
\newcommand{\bld}[1]{\textbf{{#1}}}

%================================================
% * 花文字
%================================================
\newcommand{\flwA}{\mathcal{A}}
\newcommand{\flwB}{\mathcal{B}}
\newcommand{\flwC}{\mathcal{C}}
\newcommand{\flwD}{\mathcal{D}}
\newcommand{\flwE}{\mathcal{E}} % 起電力
\newcommand{\flwF}{\mathcal{F}} % 一般化された力とか
\newcommand{\flwH}{\mathcal{H}} %
\newcommand{\flwp}{\mathcal{p}} % 一般化された運動量
\newcommand{\flwL}{\mathcal{L}}

%================================================
% * ギリシア数字作成
%================================================
\newcommand{\I}{I}
\newcommand{\II}{I\hspace{-.1em}I}
\newcommand{\III}{I\hspace{-.1em}I\hspace{-.1em}I}
\newcommand{\IV}{I\hspace{-.1em}V }
\newcommand{\V}{V}
\newcommand{\VI}{V\hspace{-.1em}I}
\newcommand{\VII}{V\hspace{-.1em}I\hspace{-.1em}I}
\newcommand{\VIII}{V\hspace{-.1em}I\hspace{-.1em}I\hspace{-.1em}I }
\newcommand{\X}{X}
%再定義1(数式番号の表現 chap-sub-subsub)
%\makeatletter
% \renewcommand{\theequation}{%
  % \thesection.\arabic{equation}}
  %\@addtoreset{equation}{section}
 %\makeatother

%================================================
% * 新しいカウンタの生成
%================================================
% * 宣言
\newcounter{Memo}           %
\newcounter{Snumber}        % "memo" の番号
\newcounter{myshade}        % 物理部分の重要事項の番号
\newcounter{PreAttentionNo} % 記入方法の諸注意の番号
\newcounter{AboutThisNoteNo}% 「このノートにいて」の番号
% * 初期値設定
\setcounter{Memo}{1}
\setcounter{Snumber}{1}
\setcounter{myshade}{1}
\setcounter{PreAttentionNo}{1}
\setcounter{AboutThisNoteNo}{1}

%================================================
% * カウンタの再設定
%================================================
% * 目次の表示の深さを設定する
\setcounter{secnumdepth}{6}
\setcounter{tocdepth}{4}
% * 立て揃え で改行を許す:0から4まで
%    0      ; 絶対改行しない
%    1から3 ; 適度に改行する.数字が高いほど,
%             基準が緩くなる
%    4      ; 必ず改行する
\allowdisplaybreaks[4]

%================================================
% 図のcaptionの : をとる
%================================================
\makeatletter
\long\def\@makecaption#1#2{%
\vskip\abovecaptionskip%
%\sbox\@tempboxa{#1: #2}% <--- original
\sbox\@tempboxa{#1\;\; #2}
\ifdim \wd\@tempboxa >\hsize%
%#1: #2\par <--- original
#1 #2\par
\else
\global \@minipagefalse
\hb@xt@\hsize{\hfil\box\@tempboxa\hfil}%
\fi
\vskip\belowcaptionskip}
\makeatother
%------------------------------------------------

%******************************************************************************
%  Newtheorem
%******************************************************************************
%定義や定理の出力は,newtheorem 環境を使用します.
%定義の場合,プリアンブルに次の内容を記述します.
%
%\newtheorem{dfn}{Definition}[引数]

%「定義」と表示させたい場合は,
%Definition を 定義 に書き換えます.
%
%定義番号に (2.1) のように章番号をつけたい場合は,
%引数に,[chapter]を入力します.
%章立ての文書で (1.3.2) のように節番号を付けたい場合は,
%引数にsection を入力します.
%
%% 使い方:定理の場合
%\begin{thm}
%集合 $A, B$ について,$A$ から $B$ への単射および $B$ から $A$ への単射がともに
%存在すれば,$A$ と $B$ の濃度が等しい.
%\end{thm}
\theoremstyle{plain}
\newtheorem{thm}{定理}[section]
\newtheorem{lem}{補題}[section]
\newtheorem{cor}{系}[section]
\newtheorem{prp}{命題}[section]
%
\theoremstyle{definition}
\newtheorem{dfn}{定義}[section]
%
\theoremstyle{remark}
\newtheorem{rem}{注意}
\newtheorem{prf}{証明}

%******************************************************************************
% Newenvironment
%******************************************************************************
% * Environment: \memo
\newenvironment{memo}[1]
{ % before
    \addcontentsline{toc}{subsubsection}{\Shrps\textit{memo} No.\theSnumber : #1} % 目次
    \subsubsection*{\Shrps\textit{memo} No.\theSnumber : \emph{#1}}
    %\addcontentsline{toc}{subsubsection}{\;\;\;\Shrps\textit{memo} \; #1}
    %\subsubsection*{\;\;\;\Shrps\textit{memo} \;\;\; \emph{#1}}
    \stepcounter{Snumber}
    \small
}
{ % after
    \par \normalsize
}

% * Environment: \myshadebox
\newenvironment{myshadebox}[1]
{ % before
    %\addcontentsline{toc}{subsubsection}{\;\S\;Point No.\themyshade :\; #1} % 目次
    \leavevmode \\  \leavevmode \\
    \begin{shadebox}
        %\;\textbf{\ding{42}\;Point No.\themyshade :\, #1} \leavevmode \\
                \textbf{\;Point \themyshade :\, \textbf{#1}} \vspace{1mm} \\
        \leavevmode
        \stepcounter{myshade}
}
{ % after
    \end{shadebox}
    \leavevmode \\
}

% * Environment: \comment
\newenvironment{mycomment}
{ % before
    \par \small
    \textbf{コメント$\quad$}
}
{ % after
    \newline \par \normalsize
}

% * Environment: \mysmallsec
\newenvironment{mysmallsec}[1]
{ % before
%    \addcontentsline{toc}{subsubsection}{\$ #1} % 目次
%    \subsubsection*{\$ #1}
    \subsubsection*{\P \; #1}
}

% * Environment: \pre_attention
\newenvironment{preattention}[1]
{ % before
    \addcontentsline{toc}{section}{(\thePreAttentionNo) #1} % 目次
    \subsubsection*{(\thePreAttentionNo) \emph{#1}}
    \stepcounter{PreAttentionNo}
    \label{my:preattention:#1}
}
{ % after
}

% * Environment: \aboutthisnote
\newenvironment{aboutthisnote}[1]
{ % before
    %\addcontentsline{toc}{section}{\$\theAboutThisNoteNo #1} % 目次
    \subsubsection*{\$\theAboutThisNoteNo \emph{#1}}
    \stepcounter{AboutThisNoteNo}
    \small
}
{ % after
    \par \normalsize
}
