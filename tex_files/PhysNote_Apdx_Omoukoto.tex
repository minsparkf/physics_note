\normalsize
%%**************************************************************************************************
%%
%% File Name : PhysNote_MessageFirst.tex
%% 説明      : 物理学の導入をかねた,古典力学(ニュートン力学と解析力学)を学習する.
%%
%%**************************************************************************************************

%===================================================================================================
%  Chapter : 素朴な疑問
%  説明    : 「考えるとはどういうことか」とかを考えてみる.
%===================================================================================================
    \chapter{素朴な疑問}
%   %-----------------------------------------------------------------------------------------------
%   %  Input
%   %    File Name : PhysNote_Math_MsgFirst.tex
%   %    説明      : 考えるって何だろう.
%   %-----------------------------------------------------------------------------------------------
        %===================================================================================================
%  Chapter : はじめに
%  説明    : 「考えるとはどういうことか」とかを考えてみる.
%===================================================================================================
%   %==========================================================================
%   %  Section
%   %==========================================================================
        \section{最も基本的なこと}
            学問を学ぶにあたって,“最も基本的で信頼できること”を
            基礎にして,その上で学習を進めることは,当たり前のこと
            である.では,その“最も基本的で信頼できること”とは何
            だろうか.

            これは多分に哲学的な問題提起であるが,ここでは,今後,
            物理学の学習を進めていくにあたり,思想の最も基本的なよ
            りどころを確認するためのもので,哲学に深く介入すること
            はしない
                \footnote{
                    正直に書こう.哲学に介入することは,私のような
                    低レベルの頭では,不可能である.開き直って,言
                    うならば,そこまで深く考えてもしょうがない,思
                    うところもある.
                }.

            私は,最も基本的で信頼できることとして,「考えること」
            をあげたい.これは独我論てきな思想である.独我論とは,
            極端に言えば,この世界に存在を確信できるのは私の思考の
            みであり,私が今感じている温度や光などは,私の思考によ
            って感じていると錯覚しているのであって,実在しているの
            ではない,という考え方である.こう考えると,自分以外の
            人間とは,私の思考が作り出した幻想であり,実際にそこに
            いるわけではないということになる.そう,信じられるのは,
            今考えている私がここにあるということである.


%   %==========================================================================
%   %  Section
%   %==========================================================================
        \section{私の思想の根本}
            では,もう一段階突っ込んでみよう.「考える」ということ
            とは,どうすることなのだろうか.「考える」という動詞の
            使い方は,おおよそ次の様だろう.
                今晩の献立を考える.
                人の気持ちを考える.
                将来の進路を考える…
            などなど.
            考えるという作業を行っているとき,「言葉」を道具として
            使う.また,時には「図」を使って考えることもあるだろう
            が,これは単に言葉で考えるよりも図を用いた方が考えやす
            いからであり,言葉では思考不可能であるということではな
            い.思考はすべて言葉で表現できる(と信じる).

            私の,最も信頼できる唯一の基本的なことである,私の思考
            は言葉を用いて実行される.では,その次の疑問として,
            「言葉」とは何か,ということが生まれてこよう.

            この章では,「考えるとは何か」について,私が考えること,
            というか,思っていることを記述する.


%   %==========================================================================
%   %  Section
%   %==========================================================================
        \section{思考の道具}
            考えるという動作は,言葉を用いて行っていることを確認し
            た.言葉を用いて考えているので,この言葉の使用限界が思
            考の限界であるということになる
                \footnote{
                    Wittgensteinは「論理哲学論考」という著
                    作で,このことを詳しく論じている.後に,彼自身
                    によってこの著作は間違いであるとされてしまうの
                    であるが…このノートでは,そこまで深く入ら
                    ない.だって,とっても難しいから.
                }.

            単に「言葉」といっても,それは様々な形で存在する.英語
            やドイツ語,フランス語,イタリア語などたくさんだ.各国
            の人々は,自国のあるいは使い慣れた言葉で考えていること
            だろう.ここでは,どのような国の言葉も,その適用限界
            は変わらないと仮定して,話を進めて生きたい.多少,言葉
            の適用限界があったとしても,それは話にならないくらい,
            細かいことに過ぎないと信じる
                \footnote{
                    実際,各国の人々が同じように「考えて」いるとい
                    う現状からこのような仮定を設けてもよいと考えて
                    いる.ただし,ここでは,「考える」ということに
                    関して,文化や伝統,生活習慣などの影響は無視す
                    る.
                }.

             もちろん,言葉で説明できないこともある.ある種の“ひらめき”
             とか,もろもろの感情とかを言葉で表現することは難しいこ
             とである.いや,不可能といってもよい.しかし,考えると
             いうことに関しては,言葉のもつ機能は十分である考える.私
             は,「どんな思考も言葉にできる」と信じて,こ
             のノートを作成する.


%   %==========================================================================
%   %  Section
%   %==========================================================================
        \section{言語の曖昧さ}
            思考を言葉で記述できたとしよう.その次に問題となるのは,どれ
            だけ正確に思考できるか,ということだろう.いや,視点を変えて
            言い換えよう.私たちが行う多くの思考の中で,正しい思考とはど
            ういうものなのかを,整理しなければいけない.わけのわからない
            思考や,意味を成さない思考などを排除したいのである.


%   %==========================================================================
%   %  Section
%   %==========================================================================
        \section{日常言語}
            普段の生活で使っている言語は,曖昧な表現をすることが多い.曖昧
            表現というのは,人によって解釈が異なってしまう表現のことである.
            「美しい景色」だの,「大きな木」だのと言って,すべての人が同じ
            情景を浮かべることはまずない.これでは正しい思考が,十分ではな
            い他の人間に伝わることはない.

            しかし,このような例から,普段使っている言語は正しい思考に適し
            ていないと,判断してはいけない.事実,過去の多くの頭のいい学者
            さん達は,言語を用いて正しい思考をし,様々な学問を作り上げてい
            る.大切なのは,曖昧な表現を避けることである.ただ,言語には使
            い方によって,曖昧に表現できてしまうだけなのである.

            では,言語の曖昧な表現を使わないようにするには,どうしたらよい
            だろうか.まず考えつくのは,日常言語から万人が認める最も基本的
            な部分を抽出し,それを元に思考をすればよいことである.言語の最
            も基本的な部分とは,「論理」である.次に,論理について簡単に触
            れよう.




%===================================================================================================
%  Chapter : 論理学とか,数学とか
%  説明    : 物理学より,思考方法としてより基礎的なことである,論理・数学について考えてみる.
%===================================================================================================
    \chapter{論理学とか,数学とか}
%   %-----------------------------------------------------------------------------------------------
%   %  Input
%   %    File Name : PhysNote_CM_LogicMathEtc.tex
%   %    説明      : 数学とか論理学について,物理学のもっと基礎にある学問についての断片的な学習メモ.
%   %-----------------------------------------------------------------------------------------------
        %===================================================================================================
%  Chapter : 論理学とか,数学とか
%  説明    : 物理学より,思考方法としてより基礎的なことである,論理・数学について考えてみる.
%===================================================================================================

%   %==========================================================================
%   %  Section
%   %==========================================================================
        \section{論理}
            私たちが普段の生活で使っている言語のうち,曖昧な使い方を避けて
            ,万人に共通に伝わるようにしたい.そのためには,\textbf{論理}
            と言うものを考える必要がある.論理は,日常言語の中の一部にある
            .相手に自分の考えを伝えるとき,物事を筋道立てて伝えようとする
            だろう.自分の考えをできるだけ性格に相手に知ってほしいからであ
            る.さて,このとき私たちは,論理を使っているのである.論理は,
            何個かの公理
                \footnote{
                    公理:万人が認める事実のこと.何の反論なしに認めなけれ
                    ばならないことである.ある仮説を検証するとき,その仮説
                    の根拠をどんどん探ることになる --- AはBとCからなってい
                    て,BはDから,CはEを基にしている・・・と言うように---.
                    しかし,いつまでもこれが続くわけではない.いつかは,ど
                    うしようもなく“当たり前すぎて”,説明ができないことに
                    たどり着く.公理とは,その当たり前の事実を明示するもの
                    である.
                }
            と推論規則
                \footnote{
                    推論規則:ある仮定から,別の仮説を作り出せる規則のこと
                    .公理と同様,有無を言わさず認めさせられるものである.
                }
            を基に構成される.少数の公理と推論規則から,主張したいことがす
            べて主張できる体系を作ることが学問の目的である.これを思考経済
            と言ったりする.公理と推論規則は,少なければ少ないほどよい.



%   %==========================================================================
%   %  Section
%   %==========================================================================
        \section{論理学}
            この論理について,詳しく研究する学問に \textbf{論理学} がある
            .ある仮説を記述した文で,本当か嘘かをはっきりと区別できる文の
            ことを,\textbf{命題} という.論理学はいくつかの必要最小限の公
            理と推論規則を組み合わせ,命題の証明を繰り返し発展させていくも
            のである.命題と推論規則の組み合わせのことを,\textbf{公理系} と
            言う.この公理系には,次の3つの性質が備わっている必要がある.一
            つは \textbf{独立性} で,公理形の中のどの1つの公理を選んでも,
            他の公理からその公理を証明できないような性質である.
            二つ目は \textbf{完全性} と言われるもので,主張したい命題が,そ
            の公理系からすべて導けることである.三つ目は \textbf{無矛盾性} と
            言われるもので,公理系に互いに矛盾する公理を含んでいないことで
            ある.

            論理学とは考察の足固めである公理系を設計し,体系を作り上げて
            いく学問でもあるのだ.どれだけ詳しく説明できるかと言う疑問の最
            も根本的な部分の研究がここでなされる.



%   %==========================================================================
%   %  Section
%   %==========================================================================
        \section{数学}
            数学とは,論理に複素数を組み合わせた学問であると言える.その研究
            の対象はおもに,複素数である.複素数の一部には,自然数が含まれて
            いる.残念なことに,自然数を含む公理系には,完全性,無矛盾性が常
            に保たれていると言う保障がないことが知られている.この事実は
            G\"{o}delの \textbf{不完全性定理} とよばれている
                \footnote{
                    参考文献:廣瀬 健,横田 一正 [著],「ゲーデルの世界」,
                    鳴海社,2004
                }.

            集合論により無限を扱えるようになってきたころ,この無限を起因とし
            て様々なパラドクスが発見されることとなった.「自分自身を要素とし
            て含まないすべての集合」がその最も有名なものである.ちょっと考え
            てみよう.
                \begin{equation*}
                    \omega:\;\;\mbox{自分自身を要素として含まないすべての集合}
                \end{equation*}
            と定義しよう.そして,次の問題,すなわち
                \begin{equation*}
                    \mbox{問題}:\;\;\omega \mbox{自身は,} \omega \mbox{に含まれるか否か}
                \end{equation*}
            を考えてみよう.すぐに明らかな矛盾が見えてくるだろう.

            $\omega$ は自分自身に含まれると仮定してみる.すると,$\omega$ の
            定義「自分自身を要素として含まない」という仮定に矛盾する.では,
            $\omega$ は自分自身に含まれないと仮定したらどうか.実はこれでも
            矛盾が生じる.なぜなら,「自分自身要素としてを含まない」という
            定義上,仮定で「自分自身は含まない」言っているので,自分自身を
            含むべきだと言う結論が出てしまう.肯定的に仮定しようが,否
            定的に仮定しようが,どちらにしても結果はその仮定と矛盾するので
            ある.

            Russellらは,この問題を解決しようと階という概念を導入し,命題
            に自分自身を含むことのないように制限を加えた.しかし,問題はこれ
            だけではとどまらず,山のように残されていた.

            Hilbertはこのような問題の山を,数学の危機であると自覚し,これを
            解決しようと計画した.Hilbertはこの数学の危機と言われる問題を23
            個の命題にまとめた.そして彼は,この23の問題を証明し解決しようと
            呼びかけた.G\"{o}delの不完全性定理はこの23の問題のうちの一つ
                \footnote{
                    第2番目に掲げられていた問題だった.
                }
            の否定的な回答であった.

            だから,物理学に公理系を作成して,論理的に構成しようとしても,無駄
            である.しかし,物理学は自然の構成を探る学問だから,この点に関して
            はあまり気にすることはないと思う.


%   %==========================================================================
%   %  Section
%   %==========================================================================
        \section{物理学}
            物理学は,自然がどのようになっているかを探る学問である.
                \footnote{
                    "なぜ"自然が私たちの感じているようになっているのかを探る学
                    問では\textbf{ない}.
                }
            「なぜ(Why)」を問うのではなく,「どのように(How)」を問うのであ
            る.

            なぜ自然がこのように
                \footnote{
                    普段の生活で,私たちが感じている自然を思い浮かべてみて.
                }
            なっているのか,とか,なぜ宇宙があるのかとかを考えるのは哲学であって,
            物理学ではない.物理学ではこういう,"なぜ"を問うような疑問には答えられない
                \footnote{
                    ただ,"なぜ"という問を深めていく事は可能で,実際に,物理学の発展は
                    その繰り返しである.その様子は.これからの学習で実際に感じることになる.
                }.

            物理学とは,自然の性質を見つけるものである.この「性質」という言葉は
            物理学では \textbf{法則} と呼ばれている
                \footnote{
                    \textbf{法則}:後で詳しく記述する.
                }.
            自然の法則を,実験や数式を通して見つけ出すことが物理学の目的なのだ.

            自然はデタラメに変化しているのではなく,何か一定の法則に従ってい
            るということは,経験上理解できることと思う.例えば,特別に力を加え
            ない限り,高いところから低いところへ,物体は落ちていく.落とした消し
            ゴムは,拾わないと手元に戻ってこないのである.何でだろうか.この原因
            を探り,「法則」として記述するのである.

            このことについては,物理学を学び始めた段階ではまだ実感がわかないかも
            知れない.学習していく過程で,だんだんとわかることだろう.

            このノートでは物理学を学習することが目的である.自然はどのような法則
            に従って変化しているのだろうか.少しずつ考えることにしよう.




%===================================================================================================
%  Chapter : 他愛のない,思ったこと
%  説明    : 他愛のない,思ったことをメモしおこう.
%===================================================================================================
    \chapter{他愛のない,思ったこと}
%   %-----------------------------------------------------------------------------------------------
%   %  Input
%   %    File Name : PhysNote_Apdx_Omoukoto_TaaiNaiOmoukoto.tex
%   %    説明      : 
%   %-----------------------------------------------------------------------------------------------
        %===================================================================================================
%  Chapter : 他愛のない,思ったこと
%  説明    : 
%===================================================================================================

%   %==========================================================================
%   %  Section
%   %==========================================================================
        \section{生まれ変わる?}
            「生まれ変われるとしたら,次は何になっていたい?」と聞かれることが
            ある.この質問には,何も考えずに答えることが,コミュニケーションを
            円滑にする為に良いのだが,やはり引っかかる部分がある.

            引っかかることとは,「生まれ変わる」ということの定義である.生まれ
            変われるか否かということも当然疑問なのだが,それよりももっと疑問な
            ことがある.生まれ変われることが可能かどうかという疑問には,おそら
            く答えることは不可能だろう.生まれ変わることが不可能であるとした
            ら,話はそれで終わりであるので,ここでは,生まれ変われることができ
            るものとして,話を進めたいと思う.

            疑問というのは,“今の記憶が忘れ去られていたとしても,生まれ変わったと言えるか”と
            いうものである.以前までの記憶がない以上,当然,自分自身には生まれ変わ
            ったという意識は生まれない.第三者的な立場にたって,他人の生まれ変
            わる瞬間を見たとしよう.その場合,生まれ変わることが可能だと,納得
            することだろう.しかし,その他人には以前までの記憶がなく,やはり,
            その人にとって,生まれ変わったという意識はないはずである.たとえ,その瞬間を見てい
            たと教えてやったとしても,その人は生まれ変わったのだと明確に認識することは不可能である.

            生まれ変わって以前までの記憶がない以上,たとえ本当に生まれ変わったのだとしても,
            自分自身にとっては,別人であると意識せざるを得ないと考えるのが,自然なのではなかろうか.
            実際,今の私には,こう考えることが一番妥当だと思っている.
            たとえ,生まれれ変わっていたとしても,記憶が残っていない以上,
            それはその人にとって別ものなのだと考えたい.

            まとめよう.求める答えは,生まれ変われるか否か,ということだったが,これには,
            答えることはできない.
            生まれ変わることができないのであれば,話はここで終了になる.もし,
            生まれ変われることがわかったとしても(他人が生まれ変わったことを見るなどして),
            自分自身では認識できないのであれば,それは生まれ変わったと考えるべきではない,
            と思う.

%   %==========================================================================
%   %  Section
%   %==========================================================================
        \section{教科書に書かれていること}
            物理学や化学$\cdot$生物学$\cdot$天文学などの自然現象を説明すべく,それを
            文字として記述し,本という形で記録できる.
            世の中には多くの専門書,教科書,解説書がある.しかし,
            どれをとっても自然現象をすべて説明するものはない.
            つまり,本を読んだところで,世界を理解できるわけではない.
            本を読んでわかることは,先人たちが苦心して築きあげてきた
            壮大な理論体系ではあるものの,自然現象についてのほんの僅かな
            ことでしかない.

            物理学を学ぶということは,物理学の論文や専門書,教科書を
            読むということではなく,実際の自然現象に触れるということ
            である.そして,なぜだろうと疑問に感じることであり,
            さらに,それを解き明かしたいと思うことである.

            物理学を学びたいから物理学の教科書を読む,なんてことは,
            甚だ見当違いである.物理学は自然現象を説明する理論であり,
            つまり,実際に起こっている現象を説明しようとするものである.
            重要なのは,現象に触れること.そして,その現象について,
            その特徴をできるかぎり詳しくしらべること.そうしてやっと,
            現象の特徴はどのような法則に従っているかといった,理論的
            研究に入るのである.

            物理学の本を読むということは,今知られている理論を把握
            するということであり,物理を追求するという行為ではない.
            あくまでも,先人の得た知恵を吸収するということである.
            しかし,それは,探求の第一歩ではない.
            物理学の本を読んで,物理をわかった気になっているとしたら,
            とても残念なことである.


%   %==========================================================================
%   %  Section
%   %==========================================================================
        \section{心配レベル}
            心配という心情には,4つの段階があると思う.それは,次の通りだ.
                \begin{enumerate}
                    \item 心配
                    \item 不安
                    \item 恐怖
                    \item 絶望
                \end{enumerate}
            下に行くほど(数字が大きくなるほど),心配レベルが上がる.
            
            具体例を示してみよう.
            いつもそばにいる大切な人が,一週間の間,自分の前からはなれることに
            なることを考えてみる.海外旅行にでも出かけることにでもしておこう.
            
            その時,あなたは大切な人が,目の前から離れることで,心配になる(はずだ).
            交通事故に遭わないだろうか,悪い人に騙されたりしないだろうか,等々.
            これが,心配という心情だ.この段階では,心配するだけで,何も行動を
            起こさないことだろう.

            一週間たっても,1日,2日,大切な人が帰って来なかったとしよう.
            あなたは不安に陥ることだろう.何があったのか気になって仕方がない.
            こうなると,あなたは,どうにかして,連絡を取ろうと必死になるはずである.
            これが不安という心情.

            どうしても連絡がつかなかったら,その不安は恐怖になる.
            事件に巻き込まれたとか,事故にあったのではないかとかと,考え始める.
            この時,大切な人が傷ついているかもしれないという,恐怖を覚える.

            その恐怖が,最悪の形で現実出会ったとしよう.このとき,あなたは為す術がなく,
            絶望に至る.その後,自分のできる限りの行動を,世の中に対して必死に働きかける
            ことになる.



%   %==========================================================================
%   %  Section
%   %==========================================================================
        \section{"分からないこと" と "知らないこと"}
            突然,新しい環境に放り込まれたとしよう.
            このとき,大変幸福なことに,近くにその環境に詳しい人がいるとする.しかし,
            その人は,私に対して,その環境のことを説明することを
            あまりしない.そのひとは,
            「わからないことがあったら,何でも質問してください」と言う.
            新参者の私にとっては,その環境に詳しい人は,唯一頼りにできる
            大変ありがたい存在である.
            しかし,私がその人に質問するには,時間がかかる.
            自分が分からないことを把握しなければならないからである.
            わからないことを把握するには,知らないことをリストアップしていく
            必要がある.知らないことは,当然,質問できないからだ.例えば,
            「不確定性原理」という言葉を知らない人は,それについて詳しい人が
            そばに居たとしても,質問することはない.
            質問できないのである.

            何が言いたいかというと,教わる側の人間にが取るべき行動は,
            その周囲の環境を,できる限り,見て聞いて把握することである.
            そして,教える側の人間が取るべき行動は,その新参者が知らないことを
            示してあげる事である.

            わからないことが質問できないと言って,嘆く必要はない.
            そんな場合は,周囲の環境を執拗に見たり聞いたりして,
            できる限り早く把握する,という目標があるのだから,
            それを行えば良い.それができなければ,諦めて,
            別の場所に行くより他はない.

