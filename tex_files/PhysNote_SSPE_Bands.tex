%===================================================================================================
%  Chapter : バンド理論
%  説明    : 物体の運動の表現方法について,記述する.
%===================================================================================================
%   %==========================================================================
%   %  Section
%   %==========================================================================
    \section{結晶構造}
        \begin{mycomment}
            物体は原子より構成される.原子がたくさん集まってできた塊を,私たちは
            目で見て,物体を認識しているわけだ.物体を構成している多くの原子は,
            デタラメに並んでいるわけではなく
                \footnote{
                    原子の並びを,\textbf{原子配列} という.配列とは,
                    ものが順番通りに並んでいる状態のことである.
                },
            何か一定の規則に基いている.本節では,この規則について,説明する.
        \end{mycomment}

%       %======================================================================
%       %  SubSection
%       %======================================================================
        \subsection{結晶}
            物体を構成する原子は,多くの場合,ある規則に従って並んでいる.
            この並びを,\textbf{原子配列} という.または,原子配列は \textbf{結晶構造} とも
            言われる.原子配列にはいろいろなパターンがある.以下に,その幾つかの例を挙げる.

%       %======================================================================
%       %  SubSection
%       %======================================================================
        \subsection{単位格子}

%       %======================================================================
%       %  SubSection
%       %======================================================================
        \subsection{ミラー指数}

%       %======================================================================
%       %  SubSection
%       %======================================================================
        \subsection{ブラックの回折条件}

%       %======================================================================
%       %  SubSection
%       %======================================================================
        \subsection{逆格子}

%       %======================================================================
%       %  SubSection
%       %======================================================================
        \subsection{結晶構造因子}


%   %==========================================================================
%   %  Section
%   %==========================================================================
    \section{エネルギーバンドの導出}


%       %======================================================================
%       %  SubSection
%       %======================================================================
        \subsection{トンネル効果のおさらい}

%       %======================================================================
%       %  SubSection
%       %======================================================================
        \subsection{1次元結晶}

%       %======================================================================
%       %  SubSection
%       %======================================================================
        \subsection{ペニー=クローニヒのモデル}

%       %======================================================================
%       %  SubSection
%       %======================================================================
        \subsection{ブロッホ関数}

%       %======================================================================
%       %  SubSection
%       %======================================================================
        \subsection{ブロッホの定理}

%       %======================================================================
%       %  SubSection
%       %======================================================================
        \subsection{バンドの導出}
