%===================================================================================================
%  Chapter : 論理学とか,数学とか
%  説明    : 物理学より,思考方法としてより基礎的なことである,論理・数学について考えてみる.
%===================================================================================================

%   %==========================================================================
%   %  Section
%   %==========================================================================
        \section{論理}
            私たちが普段の生活で使っている言語のうち,曖昧な使い方を避けて
            ,万人に共通に伝わるようにしたい.そのためには,\textbf{論理}
            と言うものを考える必要がある.論理は,日常言語の中の一部にある
            .相手に自分の考えを伝えるとき,物事を筋道立てて伝えようとする
            だろう.自分の考えをできるだけ性格に相手に知ってほしいからであ
            る.さて,このとき私たちは,論理を使っているのである.論理は,
            何個かの公理
                \footnote{
                    公理:万人が認める事実のこと.何の反論なしに認めなけれ
                    ばならないことである.ある仮説を検証するとき,その仮説
                    の根拠をどんどん探ることになる --- AはBとCからなってい
                    て,BはDから,CはEを基にしている・・・と言うように---.
                    しかし,いつまでもこれが続くわけではない.いつかは,ど
                    うしようもなく“当たり前すぎて”,説明ができないことに
                    たどり着く.公理とは,その当たり前の事実を明示するもの
                    である.
                }
            と推論規則
                \footnote{
                    推論規則:ある仮定から,別の仮説を作り出せる規則のこと
                    .公理と同様,有無を言わさず認めさせられるものである.
                }
            を基に構成される.少数の公理と推論規則から,主張したいことがす
            べて主張できる体系を作ることが学問の目的である.これを思考経済
            と言ったりする.公理と推論規則は,少なければ少ないほどよい.



%   %==========================================================================
%   %  Section
%   %==========================================================================
        \section{論理学}
            この論理について,詳しく研究する学問に \textbf{論理学} がある
            .ある仮説を記述した文で,本当か嘘かをはっきりと区別できる文の
            ことを,\textbf{命題} という.論理学はいくつかの必要最小限の公
            理と推論規則を組み合わせ,命題の証明を繰り返し発展させていくも
            のである.命題と推論規則の組み合わせのことを,\textbf{公理系} と
            言う.この公理系には,次の3つの性質が備わっている必要がある.一
            つは \textbf{独立性} で,公理形の中のどの1つの公理を選んでも,
            他の公理からその公理を証明できないような性質である.
            二つ目は \textbf{完全性} と言われるもので,主張したい命題が,そ
            の公理系からすべて導けることである.三つ目は \textbf{無矛盾性} と
            言われるもので,公理系に互いに矛盾する公理を含んでいないことで
            ある.

            論理学とは考察の足固めである公理系を設計し,体系を作り上げて
            いく学問でもあるのだ.どれだけ詳しく説明できるかと言う疑問の最
            も根本的な部分の研究がここでなされる.



%   %==========================================================================
%   %  Section
%   %==========================================================================
        \section{数学}
            数学とは,論理に複素数を組み合わせた学問であると言える.その研究
            の対象はおもに,複素数である.複素数の一部には,自然数が含まれて
            いる.残念なことに,自然数を含む公理系には,完全性,無矛盾性が常
            に保たれていると言う保障がないことが知られている.この事実は
            G\"{o}delの \textbf{不完全性定理} とよばれている
                \footnote{
                    参考文献:廣瀬 健,横田 一正 [著],「ゲーデルの世界」,
                    鳴海社,2004
                }.

            集合論により無限を扱えるようになってきたころ,この無限を起因とし
            て様々なパラドクスが発見されることとなった.「自分自身を要素とし
            て含まないすべての集合」がその最も有名なものである.ちょっと考え
            てみよう.
                \begin{equation*}
                    \omega:\;\;\mbox{自分自身を要素として含まないすべての集合}
                \end{equation*}
            と定義しよう.そして,次の問題,すなわち
                \begin{equation*}
                    \mbox{問題}:\;\;\omega \mbox{自身は,} \omega \mbox{に含まれるか否か}
                \end{equation*}
            を考えてみよう.すぐに明らかな矛盾が見えてくるだろう.

            $\omega$ は自分自身に含まれると仮定してみる.すると,$\omega$ の
            定義「自分自身を要素として含まない」という仮定に矛盾する.では,
            $\omega$ は自分自身に含まれないと仮定したらどうか.実はこれでも
            矛盾が生じる.なぜなら,「自分自身要素としてを含まない」という
            定義上,仮定で「自分自身は含まない」言っているので,自分自身を
            含むべきだと言う結論が出てしまう.肯定的に仮定しようが,否
            定的に仮定しようが,どちらにしても結果はその仮定と矛盾するので
            ある.

            Russellらは,この問題を解決しようと階という概念を導入し,命題
            に自分自身を含むことのないように制限を加えた.しかし,問題はこれ
            だけではとどまらず,山のように残されていた.

            Hilbertはこのような問題の山を,数学の危機であると自覚し,これを
            解決しようと計画した.Hilbertはこの数学の危機と言われる問題を23
            個の命題にまとめた.そして彼は,この23の問題を証明し解決しようと
            呼びかけた.G\"{o}delの不完全性定理はこの23の問題のうちの一つ
                \footnote{
                    第2番目に掲げられていた問題だった.
                }
            の否定的な回答であった.

            だから,物理学に公理系を作成して,論理的に構成しようとしても,無駄
            である.しかし,物理学は自然の構成を探る学問だから,この点に関して
            はあまり気にすることはないと思う.


%   %==========================================================================
%   %  Section
%   %==========================================================================
        \section{物理学}
            物理学は,自然がどのようになっているかを探る学問である.
                \footnote{
                    "なぜ"自然が私たちの感じているようになっているのかを探る学
                    問では\textbf{ない}.
                }
            「なぜ(Why)」を問うのではなく,「どのように(How)」を問うのであ
            る.

            なぜ自然がこのように
                \footnote{
                    普段の生活で,私たちが感じている自然を思い浮かべてみて.
                }
            なっているのか,とか,なぜ宇宙があるのかとかを考えるのは哲学であって,
            物理学ではない.物理学ではこういう,"なぜ"を問うような疑問には答えられない
                \footnote{
                    ただ,"なぜ"という問を深めていく事は可能で,実際に,物理学の発展は
                    その繰り返しである.その様子は.これからの学習で実際に感じることになる.
                }.

            物理学とは,自然の性質を見つけるものである.この「性質」という言葉は
            物理学では \textbf{法則} と呼ばれている
                \footnote{
                    \textbf{法則}:後で詳しく記述する.
                }.
            自然の法則を,実験や数式を通して見つけ出すことが物理学の目的なのだ.

            自然はデタラメに変化しているのではなく,何か一定の法則に従ってい
            るということは,経験上理解できることと思う.例えば,特別に力を加え
            ない限り,高いところから低いところへ,物体は落ちていく.落とした消し
            ゴムは,拾わないと手元に戻ってこないのである.何でだろうか.この原因
            を探り,「法則」として記述するのである.

            このことについては,物理学を学び始めた段階ではまだ実感がわかないかも
            知れない.学習していく過程で,だんだんとわかることだろう.

            このノートでは物理学を学習することが目的である.自然はどのような法則
            に従って変化しているのだろうか.少しずつ考えることにしよう.


