%******************************************************************************
%  The Bibliogeaphy
%******************************************************************************
\begin{thebibliography}{99}
\bibitem{bib:refbook_LaTeX_1}   藤田 真作 [著], 『\LaTeXe コマンドブック』, ソフトバンクパブリッシング, 2003
\bibitem{bib:refbook_LaTeX_2}   吉永 徹美 [著], 『独習\LaTeXe』, 翔泳社, 2008
\bibitem{bib:refbook_mth_1}     宮腰 忠 [著], 『高校数学 $+\alpha$ \small{基礎と理論の物語}』, 共立出版, 2009
\bibitem{bib:refbook_mth_2}     小平 邦彦 [著], 『解析入門』(軽装版), 岩波書店, 2006
\bibitem{bib:refbook_mth_3}     戸田 盛和 [著], 理工系の数学入門コース3『ベクトル解析』, 岩波書店, 2006
\bibitem{bib:refbook_Fig0}      数研出版編集部 [編著], 『視覚でとらえる フォトサイエンス 物理図録』,
\bibitem{bib:refbook_mech_1}    大貫 義郎, 吉田 春夫 [著], 岩波講座 現代の物理学第1巻 『力学』, 岩波書店, 1994
\bibitem{bib:refbook_mech_2}    阿部 龍蔵 [著], 岩波基礎物理シリーズ① 『力学・解析力学』, 岩波書店, 2005
\bibitem{bib:refbook_mech_3}    藤原 邦男 [著], 『物理学序論としての 力学』, 東京大学出版会, 2004
\bibitem{bib:refbook_em_1}      山田 直平, 桂井 誠 [著], 電気学会大学講座 『電気磁気学』 3訂版, Ohm社, 2004
\bibitem{bib:refbook_em_2}      永田 一清 [著], 基礎の物理4『電磁気学』, 朝倉書店, 1998
\bibitem{bib:refbook_em_3}      砂川 重信 [著], 物理テキストシリーズ4『電磁気学』, 岩波書店, 2003
\bibitem{bib:refbook_em_4}      川村 清 [著], 岩波基礎物理学シリーズ③『電磁気学』, 岩波書店,
\bibitem{bib:refbook_em_5}      太田 浩一 [著], 『マクスウェル理論の基礎 \small{相対論と電磁気学}』, 東京大学出版会, 2009(第4版)
\bibitem{bib:refbook_em_12}     太田 浩一 [著], 『マクスウェルの渦 アインシュタインの時計 \small{現代物理学の源流}』, 東京大学出版会, 2005(初版)
\bibitem{bib:refbook_em_6}      太田 浩一 [著], 『電磁気学の基礎\I』, 東京大学出版会, 2008
\bibitem{bib:refbook_em_7}      太田 浩一 [著], 『電磁気学の基礎\II』, 東京大学出版会, 2008
\bibitem{bib:refbook_em_8}      野田 二次男, 菅野 正吉 [著], (理工学のための)『電磁気学入門』, 培風館
\bibitem{bib:refbook_em_9}      加藤 正昭 [著], 基礎物理学3『電磁気学』, 東京大学出版会, 1987
\bibitem{bib:refbook_em_10}     長岡 洋介 [著], 物理入門コース4『電磁気学\II ‐変動する電磁場‐』, 岩波書店, 2006
\bibitem{bib:refbook_em_11}     藤村 哲夫 [著], 『電気発見物語』, 講談社ブルーバックス,2002
\bibitem{bib:refbook_em_13}     ウィリアム.H.クロッパー [著],『物理学天才外伝』, 講談社ブルーバックス, 2009
\bibitem{bib:refbook_rel_1}     アインシュタイン [著], 内山 龍雄 訳, 『相対性理論』, 岩波書店(岩波文庫), 2005
\bibitem{bib:refbook_rel_2}     砂川 重信 [著], 物理の考え方5『相対性理論の考え方』, 岩波書店, 2006
\bibitem{bib:refbook_rel_3}     中野 董夫 [著], 物理入門シリーズ9『相対性理論』, 岩波書店, 2006
\bibitem{bib:refbook_rel_4}     佐藤 勝彦 [著], 岩波基礎物理シリーズ9『相対性理論』, 岩波書店, 2006
\bibitem{bib:refbook_qm_00}     片山 泰久 [著], 『量子力学の世界』, 講談社ブルーバックス, 1998
\bibitem{bib:refbook_qm_01}     山田 克哉 [著], 『量子力学のからくり --「幽霊波」の正体--』, 講談社ブルーバックス, 2003
\bibitem{bib:refbook_qm_02}     小川岩雄 [著],物理学One Point --- 29 『原子と原子核』,共立出版,2008
\bibitem{bib:refbook_qm_1}      阿部 龍蔵 [著], 物理テキストシリーズ6『量子力学入門』, 岩波書店, 2004
\bibitem{bib:refbook_qm_2}      佐川 弘幸, 清水 克多郎 [著], 物理学スーパーラーニングシリーズ『量子力学』, シュプリンガーフェアラーク東京, 2005
\bibitem{bib:refbook_qm_3}      E.シュポルスキー [著], 玉木 英彦, 細谷 資明, 井田 幸次郎, 松平 升 訳, 『原子物理学\I』, 東京図書, 2000
\bibitem{bib:refbook_qm_4}      原島 鮮 [著], 『初等量子力学』, 裳華房, 2007
\bibitem{bib:refbook_qm_5}      小出 昭一郎 [著], 『量子力学1』, 裳華房, 2006
\bibitem{bib:refbook_qm_6}      関根 松夫 [著], 『量子電磁気学』, コロナ社, 1996
\bibitem{bib:refbook_SC_1}      A.C.ローズ--インネス, E.H.ロディリック [著], 島本 進, 安河内 昴 訳,『超電導入門』, 産業図書, 1999
\bibitem{bib:refbook_phys_1}    広江 勝彦 [著], 『趣味で物理学』, 理工図書, 2007
\bibitem{bib:refbook_phys_2}    矢野 健太郎 [著], 『アインシュタイン』, 講談社(講談社学術文庫), 1991
\bibitem{bib:refbook_phys_3}    中谷 宇吉郎 [著], 『科学の方法』, 岩波書店(岩波新書), 1998
\bibitem{bib:refbook_phys_4}    湯川 秀樹 [著], 『目に見えないもの』, 講談社(講談社学術文庫), 2007
\bibitem{bib:refbook_phys_5}    S.ワインバーグ [著],本間三郎 [訳],『新版 電子と原子核の発見』,筑摩書房(ちくま学芸文庫),2009
\end{thebibliography}

