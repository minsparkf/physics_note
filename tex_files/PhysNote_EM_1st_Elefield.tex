%===================================================================================================
%  Chapter : 電場
%  説明    : 電場の概念を導入るすことで,遠隔作用のクーロンの法則を,近接作用的に書き換える
%===================================================================================================
%       %======================================================================
%       %  Section
%       %======================================================================
            \section{作用の伝わり方}
            \subsection{遠隔作用}
            2つの点電荷が存在する場合を考える.
            この2つの電荷は距離 $r$ を隔てて固定されているものとする.
            このように設定された2つの電荷間には,$r$ の距離を通してクーロン力が
            伝わると考えられる.このようなことを,
            「クーロン力は \textbf{遠隔作用} で伝わる」という.
            この考え方によると,クーロン力は一瞬にして伝わるとされる.点電荷間の距離が
            どんなに大きくても,クーロン力は一瞬で伝わるのである.この考え方は
            納得がいかないことだろう.実際,クーロン力が伝わるには時間が掛かる
            ことが示されている.従って,クーロンの法則は,力の遠隔作用という
            点で問題を抱えていることになる.

            \subsection{近接作用}
            遠隔作用であるクーロン力を,より直感的に馴染む \textbf{近接作用} となるように
            書き換える.近接作用は,その名の通り,電荷はその隣りの空間から影響を受ける
            という考え方で,遠くにある電荷から瞬間的にクーロン力が伝わるのではなく,
            だんだんとクーロン力が伝わってくると考えるのである.しかし,近接作用を
            採用するとなると,そのクーロン力を伝えるための「何か」が必要になってくる.
            例えば,音波は空気を通して伝わるように,クーロン力も音波に対する空気のような,
            それを伝えるための媒質があるとすべきだ.
            そのために導入するのが \textbf{電場} という概念である.クーロン力は,
            電場を通して伝達するのである.以下で,この電場という考え方を
            説明していく.


%       %======================================================================
%       %  Section
%       %======================================================================
            \section{電場(1個の点電荷)}
%   %==========================================================================
%   %  Subsection
%   %==========================================================================
    \subsection{2つの点電荷間のクーロン力}
                これから,\textbf{電場} という概念を説明したいのだけど,初めて学習
                する場合に,いきなり一般的な定義を提示していまうと,数学的な演算のみ
                に思考が傾いてしまいがちだし,もしかすると,“難しい”概念なのだ感じ
                てしまうかもしれない.そこで,このノートでも,他の多くの教科書と同様
                に,段階を踏んで電場という概念を説明していきたい.

                最初に考えるのは,2つの点電荷のみが存在する場合についてである.

                クーロンの法則は,一方の点電荷が他方の点電荷にクーロン力を与える,
                というものであった.従って,クーロン力とは2つ以上の点電荷が存在し
                て初めて観測される力である.簡単のために,2つの点電荷だけが存在す
                る場合を考える.この2つの点電荷のうちの1つの点電荷を空間に固定して,
                残りの電荷は人間がいつでも好きな場所におくことができるようにする.
                この自由に動かせる電荷を固定されている電荷の周りの様々な場所に置い
                てみると,置く場所によって受けるクーロン力が異なってくる.なぜなら,
                点電荷間の距離が異なるからである.しかし,自由に動かせる点電荷の場
                所を1つだけ指定すれば,この点電荷の受けるクーロン力は一意に決まる
                    \footnote{
                        「一意に決まる」というのは,解が必ずひとつに定まることをいう.
                    }.

                これが,「電場」という発想の源となる.自由に動かせる点電荷を利用し
                て,『固定された点電荷が,その周りの空間に作る電気的な世界を見よう』
                というのだ.自由に動かせる点電荷を様々な場所に置いてみて,その場所
                で固定された電荷から受けるクーロン力を記録していくのである.もちろ
                ん,全ての場所に自由に動かせる点電荷を置いていく.実際は無理だが,
                頭の中では簡単にできることである.一種の思考実験であると考えればよ
                い.このようにして作った記録は,点電荷特有のものになる.この記録の
                ことを点電荷の電場とよぶことにしようというのである.
                    \begin{figure}[hbt]
                        \begin{tabular}{cc}
                            \begin{minipage}{0.5\hsize}
                                \begin{center}
                                    \includegraphicsdouble{denba_intro.pdf}
                                    \caption{試験電荷の受ける力}
                                    \label{fig:siken_denka_ukerutikara}
                                \end{center}
                            \end{minipage}
                            \begin{minipage}{0.5\hsize}
                                \begin{center}
                                    \includegraphicsdouble{denba_intro2.pdf}
                                    \caption{試験電荷の受ける力の記録}
                                    \label{fig:denba_intro2}
                                \end{center}
                            \end{minipage}
                        \end{tabular}
                    \end{figure}

%   %==========================================================================
%   %  Subsection
%   %==========================================================================
    \subsection{定量化}
                定量化してみよう.固定された点電荷の電気量を $q_{0}$ とし,
                位置を $\br_{0}$ とする.また,
                自由に動かせる点電荷の電気量を $q_{x}$ とし,
                位置を $\br_{x}$ とする.
                このとき,自由に動かさせる電荷が 固定された電荷から受けるクーロン力は
                        \begin{align}
                            \bF(\br_{x})=\frac{1}{4\pi\varepsilon_{0}}
                            \frac{q_{x}q_{0}}{|\br_{0}-\br_{x}|^{2}}
                            \frac{\br_{0}-\br_{x}}
                                 {|\br_{0}-\br_{x}|}
                        \end{align}
                と書ける.ここで,$\bF(\br_{x})$ と
                書いたのは,$\br_{x}$ が自由に動かせることを強調するためである.
                ここで,この式を眺めていると,以下のように変形しても示されているよさそうであ
                ることに気付く.
                        \begin{align}
                            \bF(\br_{x})=q_{x}\left(\frac{1}{4\pi\varepsilon_{0}}
                            \frac{q_{0}}{|\br_{0}-\br_{x}|^{2}}
                            \frac{\br_{0}-\br_{x}}
                                 {|\br_{0}-\br_{x}|}\right).
                        \end{align}
                この式の 括弧の中身は $\br_{x}$ の関数である.
                だから,括弧の中身を $\bE_{0}(\br_{x})$ とおくことができる.
                        \begin{align}
                            \bE_{0}(\br_{x})=\frac{1}{4\pi\varepsilon_{0}}
                            \frac{q_{0}}{|\br_{0}-\br_{x}|^{2}}
                            \frac{\br_{0}-\br_{x}}
                                 {|\br_{0}-\br_{x}|}.
                        \end{align}
                ここで,関数の添え字に「固定」とつけた理由は,固定された点電荷が作るも
                のであることを忘れないようにするためである.
                このように定義された関数は,固定された点電荷特有の関数である.従ってこ
                の関数は,固定された電荷が
                その周りの空間に作る電気的な世界を記述していると考えられる.このような
                関数を,点電荷の \textbf{電場} と
                いうのである.電場を用いると
                        \begin{align}
                            \bF(\br_{x})
                            =q_{x}\bE_{0}(\br_{x})
                        \end{align}
                とできる.

%   %==========================================================================
%   %  Subsection
%   %==========================================================================
    \subsection{単位電荷が及ぼすクーロン力}
                $q_{x}=1$ とすると,
                        \begin{align}
                            \bF(\br_{x})
                            =\bE_{0}(\br_{x})
                        \end{align}
                となって,自由に動かせる点電荷に働く力が,電場に等しくなる.
                このことは,電荷の単位が [C] であったことを思い出せば,
                『電場は 1[C]の電荷に働くクーロン力に等しい』と言える.
                従って,今までは点電荷の作る電場を考えてきたが,たとえ電場の
                関数の具体的な形が分からなくても,1[C] の電荷を様々な場所に置いて
                その場所でのクーロン力を測ることによって,
                電場を知ることができるのである.このように使われる「1[C] の電荷」のことを,
                \textbf{試験電荷} という.何となくではあるが,
                電場のイメージができたところで,
                以下で一般的な電場の定義を与えることにする.

                $'$ が付いた記号は固定電荷についての情報を表し,
                何も付いていない記号は試験電荷についての情報を表す.
                すると,次のように表現を改め直せる
                    \footnote{
                        記号が変わっただけで,
                        書いていることは同じなのだけど.
                        こう表したほうが,
                        カッコイイし,見ためもスッキリとしていて見やすい.
                    }.
                    \begin{align*}
                        \bE(\br) := \frac{1}{4\pi\varepsilon_{0}}
                                    \frac{q'}{|\br-\br'|^{2}}
                                    \frac{\br-\br'}{|\br-\br'|}.
                    \end{align*}
                    \begin{myshadebox}{電場(1個の点電荷)}
                        1個の点電荷のつくる電場 $\bE(\br)$ は次式で表現される.
                        \begin{align*}
                            \bE(\br) := \frac{1}{4\pi\varepsilon_{0}}
                                        \frac{q'}{|\br-\br'|^{2}}
                                        \frac{\br-\br'}{|\br-\br'|}.
                        \end{align*}

                        ここで,
                            $\br$ は試験電荷を置く位置(任意の位置),
                            $q'$ は固定点電荷のもつ電気量,
                            $\br'$ は固定電荷の位置
                        である.
                    \end{myshadebox}


%       %======================================================================
%       %  Section
%       %======================================================================
            \section{電場($N$ 個の点電荷)}
            状況を少し一般化させて,$N$ 個の固定された点電荷が作る電場を考える.
            電場の定義がクーロン力を基にすることには変わらない
                \footnote{
                    そもそも,電場はクーロンの法則をより直感てきに
                    なじむように発展させた概念なのである.
                }.
            だから,クーロン力が重ね合わせの原理に従う以上,
            これに付随して電場も重ね合わせの原理に従わねばならない.
            つまり,点電荷の個数が1個から $N$ 個に増えようと,
            別に新しい考え方を導入する必要はない.単に,試験電荷が
            一つひとつの固定電荷がつくる電場を計算し,最後にそれらを全て
            加えあわせればよいだけである.
            つまり,固定された $N$ 個の点電荷が作る電場は次式で表現できる.
                \begin{align}
                    \bE(\br)&=\sum_{i=1}^{N}\frac{1}{4\pi\varepsilon_{0}}
                    \frac{q'_{i}}{|\br-\br'_{i}|^{2}}
                    \frac{\br-\br'_{i}}
                         {|\br-\br'_{i}|}
                \end{align}
            と書ける.それは,クーロン力が重ね合わせの原理を満たしてい
            ることからわかる.
                \begin{myshadebox}{電場($N$ 個の点電荷)}
                    $N$ 個の点電荷のつくる電場 $\bE(\br)$ は次式で表現される.
                    \begin{align}\label{denba_huku}
                        \bE(\br)&=\sum_{i=1}^{N}\frac{1}{4\pi\varepsilon_{0}}
                        \frac{q'_{i}}{|\br-\br'_{i}|^{2}}
                        \frac{\br-\br'_{i}}
                             {|\br-\br'_{i}|}
                    \end{align}

                    ここで,
                        $\br$ は試験電荷を置く位置(任意の位置),
                        $q'_{i}$ は固定点電荷のそれぞれのもつ電気量,
                        $\br'_{i}$ は固定電荷のそれぞれの位置
                    である.
                \end{myshadebox}

%       %======================================================================
%       %  Section
%       %======================================================================
            \section{電場(電荷の連続分布)}
            電荷が連続分布している場所において,電気量 $q$ をもつ電荷
            が受ける力は,式(\ref{coulomb'slow2})によって,
                \begin{align}
                    \bF(\br)
                    &=\int_\Omega \frac{1}{4\pi\varepsilon_{0}}
                    \frac{q\rho(\br^{*})}{|\br-\br^{*}|^{2}}
                    \frac{\br-\br^{*}}
                         {|\br-\br^{*}|}\df V^{*}
                \end{align}
            のように書かれる.この $q$ 電荷は \textbf{試験電荷} である.
            試験電荷 $q$ を用意して,この試験電荷が各点で
            受ける力を考えることによって,周り電気的様子を観測するのである.
            この式を,以下のように変形する.$q$ は積分には関係のない定数であるので,
            で積分記号の前に出せて
                \begin{align}
                    \bF(\br)
                    &=q\int_\Omega \frac{1}{4\pi\varepsilon_{0}}
                    \frac{\rho(\br^{*})}{|\br-\br^{*}|^{2}}
                    \frac{\br-\br^{*}}
                         {|\br-\br^{*}|}\df V^{*}
                \end{align}
            と書ける.ここで,以下の量を定義する.

            点電荷が連続的に分布している場合,電場 $\bE(\br)$ は電荷密度 $\rho(\br^{*})$ を
            用いて,以下の数式で定義できる.
                \begin{align}
                    \bE(\br)
                    :=\int_\Omega \frac{1}{4\pi\varepsilon_{0}}
                    \frac{\rho(\br^{*})}{|\br-\br^{*}|^{2}}
                    \frac{\br-\br^{*}}
                    {|\br-\br^{*}|}\df V^{*}.
                \end{align}
             ここに,上付きのアスタリスク ${}^{*}$ が付いている
             変数について積分を実行する
                 \footnote{
                     アスタリスクなしの $\br$ は任意の位置を
                     示す,関数の変数である.上付きのアスタリスクが
                     ついた $\br^{*}$ は電荷が分布している場所を表す
                     積分変数である
                     (積分変数はどんな記号を用いても結果にはなんの影響も
                     与えないが,位置についての積分であることを強調したいため,
                     $\br^{*}$ という書き方をした).
                 }.

            このように電場 $\bE(\br)$ を定義することで,クーロンの法則は
                \begin{equation*}
                    \bF(\br) = q \bE(\br)
                \end{equation*}
            と書ける.特に,単位電荷 $q=1$[C] の場合,
                \begin{equation*}
                    \bF(\br) = \bE(\br)
                \end{equation*}
            となり,クーロン力 $\bF(\br)$ がそのまま電場 $\bE(\br)$ を表す式になる.
                \begin{myshadebox}{電場(電荷の連続分布)}
                    点電荷が連続的に分布している場合,電場は電荷密度 $\rho(\br^{*})$ を
                    用いて,以下の数式で定義できる.
                    \begin{align}\label{denba}
                        \bE(\br)
                        :=\int_\Omega \frac{1}{4\pi\varepsilon_{0}}
                        \frac{\rho(\br^{*})}{|\br-\br^{*}|^{2}}
                        \frac{\br-\br^{*}}
                        {|\br-\br^{*}|}\df V^{*}.
                    \end{align}
                    ここに,上付きのアスタリスク ${}^{*}$ が付いている
                    変数について積分を実行する.
                \end{myshadebox}


%       %======================================================================
%       %  Section
%       %======================================================================
            \section{電場(一般化)}
            一般に,電気量 $q$ をもつ点電荷は周囲のその他の
            電荷,あるいは,電荷密度からクーロン力を受ける.
            このクーロン力 $\bF(\br,q)$ は
                \footnote{
                    ここで,独立変数として,点電荷の位置 $\br$ だけ
                    ではなく,点電荷の電気量 $q$ もすぐ後の都合で,
                    明記しておく(あとで,$q$ を0の極限に持ってい
                    く必要が出てくる).
                },
                \begin{align*}
                    \bF(\br,q)=q\bE(\br)
                \end{align*}
            と表現できる
                \footnote{
                    $\bE(\br)$ は $\br$ のみを独立変数に持つ
                    関数である.
                }.

            電場はクーロンの法則を満たすように定義されることに注意する.
            具体的には,式(\ref{denba})である.クーロンの法則で表せば,
                \begin{align}
                    \bE(\br)=\frac{\bF(\br)}{q}
                \end{align}
            である.電荷分布が決定されれば,各位置でのクーロン力は
            一意に決まる.従って,電場についても同様なことがいえる.

            さらに細かいことをいえば,電気量 $q$ の電荷は,
            自身から生じる電場により周囲の電場を歪ませてしまう.
            そこで,電気量 $q$ を 0 に近づける.すると,
            これは以下のように表現できる.
                \begin{align*}
                    \bE(\br)=\frac{\rd\bF(\br,q)}{\rd q}.
                \end{align*}

            ここでは,時間変化しないクーロン力で電場を考えている.
            クーロンの法則は,点電荷の位置が時間変化しないという条
            件の下で成り立つ法則である.従って,このクーロン力によって
            定義された電場もまた,時間変化を考慮していない.
            時間変化する電場については後述する.
                \begin{myshadebox}{電場(一般的な定義)}
                一般に,電気量 $q$ をもつ点電荷が周囲から受ける
                クーロン力 $\bF(\br,q)$ を用いて,次式で \textbf{電場} を
                定義する.
                    \begin{align}\label{denba_teigi}
                        \bE(\br, t)
                        :=\frac{\rd\bF(\br,q)}{\rd q}.
                    \end{align}
                \end{myshadebox}


                \begin{memo}{(例)点電荷の作る電場}
                    電場の定義の式(\ref{denba_teigi})を用いて,
                    点電荷 $q$ の作る電場を定義から求めてみる.
                    試験電荷の電気量を $q'$ とする.
                    すると,クーロンの法則により,クーロン力は
                            \begin{align}
                                \bF=\frac{1}{4\pi\varepsilon_{0}}
                                \frac{qq'}{|\br-\br'|^{2}}
                                \frac{\br-\br'}
                                     {|\br-\br'|}
                            \end{align}
                    と書ける.ここで,$\br'$ は試験電荷 $q'$ の位置である.このクーロン力を用いて,
                    電場は以下のように計算される.
                        \begin{align}\label{tendenka_denba}
                            \bE(\br)
                            &= \frac{\rd\bF(\br,q')}{\rd q'} \notag \\
                            &= \frac{\rd}{\rd q'}\left(\frac{1}{4\pi\varepsilon_{0}}
                                \frac{qq'}{|\br-\br'|^{2}}
                                \frac{\br-\br'}
                                     {|\br-\br'|}\right)\notag \\ \notag \\
                            &= \frac{1}{4\pi\varepsilon_{0}}
                                \frac{q}{|\br-\br'|^{2}}
                                \frac{\br-\br'}
                                     {|\br-\br'|}\notag \\ \notag \\
                            \therefore \quad
                            \bE(\br)
                            &= \frac{1}{4\pi\varepsilon_{0}}
                                \frac{q}{|\br-\br'|^{2}}
                                \frac{\br-\br'}
                                     {|\br-\br'|}
                        \end{align}
                    この式(\ref{tendenka_denba})が,点電荷の作る電場を表す式である.
                    この点電荷の存在する場所に対して,電場は点対称であることが確認できる.
                \end{memo}

%       %======================================================================
%       %  Section
%       %======================================================================
            \section{時間変化する電場}
                例えば,電荷密度が常に均一でなく,時間的に変化して電荷の存在する
                場所に局所的な偏りが生じる場合,当然として,その電荷分布より生じる
                電場も時間的に変化する.

                クーロン力の時間変化の原因は,電荷密度 $\rho$ の時間変化であり,
                この他に時間変化の原因となるものはない.従って,電荷密度の
                独立変数として,時間 $t$ を書き加えてやれば良い.つまり,
                    \begin{equation*}
                        \rho := \rho(\br, t)
                    \end{equation*}
                とする.このとき,時間変化するクーロン力は,時間を表す
                変数 $t$ をその独立変数を明示して,$\bF(\br, t)$ と
                書くことにすれば,
                    \begin{align}
                        \bF(\br, t)
                        &=q\int_\Omega \frac{1}{4\pi\varepsilon_{0}}
                        \frac{\rho(\br^{*}, t)}{|\br-\br^{*}|^{2}}
                        \frac{\br-\br^{*}}
                             {|\br-\br^{*}|}\df V^{*}
                    \end{align}
                とかける.

                すると,時間変化する電場 $\bE(\br, t)$ は,自然と以下のように定義できる.
                    \begin{align}
                        \bE(\br, t)
                        :=\int_\Omega \frac{1}{4\pi\varepsilon_{0}}
                        \frac{\rho(\br^{*}, t)}{|\br-\br^{*}|^{2}}
                        \frac{\br-\br^{*}}
                        {|\br-\br^{*}|}\df V^{*}.
                    \end{align}

                この時間変化する電場を使うと,クーロン力は
                    \begin{equation*}
                        \bF(\br,t) = q\bE(\br, t)
                    \end{equation*}
                 で表せる.単純に,独立変数に時間 $t$ を書き加えるだけで済む.

                電場が時間的に変化する場合でも,上に説明したような,電場の一般
                的な定義が成立する
                    \footnote{
                        だから「一般的な」という副詞をつけられる.
                    }.
                数式は,以下のようになる.
                    \begin{align}\label{denba_teigi_3}
                        \bE(\br)
                        :=\frac{\rd\bF(\br, q, t)}{\rd q}.
                    \end{align}
                これも単に独立変数に $t$ を明記しただけにすぎない.
                \begin{myshadebox}{電場(時間変化する場合)}
                一般に,電気量 $q$ をもつ点電荷が周囲から受ける
                クーロン力 $\bF(\br,q)$ を用いて,次式で \textbf{電場} を
                定義する.
                    \begin{align}\label{denba_teigi_4_t}
                        \bE(\br)
                        :=\frac{\rd\bF(\br,q, t)}{\rd q}.
                    \end{align}
                \end{myshadebox}


%       %======================================================================
%       %  Section
%       %======================================================================
            \section{電場の定性的なイメージ}
%       %======================================================================
%       %  Subsection
%       %======================================================================
        \subsection{イメージ}
                静電場はある特定の位置を指定すると,決まった方向に決まった強さ
                を示す.この性質は,電場を流体のようにイメージすることを可能に
                する.流体とは,水とか空気とかのことである.つまり,水や空気の
                流れのように電場のイメージをするのである.電場は目に見えないの
                で,このように考えるより方法がない.また,このイメージで注意す
                ることは,「電場には水のような媒質がない」ことである
                     \footnote{
                        アインシュタインらによる,「エーテル存在の否定」をこのノートで
                        は受け入れる.今でレこれが当たり前.
                     }.
                これは,電場と流体の大きな違いの1つである.水の流れとは,要す
                るに多くの水分子($\mathrm{H_{2}O}$)の移動だが,電場にはこの分子
                に当るものは存在しないことである.ここが電場のイメージが難しい
                ところである.でも,電荷を電場内に置いたときに,その電場から電
                荷がクーロン力を受けて運動するのだから,そこには何らかの「流れ
                的なもの」があると考えてよいだろう.その流れのようなものが,電場で
                あると解釈するのである.

                すると,「何が電場を発生させているのか」という疑問が生まれる.
                この疑問に答えるのが,後の章で考える 電場に対するガウスの法則
                である.
                \begin{figure}[hbt]
                    \begin{center}
                        \includegraphicsdouble{denba_image.pdf}
                        \caption{電場の流線(電気力線)のイメージ}
                        \label{fig:denba_image}
                    \end{center}
                \end{figure}


%       %======================================================================
%       %  Subsection
%       %======================================================================
        \subsection{電気力線(電場の可視化)}\label{subsec:dennki_rikisenn}
            電場は直接見ることのできない,抽象的な概念である.しかし非常に重要な
            概念であり,この電場という言う概念があるからこそ,電磁気現象を統一的に
            把握できる.ならば,“どうにかして電場を可視化したい”と思う
            ことだろう.実際に,ファラデーはこれを可視化することを試みていて,
            これは今日の形で言うと,\textbf{電気力線} とよばれる概念になる.
            あまりにも複雑な電場を想定しても,ただ話が複雑になるだけなので,
            1個の点電荷より生じる電場の電気力線を見てみることにしよう.
            \begin{figure}[hbt]
                \begin{tabular}{cc}
                    \begin{minipage}{0.3\hsize}
                        \begin{center}
                            \includegraphicsdouble{denriki_gazoukensaku.pdf}

                            (A) 砂鉄を使う
                        \end{center}
                    \end{minipage}
                    \begin{minipage}{0.7\hsize}
                        \begin{center}
                            \includegraphicsdouble{denriki_gazoukensaku_2.pdf}

                            (B) 電気力線
                        \end{center}
                    \end{minipage}
                \end{tabular}
                \caption{点電荷が作る電気力線(平面)\label{fig:denriki1111}}
            \end{figure}

            点電荷が作る電場の平面的イメージは図\ref{fig:denriki1111}(A),(B)のようである
                \footnote{
                    図\ref{fig:denriki1111}(A)は
                        \url{http://www.kleen-tec.co.jp/elec/elec.htm}
                    より(2008.08.23現在),
                    図\ref{fig:denriki1111}(B)は
                        \url{http://www.keirinkan.com/kori/kori_physics/kori_physics_1_kaitei/index.html}
                    より(2008.08.23現在).
                }.
            左の写真では,電場の向きを捉えることはできないが,ここに試験電荷を置いたときに,
            正電荷から負電荷に向かう方向に力を受けることから,電場には向きがあることが確かめられる.

            3次元ではどうなっているのだろうか.2次元の例から用意に想像がつくが,確認しておこう.
                \begin{figure}[hbt]
                    \begin{tabular}{cc}
                        \begin{minipage}{0.5\hsize}
                            \begin{center}
                                \includegraphicsdouble{E_FILED_Point.pdf}

                                (A) 点電荷の電気力線
                            \end{center}
                        \end{minipage}
                        \begin{minipage}{0.5\hsize}
                            \begin{center}
                                \includegraphicsdouble{E_FILED_dipole.pdf}

                                (B) 異極電荷同士
                            \end{center}
                        \end{minipage}
                    \end{tabular}
                    \caption{点電荷が作る電気力線(平面)\label{fig:E_FILED_Point}}
                \end{figure}
            立体的イメージは図\ref{fig:E_FILED_Point}(A),(B)のようになる
            \footnote{
                \url{http://www15.wind.ne.jp/~Glauben_leben/Buturi/Denjiki/Denjikibase1.htm}より(2008.08.23現在).
            }.

            電気力線はあくまでも,現象を上手く説明するための方法にすぎないことに注意する必要がある
            .というのも,実際に電気力線が電荷から生じているということを確かめる術はないからである
            .電気力線は,人間が電気現象を科学的に捉えたときに,はじめて意味をなす.本来の自然の中
            の電荷は,もしかしたら,電気力線,つまり電場を生んでおらず,何か私達の考えもつかないよ
            うな機構によって,電気現象を生じているのかもしれない.今私が分かることは,“電荷が電場
            を生じていて,それは電気力線によって視覚的に表現できる”ということである.図でイメージ
            することは大変重要なことだが,このイメージが自然現象そのものであるという,勘違いを起こ
            しやすい.このようなイメージは,実験を理論的に説明しようとしたときに役に立つというもの
            であり,つまりは,あくまでもイメージである.


