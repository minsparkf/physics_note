%===================================================================================================
%  Chapter : 電磁気学の基本概念
%  説明    : 電荷の存在や,電流の存在などを確認する
%===================================================================================================
%======================================================================
%  Section
%======================================================================
\section{電流と電荷の関係}
\begin{mycomment}
    \textbf{電荷} とは,電磁気現象の発生原因であり,この存在は有無をいわさず
    受け入れさせられるものである.\textbf{電流} とは,一つ,あるいは複数(多数)の
    電荷が,平均的に一方向に移動しているような現象をいう.従って,
    電流とは,電荷と観測者の相対的な速度に依存していると考えられる.つまり,
    観測者が電荷を見ているとき,その観測者の速度により,電流が生じているのか,
    単に電荷が運動せずにその場に存在しているのかが,変わってきてしまう.
    そこで,以降では,観測者の速度を0として,扱うことにする.
\end{mycomment}
%======================================================================
%  SubSection
%======================================================================
    \subsection{大局的な電荷保存則}
        \begin{mysmallsec}{電荷は突然現れることはない}
        ある領域に電荷が多数存在していることを想定しよう(図\ref{fig:Denryu_Denka_intoro}(A)参照).
        この多数の電荷の電気量の総和を,$Q(t)$ と書くことにする
            \footnote{
                電荷が多数存在するが,その個数は有限であることを想定する.
                このとき,電荷に番号付けをして,$q_{0}$,$q_{1}$,$q_{2}$,$\cdots$ の
                ように書けば,その総和 $Q(t)$ は $Q(t):=\sum_{i}^{N(t)}q_{i}$ で表せる.
                左辺の総電気量 $Q(t)$ の独立変数 $t$ は,右辺の電荷の個数が時間変化する場合($N(t)$)を
                表したものである.
            }.
        この状態から,ある程度時間が経過して,領域内部の電気量の総和が変化したとしよう.
        つまり,$Q$ の時間微分が0ではない値をとるということになる
            \footnote{
                時間変化がないということは,時間微分して0であるということである.
                例えば,速さ $v(t)$ は $v(t):=\df x(t)/\df t$($x$ は位置,$t$ は時間を表す)で
                定義されるが,位置に時間変化がない場合 $x(t)$ は一点に止まっているので,
                $x(t)=X_{\mathrm{const}}$ となり,時間によらない定数 $X_{\mathrm{const}}$ で表せる.
                この時の速度を考えると,$v(t):=\df x(t)/\df t=\df X_{\mathrm{const}}/\df t = 0$ と
                なり,位置の時間微分は0である.つまり,位置が時間変化しない(動かない)物体の
                一の時間微分は0になる.これは逆に,位置の時間微分が0であれば,
                その物体は動いていない,と言うこともできる.さらに,物体の位置の時間微分が0でない値
                を取るならば,その物体は動いていると言える.

                今回の場合,時間変化するのは領域内の総電気量 $Q(t)$ である.
            }.
        総電気量が変化したということは,その領域内部の電荷の量が変化したということである.
        つまり,個々の電荷が領域外部に出て行ったり,あるいは逆に,領域外部から電荷が入ってきた
        ということである.すなわち,
            \begin{equation*}
                \frac{\df Q(t)}{\df t} \neq 0.
            \end{equation*}
        ここで,出たり入ったりする電荷の,正味の電気量を $I(t)$ と書けば,
            \footnote{
                ここで記号として $I(t)$ を書いたのは,電流 $I(t)$をあとで定義する
                ためである.独立変数として,時間 $t$ を明示したのは,時刻によって
                生じる電流が,異なる場合を想定したからである.
            },
            \begin{equation*}
                \frac{\df Q(t)}{\df t} + I(t) = 0.
            \end{equation*}
        となるような $I(t)$ が存在することになる.要は,領域内部の総電気量 $Q(t)$ の
        時間変化に,正味の電荷の出入り $I(t)$ を足し合わせれば0であるということである.
        もっと簡単に言うと,領域内部の総電気量の変化は,外部との電荷のやり取りで
        生じるのであり,その領域内部でいきなり電荷が現れたり消えたりしない,という
        ことである(図\ref{fig:Denryu_Denka_intoro}(B)参照).この電荷の出入りを表す $I(t)$ が
        電流である.端的に言えば,ある領域内の総電気量が変化したことと,その領域に電流が生じ
        ていることとは,等価である.
        \end{mysmallsec}

        \begin{mysmallsec}{電荷保存の法則}
        実は,この考え方こそが,\textbf{電荷保存の法則} であり
            \footnote{
                略して,\textbf{電荷保存則} といわれることのほうが,一般的である.
            },今の場合は,
        \textbf{大域的(マクロ)な}視点から見た電荷保存則(脚注参照)である.後ほど,一般化して,
        \textbf{局所的な}電荷保存則も紹介することになる.

        改めて,大局的な電荷保存則を記述しておこう.
            \begin{myshadebox}{(大局的な)電荷保存則}
                ある領域内部の総電気量を $Q(t)$ とし,その内部から外部へ向かって電流 $I(t)$ が
                生じている場合,
                \begin{align}
                    \frac{\df Q(t)}{\df t} = - I(t).
                \end{align}
                という関係式が成立する.これを,
                大局的な \textbf{電荷保存の法則}(あるいは略して,\textbf{電荷保存則})という.
            \end{myshadebox}

        ここで,電流 $I(t)$ を右辺に移項した.この表現の方が,総電気量の時間変化と
        電流が等価であることを,イメージしやすいからである.また,多くの教科書で,
        この書き方がなされている.電流の符号は,領域内の総電気量が減る場合には正
        (電荷が領域内部から飛び出して,それが電流となる),
        領域内の総電気量が増える場合には負(領域内に電流が入ってきて,内部の電荷個数が増える)とする
            \footnote{
                簡単に言うと,領域から電流が外向きに生じているときに正,
                領域に電流が吸収される場合に負とする.
            }.
        \end{mysmallsec}
                \begin{figure}[hbt]
                    \begin{tabular}{cc}
                        \begin{minipage}{0.5\hsize}
                            \begin{center}
                                \includegraphicsdouble{Denryu_Denka_intoro001.pdf}

                                (A) 領域内の電荷
                            \end{center}
                        \end{minipage}
                        \begin{minipage}{0.5\hsize}
                            \begin{center}
                                \includegraphicsdouble{Denryu_Denka_intoro002.pdf}

                                (B) 電荷の出入り
                            \end{center}
                        \end{minipage}
                    \end{tabular}
                        \caption{電流と電荷}
                        \label{fig:Denryu_Denka_intoro}
                \end{figure}

%======================================================================
%  SubSection
%======================================================================
    \subsection{局所的な電荷保存則}
        電荷保存則は,局所的にも成立する法則である.理論物理学では,
        大局的表現よりも,これから説明する局所的な電荷保存則の表現
        がよく使われる.大局的な表現がすでに得られているので,そこ
        から,局所的表現を導出しよう.

        まず,電流 $I(t)$ と総電気量 $Q(t)$ を,それぞれ電流密度 $\bi$ と
        電荷密度 $\rho$ で書きなおしておこう.
            \begin{align*}
                I(t)  &=  \sint_{S} \bi \cdot \bn \df S. \\
                Q(t)  &=  \vint_{\Omega_{S}} \rho \df V.
            \end{align*}
        電流道度 $\bi$ と電荷密度 $\rho$ は,位置と時間の関数であることに注意.
            \begin{equation*}
                \bi := \bi(\br,\,t)\,,\quad \rho := \rho(\br,\,t).
            \end{equation*}
        ちなみに,電流 $I(t)$ と総電気量 $Q(t)$ の独立変数に位置 $\br$ が
        ないのは,位置で積分してしまうためである
            \footnote{
                積分変数は積分後には残らない.大局的視点から見るので,
                細かな位置を知る必要はなく,この視点で重要なのは,考察範囲
                全体の電流量や電気量なのである.これから考える局所的な量は,
                その位置も重要になる.大局的視点と局所的視点の違いは意識して
                おくべきことだろう.
            }.
        電流道度 $\bi$ と電荷密度 $\rho$ を用いると,電荷保存則は次のように書ける.
            \begin{equation*}
                 \frac{\rd}{\rd t} \left( \vint_{\Omega_{S}} \rho \df V \right)
               = - \sint_{S} \bi \cdot \bn \df S.
            \end{equation*}

        ここで,上式の右辺にガウスの定理
            \footnote{
                任意のベクトル $\bX$ に対して,
                \begin{equation*}
                      \sint_{S} \bX \cdot \bn \df S
                    = \vint_{\Omega_{S}} \ddiv \bX \cdot \bn \df V.
                \end{equation*}

                後で簡単に復習するので,そこを参照のこと.
                それでもわからなければ,数学の解説の部分を
                読むこと.さらにそれでもわからなかったら,
                ベクトル解析教科書を別途お読みください.
            }
        を適用する.
            \begin{equation*}
                  \vint_{\Omega_{S}} \frac{\rd \rho}{\rd t} \df V
                = - \vint_{\Omega_{S}} \ddiv \bi \cdot \bn \df V.
            \end{equation*}
        この式変形で,左辺の空間微分と時間微分の可換性
            \footnote{
                空間に関する微分と,時間に関する微分は計算順序を入れ替えても,
                結果は変わらないということ.
            }
        を利用した.
        この式の両辺を見ると,積分範囲が同じ体積分になっている.この等式が一般的に成り立つのは,
        両辺の被積分関数が等しいときである
            \footnote{
                数学的に示すべきことだろうが,ここでは割愛する.
            }.
            \begin{equation*}
                  \frac{\rd \rho}{\rd t} = -\ddiv \bi.
            \end{equation*}
        慣習に従って,次のように書き換える.
            \begin{align}\label{eq:bisiteki_denkahozonsoku00}
                \ddiv \bi = -\frac{\rd \rho}{\rd t}.
            \end{align}
        この式(\ref{eq:bisiteki_denkahozonsoku00})が,
        局所的な \textbf{電荷保存則} である.

        ある局所的領域
            \footnote{
                局所的領域:可能なかぎり小さくした領域のこと.あくまでも,
                直感的な言葉であり,「可能なかぎり」に特別な意味は込めていない.
                日常言語的な捉え方をしてもらいたい.
            }
        から電流が湧き出る($\ddiv \bi$)ならば,その領域内部の電荷密度は
        減少する($-(\rd \rho/\rd t)$)ということを,式で表現できている.

        改めて,まとめておこう.
            \begin{myshadebox}{(局所的な)電荷保存則}
                ある局所的領域から電流が湧き出る($\ddiv \bi$)ならば,
                その領域内部の電荷密度は減少する($-\rd \rho/\rd t$).
                \begin{align}
                    \ddiv \bi = -\frac{\rd \rho}{\rd t}.
                \end{align}
                これは,電荷保存則を局所的に表現したものである.
            \end{myshadebox}

            \begin{memo}{ガウスの定理(復習)}
                ガウスの定理:ガウスの法則とは別のもの.ガウスの定理は数学上の
                              定理である.この定理により,以下の等式が成立する.

                                任意のベクトル $\bX$ に対して,
                                \begin{equation*}
                                      \sint_{S} \bX \cdot \bn \df S
                                    = \vint_{\Omega_{S}} \ddiv \bX \cdot \bn \df V.
                                \end{equation*}

                                言葉で説明するならば,次のようなイメージになろう.
                                「任意の閉曲面 $S$ で囲まれた領域 $\Omega_{S}$ より
                                湧き出る $\ddiv \bX$ の総和は,閉曲面 $S$ の表面から
                                抜け出る正味の流出量に等しい」.
            \end{memo}

