%===================================================================================================
%  Chapter : 2つの基本原理
%  説明    : 特殊相対性原理と,光速不変の原理を説明する
%===================================================================================================
%   %==========================================================================
%   %  Section
%   %==========================================================================
    \section{特殊相対性原理}
%       %=======================================================================
%       % SubSection
%       %=======================================================================
        \subsection{物理法則と座標変換}
            物理現象を観測するのは,人間ひとりひとりである.
            当然ながら,ひとりの人間が同時に二つの視点に立って,
            同じ物理現象を観測することは不可能である
                \footnote{
                    「分身の術」なんてのは,考えない.
                }.

            しかし,二人の人間が,同時に,同じ物理現象を
            観測することは可能である.その場合,もちろん,
            二人の観測者の位置は異なっている.二人の観測者を
            A,Bとしよう.ある物理現象を,観測者Aの視点で見て,
            そして,物理法則を見出す.同様に,観測者Bの視点で見て,
            物理法則を見出す.観測者Aが発見した物理法則と,
            観測者Bが発見した物理法則に違いはあるだろうか.
            当然のことながら,両者の視点の違いによる差異はある.
            しかし,これは \textbf{座標変換} という操作で,両方の
            視点に行ったり来たりできる,ということを考えれば,その分の
            差異はなくなる(例えば,ガリレイ変換).
            そして,アインシュタインは,
            「観測者Aと観測者Bが発見する物理法則
            は,全く同じ形(全く同じ式)で,表されるはずだ」と言う.

        \begin{memo}{慣性系の存在 と ガリレイ変換(復習)}
            まず,慣性系が少なくとも1つ存在することを仮定する.
            実際には完全な慣性系を発見することは難しいだろうが
                \footnote{
                    ニュートンの万有引力の法則によれば,質量が存在
                    する場所では重力が存在するので,従って質量
                    付近では慣性系は存在しえない.しかし,ここ
                    ではこのようなことは無視して考える.気分が
                    スッキリしないだろうが,ここは一般相対性理
                    論への第一歩と考えて,この仮定を受け入れて
                    もらいたい.まあ,実際この仮定を受け入れて
                    ニュートン力学を考えてきたので(慣性の法則)
                    すんなり受け入れられるかもしれない.
                },
            ここではこれを理想化して考える.

            さて,慣性系が1つ存在するならば,ガリレイ変換によって,
            いくつもの慣性系が存在すると考えられる.それは簡単に
            示せる.最初に存在を認めた第1の慣性系を $S$ と表現す
            ることにし,この慣性系 $S$ を基準に取りこの速度を0と
            して考える.この基準慣性系 $S$ の位置座標を $\br$ と
            表現する.慣性系 $S$ に対して等速直線運動している慣
            性系を,ガリレイ変換によって考えることができて,この
            慣性系を $S'$ と表現する.慣性系 $S'$ の位置座標
            を $\br'$ と表現する.これら2つの座標系 $S$,$S'$ の
            位置座標には,ガリレイ変換により,
                \begin{align}\label{eq:gari1}
                    \br' =  \br+\bv t
                \end{align}
            という関係がある.ここに,$\bv$ は $S'$ の $S$ を基
            準とした相対速度であり,$t$ は時刻を表現する.この
            相対速度 $\bv$ を様々な具体的な速度で考えられるので,
            複数の慣性系を考えられるというわけである.

            次に,このガリレイ変換の逆変換を考える.逆変換とは慣
            性系 $S'$ から見た 基準慣性系 $S$ の運動を記述するこ
            とである.慣性系 $S'$ は基準慣性系 $S$ から見て相対
            速度 $\bv$ で運動しているので,慣性系 $S'$ から見れ
            ば基準慣性系 $S$ は $-\bv$ の相対速度で運動していな
            ければならない.従って,以下の式が導かれることになる.
                \begin{align}
                    \br =  \br'-\bv t.
                \end{align}
            逆変換における式操作については,“$\br$ と $\br'$ の
            場所を形式的に入れ替えて,速度を $-\bv$ と置き換える”
            ということをする.実際,この式は,式(\ref{eq:gari1})
            と矛盾していない(代数学的式変形で同じ式を導くことが
            できるということである).
            \end{memo}

%       %=======================================================================
%       % SubSection
%       %=======================================================================
        \subsection{特殊相対性原理}
            物理法則はどのような座標系からみても,同じ法則として表されるべきである.
            座標系が異なったら別の物理法則になってしまうのでは,物理法則とはいえない.
            この考え方を根本原理として掲げ,これを \textbf{特殊相対性原理} という.
            "特殊"とつくのは,特殊相対性理論の枠内での原理であることを明示するためである
                \footnote{
                    一般相対性理論ではこの原理は消えてなくなる.
                    「物理法則はどの座標系でも同じ」という \textbf{一般相対性原理} として
                    表現がより強く改められる.
                }.

            後で学習することだが,特殊相対性理論は慣性系での理論であり,一般的な加速度系
            では成立しない
                \footnote{
                    特殊な状況であれば,の加速度系で,特殊相対論を議論することは可能.
                }.

            特殊相対性原理はローレンツ変換
                \footnote{
                    ローレンツ変換とは,ニュートン力学で言うところこの,ガリレイ変換に相当する
                    座標変換である.速度が光速に近い物体を扱う場合には,物体の座標変換は
                    ガリレイ変換に従わず,ローレンツ変換に従うことが明白になる.
                    ガリレイ変換はローレンツ変換の特殊な場合に当たるもので,
                    物体の速度が光速に比べて非常に遅い場合に成り立つものである.
                    ローレンツ変換についても,このあとで学習することになる.
                }
            に対して不変であることを要請する原理であるともいえる.ローレンツ変換については
            このあとの記述する.


%   %==========================================================================
%   %  Section
%   %==========================================================================
    \section{光速不変の原理}
%       %=======================================================================
%       % SubSection
%       %=======================================================================
        \subsection{エーテル(電磁波と光)}
            電磁気学において,電場の波動方程式や磁束密度の波動方程式を
            導出したときに,電磁波が光速で伝わるということを確認した.
            これにより,\textbf{光は電磁波の一種である} ということが予
            言され,実際に実験によって確認されている
                \footnote{
                    このことについては電磁気学の章を参照.
                    また,量子力学によれば,光は光子(photon)
                    とよばれる“粒子”の一種でもあり,こ
                    れによって光は波動性と粒子性とをあわ
                    せもつものと考えられているが,このこ
                    とはここでは考えないことにする.
                }.

            さて,光が電磁波の一種であることが確認されたのであれば,光は
            波動であるということになるので,この光を伝える媒質が存在する
            と考えることは当然のことである
                \footnote{
                    例えば音は波の一種であり,その媒質は空気である.ま
                    た別の例を挙げれば,水面の波を考えられる.もちろん,
                    この水面波の媒質は水である.このように,波である以
                    上は何らかの媒質によってその変化が伝えられると考え
                    るのである.電磁波も波動現象であることが確認されて
                    いるので,電磁波を伝えるような媒質を見つけようとす
                    るのである.
                }.
            実際には,光を伝える媒質は存在しないことがアインシュタインによって
            宣言されるが,ここではこのような媒質を仮定して,この媒質の
            ことを \textbf{エーテル} ということにする.存在しないものに名前を
            つける理由は,歴史的にこれが存在すると考えられていたことも
            あるが,ここでは,エーテルの存在を否定するような実験を確認
            したいからということである.

%       %=======================================================================
%       % SubSection
%       %=======================================================================
        \subsection{マイケルソンとモーレイの実験}
            \begin{mycomment}
                ここでは,エーテルの存在を確認しようとす
                る実験である \textbf{マイケルソンとモーリーの実験} を
                見ていきたいと思う.これは光速度不変の原
                理を説明する1つの実証としてもみることがで
                きる.まず,光とは何かについての復習から
                始めていきたい.
            \end{mycomment}

            マイケルソン
                \footnote{
                    Albert Abraham Michelson (1852-1931,アメリカ)
                }
            と
            モーリー
                \footnote{
                    Edward Williams Morley(1838-1923,アメリカ)\;\;
                    この人物の片仮名表記は,「モーレー」,「モーリー」
                            等,複数存在している.
                }
            は光を伝える媒質であるエーテルの存在を実験によって確かめようとした.

            ここでは,その実験そのものを考えることはせず,
            この実験の要点を確認するだけにとどめる.
            図\ref{fig:MM_J}の装置が,
            エーテル内をABDに対して平行で等速度運動している
            とする.速度の向きはAからDへの向きとする.
                \begin{figure}[hbt]
                    \begin{center}
                        \includegraphicsdefault{MM_J.pdf}
                        \caption{マイケルソンとモーリーの実験1}
                        \label{fig:MM_J}
                    \end{center}
                \end{figure}

            この図\ref{fig:MM_J}のCとDの部分には
            ミラー(鏡)を置いてある.
            Cにあるミラーの傾きは進行方向に対して平行であり,
            Dにあるミラーの傾きは進行方向に対して垂直である.
            Bの部分には45°傾いている
            ハーフミラー
                \footnote{
                        光を半分反射し,
                                残りの半分をそのまま通すもの.
                }
            を置いてある.これによって,
            Aの部分にある光源から出た光は,
            Cへ向かうものとD向かうものに
            分割される.

            さて,この装置を用いて,Eへ到着する
            光を考えてみる.Eへ到着する光の経路は2つあり,
            それは A$\rightarrow$B$\rightarrow$C$\rightarrow$E の
            青色の線の経路(「青経路」ということにする)と,A$\rightarrow$D$\rightarrow$B$\rightarrow$E の
            赤色の線の経路(「赤経路」ということにする)である.すなわち,Eで観測される光は,
            青経路を通った光と
            赤経路を通った光が
            干渉したものになる.

            そこで,この干渉について
            式を使って考えてみる.
            光速を $c$ ,
            また装置の速度を $v$ と表すことにする.装置が速度 $v$ で運動しているので,
            実際の光の経路は以下の図\ref{fig:MM_J2}のようになる
                \footnote{
                    図\ref{fig:MM_J}は実験の内容をわかりやすくするためものであった.
                            従って,図\ref{fig:MM_J}は間違いであり,
                            より正しいのは図\ref{fig:MM_J2}である.
                }.
            但し注意してもらいたいのは,
            図\ref{fig:MM_J2}ではBD間が
            装置の速度の方向に縮まっているように見えてしまうが,
            装置は全体的に同じ速度で運動しているので,
            “BD間の距離が常に一定値 $L$ をとっている”
            ということである.
            \begin{figure}[hbt]
                \begin{center}
                    \includegraphicsdefault{MM_J2.pdf}
                    \caption{マイケルソンとモーリーの実験2}
                    \label{fig:MM_J2}
                \end{center}
            \end{figure}

            ここで大事なのは,この図\ref{fig:MM_J2}のB$\rightarrow$C$\rightarrow$B$'$を通る
            青径路の光と,B$\rightarrow$D$\rightarrow$B$'$を通る
            赤径路の光である.この 青径路を通った光 と 赤径路を通った光 が
            干渉すれば,エーテル
                \footnote{
                    光を伝える媒質のこと.
                }
            が存在するという実証になる.

            次に,この干渉を定量的に数式を用いて考えるために,
            2つの径路を通った光の時間差を計算してみよう.

            まず青径路の光を考える.
            より簡略化したものを以下の図\ref{fig:MM_J3}に描く.
                \begin{figure}[hbt]
                    \begin{center}
                        \includegraphicsdefault{MM_J3.pdf}
                        \caption{マイケルソンとモーリーの実験3}
                        \label{fig:MM_J3}
                    \end{center}
                \end{figure}

            光速 $c$ と装置の速度 $v$ によって,
            BからCへ向かう光の速度の成分は,$\sqrt{c^{2}-v^{2}}$ である.
            これを用いると,BC間の距離が $L$ であるとき,BからCに着くのに要する時間 $T$ を
            考えれば,$T=L/\sqrt{c^{2}-v^{2}}$ であるので,
            結局,BからB$'$へ着くまでにはその2倍の時間がかかっているはずである.
            この時間を $t_{1}$ (BからB$'$に着くのにかかる時間)とすれば,
                \begin{align}\label{eq:MM1}
                    t_{1}=\frac{2L}{\sqrt{c^{2}-v^{2}}}
                \end{align}
            であることがわかる.

            次に,赤径路を通る光について考える.まず,
            簡略した図を図\ref{fig:MM_J4}に描く.
                 \begin{figure}[hbt]
                    \begin{center}
                        \includegraphicsdefault{MM_J4.pdf}
                        \caption{マイケルソンとモーリーの実験4}
                        \label{fig:MM_J4}
                    \end{center}
                 \end{figure}

            反射前の光は,装置の速度$v$だけその速度が増して,
            このときの光の速さは $c+v$ である.従って,
            反射前の光がBからDに到達するのにかかる時間 $t_{\mbox{反射前}}$ は,
            BD間の距離が $L$ であるので,
                \begin{align}\label{eq:MM2}
                    t_{\mbox{反射前}}=\frac{L}{c+v}
                \end{align}
            である.


            光が反射するとき,図\ref{fig:MM_J4}では光の進む距離が短くなっているが,
            実際は装置全体が同じ速度で動いていることから,
            BD間の距離は変わることがない.このことに注意しながら,
            反射後の光について考える.
            光は反射後の光は装置の進行方向とは逆向きに進むので,
            このときの光の速さは $c-v$ である.従って,
            販社後の光がD$'$からB$'$に到達するのにかかる時間 $t_{\mbox{反射後}}$ は,
            B$'$D$'$間の距離が $L$ であるので,
                \begin{align}\label{eq:MM3}
                    t_{\mbox{反射後}}=\frac{L}{c-v}
                \end{align}
            である.

            従って,B$\rightarrow$D$\rightarrow$B$'$の赤径路
            を進む光が,この赤径路を往復するのにかかる時間 $t_{2}$ は,
            式(\ref{eq:MM2})の $t_{\mbox{反射前}}$ と式(\ref{eq:MM3})の $t_{\mbox{反射後}}$ の
            和であり,
                \begin{align*}
                    t_{2}   &= t_{\mbox{反射前}}+t_{\mbox{反射後}} \notag \\
                            &= \frac{L}{c+v} +\frac{L}{c-v} \notag \\
                            &= \frac{2cL}{c^{2}-v^{2}}.
                \end{align*}
            よって,
                \begin{align}\label{eq:MM4}
                    t_{2}&=\frac{2cL}{c^{2}-v^{2}}
                \end{align}
            となる.

            さて,以上で計算してきた結果をまとめてみる.青径路を
            往復するのにかかる時間 $t_{1}$ と赤径路を往復するの
            にかかる時間 $t_{2}$ の差をとると,
                \begin{align}\label{eq:MM_10}
                   t_{1}-t_{2}&=\frac{2L}{\sqrt{c^{2}-v^{2}}}-\frac{2cL}{c^{2}-v^{2}}\notag \\ \notag \\
                   &\cong\frac{2L}{c}\left( 1+\frac{v^{2}}{c^{2}}\right)
                   -\frac{2L}{c}\left( 1+\frac{v^{2}}{2c^{2}}\right) \notag \\
                   &=\frac{L}{c}\frac{v^{2}}{c^{2}}
                \end{align}
            となる.
            式変形の2段目の等号で,近似を用いていることに注意してほしい.
            すなわち,時間差が生じて,干渉縞がEに現れるのである.

            さて,理論上では,干渉縞が観測されるという予測が立てられた.
            ところが,この実験結果は“干渉縞が観測されない”ということである.
            すなわち,
                マイケルソンとモーリーの実験では \textbf{エーテルの存在を確認することができなかった} とい
            うことになる.残念なことに,理論的予測とその実証のための実験結果が矛盾してしまったことになる.

            次の項目で,この理論と実験結果との矛盾の解決の
            一例を考える.但し,完全な解決とはえないものだが$\cdots$.それでも確認する
            理由は,結論に出てくる式は大事であるということと,その変換式の名前(ローレンツ変換式)の由来を
            知るためである.


                \begin{memo}{式(\ref{eq:MM1})の解説}
                    距離 $X$,速度 $V$,時間 $T$ の関係は $X=VT$ であった.
                    ここでは,距離が $X=2L$,速度が $V=\sqrt{c^{2}-v^{2}}$,
                    時間が $T=t_{1}$ であったので,
                        \begin{align*}
                            2L=\sqrt{c^{2}-v^{2}}t_{1}
                        \end{align*}
                    が成立している.これを $t_{1}$ について
                    解くことで,式(\ref{eq:MM1})を得る.
                \end{memo}

                \begin{memo}{式(\ref{eq:MM_10})の近似式の解説}
                    この式変形に用いた近似は以下の近似公式によるものである.
                        \begin{align}
                            (1+x)^{n}=1+nx
                        \end{align}
                    ここに,$x$,$n$ は実数である.
                    まず,式(\ref{eq:MM_10})の式変形における
                    第1の等式の第1項は以下のように表現してもよい.
                    分母と分子をそれぞれ $c$ で割って,式を
                    近似式を適用できる形にしていく.
                        \begin{align*}
                            \frac{2L}{\sqrt{c^{2}-v^{2}}}
                            &=\frac{2L/c}{(\sqrt{c^{2}-v^{2}})/c} \notag \\
                            &=\frac{2L/c}{\sqrt{(c^{2}-v^{2})/c^{2}}} \notag \\
                            &=\frac{2L}{c}\frac{1}{\sqrt{1-\displaystyle\left(\frac{v}{c}\right) ^{2}}} \notag \\
                            &=\frac{2L}{c}\left( 1-\displaystyle\left(\frac{v}{c}\right)^{2} \right)^{-1/2}
                        \end{align*}
                    この式に先ほどの近似式を考慮すれば,
                        \begin{align*}
                            \frac{2L}{c}\left( 1-\displaystyle\left(\frac{v}{c}\right)^{2} \right)^{-1/2}\cong
                            \frac{2L}{c}\left( 1+\displaystyle\frac{1}{2}
                            \displaystyle\left(\frac{v}{c}\right)^{2} \right)
                        \end{align*}
                    を得る.これを式(\ref{eq:MM_10})の式変形における
                    第1の等式の第1項に考慮すると,上のような近似式を得る.
                \end{memo}


%       %=======================================================================
%       % SubSection
%       %=======================================================================
        \subsection{光速不変の原理}
            マイケルソンとモーリーの実験でも推察されるように,
            エーテルという物質は存在しないと考える方がよ
            いのである.これは \textbf{光の速さが,その光源もつ運動の速度にかかわらずに,
            一定の値 $c$ をとる}と解釈してもよい.
            これが \textbf{光速度不変の原理} である.
            このように考えれば,干渉縞が
            現れないのも当然のこととなる.

            アインシュタインはこの考えを,
            特殊相対性理論の論文で,宣言している
                \footnote{
                    ポアンカレも『科学の価値』で
                    この考えを提案している.
                }.

            しかし,光速不変の定理という言い方には
            少し注意が必要である.というもの,
            この光速度不変の原理は,
            慣性系で成り立つものだからである.
            つまり加速度をもった系では,もはやこの法則は
            成立しない.この問題は,
            一般相対性理論によって解決される.
            とにかく,光速度不変の原理は
            慣性系という特殊な状況でのみ成立し得る法則であるということだ.



            \textbf{光速度不変の原理} を以下に書き表す.
            \begin{myshadebox}{光速度不変の原理}
                    『1つの静止系を基準に取った場合,
                    いかなる光線も,それが静止している物体,
                    あるいは運動している物体のいずれから放射されてかには関係なく,
                    常に一定の速さ $c$ をもって伝播する.
                    』(アインシュタイン 著,内山 龍雄 訳,『相対性理論』より)
            \end{myshadebox}

            この光速度不変の原理で注意すべきことは,
            \textbf{観測者の速度が異なる場合,光速は一定値をとることを主張していない}ことである.
            例えば,二人の観測者が異なる速度で運動しているとしよう.この二人の観測者が,
            それぞれ同じ光源から発せられる光の速度を測定したとき,その値が $c$ で一致するとは,
            この光速度不変の原理は,主張しない.

            では,二人の観測者が異なる速度で運動している場合,それぞれ観測する光速は違うのだろうか.
            結論からいえば,観測者がどのような速度で運動していようとも,観測される光速 $c$ で一定値を
            とる,と言える.これを確認しておこう.

            二人の観測者が測定した光速を,それぞれ $c_{1}$,$c_{2}$ とする.$c_{1}$ も $c_{2}$ も
            一定の値であることから,
                \begin{equation*}
                    c_{1}=\phi(v) c_{2}
                \end{equation*}
            となる,$\phi(v) $ が存在するはずである
                \footnote{
                        例えば, $c_{1}=6$,$c_{2}=3$ の場合,$\phi(v)  =2$ である.
                }.
            ここに,$v$ は二人の相対速度であり,$\phi(v)$ の変数である.
            二人とも,同じ方向に進む光の速度を観測すれば,$\phi(v)$ は正の値をとる
            はずである.

            ところで,特殊相対性原理によれば,どんな慣性系でも物理法則は全く同じである.
            つまり,
                \begin{equation*}
                    c_{2}=\phi(-v)  c_{1}
                \end{equation*}
            も成立している.  速度が $-v$ となることに注意.この2式から,
                \begin{equation*}
                    c_{1}=\phi(v)  c_{2}=\phi(v) \, \{\phi(-v)  c_{1} \}
                \end{equation*}
                \begin{equation*}
                    \Leftrightarrow \quad \phi(v) \, \phi(-v)  = 1
                \end{equation*}
                \begin{equation*}
                    \therefore \quad \phi(v)  =1
                \end{equation*}
            となる.つまり,
                \begin{equation*}
                    c_{1}=c_{2}
                \end{equation*}
            である.従って,観測者がどのような速度で運動していようが,光速は
            一定の値をとることがわかる.その光速の値は,1つの慣性系で $c$ を
            示していることから,$c:= c_{1}=c_{2}$ である.再度注意しておくが,
            この結果は,光速度不変の原理が直接示しているのではなく,
            特殊相対性原理との論理的結果であるということである.
            観測者の速度にかかわらず,光速が一定値をとるということは,
            数学でいう定理のようなものである.
            \begin{figure}[hbt]
                \begin{center}
                    \includegraphicsdefault{RT_kousoku_ittti.pdf}
                    \caption{速度の異なる慣性系において,同一光源から生じる光の速度を測定}
                    \label{fig:RT_kousoku_ittti}
                \end{center}
            \end{figure}
