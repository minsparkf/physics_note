\section*{記述する内容(予定)}

学習の内容(予定)は,以下の通りである.
\begin{center}
\begin{description}
    \item[第1部「{\bf 古典力学}」] \mbox{} \\
        \textbf{古典力学} とは,\textbf{ニュートン力学} と \textbf{解析力学} の
        総称である.第1章のニュートン力学では,物体の運動法則とその記述方法,推
        論方法を学ぶ.第2章の解析力学は,ニュートン力学を数学的に整理したもので
        ある.

    \item[第2部「{\bf 電磁気学}」] \mbox{} \\
        電磁気学は,その名の通り,電気と磁気に関する学問である.
        電気と磁気の関係は深い.このことについて学習する.

    \item[第3部「{\bf 特殊相対性理論}」] \mbox{} \\
        自分が光の速さで動けるとしよう.このとき,どのような世界が待っているだろうか.

    \item[第4部「{\bf 熱・統計物理学}」] \mbox{} \\
        熱とは何か.熱をどのように捉えるべきか.

    \item[第5部「{\bf 量子力学}」] \mbox{} \\
        原子のスケール($10^{-10}$[m])で物理法則を考えるとき,
        古典力学では扱えない現象が生じる.原子レベルの世界の
        物理法則をここで確認する.

    \item[第6部「{\bf 一般相対性理論}」] \mbox{} \\
        特殊相対性理論では慣性系を仮定しているが,
        実際は重力の存在のために,完全な慣性系は存在しない.
        重力の存在する場を扱う理論.

\end{description}
\end{center}

なお,参考にした教科書等については,
このノートの最後にまとめて書いておく.

