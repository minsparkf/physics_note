%   %==========================================================================
%   %  Section : ベクトルの定義
%   %==========================================================================
        %==================================================================
        %  SubSection
        %==================================================================
            \subsection{図形的(幾何学的)なベクトル}
                物理を考える上で,\textbf{矢印} はとても有用である.
                力の方向と大きさを直感的に表現するのに,矢印は欠かせない.
                この矢印は,数学的に扱うことができて,数学の世界では,
                矢印のことを \textbf{ベクトル} とよんでいる.

                矢印のトンガリがない方を,矢印の \textbf{始点} といい,
                トンガっている方を,矢印の \textbf{終点} という.
                これにより,方向は矢印の向きで表現できるし,その大きさは
                長さに比例するように描くと約束すれば,大きさも表現できる.
                イメージは,図\ref{fig:yajirusi_vector}に描いたようになる
                    \footnote{
                        当たり前すぎて,描くまでもなかったかな.
                    }.
                        \begin{figure}[hbt]
                            \begin{center}
                                \includegraphicsdefault{yajirusi_vector.pdf}
                                \caption{ベクトル:図形,矢印}
                                \label{fig:yajirusi_vector}
                            \end{center}
                        \end{figure}

                これを文字で表すと,始点を $\mathrm{O}$ とし,また,終点を $\mathrm{P}$ とする
                線分 $\overline{\mathrm{OP}}$ に向きをつけたと考えて,
                    \begin{equation*}
                        \overrightarrow{\mathrm{OP}}
                    \end{equation*}
                と書ける.

            \begin{memo}{ベクトルの位置は不問である}
                同じ大きさで,同じ向きをもつベクトルがいくつか存在する
                場合を考える.これらのベクトルの位置が異なる場合,
                異なるベクトルと考えるべきなのだろうか.
                        \begin{figure}[hbt]
                            \begin{center}
                                \includegraphicsdefault{OnajiVector.pdf}
                                \caption{ベクトル:図形,矢印}
                                \label{fig:OnajiVector}
                            \end{center}
                        \end{figure}

                結論から書くと,これらのベクトルは,同一とみなされる.
                つまり,ベクトルの存在する場所は,そのベクトルの性質
                には何ら関係がないということだ.そもそも,ベクトルと
                は大きさと向きのみをもつ概念であるから,当然,ベクトルが
                同一であるということは,大きさと向きが同じであるということ
                である.ベクトルが存在する場所は,定義には含まれておらず,
                不問とされるのである.

                ベクトルという概念は,力学を数式化するために整備されたものである.
                ベクトルを用いることで,物体にかかる力を数式で表現できるように
                なるのだ.例えば,自分が誰かから引っ張られる場合を想像してみよう.
                同じ大きさの力で,同じ向きに引っ張られるのであれば,引っ張られる
                場所は関係ない.引っ張られる場所が,公園だろうが,公民館だろうが,
                あるいはトイレだろうが,引っ張られるときの感覚は同一のはずである.
                つまり,力そのものは,その発生場所には関係がないのである.ベクトル
                という概念は力学の記述に適するようにつくられたので,力と同様に,
                その存在する場所には依存しないのである.というか,依存しないと定義
                付けるのである
                    \footnote{
                        ベクトルの定義では,場所に依存しないという直接的な記述はないが,
                        定理として成り立つ性質として,このことが保証される.場所に
                        依存しないように定義付けを行っているんだから,当たり前だ.
                    }.
                いつどんな場所でも,
                同じ物体に同じ力をかければ,その位置の変化の仕方もまた同じなのである.
                いや,結果が同じになるように,ベクトルという概念を定義してしまうのだ.
            \end{memo}

        %==================================================================
        %  SubSection
        %==================================================================
            \subsection{代数的なベクトル}
            \begin{mycomment}
                ベクトルを数式で扱えるように,是非とも,
                文字を使ってベクトルを表現したい.
                ベクトルは代数的に表現可能なのだが
                    \footnote{
                        「代数的に表現可能」とは,文字の列として表現
                        することが可能であるということである.
                    },
                2つの書き方がある.書き方の違いにより,次のように
                言葉を割当て,区別する.すなわち,
                    \begin{itemize}
                        \item 横ベクトル
                        \item 縦ベクトル
                    \end{itemize}
                なぜこう言われるかは,後の記述で分かることなので,
                今は気にせず,話を進めよう.
                横ベクトル$\cdot$縦ベクトル共に,どちらも同じように頻繁に使われる.
                どっちも大切である.どちらか一方を採用したら,他方を捨て去るということ
                はない.
            \end{mycomment}

            %==============================================================
            %  SubsubSection
            %==============================================================
                \subsubsection{横ベクトル}
                ベクトルを代数的に表現する方法は2通りあると,先に記述した.
                ここではそのうちの,\textbf{横ベクトル} について,説明する.

                ベクトルをどのように代数的に表現するのか.実は,
                その考え方は簡単で,そのベクトルの始点に原点 $O$ を合わせた
                座標を張ればよい.図\ref{fig:yajirusi_vector_daisu}では
                直交直線座標を描いた.これより,終点の座標を記述する
                ことで,ベクトルを表現できるのである.
                        \begin{figure}[hbt]
                            \begin{center}
                                \includegraphicsdefault{yajirusi_vector_daisu.pdf}
                                \caption{ベクトル:図形,矢印}
                                \label{fig:yajirusi_vector_daisu}
                            \end{center}
                        \end{figure}

                代数的なベクトルは,以下のように記述される.
                    \begin{align}
                        \br = \left[\,x\,,y\,\right].
                    \end{align}
                上の表現では,2次元の平面に存在するベクトルが記述されている.より高次元の
                ベクトル $\bx$ を考える場合,その次元数を $n$ として,
                    \begin{align}
                        \bx = \left[\,x_{1}\,,\,\,x_{2}\,,\,\,\cdots\,,\,\,x_{n}\,\right].
                    \end{align}
                ここで,座標を表すこ記号を,$x$ に統一して添字により区別すよう,
                表現方法を変えた.

                このように,成分を横書きしたベクトル表記を,\textbf{横ベクトル} という.

                ベクトルを表現するのに,太字を用いているのは,単なる実数と
                区別するためである.

            %==============================================================
            %  SubsubSection
            %==============================================================
                \subsubsection{縦ベクトル}
                    横ベクトルが成分を横書きしたベクトル表現だとしたら,
                    \textbf{縦ベクトル} とは成分を縦書きしたベクトル表現である.
                    なんとも安易な考え方だ.縦ベクトルは以下のように記述される.
                    \begin{align}
                        \bx
                        =
                        \left[
                            \begin{array}{c}
                                x_{1} \\
                                x_{2} \\
                                \vdots \\
                                x_{n} \\
                            \end{array}
                        \right].
                    \end{align}

        %==================================================================
        %  SubSection
        %==================================================================
            \subsection{ベクトルの大きさ}
                ベクトルの大きさは,三平方の定理により定義できる
                    \footnote{
                        三平方の定理について一言コメントしておこう.
                        これは数学的には三角関数の余弦定理の特殊な場合である.
                        しかし,ここでは,三平方の定理を距離を定めるひとつの要請
                        として扱うことにする.
                    }.

                まず,2次元ベクトル $\br=\left[\,x\,,y\, \right]$ の場合,このベクトルの
                大きさは三平方の定理によって定められ,
                    \begin{align}
                        |\br| = \sqrt{x^{2} + y^{2}}
                    \end{align}
                である.
                        \begin{figure}[hbt]
                            \begin{center}
                                \includegraphicsdefault{SanHeihouNoTeiri_2D_01.pdf}
                                \caption{三平方の定理(2次元)}
                                \label{fig:SanHeihouNoTeiri_2D_01}
                            \end{center}
                        \end{figure}

                2次元のベクトルの大きさを,3次元ベクトルに拡張しよう.
                図\ref{fig:SanHeihouNoTeiri_3D_01}の色を塗った部分の
                直角三角形に着目する.このとき,2次元の三平方の定理から,
                    \begin{equation*}
                        |\br| = \sqrt{\left(\sqrt{x^{2} + y^{2}}\right)^{2} + z^{2}}
                    \end{equation*}
                が成立している.つまり,
                    \begin{align}
                        |\br| = \sqrt{{x}^{2} + {y}^{2} + {z}^{2}}
                    \end{align}
                という関係があるということだ.もはや変数は4つであり,
                "三平方"という語彙と食い違ってしまったが,とりあえず,
                ここでは,3次元の三平方の定理と表現しておこう.
                        \begin{figure}[hbt]
                            \begin{center}
                                \includegraphicslarge{SanHeihouNoTeiri_3D_01.pdf}
                                \caption{三平方の定理(3次元)}
                                \label{fig:SanHeihouNoTeiri_3D_01}
                            \end{center}
                        \end{figure}

        %==================================================================
        %  SubSection
        %==================================================================
            \subsection{ベクトルの次元}
            私たちは,3次元の世界に住んでいるから,当然,縦$\cdot$横$\cdot$高さ
            の3方向しか把握できない.4つ以上の次元で成り立つ世界を見ることは
            不可能である.しかし,数学的には,4つ以上の方向を考えても,理論的
            に矛盾することはない.つまり,数学的には,より多くの方向をもつベクトルを
            扱えるのである
                \footnote{
                    ただし,4つ以上の方向を肌で感じたり,見たりできない以上,
                    図示することは不可能であるから,4次元以上の世界は完全に
                    文字(数式)だけの記述のみになってしまう.
                }.
            そこで,4つ以上の方向をもつベクトルについて,考える.

            まず,今まで日常生活的に使用してきた,\textbf{次元} という語彙を,
            数学用語として改めてその意味を明確にしておきたい.次元とは,ベクトルの
            成分の個数と定める.ただし,物理学では単位のことを次元と表現するが
                \footnote{
                    物理学では「次元解析」という言葉が使われる.これは,物理学的な単位
                    同士の関係を調べることであり,特に,数式の右辺と左辺の単位に矛盾が
                    ないかを確認することで利用される.しかし,ここで考えている次元とは,
                    意味が異なる(完全に異なるわけではないが)ものとして,考えてもらいたい.
                    少なくとも,いま考える次元とは,空間の方向の数であり,ベクトルの
                    成分の個数のことである.
                },
            それとは別物である.

            あるベクトル $\br$ があり,その成分が $n$ 個であるとしよう.つまり,
                \begin{equation*}
                    \br = \left[r_{1},\,r_{2},\,r_{3},\,\cdots,\,r_{n}\right]
                \end{equation*}
            と成分表示されるとする.この時,$\br$ のことを \textbf{$n$ 次元ベクトル} と
            いう.

                \begin{memo}{$n$ 次元の三平方の定理}
                    2次元と3次元の両方の三平方の定理から推察して,
                    より高次元である $n$ 次元に定理を拡張できる.
                    もちろん,$n$ は任意の自然数である.
                    この拡張に伴って,この定理に改めて名前をつける
                    ことにしよう
                        \footnote{
                            これから定義しようとするのは,$n$ 次元での定理
                            であり,それには $n+1$ 個の数が絡んでくるから,
                            「三平方の定理」では定理の内容と一致しなくなってしまう
                            のだ.
                        }.
                    \textbf{距離の公理} と名付ける.

                    $n$ 次元への拡張は以下のようにして行われる.
                    \begin{align}
                        r := |\br|
                           = {q_{1}}^{2} + {q_{2}}^{2} + \cdots + {q_{n}}^{2}
                           = \sqrt{\sum_{i=1}^{n} {q_{i}}^{2}}.
                    \end{align}
                    ここに,$q_{i}$ はそれぞれ座標を表す.
                \end{memo}

        %==================================================================
        %  SubSection
        %==================================================================
            \subsection{ベクトル空間の定義}
                ベクトルをより一般的に定義しておきたい.そこで,はじめに,
                ベクトルの集合を定義する.ベクトルの持つ性質を列挙して,
                それを満たすものを,\textbf{ベクトル空間} とよぶ.そして,
                ベクトルをベクトル空間の要素として定義することで,ベクトル
                を,数学的に曖昧さのない概念として,扱うことが可能になる.
                    \\
                    \begin{itembox}[l]{\textbf{ベクトル空間}}
                        \begin{dfn}
                            ある集合 $\bU$ が次の条件全てを満たすとき,$\bU$ を \textbf{ベクトル空間} という.

                                集合 $\bU$ の任意の要素 $\ba$,$\bb$,$\bc$ に対して,
                                \begin{description}
                                    \item[\;\;\;1]\;\;\; $(\ba + \bb) + \bc = \ba + (\bb + \bc) $
                                    \item[\;\;\;2]\;\;\; $\ba + \bb = \bb + \ba$
                                    \item[\;\;\;3]\;\;\; $\bo + \ba = \bo$ を満たす $\bo$ がただ一つ存在する.
                                    \item[\;\;\;4]\;\;\; $\ba + \ba' = \bo$ を満たす $\ba'$ がただ一つ存在する.
                                \end{description}
                                が成立する.

                                さらに,任意の実数 $m$,$n$ に対して,
                                \begin{description}
                                    \item[\;\;\;5]\;\;\; $(m + n) \ba = m\ba + n\ba$
                                    \item[\;\;\;6]\;\;\; $m( \ba + \bb )$
                                    \item[\;\;\;7]\;\;\; $(mn)\ba = m(n\ba)$
                                    \item[\;\;\;8]\;\;\; $1\ba = \ba$
                                \end{description}
                                が成立する.

                            特に,$\bo = (\,0,\,0,\, \cdots,\,0\,)$ であり,\textbf{ゼロベクトル} という.
                        \end{dfn}
                    \end{itembox}
                    \\

        %==================================================================
        %  SubSection
        %==================================================================
            \subsection{ベクトルの定義}
                ベクトルを定義しよう.
                \\
                \begin{itembox}[l]{\textbf{ベクトル}}
                    \begin{dfn}
                        ベクトル空間 $\bU$ の要素のことを \textbf{ベクトル} という.
                    \end{dfn}
                \end{itembox}
                \\
