%   %==========================================================================
%   %  Section
%   %==========================================================================
    \section{ファラデーの実験}
%   %==========================================================================
%   %  Subsection
%   %==========================================================================
    \subsection{起電力}
        「起電力」とは電流を発生させるためのエネルギー源である.
        具体的には 電池 と考えてよい.とにかく,電流を発生させる
        ものである.実際は,電流は電荷の移動であるので,従って,
        「起電力」とは電荷を動かすものである.電荷を動かすものと
        いえば,電場である.すなわち,起電力とは電場を導体内に発
        生させるエネルギー源であると言える.エネルギーと仕事の関
        係から,起電力は「導体内において,単位電荷を周させる仕事
        」と考えられる.従って,起電力を $V_{l}$ と書くと,(回路
        として閉ループ $l$ を想定するために,このような添え字を
        つけた.)
            \begin{align}
                V_{l}=\oint_{l} \bE(\br,t)\cdot\bt(\br,t) \,\df l
            \end{align}
        なる関係式を得ることができる.以後,\textbf{起電力} とは
        この $V_{l}$ のこという.


%   %==========================================================================
%   %  Subsection
%   %==========================================================================
    \subsection{磁束}
        イメージを先に書くと,「磁束蜜度の束」である.
        これは以下のように表現される.磁束密度を
        閉ループ $l$ を縁とする面 $S_{l}$ で面積分して,
            \begin{align}
                \Phi_{l}=\int_{S_{l}} \bB(\br,t)
                \cdot\textit{\textbf{n}}(\br,t) \,\df S_{l}
            \end{align}
        である.この $\Phi_l{}$ を \textbf{磁束} という.
            \begin{figure}[hbt]
                \begin{center}
                    \includegraphicsdefault{jisoku_image.pdf}
                    \caption{磁束のイメージ}
                    \label{fig:jisoku_image}
                \end{center}
            \end{figure}

%   %==========================================================================
%   %  Subsection
%   %==========================================================================
    \subsection{電磁誘導の法則}
            前にもビオ$=$サバールの法則の部分で書いたが,エルステッドは電流が生じるとそ
            の周りには磁束密度ができることを発見した
                \footnote{
                    電流を流している導線の近くに,方位磁針をいくつか置いてみると,
                    方位磁針は電流の作る磁束密度の方向に振れる.
                }.
            この現象はビオ$=$サバールの法則によって数学的に表現され,さらに,
            アンペールの法則と磁束密度に対するガウスの法則に形を変えている.

            ファラデーは電流が磁束密度を作るのならば,その逆作用として,磁束密度の
            中に置いた回路に電流が生じるのではないかと考えた.その実験の中で,彼
            は磁束密度の時間変化が電場を発生させることを発見した.そして,ノイマン
                \footnote{
                    Frantz Ernst Neumann:(1798 - 1895, ドイツの物理学者,鉱物学者)
                }
            は,この電磁誘導の法則を次のように定式化した.\\
                \begin{center}
                    \begin{itembox}[l]{\textbf{ファラデーの電磁誘導の法則}}
                        閉曲線 $l$ が張る曲面 $S_{l}$ を貫く磁束 $\Phi_{l}$ が時間変化すると,
                        この閉路 $l$ に起電力 $V_{l}$ が生じる.
                        この起電力 $V_{l}$ の大きさは,磁束 $\Phi_{l}$ の
                        時間変化率$\rd \Phi_{l}/\rd t$ に比例する.
                        また,起電力 $V_{l}$ の向きは,
                        この起電力によって閉路 $l$ に電流が生じるときに,
                        この電流が作る磁束がはじめの
                        磁束の変化を打ち消すような向きである.
                        起電力の向きに関すること
                        は \textbf{レンツの法則} とよばれる.
                        磁束 $\Phi_{l}$ の時間変化による閉路 $l$ 内に
                        生じる起電力 $V_{l}$ を,\textbf{誘導起電力} という.
                        ファラデーの電磁誘導の法則は,次式によって表される.
                        \begin{align}\label{denjiyudo}
                        V_{l}=-\frac{\rd \Phi_{l}}{\rd t}
                        \end{align}
                        誘導起電力によって回路に電流が生じるときに,
                        この電流が作る磁束がはじめの磁束の変化のきを正の向きとした.
                    \end{itembox}
                \end{center}

%   %==========================================================================
%   %  Subsection
%   %==========================================================================
    \subsection{法則の意味(図的イメージ)}
                電磁誘導の最も直感的なイメージを図\ref{fig:denjiyuudou},\ref{fig:denjiyuudou3}に示す
                    \footnote{図\ref{fig:denjiyuudou3}は
                        \url{http://vanity-worth.com/nature-law/lenz-1.htm}より(2008.08.24現在).
                    }.
                磁石が振動することによって,その磁石から生じている磁束が
                時間変化することになる.従って,この磁束密度の時間変化により,
                電場が生じることになる.もし,振動している磁石の周りに回路があるならば,
                回路の導線内には電場が生じ,従って起電力となる.
            \begin{figure}[hbt]
                    \begin{center}
                        \includegraphicsdefault{dennjiyuudou.pdf}
                        \caption{電磁誘導1-1}
                        \label{fig:denjiyuudou}
                    \end{center}
            \end{figure}
            \begin{figure}[hbt]
                    \begin{center}
                        \includegraphicsdefault{dennjiyuudou3.pdf}
                        \caption{電磁誘導1-2}
                        \label{fig:denjiyuudou3}
                    \end{center}
            \end{figure}

            電磁誘導の法則を別のイメージで考えてみる.原理は同じだが,次の例は
            とても面白い現象が得られる.

            コイルに電流を流すと磁束密度が生じることは,アンペールの法則から説明される.
            そこで,互いに近くに置かれた
            コイルを2つ用意し,片方(コイル1)にスイッチと電源を接続して,もう一方(コイル2)には
            何も接続しないようにする.図\ref{fig:denjiyuudou2}参照.

            まず初めの状態として,コイル1が電源に接続されスイッチが切れいている状態にする.
            このときはコイル1に電流が流れて得ておらず,従って,コイル1には磁束密度は
            生じていない.この状態からスイッチを入れてコイル1に電流を流してみると,
            この電流によってコイル1に電流が生じる.この電流は磁束密度を発生させる.
            つまり,コイル1の周りに「磁束密度の変化」があったことになる.
            ファラデーの電磁誘導の法則は,磁束密度の変化が回転する電場を生じさせるという
            ものであったので,
            この磁束密度の変化がコイル2の部分においても生じているはずであり,
            従って,コイル2に回転する電場が生じるはずである.つまり,
            コイル2に電流が生じるのである(電源がつながれていないのにもかかわらず!!).
                \begin{figure}[hbt]
                    \begin{center}
                        \includegraphicsdefault{dennjiyuudou2.pdf}
                        \caption{電磁誘導2}
                        \label{fig:denjiyuudou2}
                    \end{center}
                \end{figure}

            電磁誘導の法則(\ref{denjiyudo})を 電場 と 磁束密度 を用いて
            表現すれば,起電力 と 磁束 の項目から,次のように表現できる.
                    \begin{myshadebox}\textbf{ファラデーの電磁誘導の法則}
                        \begin{align}
                        \oint_{l} \bE\cdot\bt \,\df l =-\frac{\rd}{\rd t}\int_{S_{l}} \bB \cdot\textit{\textbf{n}} \,\df S_{l}
                        \end{align}
                        ここに,$\bE:=\bE(\br,t)$,$\bt:=\bt(\br,t)$,$\bB:=\bB(\br,t)$,$\bn:=\textit{\textbf{n}}(\br,t)$ であり,
                        位置と時間の関数である.これらが時間依存している点が重要である.
                        時間依存していなければ右辺は定数となり,静電場の方程式となる
                            \footnote{
                                この意味で,ファラデーの電磁誘導の法則は,静電場の方程式の
                                時間依存的な拡張と見ることもできる.
                            }.
                    \end{myshadebox}

%   %==========================================================================
%   %  section
%   %==========================================================================
    \section{自己誘導 / 相互誘導}
%   %==========================================================================
%   %  Subsection
%   %==========================================================================
    \subsection{自己インダクタンス}
        アンペールの法則によると,磁束密度 $B$ は電流 $I$ に比例する.
        先に見たとおり,磁束 $\Phi$ は 磁束密度 $B$ に比例するので
            \footnote{
                そのように,磁束を定義したのであった.
            },
        当然,\textbf{磁束は電流に比例する}.
            \begin{equation*}
               \Phi \propto B \propto I.
            \end{equation*}
        従って,磁束 $\Phi$ と電流 $I$ の関係式は,
        比例定数を $L$ としたときに,
            \begin{equation*}
                \Phi = LI.
            \end{equation*}
        これを電磁誘導の法則 $V=-\df \Phi/\df t$ に代入すると,
            \begin{align}
                V=-\frac{\df \Phi}{\df t}
                 =-L\frac{\df I}{\df t}
            \end{align}
        となる.この定数 $L$ を \textbf{自己インダクタンス} とよぶ.

        物理的イメージを考えてみよう.アンペールの法則により,電流は
        その周囲に磁束密度を発生させる.一方で,電磁誘導の法則によれば,
        時間変化する磁束密度の周囲には,電場が生じ,電位差が発生する.
        とすると,電流が流れていなかった導線に,突然に電流が流れ始めると,
        その周囲に磁束密度が発生するのだが,この磁束密度は時間変化するものであるので,
        同時に電位差もその周囲に生まれることになる.当然,いま流れ始めた電流は
        この電位差の影響もうけることになる.電流自身がその周囲に電位差を作り出し,
        その電位差が自身に帰ってくるのである."自己"インダクタンスと表現されている
        のは,この現象に由来している.

        \begin{memo}{磁束は電流に比例する}
            以下のように,簡略表記すると,わかりやすい.
            一重巻きのコイルを想定してみればよい.
            積分形のアンペールの法則は
            \begin{equation*}
                \oint_{l} \bB(\br)\cdot\bt(\br)\df l
                =\mu_{0}\int_{S_{l}} \bi(\br)
                \cdot\textit{\textbf{n}}(\br)\df S_{l}
            \end{equation*}
            であるから,
            \begin{equation*}
                B = \oint_{l} \bB(\br)\cdot\bt(\br)\df l
            \end{equation*}
            \begin{equation*}
                I = \int_{S_{l}} \bi(\br)\cdot\textit{\textbf{n}}(\br)\df S_{l}
            \end{equation*}
            と計算されたとき,
            \begin{equation*}
                B=\mu_{0}I.
            \end{equation*}
        \end{memo}

        \begin{memo}{ソレノイドが作る磁束密度}
            ソレノイド上の導線が作る磁束密度を考えよう.
            いま,ソレノイドを流れる電流 $I$ が定常状態であるとしたとき,アンペールの法則によって
            1回巻きの場合
            \begin{equation*}
                \oint_{l}B\,\df l =\mu_{0}I
            \end{equation*}
            である.$l$ はアンペールの法則に依れば任意の閉曲線
            であるから,ここでは図\ref{fig:sorenoido22}のように閉曲線をとってみる.閉曲線ABCDで,
            辺AB,辺CDの方向には,電流と平行な向きであるので,磁束密度は現れない.
            また,閉曲線ABCDの辺BCの部分には磁束密度は存在しない.なぜなら,この辺BCの部分は磁束密度が
            存在しない無限遠方と同じ空間でなければならないからである.つまり,
            磁束はソレノイドの内部だけに存在することになる.
            \begin{equation*}
            \oint_{l}B\,\df l=Bl
            \end{equation*}
            と計算されるから,
            \begin{align}
                Bl=\mu_{0} I
            \end{align}
            $N$ 回巻きの場合は,これを $N$ 倍すればよく,
            \begin{align}\label{sorenoidoB}
                Bl=N\mu_{0} I \\ \notag
                \therefore\quad B=\frac{N\mu_{0} I}{l}
            \end{align}
            この式(\ref{sorenoidoB})がソレノイド状のコイルに流れる電流がつくる磁束密度である.従って,
            この磁束密度を磁束 $\Phi$ に代入すると,
                \begin{align}
                    \Phi_{l}=BS=\frac{N^{2}S\mu_{0} I}{l}
                \end{align}
            である.ちなみに,この計算では巻き数 $N$ をコイルの総巻き数として計算しているが,
            単位長さあたりの巻き数 $n$ により表現すれば,$n=N/l$ から,$N=nl$ と書き換えて,
                \begin{equation*}
                    \Phi = \frac{(nl)^{2}S\mu_{0} I}{l} = \mu_{0}n^{2}lSI
                \end{equation*}
              となる.
                \begin{figure}[hbt]
                        \begin{center}
                                \includegraphicsdefault{sorenoido11.pdf}
                                \label{fig:sorenoido11}
                                \caption{ソレノイド(外観)}
                        \end{center}
                \end{figure}
                \begin{figure}[hbt]
                        \begin{center}
                                \includegraphicsdefault{sorenoido22.pdf}
                                \caption{ソレノイド(内部)}
                                \label{fig:sorenoido22}
                        \end{center}
                \end{figure}
        \end{memo}

    \begin{memo}{「インダクタ」と「インダクタンス」の違い}
        インダクタとは,現実に存在するコイルのことを意味する.
        インダクタンスと表現した場合には,理論上の比例定数のことをいう.
        似た表現であり,話すときにも混同してしまうことも多々あるが,意味は違うことを
        覚えておこう.
    \end{memo}

%   %==========================================================================
%   %  Subsection
%   %==========================================================================
    \subsection{相互インダクタンス}


%   %==========================================================================
%   %  Subsection
%   %==========================================================================
    \subsection{結合定数}

%   %==========================================================================
%   %  Subsection
%   %==========================================================================
    \subsection{変圧器の原理}
