\section*{このノートについて}
\begin{aboutthisnote}{作成方法}
    \textbf{このノートの作成にはp\LaTeXe を使用させて頂きました}.
\end{aboutthisnote}

\begin{aboutthisnote}{作成動機}
    このノートを作ることにしたのは,
    将来の自分のためです.おそらく,
    今学習している『物理学』の内容を
    忘れてしまっていると思うので,
    その記録をしておこうというものです.
\end{aboutthisnote}

\begin{aboutthisnote}{内容についての重要な注意}
    このノートの内容は
    教科書を読んだ直後に
    書いたものなので,
    教科書の内容そのままである部分が大半です.
    つまり,私のオリジナルではなく,
    様々な教科書をつぎはぎしたものです.
    しかし,(一部の引用を除いて)教科書の丸写しではなく,私なりにその内容を
    解釈して記述していますので,その教科書の内容の正確さが失われてる
    可能性があります.
    何か変な点,矛盾する箇所等があった場合は,
    すぐにその内容を修正するようにして下さい.
\end{aboutthisnote}

\begin{aboutthisnote}{言い訳}
    物理学は教科書がたくさんあり,
    また,趣味で物理学をなさっていて
    Websiteを開いている方々も多いです.
    このノートの内容はほとんどが教科書やWebによるものです.
    しかしそれでも,人から聞いたり本で読んだりした知識ではありますが,
    その知識を自分なりにノートという形でまとめることで自分の中で再定式化できれば,
    新しいものの見方を発見できるかもしれないと思い本ノートの作成を続けています.
\end{aboutthisnote}

\begin{aboutthisnote}{補足}
    このノートは物理学の理論だけを記述しており,その具体的なことは書いていないことが多いです.
    具体例を考えることで,その法則の意味をより深く理解することができますが,
    具体例は割愛しています.しかし,書かないからといって,重要でないということでは決してありません.
    具体的な問題を演習することは,例えば,定義された物理量の
    直感的な把握や,大体の大きさの見当をつけるのに役に立ちます.また,
    複数の法則の結びつき方を感じ取ることもできます.
    従って,具体例は必ず演習書等を用いて,考えなければいけません.
    理論だけ読んでいても,本当にその理論が現実と一致するかどうかは,多くの場合,
    演習を通してでしか確認できません.
    学校や科学館などでもない限り,実験して確かめるということは難しいことが多いのです.
    まあ,理論を読んで満足ができるわけがないことは当たり前なのですが(法則の現実性を
    “実感”しない限り,その法則は私にとって,確かなものではないから),
    演習をおろそかにしがちな自分に注意するために,ここに書いておきました.
\end{aboutthisnote}

\begin{aboutthisnote}{挿入図について}
    図の多くは,寺嶋容明さん
    の作られた「EPS-draw」を用いて,私が作図したものです.
    また,LibreOfficeのDrawで作図したものもあります.
    グラフの作成は,PerlやPythonにより数値計算をして,
    LibreOfficeのcalcで図式化しています.
    gnuplotで書いたところもあります.
    曲面の作成にはフリーの数式処理ソフトmaximaを使って作成
    しているところもあります.
\end{aboutthisnote}

\begin{aboutthisnote}{内容を鵜呑みにしないで下さい}
    このノートはド素人が書いたものであり,正確である保証はありません.
    むしろ,間違っている所があることは確実でしょう.
    特に,計算ミスに不安を感じます.見つけ次第,訂正していきます.
\end{aboutthisnote}

\begin{aboutthisnote}{これはノートです}
    このノートは教科書風の体裁を取っていますが,あくまでも学習ノートです.
    従って,項目の配置は論理的ではありません.メモ書き程度のものです.
\end{aboutthisnote}

\begin{aboutthisnote}{記載順}
    物理学の内容を学習する目的で,このノートをはじめから読もうとしても,
    その内容に行き着くまでには時間がかかることと思います.
    数学の学習と物理学の考え方などが,物理学の前提知識として,先立って
    説明されているからです.
    なので,そういった読み方をしてしまうと,挫折してしまうでしょう.
    このノートの使い方として,まずは目次をみて,参照したい部分を
    見つけて読む,という方法が良いと思います.
\end{aboutthisnote}

\begin{aboutthisnote}{記述精神}
    できるだけ,詳しく書こうと思います.式変形も省略なしにしたいです.
    さらにできることなら,将来,このノートを参考に,理解した物理学を論理的に自身の
    中で再構成した文書を作りたいです.
\end{aboutthisnote}

\begin{aboutthisnote}{脚注の多さについて}
    ノートを記載していた時の心境を脚注に書いておく.
    あとで,なんでそのような記述をしたかを思い起こすヒントになるはず.
    脚注が多すぎて読みにくくなってしまうが,ノートなのでいくら汚く
    書いてもOKとしたい.
\end{aboutthisnote}

\begin{aboutthisnote}{願望}
    私は,物理学を“習得”しようとは考えていません.私にとって物理学の学習は趣味です.
    物理学を勉強する理由は,それを使えるようにするためではなく,物理法則を知りたいからです.
    自然はどのように成立しているのか.自分なりに,物理学を実感したいのです.
    ですので,このノートは単に物理学の表面しか,記述していません
        \footnote{
            もしかしたら,表面もすらも記述していないかもしれない,という不安があります.
        }.

    このノートを書くことが生涯の趣味になることを期待しています.
\end{aboutthisnote}

%%%%\begin{aboutthisnote}{このノートの目標}
%%%%    最後に,学習のひそかな到達目標について.このノートの到達
%%%%    目標を“超電導”にしたいと考えています.卒研の内容が超電導現象に
%%%%    関する実験的なことだったので,理論的な事も知りたいからです.
%%%%    ただ,超電導はとても難しいので,そこに到達できるかどうかは
%%%%    わかりません...
%%%%
%%%%    また,パソコンの動作を物理学から理解したいという願望(私にとっては野望)もある.
%%%%
%%%%\end{aboutthisnote}

    以下,文の読みやすさと文字数のことを考えて,丁寧語による記述をやめます.
