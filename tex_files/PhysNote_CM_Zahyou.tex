\section{座標のとりかた}
\subsection{座標とは}
    空間内の点の位置を指定する手段として,\textbf{座標} はとても重要であり,
    座標なくしては位置を数学的に表すことはできない
        \footnote{
            これ以上に簡潔なものはない,という意味で.
        }.
    今まで主に使ってきた座標は,縦・横・高さの3つの数字であらわされる,\textbf{3次元直交座標} を使っていた.
    しかし,点の位置を指定するのに,この座標系しか考えられないということはない.
    \textbf{3次元空間内の点を表すには,適当に決めた“3つの互いに独立な変数”を用いればよい}.
    座標の取り方は3次元直交座標以外にも,例えば,\textbf{極座標} とか,\textbf{斜交座標} が
    よく取り上げられる.座標によって位置の示し方は異なるが,異なるのは示し方だけであって,どの座標を採用しても
    示される位置に違いはない.極座標を使おうが斜交座標を使おうが,それらを使ってあらわされる位置の特性に
    違いが出るわけではない.

    下手な例えかもしれないが,座標の違いは言葉の違いに似ている.
    現象を英語で表現しようが,日本語で表現しようが,その表現される現象が
    変わるわけではない.ただ,同じ現象を日本語で説明された時と,
    英語で説明された時のイメージ(ニュアンス)は異なるように,
    採用する座標によって主張する部分が変わってくる.極座標は角運動(回転運動)
    を表現するのに簡潔であることが多く,惑星の運動やコマの回転運動などを扱うときは,
    極座標がよく用いられる.

    以下では,座標の種類をいくつか見た後,それらの相互の関係について,考える.
    座標の相互関係は \textbf{座標変換} とう考え方を使って表現される.
    座標変換によって,ある座標から別の座標へ切り替えることが可能になる.
    特に,相対性理論は,座標変換による計算地獄だ.
    ここでは,将来学習する相対性理論も視野に入れつつ,
    座標そのものに対するイメージを習得していこう.

\subsection{直交座標}
    \textbf{直交座標} については説明するまでもない.
    今まで考えてきた $x$,$y$,$z$ を直線軸とし,これらの軸が
    互いに直交するようにとった座標ののことである.
        \begin{figure}[hbt]
            \begin{center}
                \includegraphicsdefault{chokkou.pdf}
                \caption{直交座標}
                \label{fig:chokkou}
            \end{center}
        \end{figure}

\subsection{斜交座標}
    直交座標は各直線軸同士の交わりが,互いに直交していた.
    しかし,直交していない ----- つまり,斜めに交わっていても ,座標として成立する.
    直交していない直線軸で表される座標は \textbf{斜交座標} とよばれる.

    まず,2次元で考えてみよう.直交座標 $x\,$---$\,y$ と,斜交座標 $x'\,$---$\,y'$ の関係
    を考えてみよう.図を描けば図\ref{fig:shakou_chokkou_zahyou}のようになる.
        \begin{figure}[hbt]
            \begin{center}
                \includegraphicsdefault{shakou_chokkou_zahyou.pdf}
                \caption{斜交座標}
                \label{fig:shakou_chokkou_zahyou}
            \end{center}
        \end{figure}

    まず $x'$ を ,$x$ と $y$ で表現することを考えよう.つまり,直交座標 $x\,$---$\,y$ から見た,
    直線 $x'$ の式を考えるのである.直線の式は簡単に書くと,$y=ax+b$ である.$a$ は
    直線の傾きで,$b$ は直線が $y$ 軸と交わる切片である.直線の傾きを知るには,直線の式を
    微分ればよい.
        \begin{equation*}
            \mbox{(直線の傾き)} \; a = \frac{\df y}{\df x}.
        \end{equation*}
    また,今回は,簡単のために
        \footnote{
            「簡単のために」とは,説明が複雑にならないように,条件を簡単にして考えたいということである.
            一般的に考えてしまうと,それ説明するときに式が複雑になり,要点がつかみにくくなってしまう.
            簡単化してしまうと,一般的な議論ができない.しかし,要点をつかむことが大事であるので,
            一般的議論を犠牲にするのである.
        },
    $b=0$ としよう.すると,関係式は次のように簡単になってしまう.
        \begin{align}
            x ' \mbox{の式} : y = \frac{\df y}{\df x}x
        \end{align}

\subsection{極座標}
            極座標の場合の座標変数の取り方は,
            自分の位置する場所を座標原点 O としたとき,
            自分からの距離 $r$ と,水平方向のある1方向を基準にしたときの角度 $\phi$,
            そして,鉛直方向の上向きを基準としたときの角度 $\theta$ の3つの変数を用いる.
            図でイメージするならば図\ref{fig:kyoku2}のようである.但し,
            直交座標との対応関係がわかるように,直交座標も点線で描いている.

           図\ref{fig:kyoku2}のように,
            極座標の3つ独立な座標変数は $(r,\,\theta,\,\phi)$ で
            表現される.それらのとり方は,書いた通りである.
            これらの変数には一応名前がついており,$r$ を \textbf{動径},$\theta$ を \textbf{天頂角},
            $\phi$ を \textbf{方位角} という.
                    \begin{figure}[hbt]
                        \begin{center}
                            \includegraphicsdefault{kyoku2.pdf}
                            \caption{極座標}
                            \label{fig:kyoku2}
                        \end{center}
                    \end{figure}
                    \begin{figure}[hbt]
                        \begin{center}
                            \includegraphicsdefault{kyoku_chokkou.pdf}
                            \caption{直交座標との対応}
                            \label{fig:kyoku_chokkou}
                        \end{center}
                    \end{figure}


            直交座標との関係は,図より
                    \begin{align}
                            \begin{cases}\label{kyoku_chokkou}
                                x=r\sin\theta \cos\varphi \\
                                y=r\sin\theta \sin\varphi \\
                                z=r\cos\theta
                            \end{cases}
                    \end{align}
            となることがわかる.



\subsection{円筒座標}
            円筒座標の場合の座標変数の取り方を説明する.
            そのイメージを図\ref{fig:entou}に描いておくので,
            この図を参照しながら読んでほしい.まず,
            自分の位置する場所を座標原点 O とする.
            そして,この座標原点 O を含むような1つの平面を
            選ぶ.この平面の選び方は複数存在するが,
            そのうちの1つを任意に選ぶとする.図\ref{fig:entou}では $x-y$ 平面に相当する
            ものである.さて,示したい点 $\textit{\textbf{P}}$ から,
            先に指定した平面に垂線を引き,
            垂線の足を $H$ とする.ここでまず一つ目の座標 $z$ を,
            点 $\textit{\textbf{P}}$ から $H$ へ引いた線分の長さとする.
             $\rho$ については座標原点 O から,$x-y$ 平面に
            沿って点 $H$ まで引いた線分の長さというようにとる.
            残りの1つの変数,$\phi$ については,極座標の場合と同様なとり方をする.

            従って,円筒座標で点の位置を示したいときは,$(\rho,\,\phi,\,z)$ と
            いう座標を用いることになる.
                \begin{figure}[hbt]
                    \begin{center}
                        \includegraphicsdefault{entou.pdf}
                        \caption{円筒座標}
                        \label{fig:entou}
                    \end{center}
                \end{figure}


            直交座標との関係は,図より
                    \begin{align}
                            \begin{cases}\label{entou_chokkou}
                                x=\rho\cos\phi \\
                                y=\rho\sin\phi \\
                                z=z
                            \end{cases}
                    \end{align}
            となることがわかる.$z$ 座標は同じである.


\section{異なる座標同士の関係}
\subsection{座標による違い}
                1つの同じ物体(質量:$m$)の運動をみて,二人の観測者がその物体の運動方程式を
                正しく記述したとしよう.運動方程式を書くときには,どのような座標をとるかを
                決める必要がある.座標には,直交座標,極座標など,いろいろな
                表し方があるが,そのどれをとってもよい.

                さて,二人のがそれぞれ書いた運動方程式を比べてみるとき,
                使用した座標が違っていたとしよう.運動方程式の記述の仕方は
                ,使用する座標によって変わってしまう.しかし,2つの運動方程式
                が表す物体の軌跡は,正しく記述されているので,全く同じになるはずである.
                しかし,その関係は,一見してわかりにくいものになるだろう.

                例として,一方の運動方程式を直交座標でかかれたものとし,もう一方を
                極座標でかかれているとしよう.簡単のために,2次元で考え,さらに,
                物体に加わる力は保存力(ポテンシャルは $V(x,\,y)$)であるとしよう.
                直交座標で書かれた運動方程式は,
                    \begin{align}
                        \begin{cases}
                            \displaystyle m\frac{\df^{2} x}{\df t^{2}} = -\frac{\rd V(x,\,y)}{\rd x} \\
                            \displaystyle m\frac{\df^{2} y}{\df t^{2}} = -\frac{\rd V(x,\,y)}{\rd y}
                        \end{cases}
                    \end{align}
                である.同じ運動方程式を,極座標で表すと,
                    \begin{align}
                        \begin{cases}
                            \displaystyle m\biggl\{ \frac{\df^{2} r }{\df t^{2}}  - r\left( \frac{\df \theta}{\df t} \right)^{2} \biggr\}
                                    = -\frac{\rd V(r,\,\theta)}{\rd r} \\
                            \displaystyle m\left( r\frac{\df^{2} \theta }{\df t^{2}} + 2\frac{\df r}{\df t}\frac{\df \theta}{\df t} \right)
                                    = -\frac{1}{r}\frac{\rd V(r,\,\theta)}{\rd \theta}
                        \end{cases}
                    \end{align}
                となる.

                運動方程式を直交座標で表そうが,極座標で表そうが,
                結果として得られる軌道は全く変わらない
                    \footnote{
                        確かに,勝手に選択した座標によって運動方程式を記述しても,結果として
                        得られる物体の運動の軌跡は,1つである.これはこれで,大変驚くべき事だと思う.
                        しかし,不満もあるだろう.運動方程式の形が,座標系によって異なるのである.
                        人間が勝手に決めた座標系によって,運動方程式の形が違うのである.
                        実はこの不満は,解析力学を見てみることで,解決をすることである.
                        ここではしばらく我慢して,異なる座標同士の関係について見ていくことにしたい.
                    }.

                座標は人間が勝手に決められるもので,そのとり方はいろいろある.どの座標を選んでも
                運動の法則を満たし,同じ結果を得ることができる.となると,この座標同士の関係はどう
                なっているかが気になってこよう.

                座標のとり方には,色々な方法があることを確認した.
                次に,それぞれの関係をで考えてみよう.

                    \subsection{直交座標 $\rightarrow$ 斜交座標}
                         直交座標と斜交座標の関係を考えよう.これは,座標同士の関係の中で,
                         最も理解しやすいものである.そしてまた,この関係は物理学ではとても重要な
                         関係である.なので,この関係はどうしても理解したい.
                         感覚的に分かるように,できるだけ細かいステップを踏んで,
                         考えていこう.

\section{座標変換}
                今までは具体的な2つの座標同士の関係を見てきた.
                実用的には,これまでの具体的な関係式のほうが,役に立つ
                ことだろう
                    \footnote{
                        実際の現象を解決するに当たっては,式は具体的に
                        書かれているほうが,使いやすい.多くある関係式
                        から,使いやすい式を選択すればよい.
                    }.
                しかし,理論物理学の目指すところは,一般的に通用する式
                を記述することである.この目標を満足させるためには,任
                意の2つの座標間の,変換関係式を記述ればよい.今までのに
                得た,具体的な関係式を参照しながら,一般的に通用する関
                係式を導いてみよう
                    \footnote{
                        発見的方法で,一般の座標変換式を導くことになる.
                        この方法をとると,「どんな2つの座標系間でも必ず
                        成り立つかどうか」を示す必要が出てくる.しかし,
                        このノートではその証明は省略する.明らかに成立
                        することが,わかるだろうから.いや,面倒臭い
                        から.知りたい場合は,座標変換に関する数学の教
                        科書や,物理学の教科書を参照しよう.
                    }.

                \subsection{座標系の表現の仕方}
                    さて,任意の2つの座標系を文字で表さないといけない.
                    この2つの座標系をそれぞれ,$x$ 座標系と $X$ 座標系
                    のように,小文字と大文字で区別することにしよう.そし
                    て,座標変換を考えるには,基準とする座標系を$x$ 座
                    標系と $X$ 座標系のどちらかに設定すると楽である.
                    ここでは,$x$ 座標系を基準の座標系としよう.基準の
                    座標系とは,もちろん,私に対して静止している座標系
                    のことである.

\section{座標変換と運動方程式}

            \begin{mycomment}
            ある慣性系で,物体の運動方程式をたてる.そして,その慣性系とは別の慣性系に移り,同じ物体の運動方程式を立てたとき,
            その運動方程式は形を変えてしまうのだろうか.結論をいうと,形を変えないのである.それを説明していく.慣性系から
            別の慣性系に移ることを,\textbf{ガリレイ変換} という.ここでいう変換とは \textbf{座標変換} のことである.座標変換をする
            ということには,「物体の観測をする立場を変えてみる」といった意味がある.観測する立場というのは,例えば,
            別の速度(等速度)で運動している状況での観測とか,加速度運動している状況での観測とかということである.
            \end{mycomment}

\subsection{ガリレイ変換}
            ある慣性系 $S_{a}$ から別の慣性系 $S_{b}$ に移ることを,\textbf{ガリレイ変換} という.
            このガリレイ変換について考える.
            系 $S_{a}$ の速度を
            $\bv_{a}$ とし,系 $S_{b}$ の速度を $\bv_{b}$ とする.
            また,系 $S_{a}$ の位置を
            $\br_{a}$ とし,系 $S_{b}$ の位置を $\br_{b}$ とする.このとき,
            系 $S_{b}$ を系 $S_{a}$ から見たときの相対速度 $\textit{\textbf{V}}_{ba}$ は
                \begin{align}
                    \textit{\textbf{V}}_{ba}=\bv_{b}-\bv_{a}
                \end{align}
            である.系 $S_{a}$ も系 $S_{b}$ も両方とも等速直線運動しているので,$\textit{\textbf{V}}_{ba}$ も
            一定の値をとる.この式を
                \begin{align}
                    \bv_{a}=\bv_{b}+\textit{\textbf{V}}_{ba}
                \end{align}
            のように変形して,両辺に時間 $t$ を掛けることによって,
                \begin{align}
                    \bv_{a}t&=\bv_{b}t+\textit{\textbf{V}}_{ba}t \notag \\
                    \Leftrightarrow\br_{a}&=\br_{b}+\textit{\textbf{V}}_{ba}t
                \end{align}
            という式が成り立つ.($\bv_{a}t$,$\bv_{b}t$ は
            絶対静止系に対する位置である.)上式は系 $S_{a}$ の位置 $\br_{a}$ から見た式である.
            ここで物体の運動方程式を作って,$m(\df^{2}\br_{a}/\df t^{2})=\bF$ とする.
            このとき,系 $S_{b}$ から物体を見ると運動方程式がどうなるかを考える.
            別の慣性系に移って,系 $S_{b}$ の位置 $\br_{b}$ から見れば,
                \begin{align}
                    \br_{b}&=\br_{a}-\textit{\textbf{V}}_{ba}t
                \end{align}
            である.この式が,系 $S_{a}$ から系 $S_{b}$ に移ったときの \textbf{ガリレイ変換} を表現する式である.

            この $\br_{b}$ について,運動方程式をたてると,
             \begin{align}
                m\frac{\df^{2}\br_{b}}{\df t^{2}}
                &=m\frac{\df^{2}\br_{a}}{\df t^{2}}-m\frac{\df^{2}\textit{\textbf{V}}_{ba}t}{\df t^{2}} \notag \\
                &=m\frac{\df^{2}\br_{a}}{\df t^{2}} \notag \\
                &=\bF
            \end{align}
            \begin{align}
                \Leftrightarrow m\frac{\df^{2}\br_{b}}{\df t^{2}}=\bF
            \end{align}
            である.この方程式は $m(\df^{2}\br_{a}/\df t^{2})=\bF$
            と同じ形をしている.
            すなわち,系 $S_{a}$ から運動を見ようが,系 $S_{b}$ から運動を見ようが,
            運動方程式は全く同じ形になることがわかる.このことを,
            ニュートンの運動方程式は \textbf{ガリレイ変換に対して不変である} という.

            今までは2つの慣性系が両方とも,速度をもつとして考えてきたが,
            これら2つの慣性系のうちどちらか一方が静止している場合を
            考えても,このガリレイ変換の式が変わることはない.
            なぜならガリレイ変換は,慣性系間の相対速度が
            重要なのであって,それら慣性系の絶対的な速度には関係がないからである.
            そこで,以下では基準となる慣性系の速度を0として考える.

            以上のことを踏まえて,ガリレイ変換を以降で使いやすい形にまとめておく.
            \begin{myshadebox}{ガリレイ変換}
                慣性系 $S$ と,$S$ に対して速度 $\bv$ で等速直線運動している慣性系 $S'$ を考える.
                ここに慣性系 $S$ の速度を0とし,基準の慣性系とする.
                基準慣性系 $S$ の位置座標を $\br$ で表現し,$S'$ の位置座標を $\br'$ で表現する.
                このとき,これらの2つの慣性系 $S$,$S'$ の座標間には以下の関係がある.
                \begin{align}
                    \br' =  \br+\bv t
                \end{align}
                ここに $t$ は時刻を表現するものである.
            \end{myshadebox}

\subsection{加速度系における物体の運動}
                2つの直交座標系を用意する.
                これらの座標系に名前を付けて,
                それぞれ系 $S_{1}$,系 $S_{2}$ とする.系 $S_{1}$ は等速直線運動しているとする.
                両方の座標系の原点が一致したとき,時刻を $t=t_{0}$ とする.
                また,各系は自転しないとする.
                今から,\textbf{系 $S_{1}$ での物体の運動を,系 $S_{2}$ から見ることを考える}.
                系 $S_{2}$ は系 $S_{1}$ に対して正方向に動いているとする.
                この変換に重要なのは位置である.座標と位置は全く同じことである.
                 位置を得れば,それを時間微分することによって,速度を得ることができるし,
                さらに時間微分することによって,加速度を得ることができる.

                「加速度をもった座標系」の図で,$\br_{1}$ と$\br_{2}$ の関係を見てほしい.
                 $\br_{1}$ が系 $S_{1}$ から見たときの 物体の位置であり,
                 $\br_{2}$ が系 $S_{2}$ から見たときの物体の位置である.$\textit{\textbf{R}}$ は
                 系 $S_{1}$ から見た,系 $S_{2}$ の原点の位置である.もちろん,系 $S_{2}$ から系 $S_{1}$ の原点を
                 見るならば,$-\textit{\textbf{R}}$ となる.

                            \begin{figure}[htbp]
                                \begin{center}
                                    \includegraphicsdefault{gari1_fix.pdf}
                                    \caption{加速度をもった座標系}
                                    \label{fig:gari1}
                                \end{center}
                            \end{figure}

                「加速度をもった座標系」の図,以下の式が成り立つ.
                    \begin{align}
                        \br_{1}=\br_{2}+\textit{\textbf{R}}
                    \end{align}
                この式を,$\br_{2}$ での位置 $\br_{2}$ について解くと,
                    \begin{align}
                        \br_{2}=\br_{1}-\textit{\textbf{R}}
                    \end{align}
                となる.この座標 $\br_{2}$ について,質量 $m$ をもつ物体の運動方程式をたてると,
                    \begin{align}\label{kasokudo3}
                        m\frac{\df^{2}\br_{2}}{\df t^{2}}
                        &= m\frac{\df^{2}\br_{1}}{\df t^{2}}-m\frac{\df^{2}\textit{\textbf{R}}}{\df t^{2}}
                    \end{align}
                と変形される.すなわち,系 $S_{2}$ では,$-m(\df^{2}\textit{\textbf{R}}/\df t^{2})$ の
                力を受けれていることになる.

                ガリレイ変換との違いは,ガリレイ変換が慣性系から慣性系への変換であったのに対して,
                今度の変換は,慣性系から加速度系への変換である ということである.
                すなわち,$\textit{\textbf{R}}$ が
                加速度をもっているために,$-m(\df^{2}\textit{\textbf{R}}/\df t^{2})$ という力が
                生じてしまったのである.
                この力のことを,\textbf{見かけの力} という.

                見かけの力は普段の生活でも経験している.
                例えば,電車の中にいるとき,電車が動き始めるときや止まるときに
                「何か」に引っ張られる感覚がある.この「何か」が,見かけの力である.

                電車の中に固定された物体の動きを考える.
                電車の外にいる人から物体を見れば,
                「電車の中の物体は,電車と共に加速度運動している」として,物体の動きを
                    \begin{align}\label{kasokudo1}
                        m\frac{\df^{2}\br}{\df t^{2}}
                        &= \bF
                    \end{align}
                と書くだろう.しかし,電車の中にいる人に言わせれば,「物体は電車に対して静止している」から,
                物体の運動方程式は,
                    \begin{align}\label{kasokudo2}
                        0 &= \bar{\bF}
                    \end{align}
                と主張する.この違いは,両者の立場が違うために生じている.電車の外にいる人は慣性系の立場であり,
                電車の中にいる人は加速度系の立場にいるのである.電車の外にいる人が書く運動方程式を
                    \begin{align}
                        \bF-m\frac{\df^{2}\br}{\df t^{2}} &= 0
                    \end{align}
                と変形すると,式(\ref{kasokudo1}),式(\ref{kasokudo2})は
                    \begin{align}
                        \bF-m\frac{\df^{2}\br}{\df t^{2}} &= \bar{\bF}
                    \end{align}
                の関係があることがみえる.この式の $-m(\df^{2}\br/\df t^{2})$ が見かけの力を
                表現しているのである.

\subsection{見かけの力}
                加速度系で現れる見かけの力と,慣性系で現れる通常の力の差異はほとんどない.
                この二つの力の違いは,
                見かけの力には\textbf{作用反作用の法則が成立しない}ということだ.
                これ以外には,見かけの力は,慣性系における力と全く同じ働きをする.

                では,作用反作用の法則が成立しない,見かけの力とは,いったい何なのか.
                アインシュタインは一般相対性理論の構築の際に,
                見かけの力と重力は等価であるを主張した.
                これは \textbf{アインシュタインの等価原理} とよばれている.
                この見かけの力について,簡単に触れておこう.

                ある基準となる座標系を設定し,系Sと名付ける.系Sは等速運動しているとする.
                この系Sに対して加速度運動している座標系を想定し,これを系Tと名付ける.


\subsection{アインシュタインの等価原理}
                アインシュタインは,慣性質量と重力質量の等価原理 $m_{\mathrm{i}}=m_{\mathrm{g}}$ から,
                非慣性系における見かけの力と重力は本質的に同じことであるという考えを指摘する.
                実際に自分が力を受けたときには,重力によるものなのか,それとも見かけの力に
                よるものなのかを区別することはできないというのである.このことは,一般相対
                性理論という分野で語られる.

                簡単に具体例で触れておこう.ここでは,地球上で落下する物体について考えてみよう.
                地球上にある物体はもちろん,地球から引力を受けていて,物体の運動方程式は
                    \begin{align}\label{E1toukgagenri}
                        m_{\mathrm{g}}\textit{\textbf{g}}=\bF
                    \end{align}
                と書ける.ここに,$m_{\mathrm{g}}$ は重力質量である.一方で,あらゆる物体は
                慣性質量をもっていて,つまりこの物体も慣性質量があり,
                これを $m_{\mathrm{g}}$ と書けば,物体の運動方程式は,
                    \begin{align}\label{E2toukgagenri}
                        m_{\mathrm{i}}\frac{\df^{2} \br}{\df t^{2}}=\bF
                    \end{align}
                である.

                式(\ref{E1toukgagenri})と式(\ref{E1toukgagenri})の両式における右辺の
                力 $\bF$ は地球が物体に及ぼす力であるから,同一のものである.従って,
                    \begin{align}\label{E1}
                        m_{\mathrm{g}}\textit{\textbf{g}}=m_{\mathrm{i}}\frac{\df^{2} \br}{\df t^{2}}
                    \end{align}
                の関係が成立している.ここで等価原理,すなわち重力質量と慣性質量が等しい
                ということを考えれば,$m_{\mathrm{i}} := m_{\mathrm{g}}$ は両辺で打ち消すことができて,
                    \begin{align}\label{E2}
                        \textit{\textbf{g}}=\frac{\df^{2} \br}{\df t^{2}}
                    \end{align}
                となる.

                式(\ref{E2})を解釈すれば,重力加速度と慣性質量によって生じた加速度は等しいということになる.
                「重量によって生じた加速度と,慣性質量の運動方程式による加速度は全く同一である」というのだ.
                もっと具体的にいうと,\textbf{慣性力と重力は同等である} と表現できる.
                この式(\ref{E2})を \textbf{アインシュタインの等価原理} という.
                この等価原理は,ニュートン力学での等価原理 $m_{\mathrm{i}} := m_{\mathrm{g}}$ からの考察で得た
                ものだから,両等価原理は同じことを主張するということなる.

                \begin{memo}{例1:電車の加速度運動}
                    例えば,駅で停車している電車が走り出すときには加速度が生じる.
                    電車の中にいる人にとっては,加速度と反対方向の力を感じることになる.
                    この力のことを特に,\textbf{見かけの力} あるいは \textbf{慣性力} という.
                    アインシュタインの主張する等価原理によれば,この見かけの力を重力と認識しても
                    間違いではない.
                    現実には,電車は目標の速度に達したときには加速を止めて等速運動に至るので,
                    見かけの力だったのだと思い直すこともできる.しかし,もし,電車が加速を止めなければ,
                    電車の中にいる人には加速による慣性力なのか新たに生じた重力なのかの見分けがつかないはずだ.
                \end{memo}

                \begin{memo}{例2:エレベータ}
                    また別の例では,人の乗ったエレベータの自由落下というのがある.
                    エレベータが自由落下しているときに,この中にいる人は“無重力”を感じるのである.
                    地上にいる人からエレベータとその中の人を見れば,
                    当然,「エレベータと人は同じ加速度で自由落下している」と主張する.
                    しかし,エレベータとともに自由落下する人にとっては事情が違う.
                    エレベータの加速度と,自身の落下加速度が等しいので,結局その
                    速度同士が打ち消されて,加速度が0となる.ニュートンの運動方程式によれば,
                    加速度が0であるということは力が加わっていない状態であるから,
                    つまり無重力を感じるということだ.
                \end{memo}

                \begin{memo}{「慣性質量」と「重力質量」の等価性とは意味が違う}
                    もう一度,確認しておこう.エトヴェシュの行った実験は,
                    慣性質量 $m_{\mathrm{i}}$ と重力質量 $m_{\mathrm{i}}$ は等価
                    であることを確かめたものである.これに対して,アインシュタインの
                    等価原理は,重力加速度と見かけの力が等価であることを主張するもの
                    である.重力質量と慣性質量の等価原理から,アインシュタインの等価
                    原理が導かれるのだが,両者の意味は異なる.
                \end{memo}

\subsection{極座標と運動方程式}
                ここでは極座標 $\left(r,\,\theta,\,\phi\right)$ での質点の運動方程式を求める.これは
                簡単であって,直交座標の $\left(x, y, z\right)$ を極座標の $\left(r,\,\theta,\,\phi\right)$ に
                書き換えればよい.
                \begin{align}
                            \begin{cases}\label{un_hou_kyoku}
                                \displaystyle m\frac{\df^{2} r}{\df t^{2}}=F_{r} \vspace{2mm} \\
                                \displaystyle m\frac{\df^{2} \theta}{\df t^{2}}=F_{\theta} \vspace{2mm} \\
                                \displaystyle m\frac{\df^{2} \phi}{\df t^{2}}=F_{\phi}
                            \end{cases}
                \end{align}

                問題となるのは,直交座標における運動方程式 と 極座標における運動方程式 の
                関係である.一般に両者の形は異なった形をとる.
                もちろん,一方の方程式から得る解
                と,もう一方から得る解は同じである.これは,式(\ref{kyoku_chokkou})の関係式を考慮することで
                確認できる
                \footnote{
                どちらの座標をとろうが,得られる解は同じであるので,最初に設定する座標系は
                後の具体的な計算のことを考えれば,その計算が楽になるほうをとるべきだ.
                }.

                具体的に考えてみる.自由に運動できる
                \footnote{
                「自由に運動できる」とは,ポテンシャルが存在しないことを意味する.
                }
                質点の運動方程式を,直交座標系と極座標系の両座標系でそれぞれ
                立ててみて,各座標系で立てた運動方程式の解を求めてみる.そして,
                式式(\ref{kyoku_chokkou})の関係式から,得られた解が等しいかを
                考える.

                直交座標系 $\br=\left(x,\,y,\,z\right)$ での質点運動方程式は,
                各座標成分に分割して書けば
                \begin{align}
                            \begin{cases}\label{un_hou_chok}
                                \displaystyle m\frac{\df^{2} x}{\df t^{2}}=F_{x} \vspace{2mm} \\
                                \displaystyle m\frac{\df^{2} y}{\df t^{2}}=F_{y} \vspace{2mm} \\
                                \displaystyle m\frac{\df^{2} z}{\df t^{2}}=F_{z}
                            \end{cases}
                \end{align}
                であった.
