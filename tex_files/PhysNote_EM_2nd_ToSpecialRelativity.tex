%===================================================================================================
%  Chapter : マクスウェル方程式のポテンシャル表示
%  説明    : マクスウェル方程式を,ベクトルポテンシャルA とスカラーポテンシャルphi を
%            用いて表す.また,ゲージ変換についても考える.特殊相対性理論への導入も,
%            マクスウェル方程式のガリレイ変換ができないことを通して,行う.
%===================================================================================================
%   %==========================================================================
%   %  Section
%   %==========================================================================
    \section{マクスウェル方程式のガリレイ変換}
        \subsection{マクスウェル方程式に対してガリレイ変換は適用できない}
            マクスウェル方程式をガリレイ変換すると,式の形が変わってしまう.
            つまり,残念ながら,マクスウェル方程式はガリレイ変換に対して不変でない.

            式で示すと,S座標系と S'座標系で式の形が違うということであり,
                \[
                {\nabla}^{2} -\frac{1}{{c}^{2}}\frac{{\rd}^{2}}{{\rd t}^{2}}
                \neq
                {\nabla '}^{2} -\frac{1}{{c}^{2}}\frac{{\rd}^{2}}{{\rd t'}^{2}}
                \]
            ということ.要するに,波動方程式はガリレイ変換にできないので,
            波動方程式をその内部に含むマクスウェル方程式もガリレイ変換に従わないということだ.

            しかし,マクスウェル方程式が間違っているのではない.
            マクスウェル方程式は電磁波など現象を予言できるし,
            電磁気現象を十分に説明できる方程式であり,誤っているとは考えにくい.

            では,何がおかしいのか.実は,そもそも,ガリレイ変換がおかしいのである.
            特殊相対性理論により,ガリレイ変換は,光速よりも十分に遅く等速運動する物体に
            対して有効な変換であることが分かった.より一般的な変換法則はローレンツ変換
            であり,ガリレイ変換はローレンツ変換の特殊な場合
                \footnote{
                    ここでいう特殊な場合とは,光速に対して,とても遅く運動するという状況である.
                }
            に過ぎない.そして,マクスウェル方程式はローレンツ変換に対しては不変である.

            以下で,このことを数式を使いながら考えていこうと思う.
            よく教科書では,
                1) ガリレイ変換を偏微分の公式に形を変えて,
            その後に,
                 2) 電磁波の波動方程式がその偏微分で表されたガリレイ変換の式に不変でないことを示している.
            このノートでも同じような手法をとる.計算課程も詳しく記載しておこう
                \footnote{
                    多くの教科書では,計算が当たり前すぎるためなのか,紙面の都合上の問題なのか,
                    計算過程が示されておらず,結果のみが記されている.
                }.

        \subsection{ガリレイ変換と偏微分演算子}
            \subsubsection{時間微分の計算}
                座標変換により,$S=S'(x',\,t')$ と変換できるとする.
                このとき,合成関数の微分を用いると,\textbf{時間微分} は以下の通り.
                    \begin{align*}
                        \frac{\rd S }{\rd t'} &=   \frac{\rd S }{\rd x } \frac{\rd x }{\rd t'}
                                                + \frac{\rd S }{\rd t } \frac{\rd t }{\rd t'} \\
                                            &=   \frac{\rd x }{\rd t'} \frac{\rd S }{\rd x }
                                                + \frac{\rd t }{\rd t'} \frac{\rd S }{\rd t } \\
                                            &=   \left(
                                                \frac{\rd x }{\rd t'} \frac{\rd   }{\rd x }
                                                + \frac{\rd t }{\rd t'} \frac{\rd   }{\rd t }
                                                \right) S.
                    \end{align*}

                ガリレイ変換の場合,空間座標は $x=x'+Vt'$($V$ は S系と S'系の相対速度)なので,
                    \begin{align*}
                        \frac{\rd x }{\rd t'} = \frac{\rd }{\rd t'} \left( x'+Vt' \right)
                                              = V.
                    \end{align*}
                さらに,
                    \begin{align*}
                        \frac{\rd x}{\rd x'}  = \frac{\rd }{\rd x'} \left( x'+Vt' \right)
                                              = 1
                    \end{align*}
                が成立する.従って,
                    \begin{align*}
                        \frac{\rd S }{\rd t'} &=   \left(
                                                  \frac{\rd x }{\rd t'} \frac{\rd   }{\rd x }
                                                + \frac{\rd t }{\rd t'} \frac{\rd   }{\rd t }
                                                  \right) S \\
                                              &=   \left(
                                                  V \frac{\rd   }{\rd x }
                                                + 1 \frac{\rd   }{\rd t }
                                                  \right) S.
                    \end{align*}
                微分演算子の部分を抽出すると(両辺の $S$ の記述を省略すると)
                    \begin{align}
                        \frac{\rd  }{\rd t'} =  V \frac{\rd   }{\rd x } + \frac{\rd   }{\rd t }
                    \end{align}
                を得る.よく見る式が現れた.

                $y$ と $z$ も同様に(書くまでもない気がするが),
                    \begin{align}
                        \frac{\rd  }{\rd t'} &=  V \frac{\rd   }{\rd y } + \frac{\rd   }{\rd t } \\
                        \frac{\rd  }{\rd t'} &=  V \frac{\rd   }{\rd z } + \frac{\rd   }{\rd t }
                    \end{align}
                である.
            \subsubsection{空間微分の計算}
                同じように,\textbf{空間微分} は以下のようになる.
                    \begin{align*}
                        \frac{\rd S }{\rd x'} &=   \frac{\rd S }{\rd x } \frac{\rd x }{\rd x'}
                                                + \frac{\rd S }{\rd t } \frac{\rd t }{\rd x'} \\
                                              &=   \frac{\rd x }{\rd x'} \frac{\rd S }{\rd x }
                                                + \frac{\rd t }{\rd x'} \frac{\rd S }{\rd t } \\
                                              &=   \left(
                                                  \frac{\rd x }{\rd x'} \frac{\rd   }{\rd x }
                                                + \frac{\rd t }{\rd x'} \frac{\rd   }{\rd t }
                                                  \right) S.
                    \end{align*}
                ここで,さっき計算した $\rd x/\rd x' = 1$ であることと,
                        \begin{align*}
                            \frac{\rd t }{\rd x'} = 0
                        \end{align*}
                であることを考慮すれば,
                        \begin{align*}
                            \frac{\rd S }{\rd x'} &=   \frac{\rd x }{\rd x'} \frac{\rd S }{\rd x }
                                                    + \frac{\rd t }{\rd x'} \frac{\rd S }{\rd t } \\
                                                  &=   \left(
                                                      1 \frac{\rd   }{\rd x }
                                                    + 0 \frac{\rd   }{\rd t }
                                                      \right) S \\
                                                  &= \frac{\rd   }{\rd x } S
                        \end{align*}
                を得る.微分演算子の部分を抽出すると(両辺の $S$ の記述を省略すると)
                        \begin{align}
                            \frac{\rd   }{\rd x'} =  \frac{\rd   }{\rd x }
                        \end{align}
                を得る.これもまた,よく見る式だ.

                $y$ と $z$ に関しても同様に計算して,
                        \begin{align}
                            \frac{\rd   }{\rd y'} &=  \frac{\rd   }{\rd y } \\
                            \frac{\rd   }{\rd z'} &=  \frac{\rd   }{\rd z }
                        \end{align}
                となる.



%   %==========================================================================
%   %  Section
%   %==========================================================================
    \section{ローレンツ変換}

%   %==========================================================================
%   %  Section
%   %==========================================================================
    \section{共変形式にむけて}
        \subsection{4元電流}
            今までの電流密度 $\bi$ は3次元ベクトルとして考えていた.
            これを以下のように,4次元に拡張する.
            4次元に拡張した電流のことを \textbf{4元電流} とよぶことにしよう
                \footnote{
                    細かいことを言うと,\textbf{4元電流密度} である.
                }.
            このノートでは,4元電流を表す記号として,$\bj$ を使うことにする.
                \begin{align}
                    \bj = ({j}_{0},\,{j}_{1},\,{j}_{2},\,{j}_{3}) := (c\rho, \bi) = (c\rho,\,{i}_{1},\,{i}_{2},\,{i}_{3}).
                \end{align}
            ここに,$c$ は光速であり,$\rho$ は電荷密度である.

            3次元の電流密度 $\bi$ に対して,形式的に,その第1成分に $c\rho$ を追加しただけだ.

            念のため,これを電流の座標成分としてよいか,次元の確認をしておこう.
            光速 $c$ の次元は [m/s], 電荷密度 $\rho$ の次元は [C/$\mbox{m}^{3}$] なので,
                \begin{equation*}
                    \left[\frac{\mbox{m}}{\mbox{s}} \right]
                    \left[\frac{\mbox{C}}{\mbox{m}^{3}}\right]
                    =
                    \left[\frac{\mbox{m}}{\mbox{s}} \right]
                    \left[\frac{\mbox{A} \cdot \mbox{s}}{\mbox{m}^{3}}\right]
                    =
                    \left[\frac{\mbox{A}}{\mbox{m}^{2}} \right].
                \end{equation*}
            と計算される.電流密度の次元に一致することが確認できた.



