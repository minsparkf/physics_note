
%======================================================================
%  Section
%======================================================================
\section{ローレンツ力(電磁気力)}
    クーロン力 $\bF_{\mathrm{Coulomb}}$ と
    磁束密度に関するローレンツ力 $\bF_{\mathrm{Lorentz}}$ に
    よって,電気的な力と電荷が磁束密度より受ける力を記述する方法を得た.
        \begin{align*}
               & \bF_{\mathrm{Coulomb}} = q\bE. \\
                &\bF_{\mathrm{Lorentz}} = q\bv\times\bB.
        \end{align*}

    ところで,これまで「磁束密度に関するローレンツ力」という表現をしきりに使用
    してきた.わざわざ“磁束密度に関する”なんていう但し書きのような言い回しを
    してきたのには,理由がある.それは,単に \textbf{ローレンツ力} といった
    とき,それは
        \begin{align*}
            \bF &= \bF_{\mathrm{Coulomb}} + \bF_{\mathrm{Lorentz}} \\
                &= q(\bE + \bv\times\bB)
        \end{align*}
    のようなクーロン力と磁束密度に関するローレンツ力の和を指すからである
        \footnote{
            ちなみに,“電場に関するローレンツ力”なんてものは存在しない.ただ,
            磁束密度 $\bB$ や電荷 $q$ が0のときのような場合のローレンツ力は,
            クーロン力と等しくなり,電場に関するローレンツ力といっても間違い
            ではないとは思うが,一般的に通用する語彙ではない.
        }.
    \begin{myshadebox}{ローレンツ力}
        ある空間に電場 $\bE$ と磁束密度 $\bB$ が存在するとき,
        電気量 $q$ をもつ点電荷(以下,電荷 $q$ と書く)が
        速度 $\bv$ で運動しているならば,この電荷 $q$ には
        次式で示す \textbf{ローレンツ力} $\bF$ を受ける.
        \begin{align}
            \bF = q(\bE + \bv\times\bB)
        \end{align}
    \end{myshadebox}

\section{ローレンツ力と観測者}
        このローレンツ力は,不思議な力である.というのも,この力は
        観測者と電荷との相対的な速度 $\bv$ によっているからである.
        同じ電荷を観測していても,観測者によって電荷との相対速度
        が異なるのであれば,ローレンツ力の向きや大きさは,観測者ごとに
        異なったものとなる.なんとも不思議なことであるが,
        相対性理論を受け入れれば,何の不思議なことではなくなる.
        しかし,ここでは電磁気学を学ぶことが目標であるので,
        とりあえず,この不思議さは棚上げにしておき,話を
        先に進めることにしよう.


    \begin{memo}{電荷自身から発する磁束密度}
        電荷が磁束密度中を運動するときには,それにより磁束密度
        に関するローレンツ力を受け,速度の方向が変化してしまう.
        しかし,あとで述べるとおり,アンペールの法則によれば,
        磁束密度の発生源は電流である
            \footnote{
                アンペールの法則(※1)は
                次式で表現される.
                \begin{align*}
                    \drot\bB =  \mu_{0}\left(
                                    \bi + \frac{\rd \bE}{\rd t}
                                \right)
                \end{align*}
                言葉で表現すれば,大雑把に,
                \begin{align*}
                    \mbox{回転する磁束密度} = \mbox{電流} + \mbox{時間変動する電場}
                \end{align*}
                という感じになろう.詳しいことは,後に考える.

                (※1)正確には「アンペール$=$マクスウェルの法則」というべきだ.
            }.

        電荷が速度をもっていれば,
        それは電流と同一視できることになり,つまり,その物体自
        身がその周囲に磁束密度を生じさせていることになる.

        たしかに,はじめに考えたように,速度をもった電荷は,自身が
        発している磁束密度とは発生源が異なる磁束密度の影響をうけて,
        その方向が変化する.しかし一方で,電荷から発している磁束密度
        は,その周囲の磁束密度に影響を与えていることも事実である.

        そうであるとき,その影響はどの程度なのだろうか.それは,
        磁束密度に関するローレンツ力が $q\bv\times\bB$ と表される
        ことから,電荷の電気量 $q$ と速度 $\bv$ の大きさに依存している
        ことは明らかである.

        つまり,電荷を除いた状態での純粋な空間の磁束密度を $\bB_{\mathrm{pure}}$ と
        し,電荷から生じる磁束密度を $\bB_{\mathrm{q}}$ としたならば,電荷が
        存在する場合の正味の空間の磁束密度 $\bB$ は
            \begin{align*}
                \bB = \bB_{\mathrm{pure}} + \bB_{\mathrm{q}} \\
            \end{align*}
        である.だから,電荷が受ける磁束密度に対するローレンツ力は,
            \begin{align*}
                \bF &= q \bv \times \bB \\
                    &= q \bv \times (\bB_{\mathrm{pure}} + \bB_{\mathrm{q}}) \\
                    &= q \bv \times  \bB_{\mathrm{pure}} + q \bv \times \bB_{\mathrm{q}} \\
                \therefore\quad
                \bF &= q \bv \times  \bB_{\mathrm{pure}} + q \bv \times \bB_{\mathrm{q}}
            \end{align*}
        となり,$q \bv \times \bB_{\mathrm{q}}$ という分だけ,周囲の磁束密度を
        変化させ,それが自分の受ける磁束密度に対するローレンツ力に跳ね返ってくる.

        ここで気になるのは,$\bB_{\mathrm{q}}$ である.これは運動している電荷から
        発している磁束密度である.
        大きさはどのくらいで,どの方向に磁束密度は生じているのだろうか.
        これはビオ$=$サバールの法則から,位置 $\br$ に発生させる磁束密度 $\bB(\br)$ は
            \begin{align*}
                \bB_{\mathrm{q}}(\br) =
                \frac{\mu_{0}}{4\pi}q\bv \times
                \frac{\br-\br'}{|\br-\br'|^{3}}.
            \end{align*}
       と計算される.ここに,$\br'$ は磁束密度を観測する固定点である.
       とすれば,電荷が受ける磁束密度に対するローレンツ力は次のようになる.
            \begin{align*}
                \bF &= q \bv \times \bB_{\mathrm{pure}} + q \bv \times
                \left(
                    \frac{\mu_{0}}{4\pi}q\bv \times \frac{\br-\br'}{|\br-\br'|^{3}}
                \right) \\
                &= q \bv \times \bB_{\mathrm{pure}} + \frac{\mu_{0}}{4\pi}
                \left( q^{2} \bv \times \bv \times \frac{\br-\br'}{|\br-\br'|^{3}}
                \right).
            \end{align*}
        結果が見えてきた.ここで,ベクトル解析の公式,任意のベクトル $\bX$ に対して,
        $\bX \times \bX = \bzero$ を思い起こせば,
            \begin{align*}
                \bF &= q \bv \times \bB_{\mathrm{pure}} + \frac{\mu_{0}}{4\pi}
                \left(
                    q^{2} \bzero \times \frac{\br-\br'}{|\br-\br'|^{3}}.
                \right)\\
                &= q \bv \times \bB_{\mathrm{pure}} + \bzero \\
                \therefore\quad
                \bF &= q \bv \times \bB_{\mathrm{pure}}.
            \end{align*}
        この式の意味するところは,明白である.
        運動する電荷から発している磁束密度より受けるローレンツ力は,
        電荷自身に対して影響を及ぼさない,ということである.
        考えて見れば,簡単にイメージができることである.速度をもつ
        電荷がその周囲につくる磁束密度 $\bB_{\mathrm{q}}$ は,その速度に対して垂直な方向に
        生じる.なので,$\bB_{\mathrm{q}}$ は運動する電荷に対して,
        なんのエネルギーも与えないのだ
            \footnote{
                仕事の定義 $W = \bF \cdot \br$ を思い起こそう.物体に仕事を
                すると,それはエネルギーとして蓄えられるのだが,垂直成分は
                それに寄与しない.なぜなら,
                    \begin{equation*}
                        \bF \cdot \br = |\bF||\br| \cos \theta =  |\bF||\br| \cos(\pi/2) = 0.
                    \end{equation*}
            }.
    \end{memo}
