%===================================================================================================
%  Chapter : コンピュータの物理学的基礎
%  説明    : コンピュータの動作を,電磁気学に立ち返って説明する
%===================================================================================================

%======================================================================
%  Section
%======================================================================
    \section{考え方と方針}
    ここでは,大規模な電子回路である,コンピュータについて考える.
    一般論として,電子計算機を学ぶのも1つの方法であるが,
    このノートでは現実性を重視し,実際のコンピュータが電子回路によって,
    構成可能かどうかを検証することとしたい
        \footnote{
            「コンピュータが電子回路によって構成買うであることを“検証”したい」
            と書いたが,現実にコンピュータが存在しているので,すでに検証済みである
            と考えられるだろう.その通りである.しかし,今の私にはコンピュータが
            本当に電子回路の知識で構成可能かどうかは理解していない.わかっているのは,
            コンピュータが存在している事実であって,コンピュータが電子回路であるという
            ことはわかっているわけではない.なので,この章で
            コンピュータの動作原理を理解し,コンピュータが電子回路で構成可能である
            ということを確認したいと思う.

            しかし,コンピュータの動作原理を説明する教科書には,
            物理学的な基礎から書かれているものがない
            (見当たらない).しかし,確かに,コンピュータは現実に存在し
            ていて,\textbf{コンピュータは物理法則に則って動作している}はずである.そこで,
            コンピュータの動きを,物理法則から理解できるということを文書にまとめ,
            知識を整理することで理解を深めていきたいと思う.
         }.
    以降の議論はかなり荒削り
        \footnote{
            壊滅的に等しい.
        }
    だが,コンピュータがどのように物理法則に則って計算を行っているか,
    というイメージをもつのには十分であると思う.

    あくまでも,\textbf{目的は,コンピュータが物理法則に従って計算を行なっている
    ということを“実感すること”}である.

    \begin{memo}{注意}
        説明のために,簡易的なコンピュータを(頭の中で)つくることになるが,
        あくまでも私が考えた構成のもので,現実的なものではない.それは,上記の
        目的
            \footnote{
                目的:「コンピュータがどのように物理法則に則って計算を行っているか」
                というイメージをもつこと.
            }
        のためにつくるものであって,実際のコンピュータがそうなっているの
        ではない.「ああ,たしかに,こうやって構成すれば,コンピュータが
        作れるんだな」という感覚を抱けるようになりたいのだ.
    \end{memo}


%======================================================================
%  Section
%======================================================================
    \section{電磁気学の復習}
    \begin{mycomment}
        まずは,基礎中の基礎の確認から始めよう.
    \end{mycomment}
    %==================================================================
    % SubSection
    %==================================================================
    \subsection{電子の存在}
    \textbf{電子} がないと,電磁気現象は生じない.もともと,電磁気現象の
    根源として,\textbf{電荷} の存在が仮定(要請)されていて,これをもとに
    電磁気学が構成されている.電磁気学が成立した後,存在を仮定していた
    電荷というものが,電子という形で実在することが,実験により確かめられた.
    しかし,電子のもつ電気量は,電磁気学で定める正電気量の反対の符号を
    もつことも確かめられた.そうは言っても,電子が電磁気現象の根源
    であることは,今では,絶対に否定できない事実である.

    コンピュータも電子回路のひとつである以上,電子の存在の上に成り立って
    いるものである.コンピュータ内部の回路を移動する電子こそが,計算の動力
    源となるだ.

    %==================================================================
    % SubSection
    %==================================================================
    \subsection{「ホール(hole)」という考え方}
    \textbf{ホール} という語彙は,正確には,
    \textbf{エレクトロン$\cdot$ホール(Electron hole)} と
    いう.また,日本語に訳して,\textbf{正孔} とも言われる
        \footnote{
            正孔:「せいこう」と読む.
        }.
    このノートでは,「ホール」という言い方を採用する.次に,肝心のホールとは
    何かを,説明しよう.

    ホールとは,簡単に言えば,電子の抜け殻とでも表現できよう.
    元々電子が存在すべき場所なのだが,実際には電子が存在しない部分を,
    ホールと呼んでいる.
       \begin{figure}[hbt]
           \begin{tabular}{cc}
               \begin{minipage}{0.5\hsize}
                   \begin{center}
                       \includegraphicsdouble{Comp_ElecHole_EleMov.pdf}

                       (A)
                   \end{center}
               \end{minipage}
               \begin{minipage}{0.5\hsize}
                   \begin{center}
                       \includegraphicsdouble{Comp_ElecHole_EleMov01.pdf}

                       (B)
                   \end{center}
               \end{minipage}
           \end{tabular}
       \end{figure}
       \begin{figure}[hbt]
           \begin{tabular}{cc}
               \begin{minipage}{0.5\hsize}
                   \begin{center}
                       \includegraphicsdouble{Comp_ElecHole_EleMov02.pdf}

                       (C)
                   \end{center}
               \end{minipage}
               \begin{minipage}{0.5\hsize}
                   \begin{center}
                       \includegraphicsdouble{Comp_ElecHole_EleMov03.pdf}

                        (D)
                   \end{center}
               \end{minipage}
           \end{tabular}
           \caption{ホールの動き(実際は電子の動き)}
       \end{figure}


    なんで,こんなこと(ホール)を考える必要があるのか.
    結論から言えば,実はホールという概念は絶対に必要というわけではない.
    しかし,ホールという考え方を用いることにより,物理的イメージが
    捉えやすくなり,物理現象の説明も短くなり簡潔にまとめることができるのだ.
    だったら,これを採用しない手はない.

    \begin{memo}{炭酸の泡 --- ホールに似た現象として ---}
        似たような考え方の例として,炭酸飲料の気泡がよく挙げられる.
        この気泡が,ホールに対応するのである.たんさんの入っていない
        飲み物には,当然,泡は生じない
            \footnote{
                もちろん,振ったり,ストローか何かで内部に気体(空気)を
                入れれば泡ができるが,ここでは理想的(液体に対してなんの操作もしていない)
                な状態を想定している.
            }.
        しかし,炭酸飲料の場合には,内部から泡が生じる.これは,圧力により
        液体に溶けていた二酸化炭素などの気体が,圧力の低下に伴って,外部に
        出ようとして泡として現れるものである.
        このとき,本来ならば液体が存在する場所なのだが,実際には泡(気体)が
        眼に見えてくる.そして,この泡は液体中を,上に向かって移動する.いや,
        この言い方は,正確ではない.なぜなら,こう表現してしまうと,気体が
        重力に逆らっているように,捉えられてしまうからである.では,本当の
        ところは,どう説明すればよいのだろうか.答えは,視点を変えることである.
        泡を見るのではなく,液体を見るのである.つまり,液体が気体の下に潜り込む
        ように落ちるのである.

        現象の流れは次のように説明できる.最初の段階では,炭酸飲料には圧力
        がかけられていて,気体は内部に溶けたままである
            \footnote{
                炭酸飲料が入ったペットボトルが未開封の状態に相当する.
            }.
        そして,炭酸飲料にかける圧力を低くする
            \footnote{
                ペットボトルのフタを開ける.
            }.
        すると,液体の内部から気体が生じてくる
            \footnote{
                圧力によって,液体中に抑えつけられるように閉じ込められていた気体が,
                圧力を緩めたことによって,泡となって液体中から飛び出してくる.
            }.
        一度液体中から気体が生じれば,液体は気体より重いので,液体は気体の下の方へ
        入り込もうとする
            \footnote{
                泡に視点を合わせれば,泡は上へ移動していくように見える.
            }.
        この結果として,泡が上に移動しているように見えるのだ.屁理屈を言っている
        ような気がするが
            \footnote{
                日常会話でこんなことを説明したら,屁理以外の何ものでも
                ないだろう.
            },
        正確さを求める場合には,高説明するしかない(と思う).
    \end{memo}

    %==================================================================
    % SubSection
    %==================================================================
    \subsection{電荷保存則}
        %==============================================================
        % SubsubSection
        %==============================================================
        \subsubsection{電場に対するガウスの法則}
            \begin{align}
                \ddiv \bE = \frac{1}{\varepsilon_{0}}\rho.
            \end{align}

        %==============================================================
        % SubsubSection
        %==============================================================
        \subsubsection{アンペール$=$マクスウェルの法則}
            \begin{align}
                \drot \bB = \mu_{0}\bi + \varepsilon_{0}\mu_{0}\frac{\rd \bE}{\rd t}.
            \end{align}

        %==============================================================
        % SubsubSection
        %==============================================================
        \subsubsection{電荷保存の法則}

    %==================================================================
    % SubSection
    %==================================================================
    \subsection{電位(電圧)の定義}

%======================================================================
%  Section
%======================================================================
    \section{トランジスタの動作概要}
    %==================================================================
    % SubSection
    %==================================================================
    \subsection{半導体}
        %==============================================================
        % SubsubSection
        %==============================================================
        \subsubsection{半導体の分類}
        \begin{figure}[hbt]
            \begin{center}
                \includegraphicslarge{handoutai_bunrui.pdf}
                \caption{半導体の分類}
                \label{fig:handoutai_bunrui2}
            \end{center}
        \end{figure}

        %==============================================================
        % SubsubSection
        %==============================================================
        \subsubsection{真性半導体}

        %==============================================================
        % SubsubSection
        %==============================================================
        \subsubsection{n型半導体,ドナー}

        %==============================================================
        % SubsubSection
        %==============================================================
        \subsubsection{p型半導体,アクセプタ}

    %==================================================================
    % SubSection
    %==================================================================
    \subsection{ダイオード}

    %==================================================================
    % SubSection
    %==================================================================
    \subsection{ダイオードの電流電圧特性}

    %==================================================================
    % SubSection
    %==================================================================
    \subsection{トランジスタの物理構成}
        %==============================================================
        % SubsubSection
        %==============================================================
        \subsubsection{バイポーラトランジスタの物理構成}

        %==============================================================
        % SubsubSection
        %==============================================================
        \subsubsection{電界効果型トランジスタ(FET)の物理構成}

        %==============================================================
        % SubsubSection
        %==============================================================
        \subsubsection{MOSFETトランジスタ(MIS構造)の物理構成}

        %==============================================================
        % SubsubSection
        %==============================================================
        \subsubsection{CMOSの物理構成}

    %==================================================================
    % SubSection
    %==================================================================
    \subsection{トランジスタの動作}
        %==============================================================
        % SubsubSection
        %==============================================================
        \subsubsection{バイポーラトランジスタの動作原理}
        電流により,電流の増幅率を調整する.

        %==============================================================
        % SubsubSection
        %==============================================================
        \subsubsection{MOSFETトランジスタ(MIS構造)の物理構成}
        電圧により,電流の増幅率を調整する.

        %==============================================================
        % SubsubSection
        %==============================================================
        \subsubsection{トランジスタの動作(これだけを覚えていれば十分)}


%======================================================================
%  Section
%======================================================================
    \section{コンピュータの構成概要}
    \begin{mycomment}
        「コンピュータの構成」は現実のところ,様々である.
        それは開発される回路の数だけ存在するからである.
        同じ回路を開発するのは,骨折り損だ.しかし,
        コンピュータの構成の思想はどれも同じである.
        それは,「命令」による「データ加工」である.
    \end{mycomment}

    %==================================================================
    % SubSection
    %==================================================================
    \subsection{コンピュータの定義}
        %==============================================================
        % SubsubSection
        %==============================================================
        \subsubsection{コンピュータとは何か}
            まず始めに,明確にして置かなければならないことがある.それは,
                \begin{center}
                    「コンピュータとは何か.」
                \end{center}
            ということだ.何を今更,と思うかもしれないが,
            いざこの質問に答えようとすると,曖昧になってしまうのではなかろうか.
            この質問の答えとして,例えば,電気関係にうとい一般の方だとしたら,
            「何かものすごい計算を行うもの」だとかと答えそうである.
            また,普段仕事でパソコンを使っている人には,
            「MicrosoftのWordやExcelなど
            で文書を作成したり,表計算をするをするものである」という
            答が返ってきそうでもある.これらの答えは,大きな枠組みの範囲では
            正しい.しかし,これから議論していくには,コンピュータの定義
            としては,不十分である.では,どのように定義したらよいのだろうか.

            一般的に,コンピュータを定義することは難しい.これは
            「世界初のコンピュータは誰が開発したか」という質問に対して,
            パスカル

            だとか,
            バベッジ

            だとかと言われるように,複数の解答が寄せられることにも,
            定義の難しさ,あるいは認識の曖昧さが現れている.
            この違いは,コンピュータをどう定義の違いの現れなのだ.
            答えとして,誰を上げても,間違いはない
                \footnote{
                    もちろん,コンピュータを自分なりに定義して,
                    それを最初に創り上げた人で無いといけない.
                }.
            パスカルは四則演算を機械的に行える装置を作ったし,
            ライプニッツはそれを改良して桁上がりにも対応させた.
            さらに,バベッジはプログラムと言う概念をその計算機に折込み,
            自動で計算を行えるような装置を考案した
                \footnote{
                    実際に装置が作られたのは,彼の死後である.
                    ただ,文書として理論が残されており,これを基に
                    後の人々がその理論の正しさを実証している.
                }.

            このように,コンピュータを単なる四則演算器として捉えるならば,
            パスカルがその創始者となる.しかし,私の知る限りでは,
            コンピュータの定義として,「計算を自動的に行うもの」と
            されることが多く,その意味では,計算をプログラム制御によって
            自動化することに成功した,バベッジがその答えとされることが
            多いと感じている.

            このノートでは,コンピュータを次のように定義し,以降では,この定義を基にして話を進めていく.
                \begin{myshadebox}{(このノートの)コンピュータの定義}
                    \textbf{コンピュータ} とは,\textbf{命令} による \textbf{データ} の \textbf{加工} を
                    自動で行う装置である.
                \end{myshadebox}

            「命令」,「データ」,「加工」の定義については,後ほど詳しく
            記述することとしたいが,さしあたっては,次のように捉えておいて欲しい.
            命令とは,「四則演算を行え」ということであり,データはその四則演算の
            対象となる数であり,加工とは四則演算を実際に行うことである.

        %==============================================================
        % SubsubSection
        %==============================================================
        \subsubsection{コンピュータの構成方法}

        %==============================================================
        % SubsubSection
        %==============================================================
        \subsubsection{ノイマン型コンピュータ}

        %==============================================================
        % SubsubSection
        %==============================================================
        \subsubsection{ハーバードアーキテクチャ}

    %==================================================================
    % SubSection
    %==================================================================
    \subsection{CPU(中央処理装置)}
        %==============================================================
        % SubsubSection
        %==============================================================
        \subsubsection{制御装置}

        %==============================================================
        % SubsubSection
        %==============================================================
        \subsubsection{主記憶装置(メモリ)}

        %==============================================================
        % SubsubSection
        %==============================================================
        \subsubsection{ALU(算術論理演算装置)}

    %==================================================================
    % SubSection
    %==================================================================
    \subsection{入出力装置}
        %==============================================================
        % SubsubSection
        %==============================================================
        \subsubsection{入力装置}
        キーボード,マウス,タブレットなど.

        %==============================================================
        % SubsubSection
        %==============================================================
        \subsubsection{出力装置}
        パソコンの画面(ディスプレイ)やプリンタ,プロジェクタなど.

%======================================================================
%  Section
%======================================================================
    \section{コンピュータの基本構成回路}
    %==================================================================
    % SubSection
    %==================================================================
    \subsection{NOT回路}

    %==================================================================
    % SubSection
    %==================================================================
    \subsection{AND回路}

    %==================================================================
    % SubSection
    %==================================================================
    \subsection{OR回路}

    %==================================================================
    % SubSection
    %==================================================================
    \subsection{NOR回路}

    %==================================================================
    % SubSection
    %==================================================================
    \subsection{NAND回路}


%======================================================================
%  Section
%======================================================================
    \section{制御回路}
    %==================================================================
    % SubSection
    %==================================================================
    \subsection{クロック}

    %==================================================================
    % SubSection
    %==================================================================
    \subsection{リセット}

    %==================================================================
    % SubSection
    %==================================================================
    \subsection{制御用選択回路(マルチプレクサ)}

%======================================================================
%  Section
%======================================================================
    \section{記憶回路(簡易的なメモリの作成)}
    %==================================================================
    % SubSection
    %==================================================================
    \subsection{コンピュータにおける「記憶」の意味}

    %==================================================================
    % SubSection
    %==================================================================
    \subsection{メモリの構成}
        %==============================================================
        % SubsubSection
        %==============================================================
        \subsubsection{メモリとは}

        %==============================================================
        % SubsubSection
        %==============================================================
        \subsubsection{アドレスデコーダ}

        %==============================================================
        % SubsubSection
        %==============================================================
        \subsubsection{メモリセル}

    %==================================================================
    % SubSection
    %==================================================================
    \subsection{メモリセルの基本素子}
        %==============================================================
        % SubsubSection
        %==============================================================
        \subsubsection{ラッチ(Latch)}
        AND回路を例に,ラッチの動作を説明する.

        %==============================================================
        % SubsubSection
        %==============================================================
        \subsubsection{記憶回路の要素}
        インバータ2つのラッチを説明

        %==============================================================
        % SubsubSection
        %==============================================================
        \subsubsection{D-ラッチ}

        %==============================================================
        % SubsubSection
        %==============================================================
        \subsubsection{D-フリップフロップ}

    %==================================================================
    % SubSection
    %==================================================================
    \subsection{メモリの動作}

%======================================================================
%  Section
%======================================================================
    \section{計算回路(簡易的なALUの作成)}
    %==================================================================
    % SubSection
    %==================================================================
    \subsection{コンピュータが行う計算}
        %==============================================================
        % SubsubSection
        %==============================================================
        \subsubsection{10進法}

        %==============================================================
        % SubsubSection
        %==============================================================
        \subsubsection{2進法}

        %==============================================================
        % SubsubSection
        %==============================================================
        \subsubsection{16進法}

        %==============================================================
        % SubsubSection
        %==============================================================
        \subsubsection{2の補数(負の数の表現)}

        %==============================================================
        % SubsubSection
        %==============================================================
        \subsubsection{2進法による足し算}

    %==================================================================
    % SubSection
    %==================================================================
    \subsection{簡易ALUの構成}
        %==============================================================
        % SubsubSection
        %==============================================================
        \subsubsection{加算器}

        %==============================================================
        % SubsubSection
        %==============================================================
        \subsubsection{乗算器}

        %==============================================================
        % SubsubSection
        %==============================================================
        \subsubsection{アキュムレータ}

    %==================================================================
    % SubSection
    %==================================================================
    \subsection{簡易ALUの動作}
