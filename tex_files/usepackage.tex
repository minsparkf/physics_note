%===============================================================================
% パッケージの読み込み
%===============================================================================
%% ------------------------------------------
%% * Now Setting
%% ------------------------------------------
%\usepackage[setpagesize=false,dvipdfmx,%
%bookmarks=true,bookmarksnumbered=true,%
%bookmarksopen=false,%
%colorlinks=false,%
%pdftitle={タイトル},%
%pdfauthor={作成者},%
%pdfsubject={サブタイトル},%
%pdfkeywords={キーワード},%
%pdfborder={0 0 0},%
%linkcolor=blue,anchorcolor=blue,urlcolor=blue,
%]{hyperref}
%\RequirePackage[12tabu, orthodox]{nag}
%\usepackage{atbegshi}
%\AtBeginShipoutFirst{\special{pdf:tounicode 90ms-RKSJ-UCS2}}
%\usepackage{refcheck}              % ref auto check
\usepackage[dvipdfmx]{graphicx}
%\usepackage{mediabb}               %  pdfの図を挿入時の,大きさの設定を自動化する
%\usepackage{gradientframe}          %  図に影をつける
\usepackage[dvips]{color}           %  dvipsにて,色付文字を有効化する

\usepackage{bm}                     %  数式太字

%\usepackage{lmodern}
\usepackage{fancyhdr}

\usepackage{bbding}
\usepackage{textcomp}
\usepackage{okumacro}
\usepackage{fancybox}               %  囲みの種類の追加
%\usepackage{calrsfs}               %  文字の種類の追加 1
%\usepackage{rsfso}                 %  文字の種類の追加 2
\usepackage{ascmac}                 %  数式環境の拡張
\usepackage{amsfonts}               %  数式の拡張1
\usepackage{amsthm}                 %  数式の拡張2
\usepackage{amsmath,amssymb}        %  数式の拡張3
\usepackage{mathtools}                                %  数式の拡張4
\usepackage[sans]{dsfont}
\usepackage[T1]{fontenc}
\usepackage{pifont}
%  \mathtoolsset{showonlyrefs=true}  %    参照していない式の番号は付加しない
\usepackage{enumitem}
\usepackage[all, warning]{onlyamsmath}
%\usepackage{esvect}                %  その他のベクトル表現
\usepackage{url}                    %  \url のために必要
\usepackage{makeidx}                %  索引のために
%\usepackage{mediabb}               %  PDFファイル読み込みのために
%\usepackage[dvipdfmx]{hyperref}    %  文字化け
%\usepackage[hypertex]{hyperref}    %  動作がおかしい
\usepackage[cc]{titlepic}           %  表紙に図を挿入可能にする
%\usepackage{harmony}               %  音符が表示できるはずだが,フォントが見つからない
%\usepackage{hangcaption}           %  長いキャプションの折り返し
%\usepackage[times]{quotchap}        %  章のみだし体裁の変更
%\usepackage{pdfdraftcopy}          %  透かしを入れる
%  \draftstring{作成中}             %    -> 透かしの文字列 を設定
%\usepackage[brackets]{fnindent}    %  脚注の体裁を変更
\usepackage[version=4]{mhchem}
\usepackage{setspace}               % 行間調整してみる(\setstretch{1.5}, \begin{spacing}{0.8}-\end{spacing})
%\usepackage{breqn}                  % \begin{dmath}
\usepackage{wasysym}
\usepackage{layout}
\usepackage{braket}                 %  ディラックのブラケット
\usepackage{physics}                % 物理でよく使う数式ショートカット

\usepackage{pxrubrica}              % ルビ
\usepackage{plext}
\usepackage[T1]{fontenc}
\usepackage[utf8]{inputenc}
\usepackage{siunitx}
%\usepackage{kpfonts}
%\usepackage{txfonts}               %  Times フォントの使用
%\usepackage{newtxtext,newtxmath}   %  NewTimes フォントの使用
%\usepackage{fourier}               %  アルファベットフォント変更 1
%\usepackage{fouriernc}             %  アルファベットフォント変更 2
%\usepackage{kurier}                %  アルファベットフォント変更 3
%\usepackage{lmodern}               % Latine Modern
\usepackage{pxfonts}                %
% --- End Of Setting ---


%%%%%%%%%%%%%%%%%%%%%%%%%%%%%%%%%%%%%%%%%%%%%%%%%%%%%%%%%%%%%%%%%%%%%%%%%%%%%%%%
%%%%%%%%% Back Up Code %%%%%%%%%%%%%%%%%%%%%%%%%%%%%%%%%%%%%%%%%%%%%%%%%%%%%%%%%
%%%%%%%%%%%%%%%%%%%%%%%%%%%%%%%%%%%%%%%%%%%%%%%%%%%%%%%%%%%%%%%%%%%%%%%%%%%%%%%%
%% ------------------------------------------
%% * Old Setting (hyperref)
%% ------------------------------------------
%\usepackage[dvipdfmx,%
%colorlinks=true,%
%urlcolor=black,%
%linkcolor=black,%
%citecolor=black,%
%linktocpage=true,%
%bookmarkstype=toc,%
%bookmarksdepth=4,%
%bookmarksnumbered=true,%
%bookmarksopen=true,%
%bookmarks=true]{hyperref}
%%\AtBeginDvi{\special{pdf:tounicode EUC-UCS2}}%EUC
%\AtBeginDvi{\special{pdf:tounicode 90ms-RKSJ-UCS2}}%SJIS
