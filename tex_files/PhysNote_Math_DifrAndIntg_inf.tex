%   %==========================================================================
%   %  Section : 極限
%   %==========================================================================
        %==================================================================
        %  SubSection
        %==================================================================
            \subsection{数列とは}

        %==================================================================
        %  SubSection
        %==================================================================
            \subsection{等差数列}

        %==================================================================
        %  SubSection
        %==================================================================
            \subsection{等比数列}

        %==================================================================
        %  SubSection
        %==================================================================
            \subsection{等比級数}

        %==================================================================
        %  SubSection
        %==================================================================
            \subsection{無限等比級数}

        %==================================================================
        %  SubSection
        %==================================================================
            \subsection{数列の極限\;\;--「限りなく0に近づける」とは --}
                微分の定義で,$\Delta x$ を「0に近づける」という表現を用いる.
                しかし,この“近づける”という表現は,どうにも,曖昧である.“
                近づける”とはいうものの,その近づける方法がよく分からないし,
                また,どの程度まで近づけられるかもはっきりとしない.そもそも,
                「近づけられるのか」という不安もある.そこで,ここでは“近づけ
                る”という作業とはどのようにすべきかを考える.まず,具体例で考
                えよう.例として,0の“次に”大きい数を考えてみよう.
                自然数で考えるならば,これは1である.しかし,
                今考えているのは,実数の範囲だから,1ではありえない.それでは
                ,0.1か.いや,もっと小さい数の0.01がある.いやいや,0.001,
                0.0001,といくらでも考えられる.この問いには,具体的に数を提示
                して,解答することはできない.それでは,「0の次に大きい数」は
                存在しないのか.0よりも少しでも大きい数は存在するが,“0の次”
                となると,答えられなくなってしまう.この問い自体が,無意味な問
                いだからである.では,なぜこのようなことを考えたかといえば,0
                の次に大きい数を考えるときに,その答えとなる数として,1,0.1,
                0.01,$\cdots$ と0にだんだんと“近づいて”いったからである.「
                近づける」という作業をしたのだ.これを,文字を用いて表現できれ
                ば,「近づける」ということを,もっと納得の行く形で受け入れるこ
                とができよう.今の手順を,もう一度詳しく,考え直そう.最初に,
                0に近い数として,1を考えた.今回は,1を考えたが,実際は0.5でも
                ,0.3でも,その他の数でも,よかった.とりあえず,考察に先立っ
                て,基準となる数を挙げただけである.そして次に最初に挙げた数よ
                りも,より小さい数を考えた.今回は1に対して0.1を挙げたが,最初
                に挙げた数の0よりも近い値であれば,どのような数でもよい.そして
                ,さらに0に近い値,もっと0に近い値へと徐々に0に近づけていった.
                この作業は無制限に続き,終わりがないが,これを自動的に行わせる
                ように,文字で表現できれば,目標を達成できる.今の例とその解答
                手順を,もう少し一般的に扱うために,文字で表現してみよう.0に近
                い数として最初に挙げる数を $a_{0}$ としよう.そして,0と $a_{0}$ の
                値との差 $\mid a_{0}-0 \mid$ を考え,これよりもさらに小さい数が
                存在するとき,そのような数を $a_{1}(<\mid a_{0}-0 \mid)$ とおこ
                う.これを繰り返すと,$a_{0}$,$a_{1}$,$a_{2}$,$a_{3}$,のよう
                な数列が得られる.この数列を $\{ a_{n} \}$ と表現すれば,$n$ が
                大きくなるに伴い,数列 $\{ a_{n} \}$ は0に近づいていくと言える.
                あとは,この作業を式で表現することを考えるのみだ.自然数 $n$ が
                大きくなるということは,「ある任意の自然数 $n_{0}$ に対して,$n>n_{0}$ と
                なるような自然数 $n$ が存在する」
                ということである.例えば,任意の自然数を,$n_{0}=100$ としてみよ
                う.この100に対して,大きい自然数は例えば200がある.そうなれば,
                $n_{0}=200$ と書き換えて,これより大きい数500を考えられる
                .さらに $n_{0}=500$ と書き換えて,$\cdots$ のように考えていけば,
                いくらでも大きな自然数を得ることができる.そして,$n$ の増加に伴
                い,$a_{n}$ と0との差が小さくなるので,その差を $\varepsilon (>0)$ と
                表現すれば,目標とする式表現として,次のように書ける.
                \\
                    \begin{itembox}[l]{限りなく0に近づける}
                        任意の正の数 $\varepsilon$ に対して,自然数 $n_{0}$ を決めることができ,
                            \begin{align}
                                n>n_{0} \quad\Rightarrow\quad \mid a_{n}-0 \mid <\varepsilon
                            \end{align}
                        が成立するならば,数列 $a_{n}$ は0に限りなく近づくと言える.
                    \end{itembox}
                    \\

                このとき,これをもっと読みやすく簡略化した式として,
                    \begin{align}
                        \lim_{n\rightarrow \infty } \{ a_{n} \} = 0
                    \end{align}
                と表現する.

                今回挙げた例の,この $\varepsilon$ に対応するのが,1である.
                この1対応して $n_{0}=1$ と決まる.そして,$n_{0}=1$ に対して,
                これより大きい自然数 $n=2$ を与えることができる.$n=2$ とした
                ときに,$\mid a_{2}-0 \mid < 1$ となるような数 $a_{2}$ が存在
                すれば,$\varepsilon$ として[1より小さい値]が存在するとして,
                $\varepsilon$ がその値で書き換えられる.今回の例では0.1である.
                また,$n_{0}=2$ とも書き換えられる.これで一巡したが,同様に,
                $n_{0}=2$ のとき,これよりも大きい自然数 $n=3$ を与えることが
                でき,$\mid a_{2}-0 \mid < 0.1$ となるような数が存在すれば,
                $\varepsilon$ をその数に書き換える.今回の例では0.01である.
                そして,$n_{0}=3$ とも書き換える.$n_{0}=3$ に対して $n=4$ が
                存在し,$\cdots$ 以下同様.

                このような作業を無制限に続けることができるとき,
                0に近づけることができるということであり,
                また,「限りなく0に近づける」という行為でもある.
                そして,この式によって,「限りなく0に近づける」
                という行為が,いわば“自動的”に行
                .「近づける」という行動を起こさなくても,極限を
                定義することができた.
                    \begin{figure}[hbt]
                        \begin{center}
                            \includegraphicsdefault{suretu_kyokugen1.pdf}
                            \caption{数列の極限}
                            \label{fig:suretu_kyokugen1}
                        \end{center}
                    \end{figure}

                また,上の例では0に近づけるということを考えたが,
                任意の実数に近づけるという作業も
                同じように考えられる.その場合,近づけたい数を $\alpha$ とするならば,
                次のように表現を拡張できる.
                    \\
                    \begin{itembox}[l]{限りなく実数 $\alpha $ に近づける}
                        任意の正の数 $\varepsilon$ に対して,自然数 $n_{0}$ を決めることができ,
                            \begin{align}
                                n>n_{0} \quad \Rightarrow \quad
                                \mid a_{n}-\alpha  \mid < \varepsilon
                            \end{align}
                        が成立するならば,数列 $a_{n}$ は $\alpha$ に限りなく近づくと言える.
                    \end{itembox}
                    \\

                このとき,これをもっと読みやすく簡略化した式として,
                    \begin{align}
                        \lim_{n\rightarrow \infty } \{ a_{n} \} = \alpha
                    \end{align}
                と表現する.


                見ての通り,先ほどの式の0を $\alpha$ で置き換えただけである.
                ちなみに,このような $\alpha $ が存在するとき,
                数列 $\{a_{n}\}$ は $\alpha$ に \textbf{収束} するという.
                また,$\alpha$ のことを数列 $\{ a_{n}\}$ の \textbf{極限} という.
                感覚的にいってしまえば,数列の項の番号 $n$ が大きくなるともなっ
                て,項の値が極限値に近づくということである.

                    \begin{figure}[hbt]
                        \begin{center}
                            \includegraphicsdefault{suretu_kyokugen2.pdf}
                           \caption{数列の極限2}
                           \label{fig:suretu_kyokugen2}
                        \end{center}
                    \end{figure}

                最初に $\varepsilon$ を設定するのは,自然数 $n_{0}$ を必ず決定できる
                ようにするためである.自然数 $n_{0}$ を決定したとしても,$\varepsilon$ は
                全く定まらない.先に $\varepsilon$ を決定しておけば,その範囲に含まれる
                自然数があるはずであり,この内のどれかが $n_{0}$ であるとして,極限の定
                義ができる.

                    \begin{figure}[hbt]
                        \begin{center}
                            \includegraphicsdefault{suretu_kyokugen3.pdf}
                            \caption{数列の極限3}
                            \label{fig:suretu_kyokugen3}
                        \end{center}
                    \end{figure}


                この定義の最大の利点は,複数の数列の極限値に関する,
                加減乗除の定理を証明できる点にある.
                例えば,2つの数列 $\{ a_{n} \}$ と $\{ b_{n} \}$ が
                あるとき,次のような数列を作ってみよう.
                    \begin{align*}
                        \{ a_{n}+b_{n} \} =
                            a_{1}+b_{1}\,,\,\,a_{2}+b_{2}\,,
                            \,\,a_{3}+b_{3}\,,
                            \,\,\cdots\,,
                            \,\,a_{n}+b_{n}\,,\,\,\cdots
                    \end{align*}
                この数列の極限はどうなるだろうか.
                これは簡単で,2つの数列 $\{ a_{n} \}$ と $\{ b_{n} \}$ の極限がそれぞれ,
                    \begin{align*}
                        \lim_{n\rightarrow \infty } \{ a_{n} \}
                            &= \alpha\, \\
                        \lim_{n \rightarrow \infty } \{ b_{n} \}
                            &= \beta
                    \end{align*}
                であるとき,数列 $\{ a_{n}+b_{n} \}$ の極限は,
                    \begin{align*}
                        \lim_{n\rightarrow \infty } \{ a_{n}+b_{n} \}
                        &= \lim_{n\rightarrow \infty } \{ a_{n} \}
                        +  \lim_{n\rightarrow \infty } \{ b_{n} \} \\
                        &= \alpha + \beta
                    \end{align*}
                となる.これを証明するには,「近づける」という表現で極限を説明し
                ただけでは論理的に不可能である.極限を式で定義することで,この定
                義に基づいて,上の公式を証明することが可能になる.


            %==================================================================
            %  SubSection
            %==================================================================
                \subsection{関数の極限\;\;-- $\varepsilon $\,-\,$\delta$ 論法 --}
                数列の極限を考えたついでに,関数の極限も考える.
                考える関数の性質として,全ての点で連続で,全ての点で微分可能な関数を考える.
                関数が連続であるとは,関数に切れ目がないことである.微分可能であるというのは,
                全ての点がなめらかにつながっているということである.

                1変数関数について考える.
                変数 $x$ をもつ関数 $f(x)$ において,ある任意の定点 $x_{0}$ を
                関数 $f(x)$ に代入する.このとき,関数が $f(x_{0})=A$ であったとする.
                つまり,関数 $f(x)$ は,点 $x_{0}$ で,$A$ という値をとると仮定しよう.

                    \begin{figure}[hbt]
                        \begin{center}
                            \includegraphicsdefault{EpDl.pdf}
                            \caption{関数の極限}
                            \label{fig:EpDl}
                        \end{center}
                    \end{figure}

                この場合,関数の極限として,1変数関数 $f(x)$ において,
                   「 変数 $x$ を限りなく $x_{0}$ に近づけると,関数 $f(x)$ の値は,
                    A に近づく.」
                と言える.しかし,これはとても曖昧な表現である.
                数列の極限で考えたときと同じように,
                この「近づける」という行為を,式で表現することにしよう.

                変数 $x$ を定点 $x_{0}$ に近づけると,関数 $f(x_{0})$ は $A$ に近づくが,
                $A$ とは一致しない.つまり,変数 $x$ を定点 $x_{0}$ に近づけているときに
                は, $A$ とは若干異なった値の $A+\varepsilon $ を示す.ここに,$\varepsilon$ は
                任意の正の定数である.$\varepsilon $ のイメージは,とても小さな値をもつ
                定数である.

                さて最初に関数 $f(x)$ に対して,$\varepsilon $ を決めると,
                これに対応して,変数 $x$ の範囲も決まる.その $x$ の範囲を $\delta $ と書こう.
                しかし,この $\delta $ が分かったとき,さらに関数 $f(x)$ が $A$ に
                近づくような範囲 $\varepsilon $ を与えなおすことができる.そうすれば,
                この $\varepsilon $ の変更に伴って,$x$ の範囲 $\delta $ も変わる.そ
                うなれば,変更された $\delta $ に対して,さらに $f(x)$  が $A$ に
                近づくような範囲に $\cdots$.これは無制限に続けることができ,最終的に
                は,関数の極限として,$x$ を $x_{0}$ に近づけたときに,$A$ の値を得る,
                ということになる.

                このように,最初の一回だけ $\varepsilon$ を指定するだけで,
                あとは無制限に自動的に,関数の極限値を得られる.以上をもう
                少し詳しく記述し,式で表現して見よう.

                関数 $f(x)$ において,変数 $x$ を $x_{0}$ に近づけるが,
                $x_{0}$ に一致させないようにするとき,
                $\mid x-x_{0} \mid$ はある正の値 $\delta$ をもつ.
                この $\delta$ のイメージは,$x_{0}$ を
                含むような,変数 $x$ の微小な区間である.
                そして,関数 $f(x)$ には,この微小区間 $\delta$ に対応して,
                $\mid f(x) - A \mid$ という正の値 $\varepsilon$ という値域
                をもつことになる.$f(x_{0})=A$ であるので,$\varepsilon$ が
                より小さくなれば,さらに関数は $A$ からの範囲を狭めていくこ
                とになる.関数 $f(x)$ は連続な関数なので,$\varepsilon$ をさ
                らに小さくすることは可能である.$\varepsilon$ をさらに小さく
                するに伴い,$\delta$ もさらに小さくなっていく.
                もちろん,$\delta$ をいくら小さくしても,その範囲の中には $x_{0}$ が
                含まれている.

                式で書いてみよう.まず,$A$ からのずれの程度の指標である,
                $\varepsilon$ が考えられる.そして,この $\varepsilon$ が
                定まることで,$x$ の範囲もきまり,$\delta$ となるとき,
                    \begin{align}
                        \mid x-x_{0} \mid < \delta \quad
                        \Rightarrow \quad \mid f(x) - A \mid < \varepsilon
                    \end{align}
                であるならば,関数 $f(x)$ は,変数 $x$ を定点 $x_{0}$ に
                近づけたときに,$A$ に収束すると言える.

                今までは連続関数を考えてきたが,実際は,もう少し制限を緩めることができる.
                というのも,関数 $f(x)$ は区間 $I$ のみで連続と定義されていればよく,
                また一方で,$x$ も $x_{0}$ の点で定義されていなくてもかまわない.
                $f(x_{0})$ が定義されていなくても,$x$ は単に $x_{0}$ に近づけるだけ
                であり,一致させるわけではないので,不都合は生じない.

                最初に $\varepsilon$ の存在を確認したのは,このためである.
                もし $\delta$ の存在を先に認めてしまうと,もしかしたら,
                関数が定義されないような範囲を選んでしまう可能性がある.
                そうなれば,上の式は,その関数が定義されていない点に差し掛かったとき,
                極限が存在しないという結論を出してしまう.

