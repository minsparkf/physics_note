%       %======================================================================
%       %  SubSection
%       %======================================================================
\subsection{和の微分}
    1変数関数 $K(x)$ があり,この $K(x)$ が2つの関数 $f(x)$,$g(x)$ の和に分解できるとする.
    つまり,以下が成り立っている場合を考える
        \footnote{
            ごく簡単な例で言えば,$K(x)=f(x)+g(x)=5x$,$f(x)=x$,$g(x)=4x$ の場合とか.
        }.
        \begin{equation*}
            K(x) = f(x) + g(x).
        \end{equation*}
    この時,
         \begin{align}
             K'(x) = f'(x) + g'(x)
         \end{align}
    が成り立つ.

    これは微分の定義に従って式変形をすることで,確認できる.少々面倒だが,やってみよう.
        \begin{align*}
            K'(x) &= \lim_{\Delta x \to 0} \frac{K(x+\Delta x)-K(x)}{\Delta x} \\
                  &= \lim_{\Delta x \to 0}
                     \frac{\left(f(x+\Delta x)+g(x+\Delta x)\right)-\left( f(x)+g(x)\right)}{\Delta x} \\
                  &= \lim_{\Delta x \to 0}
                     \frac{f(x+\Delta x)-f(x) + g(x+\Delta x)-g(x)}{\Delta x} \\
                  &= \lim_{\Delta x \to 0}
                     \left(
                         \frac{f(x+\Delta x)-f(x)}{\Delta x}+\frac{g(x+\Delta x)-g(x)}{\Delta x}
                     \right) \\
                  &= \lim_{\Delta x \to 0} \frac{f(x+\Delta x)-f(x)}{\Delta x}
                   + \lim_{\Delta x \to 0} \frac{g(x+\Delta x)-g(x)}{\Delta x} \\
                  &= f'(x) + g'(x).
        \end{align*}

    この式変形で,関数の極限の和の公式を利用している
        \footnote{
            $\displaystyle\lim_{x \to 0}f(x)$ と $\displaystyle\lim_{x \to 0}g(x)$ が共に収束する場合,
            \begin{equation*}
                \lim_{x \to 0}(f(x)+g(x)) = \lim_{x \to 0}f(x) + \lim_{x \to 0}g(x).
            \end{equation*}
        }.

\subsection{積の微分}
    1変数関数 $K(x)$ があり,この $K(x)$ が2つの関数 $f(x)$,$g(x)$ の積に分解できるとする.
    つまり,以下が成り立っている場合を考える
        \footnote{
            ごく簡単な例で言えば,$K(x)=f(x)g(x)=5x(x+1)$,$f(x)=5x$,$g(x)=x+1$ の場合とか.
        }.
        \begin{equation*}
            K(x) = f(x)g(x).
        \end{equation*}
     この時,
         \begin{align}
             K'(x) = f'(x)g(x) + f(x)g'(x)
         \end{align}
     が成り立つ.

    これは微分の定義に従って式変形をすることで,確認できる.少々面倒だが,やってみよう.
    特に,$-f(x)g(x+\Delta x) + f(x)g(x+\Delta x) = 0$ であることに注意.
        \begin{align*}
            &K'(x) \\
            &= \lim_{\Delta x \to 0}
               \frac{K(x+\Delta x)-K(x)}{\Delta x} \\
            &= \lim_{\Delta x \to 0}
               \frac{f(x+\Delta x)g(x+\Delta x)-f(x)g(x)}{\Delta x} \\
            &= \lim_{\Delta x \to 0}
               \frac{f(x+\Delta x)g(x+\Delta x)-f(x)g(x+\Delta x) + f(x)g(x+\Delta x)-f(x)g(x)}{\Delta x} \\
            &= \lim_{\Delta x \to 0}
               \frac{\left(f(x+\Delta x)-f(x)\right)g(x+\Delta x) + f(x)\left(g(x+\Delta x)-g(x)\right)}{\Delta x} \\
            &= \lim_{\Delta x \to 0} \frac{\left(f(x+\Delta x)-f(x)\right)g(x+\Delta x)}{\Delta x}
             + \lim_{\Delta x \to 0} f(x)\frac{g(x+\Delta x)-g(x)}{\Delta x} \\
            &= \lim_{\Delta x \to 0} \frac{f(x+\Delta x)-f(x)}{\Delta x} g(x+\Delta x)
             + f(x) \lim_{\Delta x \to 0} \frac{g(x+\Delta x)-g(x)}{\Delta x} \\
            &= f'(x)g(x) + f(x)g'(x).
        \end{align*}

    この式変形で,関数の極限の積の公式を利用している
        \footnote{
            $\displaystyle\lim_{x \to 0}f(x)$ と $\displaystyle\lim_{x \to 0}g(x)$ が共に収束する場合,
            \begin{equation*}
                \lim_{x \to 0}(f(x) \cdot g(x)) = \lim_{x \to 0}f(x) \cdot \lim_{x \to 0}g(x).
            \end{equation*}
        }.
    また,式変形について,以下を考慮した.
        \begin{align*}
            f(x) &= \lim_{\Delta x \to 0} f(x) \\
            g(x) &= \lim_{\Delta x \to 0} g(x+\Delta x)
        \end{align*}
    1つ目の式はそもそも $\Delta x$ が関数 $f(x)$ にないので,$\Delta x$ に関する極限を
    とったところで変化なし.2つ目の式は関数の極限の定義そのものである.

\subsection{商の微分}
    1変数関数 $K(x)$ があり,この $K(x)$ が2つの関数 $f(x)$,$g(x)$ の商に分解できるとする.
    つまり,以下が成り立っている場合を考える
        \footnote{
            ごく簡単な例で言えば,$K(x)=f(x)/g(x)=5x$,$f(x)=20{x}^{2}$,$g(x)=4x$ の場合とか.
        }.
        \begin{equation*}
            K(x) = \frac{f(x)}{g(x)}.
        \end{equation*}
    この時,
         \begin{align}
             K'(x) = \frac{f'(x)g(x) - f(x)g'(x)}{{\left(g(x)\right)}^{2}}
         \end{align}
    が成り立つ.

    これを示すには,
        \begin{equation*}
            K(x) = \frac{f(x)}{g(x)} = f(x)\frac{1}{g(x)}
        \end{equation*}
    とみなし,積の微分公式を用いる.実際に確認してみよう.
        \begin{equation*}
            K'(x) = f'(x)\frac{1}{g(x)} + f(x)\left(\frac{1}{g(x)}\right)'
        \end{equation*}
    ここで移行の式変形の都合上
        \footnote{
            不自然に思われるかもしれないが,覚えやすく使いやすい形にしたいがための式変形であり,
            重要な部分である.
        }
    ,上式の第1項の分母と分子に $g(x)$ をかけて,
        \begin{equation*}
            K'(x) = f'(x)g(x)\frac{1}{{g(x)}^{2}} + f(x)\left(\frac{1}{g(x)}\right)'
        \end{equation*}
    としておく.また,第2項のカッコ内の微分は,定義に従って計算する
        \footnote{
            2つの分数の恒等式を使うので,補足しておこう.
            \begin{equation*}
                \frac{1}{A} - \frac{1}{B}
                = \frac{B}{AB} - \frac{A}{AB}
                = \frac{B-A}{AB}
                = - \frac{A-B}{AB}.
            \end{equation*}
            \begin{equation*}
                \frac{\displaystyle\frac{A}{B}}{C}
                = \frac{\displaystyle\frac{A}{B}B}{BC}
                = \frac{A}{BC}.
            \end{equation*}
        }.
        \begin{align*}
            \left(\frac{1}{g(x)}\right)'
            &= \lim_{x \to 0}
               \frac{\displaystyle\frac{1}{g(x + \Delta x)} - \displaystyle\frac{1}{g(x)}}{\Delta x} \\
            &= \lim_{x \to 0}
               \frac{\displaystyle\frac{-g(x + \Delta x)+g(x)}{g(x + \Delta x)g(x)}}{\Delta x} \\
            &= \lim_{x \to 0}
               -\frac{g(x + \Delta x)-g(x)}{\Delta x}\frac{1}{{g(x + \Delta x)g(x)}} \\
            &= -g'(x) \frac{1}{{\left(g(x)\right)}^{2}} \\
            &= -\frac{g'(x)}{{\left(g(x)\right)}^{2}}
        \end{align*}
    なので,
        \begin{equation*}
            K'(x) = f'(x)g(x)\frac{1}{{g(x)}^{2}}-f(x)\frac{g'(x)}{{\left(g(x)\right)}^{2}}.
        \end{equation*}
    共通な分母でまとめると,
        \begin{equation*}
            K'(x) = \frac{f'(x)g(x)-f(x)g'(x)}{{\left(g(x)\right)}^{2}}.
        \end{equation*}

\subsection{合成関数の微分}
    関数 $f(g(x))$ というもの考える.$g(x)$ は独立変数 $x$ を持つ関数であり,$f$ は $g(x)$ で
    得られた値を引数に持つ関数である
        \footnote{
            例えば,${(x+1)}^{2}$ であれば,$g(x)=x+1$ として $f(g(x))={\left(g(x)\right)}^{2}$ と
            書き表せる.
        }.
    こういった入れ子になった関数を \textbf{合成関数} という.
    $f(g(x))$ は $(g \circ f)(x)$ と書かれることもある.
        \footnote{
            表記だけの問題だが,$f(g(x))$ という表現は,見難くなる傾向にある.この程度であれば,まだ
            わかるが,これが $f(g(h(i(j(x))))$ となった場合にはカッコが多くて,読み難くくなってしまう.
            だから,$f(g(x))=(g \circ f)(x)$ と書くこともある.これなら,関数が多くなろうとも問題ない.
            さっきの例で言うと,$(j \circ i \circ h \circ g \circ f) (x)$ である.
            関数の依存関係を左のカッコ内に書き,その独立変数を右のカッコの中に書く.関数の依存関係は
            一番深いものを最左に書き,順次右に追記していく.
        }.

    ここでは,この合成関数の微分公式を確認する.
    合成関数 $(g \circ f)(x)=f(g(x))$ は $u=g(x)$ と置いて $f(u)$ と見ることができる.
    $g(x)$ が $x$ の関数だから,結局のところ $(g \circ f)(x)$ も $x$ の関数となるので,
    $\df (g \circ f) /\df x$ を考えられる.$u=g(x)$ を展開し,さらに $f(u)$ を
    展開した後で,微分を計算しても良いが,もっと賢い方法がある.次のとおりだ.
        \begin{align}
            \frac{\df (g \circ f)}{\df x}
            = \frac{\df f}{\df u}\frac{\df u}{\df x}
            = \frac{\df f(u)}{\df u}\frac{\df g(x)}{\df x}.
        \end{align}

    これが正しいことは,微分の定義から計算することで確認できる.
        \begin{align*}
            &\frac{\df (g \circ f)}{\df x} \\
            &= \lim_{\Delta x \to 0}
               \frac{(g \circ f)(x+\Delta x)-(g \circ f)(x)}{\Delta x} \\
            &= \lim_{\Delta x \to 0}
               \frac{(g \circ f)(x+\Delta x)-(g \circ f)(x)}{g(x + \Delta x) - g(x)}
               \frac{g(x + \Delta x) - g(x)}{\Delta x} \\
            &= \lim_{\Delta x \to 0}
               \frac{(g \circ f)(x+\Delta x)-(g \circ f)(x)}{\Delta g}
               \frac{g(x + \Delta x) - g(x)}{\Delta x} \\
            &= \lim_{\Delta x \to 0}
               \frac{(g \circ f)(x+\Delta x)-(g \circ f)(x)}{g(x + \Delta x) - g(x)}
               \cdot
               \lim_{\Delta x \to 0}
               \frac{g(x + \Delta x) - g(x)}{\Delta x}
        \end{align*}

    ここで,式の見やすさのため,$\Delta g = g(x + \Delta x) - g(x)$ と置く.
    また,$g(x)=\displaystyle \lim_{\Delta x \to 0} g(x+\Delta x)$ の関係から,
        \begin{align*}
            &\lim_{\Delta x \to 0} \left( g(x+\Delta x) \right) - g(x) \\
            &= \lim_{\Delta x \to 0} \left( g(x+\Delta x) - g(x) \right) \\
            &= \lim_{\Delta x \to 0} \Delta g \\
            &= 0
        \end{align*} \\
    が成り立つので,$\Delta x \to 0$ を $\Delta g \to 0$ に置き換えられることを使う.
        \begin{flalign*}
            &\frac{\df (g \circ f)}{\df x} \\
                                &= \lim_{\Delta g \to 0}
                                   \frac{(g \circ f)(x+\Delta x)-(g \circ f)(x)}{\Delta g}
                                   \cdot
                                   \lim_{\Delta x \to 0}
                                   \frac{g(x + \Delta x) - g(x)}{\Delta x}
        \end{flalign*}

    最後に,$\Delta (g \circ f) = (g \circ f)(x+\Delta x)-(g \circ f)(x) $,
    $\Delta u = \Delta g$ であることに注意すれば,
        \begin{flalign*}
            \frac{\df (g \circ f)}{\df x}
                &= \lim_{\Delta g \to 0} \frac{\Delta (g \circ f)}{\Delta g}
                   \cdot
                   \lim_{\Delta x \to 0} \frac{g(x + \Delta x) - g(x)}{\Delta x} \\
                &= \lim_{\Delta u \to 0} \frac{\Delta (g \circ f)}{\Delta u}
                   \cdot
                   \lim_{\Delta x \to 0} \frac{\Delta g}{\Delta x} \\
                &= \lim_{\Delta u \to 0} \frac{\Delta (g \circ f)}{\Delta u}
                   \cdot
                   \lim_{\Delta x \to 0} \frac{\Delta u}{\Delta x} \\
                &= \frac{\df (g \circ f)}{\df u} \frac{\df u}{\df x}
        \end{flalign*}
    を得る.大抵の場合は,$f:=(g \circ f)=f(g(x))$ と略記されるので,これに従うと,
        \begin{flalign*}
            \frac{\df f}{\df x} &= \frac{\df f}{\df u} \frac{\df u}{\df x}
        \end{flalign*}
    となって,いつもの合成関数の微分公式が見えてくる.

\subsection{部分積分}


