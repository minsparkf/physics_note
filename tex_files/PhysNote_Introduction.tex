%===================================================================================================
%  Chapter : 感覚・思想・表現
%  説明    : 物事を考える上での,根本的な思想を記述する
%===================================================================================================
\chapter{感覚$\cdot$思考$\cdot$表現}
    \begin{mycomment}
        この章は,私の個人的な考えをメモしておくものである.だから,記述内容が
        誤っているかもしれず,考え違いや,誤解が含まれていることと思う.
        誰とも議論もせずに記述することであるから,偏った考え方になりがちだし,
        最悪の場合,矛盾がふくまれているだろう.それにもかかわらず,
        この章の文章は"言い切り"の形で記述する.いちいち,「だと思う」という
        語彙を文章末尾につけてしまうと,読みにくくなってしまうし,第一,カッコ悪い.
        なので,偏った考え方や誤った主張を強く強調していると思われるかもしれないが,
        そうではなく,単に現在の私の考えをメモしているものに過ぎないと,捉えてもらいたい.
    \end{mycomment}
%   %==========================================================================
%   %  Section
%   %==========================================================================
        \section{根拠なしに,確信できること}
            根拠なしに確信を持てることは,「考えられる」ということである.
            そして,考えるという行為は,言葉
                \footnote{
                    ここで言う \textbf{言葉} とは,“書くことによる表現”と“話すことによる表現”
                    の両方を指す意味で使っている.
                }
            を用いて行われてることも,根拠なしに
            認めることができる
                \footnote{
                    「言葉なしに考える」という行為は可能だろうか.
                    少なくとも,私には実行不可能である.
                }.

            何かものを考えるときには,考えるための材料と道具が必要である.
            材料というのは経験であり,道具というのは言葉である.
            考えるという行為とは,経験を道具により整理して,その経験に対する
            理解を深めるという行為である.

            私たちの経験する全ての事柄を,\textbf{世界} と表現することにしよう
                \footnote{
                    ここでいう \textbf{世界} とは,世界の国々を表しているのではない.
                    私たちが目や耳などの,いわゆる五感で感じ取る全てのことを総合して,
                    「世界」と表現する.
                }.
            私たちは,世界を経験をできる.経験を記憶できる.
            また,経験を不思議がったり,その不思議を考えられる.
            そして,その考えを言葉や絵で表現できる.そして,このような表現を
            行うことで,自分以外の相手に自分の考えを伝えることができる.

            これが,私の「考えること」の根本的な思想である.これからの物理学の学習は,
            この思想のもとに行う.

            \begin{memo}{言葉と思考の順序}
                考えるという行為は,言葉を使って行われう行為である.もっと強い言い方を
                すると,考えるという行為は言葉なしに行うことは不可能である.つまり,
                言葉を習得する以前は,ものを考えることができないことになる.
                となると,言葉の見習得の赤ん坊は,ものを考えることができないのだろうか.
                この問題に対する,私が納得のできる答えは,まだ存在していない.

                しかし,これだけは言える.言葉の習得する以前から,この世界を感じている.
                世界を感じるという経験の1つに,言葉がある.経験が「思考する」という
                行為の基礎に位置するのである.しかし,この推論に確証はもてない.
                経験が思考の基礎をなすということ
                を,今から身をもって体験することができないからだ.
                言葉習得以前の状態にもどり,言語を習得しなくとも世界を感じとることが
                できるのだな,という感覚を体験できればよいが,これは不可能である.
                推論をいくら重ねたところで,あるいは,多くの言語見習得の赤ん坊を
                観察したところで,自分自身で実感できないので,納得はできない.
                納得はいかないけれど,今の私は言語を扱っている
                    \footnote{
                        少なくとも,とりあえずの不都合なく,意思疎通ができる程度に.
                   }.
                また,その習得に多くの時間を費やしたことも記憶にある.ということは,
                言語見習得の時期があったいう推論は妥当であるとも思える
                    \footnote{
                        実際に言語見習得の時期の記憶がないので,単なる推論でしかない.
                    }.
            \end{memo}

            \begin{memo}{言葉の習得}
                言葉は,人間が生まれながらにしてもっているものではない.
                言葉は,意思疎通を行うために,先人の経験により発明され,
                洗練されてきたものである.

                私たちは,生まれてから自然と母国語の文法を習得する
                    \footnote{
                        国語の時間に,強制的に習得させられることもあるだろう.
                        ひらがなや,カタカナ,漢字を覚えるのには苦労したはず.
                        また,使い慣れない語彙を使用し始める場合,単語そのものを
                        誤って使ってしまい(「つくる」を「くつる」に間違うなど),
                        大人から,その場で間違いを指摘され,修正された経験もあったことだろう.
                        しかし,その間違いの指摘は言葉によって行われたのであり,
                        それにより,誤りを自覚してそれを修正しようと努力できたはずである,
                        全ての言葉を自然に習得できるわけではないのだが,言葉の基本的な使い方
                        は誰から教わったものではなく,自分の周囲に飛び交っていた母国語を
                        聞くことにより,非言語的に習得するのである(
                        そうでなければ,言葉の使い方を間違っときの,その間違いの指摘を
                        理解することができないことになる).
                        言葉の文法を,0から
                        言葉により説明することはできない.というのも,文法を説明しようとすると,
                        その説明自体に言葉を使用せざるを得ず,
                        つまり,「文法を知る前に文法を知っていなければならない」といった,
                        自己言及的な矛盾がおこるからである.しかし,現に私たちは母国語の文法を
                        習得して使用している.
                    }.
                母国語は体系的に教わることない.常に生の言葉を聞き,そして,その時の状況を
                機能するすべての感覚器から感じ取り,言葉の使用法を身につける.論理学や数学
                で言うところの公理
                    \footnote{
                        最も基礎となる,疑いようのない事実のこと.万人が根拠なしに正しいと
                        感じること.例えば,数学で言うと,「A$=$B で B$=$C のとき,A$=$C」
                        といった感じの,最も基本的なお約束のことを言う.これは有無を言わさず
                        たたきつけられることである.その根拠を求めてはならない.公理の根拠
                        なんてものは,はじめから存在しないのだから.言い方を変えれば,何かを
                        論理的に考えようとした時に,その根拠が求められることがあるが,
                        その根拠は果てしなく問うことは不可能である.いずれは,根拠を示すことができない
                        当たり前すぎる事柄に,直面することだろう.それを公理というのである.
                    }
                があるわけでもなく,ただ漠然と,その使われ方を悟り,自分のものとしていく.
                人間には,言葉を習得する能力が生まれながらにして備わっているのだ.
                なぜだろうか.意思疎通をうまい具合に行うためだとか,いろいろ後付的な
                説明がなされることもあろうが,そんなことは確かめようがない.なんとでも言えるのだ.
                ここでは,深く考えずに,「私は言葉を使うことができる」ということを,
                素直にそのままの形で受け入れておこう.
            \end{memo}

            \begin{memo}{意思疎通}
                言葉はどんな時に使われるのだろうか.いや,おかしな疑問を投げかけてしまった.
                そんなの,意思疎通を行うために決まっているじゃないか.本当に,そうなのか.
                言葉でだけで,意思疎通が十分に可能だろうか.いや,できないはずだ.
                誤解や言い間違いなどで,正しく意思疎通ができないこともあるだろう.
                言葉だけで十分に意思疎通はできないのは,当たり前で,そのために,他の手段として,
                手や体を動かして(身振り手振り)表現することもある
                    \footnote{
                        別れ際に相手に対して手を振ったり,違うことを示すために首を横に振ったりするだろう.
                    }.
                図を使って表現することもあろう.音楽で気分を表現したりもするだろう
                    \footnote{
                        気分が良い時には,鼻歌が自然にでたり,踊ることだってするでしょ?
                    }.
                とにかく,意思疎通のための道具は,言葉だけではない.

                だけど,ここまで書いといてなんだけども,ノートで自分の考えを示すのには,
                図と言葉でしか表現できない.なんともやりづらいが,どうしようもない.
                ここは我慢して,図と言葉だけで伝わるように,記述しなければならない.
                図と言葉だけで,どれだけのことが表現できるかわからないが,頑張って考えながら,
                記述していこう.たとえ時間がかかろうとも,文が長ったらしくなろうとも,
                丁寧に記述していけば,どんなことでも言葉で伝えることができると信じて,
                記述していこう.

                私は考えることができ,それを言葉で表現でき,そして,
                その言葉によって他者に自分の考えを伝える事ができると信じる.
                そして,逆に,他人の表現する言葉を理解し,他者の考えを受け入れられることも,信じる.
                さらに,自分と他人とで会話を続けることにより,より深く正確に互いの考えを
                理解し合えると信じよう.ここのところをこれ以上疑いだすと,言語哲学的な世界に
                陥ることになり,抜け出せなくなってしまう.もうこれ以上,言葉について考える
                事はせずに,話を先に進めよう.
            \end{memo}

%   %==========================================================================
%   %  Section
%   %==========================================================================
        \section{表現}
            思考を表現する最も有効な道具が,言葉である.また,時には,言葉よりも絵に書いたほうが
            より伝わりやすいこともあろう
                \footnote{
                    宣伝看板や,ポスターなどが良い例だ.
                }.
            言葉や絵以外にも,ある規則に従った記号により,思考を
            表現することも可能である.
                \footnote{
                    特に,音楽は言葉によって表すことは難しい.原理的に不可能とは言わないが,
                    そうして表現できたものは,とても煩雑で理解しがたい表現になっていることだろう.
                    では,音楽を伝える方法はないのかというと,もちろん,そんなことはない.
                    周知の通り,\textbf{楽譜} という音楽独特の表現方法が考えだされている.
                    楽譜は絵でも言葉でもないが,人間のもっている,ある種の感覚を表現するものである.

                    また,他例を上げれば,数学や記号論理学なども,記号の羅列である.

                    「言葉」という語彙は,これらの例のような意味を含めて使われることも多い.
                    これは,文脈によって理解できるだろう(著者はそうわかるように書くべきだ).
                }.
            表現するという行為は,言うまでもなく,
            自分の考えや思いを自分以外の相手に伝えるということである.

%   %==========================================================================
%   %  Section
%   %==========================================================================
        \section{「科学」の思想}
            "科学的" という言葉は日常的に使用される.特に,最新技術という意味で
            使用されることが多いように感じる.しかし,"科学的" とはどういうことかを
            考えてみると,残念なことに,明確なイメージを描けないことに気付く.

            \begin{figure}[hbt]
                \begin{center}
                    \includegraphicslarge{SekaiKansokuBunrui01.pdf}
                    \caption{科学的に扱えることの範囲}
                    \label{fig:ThomomExpElec001}
                \end{center}
            \end{figure}

        \begin{memo}{基礎がない,考えが循環する}
            ものごとを考える前には,経験が必要である.考えは,その経験の不思議さ
            を基に行われるからである.では,この経験はどう捉えら得るのだろうか.
            当然ながら,経験は眼や耳などの五感で捉えられる.そして,その認識は,
            脳に伝わって初めて経験を実感し,記憶される.だから,最も基礎的な
            学問分野は脳科学なのだろうか.いや,これは違う.脳は生物の一部であり,
            脳科学は生物学の一分野として位置するものである.生物はその組織の
            内部で,化学反応を起こして生命を維持している.従って,化学が,生物学よりも
            基礎的な位置にあるはずである.また,化学で扱われる化学反応は,突き詰めれば,
            原子や分子の結びつきであり,原子や分子の運動は物理学により説明される.
            そして,物理学は数理的推論をその拠り所としていて,数学と論理学が物理学
            の基礎となっている.数学と論理学は,元を正せば,私達が日常生活の中で使用している
            言語の曖昧な部分を除き,正しい思考とは何かを探る学問である.そして,その思考を
            行うには,前もってその思考に関する経験が必要である.経験は五感で感じ,脳によって
            認識され,$\cdots$.

            循環する.上記のどこかに誤りがあるのだろうか.
        \end{memo}

%===================================================================================================
%  Chapter :
%  説明    :
%===================================================================================================
\chapter{論理}
