\section*{記述方法についての諸注意}
    \begin{mycomment}
        ノートの記述に先立って,数式の記述についていくつか注意しておきたい.
        意味が分からなければ,先に進んでも構わないが,この部分に数式表現
        の注意があることは覚えておいてもらいたい(必要な場合に,ここを参照
        できるように).
    \end{mycomment}


    %===========================================================
    % * 括弧の扱いについて
    %===========================================================
    \begin{preattention}{括弧の扱いについて}
        このノートでは,多くの教科書と同様に,関数の独立変数を表す括弧「例:$\phi(x)$」と
        式中の括弧「例:$a(b+c)$」は同一のものを使用している.

        しかし,この表記は,ときには,誤解をまねく.というのも,
        例えば,関数 $\phi(x)$ と $a+b$ の積を記述する場合,
            \begin{equation*}
                \phi(x) (a+b)
            \end{equation*}
        と書く.これは見方によっては,実数 $\phi$,$x$,$a+b$ の積
        であると解釈することもできる.まあ,この場合は $x$ が1つであること
        から,「$x$ は関数 $\phi$ の独立変数であり,$\phi(x)$ は関数を表すのだな」
        と読まれることだろう.しかし,上の独立変数 $x$ に定数 $a+b$ を代入すること
        きには,曖昧にな記述になってしまう.つまり,
            \begin{equation*}
                \phi(a+b) (c+d)
            \end{equation*}
        これは,著者の意図としては,「関数 $\phi$ の独立変数 $x$ に,$a+b$ を代入したものと,
        実数 $c+d$ の積」を表現したつもりだろう.しかし,読者が分からしてみれば,単に式を
        見ただけでは,「3つの実数 $\phi$,$a+b$,$c+d$ の積」と解釈するのが妥当である.
        もちろん,著者は,このような式を記述する前後で,文字の意味に対する説明を行って
        いるので,通常なら,誤解されることはない.しかしながら,意味が曖昧な式である
        ことには変わりはない.それゆえに,何かストレスを感じてしまう.

        ちなみに,「関数 $\phi$ の独立変数 $x$ に,$a+b$ を代入したものと,
        実数 $c+d$ の積」を表現したい場合には,
            \begin{equation*}
                \phi(a+b) \cdot (c+d) \quad \mbox{, あるいは,} \quad (c+d)\phi(a+b)
            \end{equation*}
        などと書かき,区別を強調するかもしれない.何れにしても,教科書に記述されている
        ことを理解するのは読者の仕事であり,臨機応変に適切に読み込まなければいけない.
        それでも尚,複数の意味として捉えられてしまうようであれば,もしそれが重要な部分
        であると感じるなら,著者に直接質問すべきだ.しかし,著者と直接対話できる
        ことは容易ではなく,また,趣味で物理学を学んでいるので実害はなく,直接質問
        することを躊躇してしまうことだと思う.そういった場合は,手っ取り早い方法として,
        他の著作も参照して見ることである.大概の場合は,この方法で解決することだろう.


    \end{preattention}

    %===========================================================
    % * 積分記号について
    %===========================================================
    \begin{preattention}{積分記号}
        おそらく,一般的な積分の表現は,
            \begin{align*}
                \int f(x) \df x
            \end{align*}
        のように,関数 $f(x)$ をインテグラル $\int$ と微分記号 $\df x$ で
        囲んだ形だろう.このノートでも,積分を表す記述方法として,上記を採用する.

        しかし,別の表記方法を採用している教科書も多い.次のような
        書き方がされることがあるのだ.
            \begin{align*}
                \int \df x f(x).
            \end{align*}
        この書き方は,\textbf{演算子} という考え方をもとにした
        表現方法である.

        演算子とは何かを,考えてみよう.微分を例にとろう.関数 $f(x)$ を $x$ で
        微分することを,次のように表現する.
            \begin{align*}
                \frac{\df f(x)}{\df x}.
            \end{align*}
        上の表現とは別に,教科書には次のようにも表現されることが,書かれている
            \footnote{
                微積分の教科書であれば,どのようなものでも記述されている.
                もっと強い言い方をすれば,この記述を紹介していないものは,
                微積分の教科書とは言えない.
            }.
            \begin{align*}
                \frac{\df}{\df x} f(x).
            \end{align*}

        微分は,関数 $f(x)$ に対するひとつの操作である.具体的に見てみよう.
        例えば,$f(x)=x^{2}$ でれば,$f(x)$ を微分した結果 $f'(x)$ は $2x$ と
        なる.$f(x)=x^{5}$ であれば,$f'(x)=5x^{4}$ だし,$f(x)=\sin x$ だったら,
        $f'(x)=\cos x$ である.こうしてみると,微分するということは,元となる
        関数 $f(x)$ に対して,「微分するという操作」を与えることで,あたらな関数
        (導関数:$f'(x)$)を作り出していると,捉え直すことができる.こう考えた場合,
        上に書いた記号で,$\df/\df x$ の部分と,$f(x)$ の部分に分けてみて,
        「$x$ に関して微分するという操作 $\df/\df x$ を,関数 $f(x)$ に対して行う」
        と読むことで,$\df/\df x$ に,「$x$ に関して微分する」
        という意味を与えることができる.
        つまり,以下の記号
            \begin{align*}
                \frac{\df}{\df x}
            \end{align*}
        が,微分の操作を象徴する記号になる.$\df/\df x$ は,関数 $f(x)$ に対して微分するという
        作用をほどこすことから,\textbf{微分作用素} とよばれる.

        積分に関しても,微分と同様に考えて,一般的な記号 $\int f(x) \df x$ を書き換えて,
            \begin{align*}
                \int \df x f(x)
            \end{align*}
        とすることにより,積分という操作 $\int \df x$ を,関数 $f(x)$ に関して行う,といった
        意味を強調できる.
    \end{preattention}

    %===========================================================
    % * 三角関数の表現
    %===========================================================
    \begin{preattention}{三角関数の表現}
        ここで上げる問題は,上記の括弧の使い方に関連するものだが,
        三角関数に関する括弧の扱いについて,誤解を生みやすいので,
        ここで特別に取り上げることとする.

        三角関数は,$\sin x$ のように記述される.$x$ は位相とよばれ,
        この関数の独立変数をになっている.問題は,この位相 $x$ の書き方
        である.例えば,物理学では,位相として,角周波数 $\omega$ と時間 $t$ の
        積 $\omega t$ が使われることが多い.
        つまり,$x=\omega t$ として,
            \begin{equation*}
                \sin \omega t
            \end{equation*}
        と記述されることがある.ここまでは,特に問題がない.しかし,
        例えば,$\omega=\omega_{0}+\omega_{1}$ というような場合,上式は
            \begin{equation*}
                \sin \left(\omega_{0}+\omega_{1}\right) t
            \end{equation*}
        と書かれる.さて,この式はどう見えるだろうか.一般的な解釈
        では,位相が $\left(\omega_{0}+\omega_{1}\right) t$ の $\sin$ 関数
        だろう.しかし,式だけを見る限り,$\sin \left(\omega_{0}+\omega_{1}\right)$ と
        時間 $t$ の積であるようにも,解釈ができる.つまり,
            \begin{equation*}
                \{\sin \left(\omega_{0}+\omega_{1}\right)\}\cdot t
            \end{equation*}
        のようにも見えてしまうのである.しかし,このように見てしまうのは,
        暗黙の了解を知らないものだけである.三角関数を記述する上での暗黙の了解とは,
        $\sin$ の直後に記述されるものが位相である,ということである.つまり,
            \begin{equation*}
                \{\sin \left(\omega_{0}+\omega_{1}\right)\}\cdot t
            \end{equation*}
        と解釈してはいけない.あくまでも,位相は $\sin$ の直後に書かれている
        もじであり,この例では,$\left(\omega_{0}+\omega_{1}\right) t$ がその
        位相にあたる.ただし,$\sin x$ 全体が括弧に囲まれていて,例えば,
            \begin{equation*}
                \{\sin \left(\omega_{0}+\omega_{1}\right) t\}x
            \end{equation*}
        と書かれていたら,$\sin \left(\omega_{0}+\omega_{1}\right) t$ と $x$ の
        積であると解釈すべきだ.

        三角関数の記述には,別の問題もある.例えば,
            \begin{equation*}
                \sin x \sin y
            \end{equation*}
        という記述である.これは間違っても以下のように解釈してはいけない.
            \begin{equation*}
                \sin(x \sin y). \qquad \mbox{この解釈は間違い}
            \end{equation*}
        正しくは,
            \begin{equation*}
                (\sin x)(\sin y)
            \end{equation*}
        と読むべきだ.

        三角関数の変数を表すとき,$\sin(x)$ のように記述すべきなのだが,
        なぜか,いちばん外側のカッコが省略されてしまい,$\sin x$ と
        かかれてしまう.おそらく,カッコが多すぎると,式が読みづらくなって
        しまうからだろう.たしかに,カッコは少ないほうが,式は簡潔になり,
        読みやすくなる.しかし,その代償として,式の意味するところが曖昧に
        なってしまう.慣れている人ならば,上に書いた暗黙の了解
        を会得しているので,なんの誤解もなく読めてしまうのだが,不慣れなものは,
        よく読み間違えをしたり,どう解釈して良いかわからなかったりする.
        話の流れから理解できることが大半ではあるが,混乱をさせないように,
        予め,この暗黙の了解について記述しておいた.
    \end{preattention}


    %===========================================================
    % * 「一般に…」という文句について
    %===========================================================
    \begin{preattention}{「一般に」という記述}
        なんの根拠の記述もなしに,「一般に$\cdots$」と書かれていたら,
        注意が必要である.
        つまり,著者が独断的に一般的であるとしていえるからである.
        なんの資料や調査もなしに,「一般に」という語彙を使用しているの
        であれば,著者の経験上のものであり,実際には一般ではないかも
        しれない.

        著者が専門家であれば,信頼できる単語だが,こんにちでは,
        非専門家による物理学のWebサイトや,解説本などがはびこっている.
        そういった場合には,ある程度疑ってかかってみたほうが良い.
        このノートでも,「一般に」という語彙は頻出語彙の1つであるが,
        これも,その意味は「(私の経験上)一般に」ということである.
        このノートを読む際には,特に注意しておいてもらいたい.
    \end{preattention}