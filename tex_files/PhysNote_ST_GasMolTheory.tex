%   %==========================================================================
%   %  Comment
%   %==========================================================================
\begin{mycomment}
    気体についての考察.気体は無数の分子から構成されていると言う仮定して,
    気体の圧力や分子の運動エネルギーについて調べてみよう.
\end{mycomment}

\section{仮定}
    気体分子運動論を考える状況について,議論の最初に,いくつか仮定を設けておく.
    \begin{itembox}[l]{気体分子運動論の仮定}
        \begin{itemize}
            \item 気体は理想気体を対象とし,分子間力は考えない
            \item 分子は数え切れないほど多く存在し,巨視的に見れば,
                  あらゆる方向に均一に運動している($x$,$y$,$z$の全方向に均一である)
            \item 分子同士の衝突はないものとする(衝突しても結果は変わらないが,状況が複雑になり計算が面倒になる)
            \item 分子と壁は弾性衝突する(衝突係数$=1$:跳ね返りによるエネルギーの散逸なし)
            \item 重力の影響はないものとする
            \item 1辺が$L$[m]の立方体の容器に入った気体分子を考える(体積は$V={L}^{3}$[m${}^{3}$])
        \end{itemize}
    \end{itembox}

\section{分子の速度とその平均}
    まず,立方体のある1つの分子に着目する.
    この分子の速度を$\bv$とする.速度は3次元であるので,$\bv$は$x$,$y$,$z$成分に分解できる.
    \begin{equation}
        \bv = ({v}_{x},\,{v}_{y},\,{v}_{z}).
    \end{equation}
    他のベクトル量も3次元であり,速度と同様に3つの成分を持つ.
    後で使うので,速度の2乗平均$\bar{{v}^{2}}$も計算しておこう.

    まず,速度ベクトルを2乗する.
    \[
        {\bv}^{2} = {v}^{2} = {{v}_{x}}^{2} + {{v}_{y}}^{2} + {{v}_{z}}^{2}
    \]
    すべての分子の平均をかんがえると,
    \begin{align*}
        \bar{{\bv}^{2}} &= \bar{{v}^{2}} \\
                        &= \bar{{{v}_{x}}^{2} + {{v}_{y}}^{2} + {{v}_{z}}^{2}} \\
                        &= \bar{{{v}_{x}}^{2}} + \bar{{{v}_{y}}^{2}} + \bar{{{v}_{z}}^{2}}
    \end{align*}
    さらに,あらゆる方向に均一に運動しているので,
    \[
        \bar{{{v}_{x}}^{2}} = \bar{{{v}_{y}}^{2}} = \bar{{{v}_{z}}^{2}}
    \]
    としてよい.一旦${v}_{x}$だけの式にする.
    \[
        \bar{{v}^{2}} = 3\bar{{{v}_{x}}^{2}}
    \]
    これより,
    \begin{equation}
        \bar{{{v}_{x}}^{2}} = \frac{1}{3} \bar{{v}^{2}}
    \end{equation}
    を得る.この速度の式は後の計算で使うことになる.

    \begin{memo}{$N$個の分子の速度の2乗平均}
        この部分の計算で,現れた速度の2乗平均の計算方法を確認しておこう.
        $N$個の分子に,1から$N$の自然数で番号付けをしておこう.そして,
        $N$個の各々の分子の速度を,${v}_{i}$としよう.添字の$i$は1から$N$までの
        いずれかの自然数である.自然数$i$をきめると,対応する分子が定まるこのとになる.
        平均とは全部(ここでは$N$個)のデータを合計して,その総数(ここでは$N$個)で割った
        値のことである.${v}_{1}$,${v}_{2}$,${v}_{3}$,$\cdots$ の値はそれぞれ異なる.
        個々の値には興味はなく,その平均が知りたい.ここで知りたいのは速度の大きさの
        平均である.速度は正負の向きを持ったベクトルなので,このまま足し合わせても
        速さの平均値を求めることはできない
            \footnote{
                極端な例を上げよう.速度には正負の向きがあるため,
                たまたま,すべてを合計したら0になるかのせいもある.
                平均の速度が0となる.ベクトルとしてはそれで正解であるが,
                速さの平均を求めたいのであるから,平均が0になっていしまうのは困る.
                各々の絶対値をとる方法も考えれるかもしれないが,二乗して必ず正の値に
                なるように細工したほうが都合が計算しやすいし,式も見やすいし良い.
                二乗しているので,計算の最終段階で平方根を求めれば良い.
            }.
        そこで,速度の2乗平均を計算することにしよう.実は,運動エネルギーも速度の
        2乗でであるため,この方法が都合が良い.

        すると,速度の2乗平均は以下のように計算される.
        \begin{align*}
            \bar{{\bv}^{2}} &= \frac{1}{N} \left( {{\bv}_{1}}^{2} + {{\bv}_{2}}^{2} + \cdots + {{\bv}_{N}}^{2} \right ) \\
                            &= \frac{1}{N} \sum_{i=1}^{N} {{\bv}_{i}}^{2}.
        \end{align*}

        速さが知りたかったら,平方根取ればいい.
        \[
            v = \sqrt{\bar{{\bv}^{2}}} = \sqrt{\frac{1}{N} \sum_{i=1}^{N} {{\bv}_{i}}^{2}}.
        \]
    \end{memo}

    \section{分子1つが壁に衝突するときに,壁に与える平均の力:$\bar{{f}_{x}}$}
    まずは$x$軸方向のみを考えよう.
    \begin{figure}[hbt]
        \begin{center}
            \includegraphicslarge{netsurikigaku_bunsi_undo_ron.pdf}
            \caption{気体分子運動論}
            \label{fig:netsurikigaku_bunsi_undo_ron}
        \end{center}
    \end{figure}

    最初に,「壁が分子から受ける力積」を計算する.この力積を${I}_{x}$とする.
    計算したいのだが,壁の速度に変化がないので,壁の運動から直接的に計算できない.
    そこで,分子と壁の作用反作用の法則を利用する.つまり,
        \[
            \mbox{分子の受ける力積} = - \mbox{壁の受ける力積}
        \]
    が成り立っているはずなので,分子の受ける力積を計算して,符号を反対にすればよい.
    力積は運動量の変化分に等しく,
        \[
            \mbox{力積} = \mbox{衝突後の運動量} - \mbox{衝突前の運動量}
        \]
    である.運動量$m{v}_{x}$で運動していた分子は壁に衝突すると,速度は反対向きになり,$m(-{v}_{x})$となる.
    また,分子が壁から受ける力積は$x$軸方向の反対なので,符号はマイナスである.
    これに当てはめると,分子が壁から受ける力積$-{I}_{x}$は
    \begin{align*}
        - {I}_{x}   &= - \bar{{f}_{x}}  t     \\
                    &= m(-{v}_{x}) - m{v}_{x} \\
                    &= -m{v}_{x} - m{v}_{x}   \\
                    &= -2m{v}_{x}
    \end{align*}
    である.$\bar{{f}_{x}}$は一回の衝突で与える$x$方向の力の大きさである.
    これより,壁が分子から受ける力積${I}_{x}$は,
    \begin{equation}
        {I}_{x} = \bar{{f}_{x}}  t = 2m{v}_{x}
    \end{equation}
    となる
        \footnote{
            ちなみに,「壁が分子から受ける力積」を言い換えれば,「分子が壁に与える力積」である.
            言葉の表現による注意も怠りなく.
        }.

    次に,「1つの分子が$t$秒間に壁に与える力積」を計算する.壁にあたった瞬間を$t=0$としたとき,
    分子は往復で$2L$の距離を速度${v}_{x}$で運動しているわけだから,次に衝突する時間を$t'$とすれば,
    \[
        2L = {v}_{x}t'
    \]
    が成立している.$t'$について解いて,
    \[
        t' = \frac{2L}{{v}_{x}}
    \]
    としておこう.分子が壁に1回衝突するのにかかる時間が$t'=2L/{v}_{x}$であることがわかった.
    であれば,$t$秒間に衝突する回数$n$は,$t/t'$を計算すればよく,
    \[
        n = \frac{t}{t'}=\frac{t}{2L/{v}_{x}}=\frac{{v}_{x}t}{2L}
    \]
    となる.
    \begin{figure}[hbt]
        \begin{center}
            \includegraphicslarge{netsurikigaku_bunsi_undo_ron2.pdf}
            \caption{壁への衝突回数}
            \label{fig:netsurikigaku_bunsi_undo_ron2}
        \end{center}
    \end{figure}

    分子が壁に与える1回あたりの力積${I}_{x}$は${I}_{x} = \bar{{f}_{x}}  t = 2m{v}_{x}$であった.これを使えば,
    $t$秒間に分子1つが壁に与える力積がわかる.以下のとおりである.
    \[
        2m{v}_{x} \times \frac{{v}_{x}t}{2L}=\frac{m{{v}_{x}}^{2}}{L}t=\bar{{f}_{x}}  t.
    \]

    したがって,最後の等式の対応見れば,壁に与える平均の力$\bar{{f}_{x}}$がわかる.
    \begin{equation}
        \bar{{f}_{x}} = \frac{m{{v}_{x}}^{2}}{L}.
    \end{equation}

    \begin{figure}[hbt]
        \begin{center}
            \includegraphicsdefault{netsurikigaku_bunsi_undo_ron_rikiseki.pdf}
            \caption{力積}
            \label{fig:netsurikigaku_bunsi_undo_ron_rikiseki}
        \end{center}
    \end{figure}

    \section{分子N個が壁に衝突するときに,壁に与える平均の力:$\bar{{F}_{x}}$}
    分子$N$個の衝突によって,壁が受ける平均の力が知りたい.これは$\bar{{f}_{x}}$を$N$倍することで
    計算できる.ただし,注意が必要である.いままでは1つの分子について着目していたため,${{v}_{x}}^{2}$は
    確かな固定値であった.しかし,今回は$N$個の分子を対象にしており,すべての分子が一律に${{v}_{x}}^{2}$という
    速度で運動しているはずがなく,このまま${{v}_{x}}^{2}$すれば,誤りとなる.ではどうするかと言うと,
    答えは簡単で,$N$個の分子の速度の平均値を考えれば良い.これを$\bar{{{v}_{x}}^{2}}$とかくことにしよう.
    \begin{equation}
        \bar{{F}_{x}} = \frac{Nm\bar{{{v}_{x}}^{2}}}{L}.
    \end{equation}


    \section{気体の圧力:$P$}
    圧力とは,単位面積あたりの力のことであった.今の場合,面積は${L}^{2}$である.
    $N$個の分子からなる気体が発生させる圧力$P$は
    \[
        P = \frac{\bar{{F}_{x}}}{{L}^{2}} = \frac{Nm\bar{{{v}_{x}}^{2}}}{L{L}^{2}} = \frac{Nm\bar{{{v}_{x}}^{2}}}{{L}^{3}}
    \]
    である.ここで,${L}^{3}$は立方体の体積であるから,$V$という文字で置き換えておこう($V:={L}^{3}$).
    \[
        P = \frac{Nm\bar{{{v}_{x}}^{2}}}{V}
    \]
    さらに,さっき計算した速度の式
    \[
        \bar{{{v}_{x}}^{2}} = \frac{1}{3} \bar{{v}^{2}}
    \]
    を適用すると,
    \[
        P = \frac{Nm \left((\frac{1}{3} \bar{{v}^{2}}\right)}{V} =\frac{Nm\bar{{v}^{2}}}{3V}.
    \]
    となる.

    \section{分子1つの並進運動エネルギーの平均値}
    理想気体の状態方程式 $PV=nRT$に関連付けたいので,両辺に$V$をかけよう.
    \[
        PV = nRT =  \frac{Nm \left(\frac{1}{3} \bar{{v}^{2}}\right)}{V} =\frac{Nm\bar{{v}^{2}}}{3}.
    \]
    分子1つの運動エネルギーは$m\bar{{v}^{2}}/2$であるから,この式から抽出してみよう.
    $m\bar{{v}^{2}}/$について解いてから,両辺に$1/2$をかけるとよいだろう.
    \begin{align*}
                                   nRT &= \frac{Nm\bar{{v}^{2}}}{3} \\
        \Leftrightarrow m\bar{{v}^{2}} &= \frac{3nRT}{N} \\
        \Leftrightarrow \frac{1}{2}m\bar{{v}^{2}} &= \frac{3}{2} \frac{n}{N}RT.
    \end{align*}
    ここで,$n/N$に注目しよう.$N$は気体分子全部の個数で,$n$は物質量([mol])である.
    つまり,$N$を$n$で割ると,1[mol]あたりの分子の個数が計算できて,これはアボガドロ定数${N}_{A}$であり,
    \[
        {N}_{A} = \frac{N}{n}
    \]
    であるから,$n/N$を$1/{N}_{A}$で置き換えられる.
    \begin{align*}
        \frac{1}{2}m\bar{{v}^{2}} &= \frac{3}{2} \frac{R}{{N}_{A}}T.
    \end{align*}
    後で詳しく学習するが,$R/{N}_{A}$はボルツマン定数${k}_{B}$であり,以下を得る.
    \begin{equation}
        \frac{1}{2}m\bar{{v}^{2}} = \frac{3}{2} {k}_{B}T.
    \end{equation}
    分子の運動エネルギーの平均値は,絶対温度$T$のみにより定まり,気体種類には依存しないことがわかる.

    \section{単原子分子からなる理想気体の内部エネルギー}
    N個の単原子分子からなる理想気体の内部エネルギー$U$は,内部の分子の運動エネルギーの平均を$N$倍すればいい.
    \begin{align*}
        U &= N \times \frac{1}{2}m\bar{{v}^{2}} \\
          &= N \times \frac{3}{2} \frac{R}{{N}_{A}}T    \\
          &= \frac{3}{2} \frac{N}{{N}_{A}}RT
    \end{align*}
    ここで,$n=N/{N}_{A}$だから,
    \begin{equation}
        U = \frac{3}{2} nRT = \frac{3}{2} PV.
    \end{equation}
    単原子分子の内部エネルギーは絶対温度で定まり,絶対温度に比例する.

    考えなければならず,この結果は多原子分子には適用できない.
    \begin{figure}[hbt]
        \begin{center}
            \includegraphicslarge{netsurikigaku_bunsi_undo_ron_tangensibunsi.pdf}
            \caption{多原子分子の場合は回転の考慮も必要}
            \label{fig:netsurikigaku_bunsi_undo_ron_tangensibunsi}
        \end{center}
    \end{figure}
