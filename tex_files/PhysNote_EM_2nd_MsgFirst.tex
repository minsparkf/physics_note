%===================================================================================================
%  Chapter : 電磁気学の再構築
%  説明    : これまでは,マクスウェル方程式を導くように電磁気学を学習してきた.しかし,現在の電磁気学
%            の体系はマクスウェル方程式を基礎にした記述である.そこで,今までの電磁気学のイメージを取
%            り払うよう,注意を呼びかける.
%===================================================================================================
%   %==========================================================================
%   %  Section
%   %==========================================================================
    \section{もう一度はじめから}
%   %==========================================================================
%   %  SubSection
%   %==========================================================================
        \subsection{帰納的な考え方}
        これまでは,マクスウェル方程式を導くように電磁気学を学習してきた.
        すでに知られている実験事実を基に,物理法則を導出するという考え方を,
        \textbf{帰納的} な考え方であるという.とにかく,今知られている事実
        から,どんなことが言えるかを,直感をもとにして探し出すのである.
        このような帰納的な考え方は,新しい概念をい知識として吸収するのに,
        有効である.

        例えば,小学校や中学校における,算数や数学の授業では,新しい概念
        を教わるとき,必ず具体的な例を見た後に,「これは一般的に成り立つ」
        というように教わるはずである.このような学習方法は確かに分かりやすい
        が,論理的であるとは言えない.何しろ直感をもとにして,説明をしている
        のだから,仕方がないのであるが,ものごとをきちんと整理して把握するの
        には,この方法は不適切である.そこで,次の段階として,帰納的に導いた
        事柄(ここではマクスウェル方程式)をもとにして,理論を再構成する作業
        が必要になってくる.
            \begin{figure}[hbt]
                \begin{center}
                    \includegraphicsdefault{KinouTekiSuiron.pdf}
                    \caption{帰納的推論}
                    \label{fig:KinouTekiSuiron}
                \end{center}
            \end{figure}



%   %==========================================================================
%   %  SubSection
%   %==========================================================================
        \subsection{演繹的な考え方}
        何かの理論を構築する際に,あるいくつかの約束事を,納得するか否かに関わ
        らずに,強制的に認めさせる.そして,この強制的に与えられた約束事を元に
        して,理論を構築する.もちろん論理的に作らないといけない.最初の約束事
        さえ認めれば,その後は,論理的推論で導き出せるように,理論を構成するの
        である.このような考え方を,\textbf{演繹的} な考え方であるという.

        演繹的な考え方は,理論を論理的に構築する際に,とても役に立つ.ただ,問
        題なのは,最初に与える約束事が認められない,あるいは疑わしさを感じ
        る場合である
            \footnote{
                集合論などの,選択公理などは,その最も有名なものではなかろうか.
            }.
        この場合,論理の最も基礎の部分に不安があるため,いくら論理的に理論を構
        築したところで,この理論の正当性も怪しまれてしまう.なので,その最も基
        礎となる約束事は,できるだけ確かな事実を採用すべきだ.それには,
        帰納的な考え方が適している.帰納的に導きだされた結果は,直感的になじみ
        やすいからである.
            \begin{figure}[hbt]
                \begin{center}
                    \includegraphicslarge{EnnekiTekiSuironn.pdf}
                    \caption{演繹的推論}
                    \label{fig:EnnekiTekiSuironn}
                \end{center}
            \end{figure}


%   %==========================================================================
%   %  SubSection
%   %==========================================================================
        \subsection{“帰納”から“演繹”へ}
        物事を論理的に構成しようとするとき,まず土台
            \footnote{
                土台:有無を言わさず,認めさせる,最初の約束事.
            }
        が必要である.その土台は,おおよそ間違いを含まず,正当性が高いものであ
        る必要がある.このような理論の土台を得るには,まず帰納的に考えるとよい.
        帰納的に考えるとは,実験をして,その結果を説明できるような法則を考える
        ということである.これには,多くの実験を行う必要があることだろう.何度も
        何度も,実験を行う.そうして,いろいろな実験結果が揃う.そして,それらを
        説明する理論も,同じくらい多くなっていることだろう.そうした段階で,少々
        立ち止まって,考えてみる.今度は,頭と鉛筆と紙で考えるのである.多くの
        実験で,多くの理論が創りだされた.しかし,この多くの理論には,いくつかの
        共通点が存在するはずであると考え,その共通点を探し出すのだ.いくら考えて
        も共通点がないかもしれないが,こればかりはやってみなければわからない.
        運良く,それらの共通点が明らかになった場合には,より簡潔な理論を作り出せ
        る可能性がある.そしたら,共通点を法則として捉え直し,演繹的な理論構築
        の土台にするのだ.たくさんの実験から得られた土台(法則)だから,それだけ
        に説得力があり,正当性の保証も大きい.

        帰納的な考え方から,演繹的な考え方に思考を切り替えるのには,それ相応の困
        難があることと思う.しかし,理論を組み上げる上で,この考え方の切り替えは
        非常に重要なことである.以下では,今までに帰納的に導いてきたマクスウェル
        方程式を土台とし,演繹的な理論構築の基礎に置き,電磁気学的な現象をとらえ
        なおしていくことにしよう.

%       %======================================================================
%       %  SubSection
%       %======================================================================
        \subsection{演繹的な電磁気理論の組み立て方}
        電磁気理論を再構築するに当たって,まず,その構成の手順を示しておくと,
        後の話の流れが把握しやすいだろう.

        構築の仕方は色いろあるだろうが,ここでは,太田浩一によって著された
        \cite{bib:refbook_em_5}の構成で学習を進めていくことにする.以下に,
        その概略を書いておこう.

        まず,力学との接点として,クーロンの法則を採用する.そして,特殊相対性
        理論の学習で導いたローレンツ変換
            \footnote{
                ローレンツ力ではないよ.
            }
        をクーロン力に対して,適用する.その適用結果を式変形すると,
        電磁場のローレンツ力と同じ形の力が導出される.これは,もちろん,
        ローレンツ力とみなすことができる.このローレンツ力をもとに,
        電場と磁束密度を定義する.最後に,マクスウェル方程式を,電場と磁束密度
        が満たす条件として,付け加える.
