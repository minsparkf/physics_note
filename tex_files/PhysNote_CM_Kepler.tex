%   %==========================================================================
%   %  Section
%   %==========================================================================
    \section{はじめに}
        ニュートンはデカルトやガリレイらの先人の考えに影響を受けながら,
        力学を構築した.この先人の中にはケプラー
            \footnote{
                Johannes Kepler ( 1571 - 1630,ドイツ )
            }
        の名前もあり,万有引力の法則は,
        ケプラーの法則を説明するために考え出されている.
        と言っても,万有引力の法則は,惑星だけの性質ではなく,
        世の中に存在する全ての物体が持つべき性質であると,理解が拡張されている.
        ニュートンは,惑星の運動と地上の物体の運動は同じ運動法則に従っている,
        と主張するのである.

        惑星の軌道に関する情報は,ケプラーよりも前の時代に,ティコ$\cdot$ブラーエ
            \footnote{
                Tycho Brahe ( 1546 - 1601,デンマーク )
            }
        が膨大な観測データを残していた.ケプラーは,このデータに規則性を見出し,
        惑星の運動に関する3つの法則を見つけ出した.今日では \textbf{ケプラーの法則} と呼ばれる,
        さらにその後,ニュートンはケプラーの3つの法則を包括するような,力学体系を
        築いた.これが現在のニュートン力学と呼ばれるものである.地上の物体
        の運動と,惑星の運動は同じ法則に従っていることを示したのである.

        惑星の運動の記述は,このように,ニュートン力学の1つのクライマックスでもある.
        惑星の運動が分からずに,ニュートン力学を学んだとは言えない.

        この章では惑星の運動の記述を考えていこう.

%   %==========================================================================
%   %  Section
%   %==========================================================================
    \section{ケプラーの法則}
            ケプラーはティコ$\cdot$ブラーエの残した,膨大な惑星の観測データから,
            惑星の運動軌道は,次の3つの法則によって,説明できることを
            発見した.
            \begin{myshadebox}{ケプラーの法則(惑星の運動に関する3つの法則)}
                惑星の運動は,以下の3つの法則に従っている.
                \begin{description}
                    \item[第1法則(楕円軌道の法則)]\mbox{}\\
                      惑星は太陽を1つの焦点とする,楕円軌道を描く.
                    \item[第2法則(面積速度一定の法則)]\mbox{}\\
                      一定時間に, 太陽と惑星を結ぶ動径が掃く面積は一定である.
                    \item[第3法則(調和の法則)]\mbox{}\\
                      惑星の公転周期の2乗と, 惑星の太陽からの距離の3乗の比は,どの惑星でも一定である.
                \end{description}
            \end{myshadebox}

            まずこれらを数式で表現してみよう.

%       %======================================================================
%       %  SubSection
%       %======================================================================
        \subsection{楕円の方程式}
            楕円を数式で表そう.
            \begin{figure}[hbt]
                \begin{center}
                    \includegraphicsdefault{daen_kepler.pdf}
                    \caption{楕円と座標・方程式}
                    \label{fig:daen_kepler}
                \end{center}
            \end{figure}


%       %======================================================================
%       %  SubSection
%       %======================================================================
        \subsection{極座標での運動方程式}
            惑星の運動は角運動であるから,極座標を採用する.
            水平方向の角度を $\phi$,高さ方向の角度を $\theta$,原点からの
            直線距離を $r$ とする.点Pは ($r$,\,$\theta$,\,$\phi$) と表現される.

%       %======================================================================
%       %  SubSection
%       %======================================================================
        \subsection{第1法則;楕円軌道の法則}
        \begin{figure}[hbt]
            \begin{center}
                \includegraphicsdefault{wakusei_kepler_low.pdf}
                \caption{惑星の楕円軌道}
                \label{fig:wakusei_kepler_low}
            \end{center}
        \end{figure}

%       %======================================================================
%       %  SubSection
%       %======================================================================
        \subsection{第2法則;面積速度一定の法則}
        \begin{figure}[hbt]
            \begin{center}
                \includegraphicsdefault{wakusei_kepler_low2_menseki.pdf}
                \caption{面積速度一定}
                \label{fig:wakusei_kepler_low}
            \end{center}
        \end{figure}

%       %======================================================================
%       %  SubSection
%       %======================================================================
        \subsection{第3法則;調和の法則}
