        %======================================================================
        %  SubSection
        %======================================================================
        \subsection{ベクトルの転置}
            横ベクトルと縦ベクトルの関係を,ここに書いておこう.
            議論の最初に,横ベクトルで表現したベクトルを,縦ベクトル
            として表現したい場合が起こる
                \footnote{
                    特に,ベクトルの内積を行列的に記述したい場合に,
                    このような要求をしたくなる.
                }

            そんな時,活躍するのが,ベクトルの \textbf{転置} という
            考え方である.

            $n$ 次元ベクトル $\bx$ を最初に導入するとき,
            横ベクトルとして定義したとする.そして,議論の途中,
            $\bx$ を縦ベクトルとして記述したくなったとしよう.
            すなわち,以下のような書き換えたいのである.
            \begin{align*}
                \left[\,x_{1}\,,\,\,x_{2}\,,\,\,\cdots\,,\,\,x_{n}\,\right]
                \quad \rightarrow \quad
                \left[
                    \begin{array}{c}
                        x_{1} \\
                        x_{2} \\
                        \vdots \\
                        x_{n} \\
                    \end{array}
                \right]
            \end{align*}
            要するに,横ベクトルの成分表示で,その成分を左から順に書いていたベクトルを,
            縦に上から順に成分を記述する方式に変更したいのだ.
            この操作を,ベクトルの \textbf{転置} という.

            この場合,以下のように記述する.
            \begin{align}
                {}^{t}\left[\,x_{1}\,,\,\,x_{2}\,,\,\,\cdots\,,\,\,x_{n}\,\right]
                =
                \left[
                    \begin{array}{c}
                        x_{1} \\
                        x_{2} \\
                        \vdots \\
                        x_{n} \\
                    \end{array}
                \right].
            \end{align}

            左辺の左上の添字 ${}^{t}$ で,ベクトルの転置を表現する.

            最初に定義されたベクトルが縦ベクトルであっても,
            ベクトルの転置を行うことで,横ベクトルにできる.
            つまり,次のようにも記述して良い.
            \begin{align}
                    \begin{array}{c}
                        {}^{t}
                        \\
                        \\
                        \\
                        \\
                    \end{array}
                \left[
                    \begin{array}{c}
                        x_{1} \\
                        x_{2} \\
                        \vdots \\
                        x_{n} \\
                    \end{array}
                \right]
                =
                \left[\,x_{1}\,,\,\,x_{2}\,,\,\,\cdots\,,\,\,x_{n}\,\right]
            \end{align}

            以上から,ベクトルの転置について,次が成立している.
            \begin{align}
               {}^{t} ({}^{t}\bx) = \bx.
            \end{align}

            ベクトル $\bx$ に2回転置をすれば,元のベクトルに戻るのである.
            以下のような感じで,巡回が起こる.
            \begin{align*}
                \mbox{縦ベクトル}
                    \xrightarrow[\mbox{\small{転置}}]{} \mbox{横ベクトル}
                    \xrightarrow[\mbox{\small{転置}}]{} \mbox{縦ベクトル} \\
                \mbox{横ベクトル}
                    \xrightarrow[\mbox{\small{転置}}]{} \mbox{縦ベクトル}
                    \xrightarrow[\mbox{\small{転置}}]{} \mbox{横ベクトル}
            \end{align*}

            もちろん,横ベクトルで表現しようとも,縦ベクトルで表現
            しようとも,同一のベクトルであれば,それが示すベクトルは同一
            である.同じベクトルを表現する方法が2種類あるということであり,
            ベクトルが2つになるわけではない.$\bx$ も ${}^{t}\bx$ も同じ
            ベクトルを表現するものであり,違うのは表現の方法なのだ.

        %==================================================================
        %  SubSection
        %==================================================================
            \subsection{ベクトルの四則演算}
            \begin{mycomment}
                ここでは,ベクトルに対して四則演算を定義する.
            \end{mycomment}

            %==============================================================
            %  SubSection
            %==============================================================
            \subsubsection{ベクトルの加法}
            3次元の2つのベクトルを,任意にもってきて,それらを
            \begin{equation*}
                \bx =
                \left[
                    \begin{array}{c}
                        x_{1} \\
                        x_{2} \\
                        x_{3} \\
                   \end{array}
                \right]\quad , \qquad
                \by =
                \left[
                    \begin{array}{c}
                        y_{1} \\
                        y_{2} \\
                        y_{3} \\
                   \end{array}
                \right]
            \end{equation*}
            と書くことにしよう.
            これら2つのベクトル $\bx$,$\by$ の和 $\bx + \by$ は,成分表示で
            以下のように示される.
                    \begin{align}
                        \bx + \by
                        =
                        \left[
                            \begin{array}{c}
                                x_{1} \\
                                x_{2} \\
                                x_{3} \\
                            \end{array}
                        \right]
                        +
                        \left[
                            \begin{array}{c}
                                y_{1} \\
                                y_{2} \\
                                y_{3} \\
                            \end{array}
                        \right]
                        =
                        \left[
                            \begin{array}{c}
                                x_{1} + y_{1} \\
                                x_{2} + y_{2} \\
                                x_{3} + y_{3} \\
                            \end{array}
                        \right].
                    \end{align}

                    つまり,成分同士を足し合わせるということである.

                    $n$ 次元ベクトルに対して,拡張しておこう.
                    任意の 2つの $n$ 次元ベクトル
                    \begin{equation*}
                        \bx =
                        \left[
                            \begin{array}{c}
                                x_{1} \\
                                x_{2} \\
                                \vdots \\
                                x_{n} \\
                           \end{array}
                        \right]\quad , \qquad
                        \by =
                        \left[
                            \begin{array}{c}
                                y_{1} \\
                                y_{2} \\
                                \vdots \\
                                y_{n} \\
                           \end{array}
                        \right]
                    \end{equation*}
                    に対して,これらの和は,成分で書くと次のようになる.
                    \begin{align}
                        \bx + \by
                        =
                        \left[
                            \begin{array}{c}
                                x_{1} \\
                                x_{2} \\
                                \vdots \\
                                x_{n} \\
                            \end{array}
                        \right]
                        +
                        \left[
                            \begin{array}{c}
                                y_{1} \\
                                y_{2} \\
                                \vdots \\
                                y_{n} \\
                            \end{array}
                        \right]
                        =
                        \left[
                            \begin{array}{c}
                                x_{1} + y_{1} \\
                                x_{2} + y_{2} \\
                                \vdots \\
                                x_{n} + y_{n} \\
                            \end{array}
                        \right].
                    \end{align}

                    また,次元の違うベクトル同士の和は定義されない.
                    つまり,次元が違うと,加法を行うことはできない.

            %==============================================================
            %  SubSection
            %==============================================================
            \subsubsection{実数とベクトルの積}
                ここでは,ひとつの任意の実数と,ひとつの任意のベクトルの
                積を定義する.ベクトル同士の掛け算も定義されるが
                    \footnote{
                        ベクトルの内積$\cdot$外積を参照.
                    },
                ここでは,実数とベクトルの掛け算のみについて考える.

                任意の実数 $a$ と任意の $n$ 次元ベクトル $\bx$ をもってきて,
                積を作る.その積を,
                    \begin{equation*}
                        a\bx
                        =
                        a\left[
                        \begin{array}{c}
                            x_{1}  \\
                            x_{2}  \\
                            \vdots \\
                            x_{n}  \\
                        \end{array}
                    \right]
                    \end{equation*}
                と書くことにする.成分で書くと,次のようになる
                    \footnote{
                        というか,こうなるように,実数とベクトルの積を
                        定義する.
                    }.
                \begin{align}
                    a\bx
                    =
                    a\left[
                        \begin{array}{c}
                            x_{1}  \\
                            x_{2}  \\
                            \vdots \\
                            x_{n}  \\
                        \end{array}
                    \right]
                    =
                    \left[
                        \begin{array}{c}
                            ax_{1}  \\
                            ax_{2}  \\
                            \vdots \\
                            ax_{n}  \\
                        \end{array}
                    \right].
                \end{align}

                これは $n$ 次元ベクトルに対して成り立つものである.
                つまり,実数とベクトルの積は,ベクトルの各成分を
                実数 $a$ 倍するということである.

            %==============================================================
            %  SubSection
            %==============================================================
            \subsubsection{ベクトルの減法}
                実数の世界では,事実上,減法が存在するが,しかしこれは公理的
                視点からは加法の一部である.つまり,2つの実数 $x$,$y$ があった
                時,減法 $x-y$ は 加法 $x+(-1 \cdot y)$ で定義されるのである.
                負の記号 $-1$ は公理により,その存在が示されているからである.
                減法を新たに定義するよりも,$-1$ の存在を主張するほうが,理論が
                単純になるり,思考経済
                    \footnote{
                        \textbf{思考経済}\quad
                        同じことを説明するのに,2つ以上の手段があるとしよう.
                        思考経済とは,この2つの説明のうちどちらを採用するかという
                        基準である.「簡潔に説明できる方を採用せよ」というものだ.
                        もちろん,最も簡潔に説明するほうがいいに決まっている(直感だけど)
                        うだうだ説明するりも,スパッと説明したほうが
                        かっこいいし,覚えることも少なくすむし,何しろ短時間で
                        説明できるのだから.しかし,世の中にはどちらも同じだけ
                        簡潔に説明できることが多くある.その場合には,時と場合に
                        よって使い分ける必要が出てくることだろう.
                    }
                に合致する.

                ベクトルに関する減法も,実数と同様に,加法の一部として,説明される
                べきものである.負の向きをもつベクトルとは,当然,正の向きに対して
                逆向きのベクトルである.負の向きのベクトルは,ベクトルに $-1$ を掛
                けることで,得られる.つまり,任意の $n$ 次元ベクトル $\by$ に対して,
                これと同じ大きさで,逆向きのベクトルは,
                \begin{equation*}
                    -1 \cdot \by = -\by
                \end{equation*}
                と書かれる.実数の場合と同じように,$-1$ 倍のときは,$1$を省略して
                書くことにする.

                これを用いて,ベクトルの減法を定めよう.
                任意の 2つの $n$ 次元ベクトル $\bx$,$\by$ に対して,差
                $\bx - \by$ とは次のように,成分表示される.
                \begin{align}
                    \bx - \by &= \bx + (- \by) \notag \\
                    &=
                    \left[
                        \begin{array}{c}
                            x_{1}  \\
                            x_{2}  \\
                            \vdots \\
                            x_{n}  \\
                        \end{array}
                    \right]
                    +
                    (-1)\left[
                        \begin{array}{c}
                            y_{1}  \\
                            y_{2}  \\
                            \vdots \\
                            y_{n}  \\
                        \end{array}
                    \right] \notag \\
                    &=
                    \left[
                        \begin{array}{c}
                            x_{1}  \\
                            x_{2}  \\
                            \vdots \\
                            x_{n}  \\
                        \end{array}
                    \right]
                    +
                    \left[
                        \begin{array}{c}
                            -y_{1}  \\
                            -y_{2}  \\
                            \vdots \\
                            -y_{n}  \\
                        \end{array}
                    \right] \notag \\
                    &=
                    \left[
                        \begin{array}{c}
                            x_{1}  +(-y_{1}) \\
                            x_{2}  +(-y_{2}) \\
                            \vdots \\
                            x_{n}  +(-y_{n}) \\
                        \end{array}
                    \right] \notag \\
                    \therefore \quad
                    \bx - \by
                    &=
                    \left[
                        \begin{array}{c}
                            x_{1}  -y_{1} \\
                            x_{2}  -y_{2} \\
                            \vdots \\
                            x_{n}  -y_{n} \\
                        \end{array}
                    \right]
                \end{align}

