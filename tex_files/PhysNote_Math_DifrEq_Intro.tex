%   %==========================================================================
%   %  Section : 微分方程式とは
%   %==========================================================================
        %==================================================================
        %  SubSection
        %==================================================================
            \subsection{微分方程式の概要}
                微分方程式とは,たとえば以下のようなものだ.
                    \begin{align*}
                        \frac{\df f(x)}{\df x} = 3x + 5
                        \quad , \qquad \frac{\df f(x)}{\df x} = 5{x}^{2} + 3x + 10 \\
                        \frac{{df}^{2}\ f(x)}{{\df x}^{2}} - 4f(x) + 10 = 0
                        \quad , \qquad \frac{{\df}^{3} f(x)}{{\df x}^{3}} + 4\frac{\df f(x)}{\df \df} = 0. 
                    \end{align*}
                つまり,独立変数とそれをもつ関数,さらにその導関数を含む方程式を \textbf{微分方程式} という
                    \footnote{
                        例に対して,$x$が独立変数,$f(x)$がその関数,$\df f(x)/\df x$ が導関数だ.
                    }.
                この例のように,独立変数が1つの場合を \textbf{常微分方程式} という.独立変数が2つ以上ある場合は,
                \textbf{偏微分方程式} という.偏微分方程式の例は以下のようだ.
                    \[
                        \frac{\rd P(x,\,y)}{\rd y} = \frac{\rd Q(x,\,y)}{\rd x}
                        \quad , \qquad \frac{{\rd}^{2} u(x,\,y)}{\rd {x}^{2}} + \frac{{\rd}^{2} v(x,\,y)}{\rd {y}^{2}} = 0.
                    \]
                
                偏微分方程式は難しい.まず,常微分方程式について勉強していこう.

                微分方程式の解は関数である.逆に,微分方程式を満たす関数を導くことを微分方程式を解くという.
                微分方程式の解くには,微分積分学の基本定理を使う.積分(多くの場合,不定積分)すればいい.

            \subsection{積分定数}
                実は,微分方程式の解くのに積分の範囲が定まらない場合は,その解は一つではない.
                不定積分を計算することになるからである.微分方程式の解には積分定数$C$の不定性が伴う.
                最も簡単な例を考えよう,$y=f(x)$ として,
                    \[
                        \frac{\df y}{\df x} = 0
                    \]
                を解いてみよう.と言っても両辺を$x$で積分するだけだ.
                    \begin{align*}
                        \int \frac{\df y}{\df x} \df x &= \int 0 \df x. \\
                        \Leftrightarrow \quad
                        \int \df y &= C \\
                        \Leftrightarrow \quad
                        y &= C.
                    \end{align*}
                ここで,$C$ は積分定数(任意の実数)である.$y=f(x)$ だから,結局,
                    \[
                        f(x) = C.
                    \]
                これが解である.微分方程式$\df y/\df x = 0$ の解は定数であることがわかった.
                実際,定数を$x$で微分すると0であることは微分を勉強したときに確認したことである.

            \subsection{簡単な例(1)($\df y/\df x = x$)}
                次の例も見ておこう.$y=f(x)$である.
                \[
                    \frac{\df y}{\df x} = x.
                \]
                これも解き方は同じ.両辺を$x$で不定積分をすればいい.
                \begin{align*}
                    \int \frac{\df y}{\df x} \df x &= \int x \df x. \\
                    \Leftrightarrow \quad
                    \int \df y &= \frac{1}{2} {x}^{2} + C \\
                    \Leftrightarrow \quad
                    y &= \frac{1}{2} {x}^{2} + C.
                \end{align*}
                つまり,
                \[
                    f(x) = \frac{1}{2} {x}^{2} + C
                \]
                である.
                
                検算してみよう.この$f(x)$を微分すると,
                \[
                    \frac{\df f(x)}{\df x} = \frac{\df }{\df x}\left( \frac{1}{2} {x}^{2} + C \right)
                    = \frac{\df }{\df x}\left( \frac{1}{2} {x}^{2}\right) + \frac{\df }{\df x} C
                    = x + 0
                \]
                よって,
                \[
                    \frac{\df f(x)}{\df x} = \frac{\df y}{\df x} = x
                \]
                となり,もとの微分方程式が導かれるので,検算もOKだ.

            \subsection{簡単な例(2)(${\df}^{2} x/{\df t}^{2} = a$)}
                この節の最後に,もう一つだけ,微分方程式を解いておこう.$x=f(t)$で$a$は実数の任意定数とする.
                \[
                    \frac{{\df}^{2} x}{{\df t}^{2}} = a.
                \]
                これは2階の微分方程式という.ニュートンの運動方程式もこの形をしている.
                この微分方程式を解くには,不定積分を2回繰り返せばよい.

                式変形を見やすくしたいので,この微分方程式の表現を改める.この微分方程式は以下のような意味であった.
                \[
                    \frac{\df \displaystyle \left(\frac{\df x}{\df t}\right)}{\df t}= a.
                \]
                ここで,
                \[
                    v := \frac{\df x}{\df t}
                \]
                とおくと,
                \[
                    \frac{\df v}{\df t}= a
                \]
                となる.

                では,微分方程式を解こう.両辺を$t$で不定積分する.
                \begin{align*}
                    \int \frac{\df v}{\df t} \df t &= \int a \df t \\
                    \Leftrightarrow \quad v &= at + {v}_{0}
                \end{align*}
                ここで,${v}_{0}$は積分定数である.$v=\df x/\df t$であるから,さらに不定積分する.$v$を元に戻して,
                \[
                    \frac{\df x}{\df t}  = at + {v}_{0}.
                \]
                両辺を$t$で不定積分しよう.
                \begin{align*}
                    \int \frac{\df x}{\df t} \df x &= \int \left( at + {v}_{0} \right) \df t \\
                    \Leftrightarrow \quad x &= \frac{1}{2} {at}^{2} + {v}_{0}t + {x}_{0}.
                \end{align*}
                ここで,${x}_{0}$は積分定数である.これが求める解である.

            \begin{memo}{補足}
                先の例,つまり,$x=f(t)$としたときの
                \[
                    \frac{{\df}^{2} x}{{\df t}^{2}} = a.
                \]
                の計算で,以下のような文字の置き換えを行った.
                \[
                    v := \frac{\df x}{\df t}
                \]
                ここでは,この置き換えをしない場合の式変形を記述しておく.
                先にも書いたように,2階微分を省略しない表現で書くと,以下のとおりである.
                \[
                    \frac{\df \displaystyle \left(\frac{\df x}{\df t}\right)}{\df t}= a.
                \]
                この式の両辺を$t$で不定積分する.
                \begin{align*}
                    \int \frac{\df \displaystyle \left(\frac{\df x}{\df t}\right)}{\df t} \df t &= \int a \df t. \\
                    \int \df \displaystyle \left (\frac{\df x}{\df t}\right) &= at + {C}_{0}.
                \end{align*}
                この式の左辺を見たときに,どう計算すればよいか,迷うかもしれない.先程の例だと,文字の置き換えを
                行うことでこれがカモフラージュされて,なんの違和感もなく自然に積分ができたと思う.
                どんなふうに計算していたかというと,$\df x/\df t$ を微分の式とみなさずに,一つの変数として計算していたのである.

                ちょっと脱線して,改めて
                \[
                    \int a \df t = 0.    
                \]
                という積分を計算してみよう.これは簡単で,
                \[
                    at = C
                \]
                と解けるはずだ.$C$は積分定数である.この$t$が$\df x/\df t$の場合も,形式的に同じように計算してよいのだ.すなわち,
                \[
                    \int \df \displaystyle \left( \frac{\df x}{\df t} \right)= 0
                \]
                という積分のような式は,$\df x/\df t$を一つの積分変数とみなして,
                \[
                    \frac{\df x}{\df t} = C
                \]
                とできる.あとは詰まることなく計算できるだろう.
                話をもとに戻して,
                \begin{align*}
                    \int \df \displaystyle \left (\frac{\df x}{\df t}\right) &= at + {C}_{0} \\
                    \Leftrightarrow \quad \frac{\df x}{\df t} &= at + {C}_{0} \\
                    \Leftrightarrow \quad \int \frac{\df x}{\df t} \df t &=  \int \left(at + {C}_{0} \right)\df t\\
                    \Leftrightarrow \quad \int \df x &= \frac{1}{2}a{t}^{2} + {C}_{0}t + C{1} \\
                \end{align*}
                となる.${C}_{0},\,{C}_{1}$は積分定数である.よって,
                \[
                    x = \frac{1}{2}a{t}^{2} + {C}_{0}t + C{1}
                \]
                である.
 



            \end{memo}


