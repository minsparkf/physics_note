%   %==========================================================================
%   %  Section : 特殊なベクトル
%   %==========================================================================
        %==================================================================
        %  SubSection
        %==================================================================
            \subsection{単位ベクトル}
                ベクトルは,大きさと向きをもつ.しかし,ベクトルを成分
                表示したときに,その大きさと向きが同時に記述されていて,
                成分表示を見ただけでは,簡単にはその大きさや向きを把握
                することはできない
                    \footnote{
                        計算(暗算)が非常に得意な人や,ベクトルのスペシャリスト
                        でない限り,ベクトルの成分表示を見た瞬間に,その大きさと
                        向きが想像できることはないと思う.少なくとも,私はそのような
                        マネはできない.
                    }.
                そこで,登場するのが,\textbf{単位ベクトル} という概念である.

                自然数でいえば,数字の1がその単位元であり,2以上のすべての自然数は,
                この単位元1の倍数として記述できる.
                    \begin{equation*}
                        2 = 1 \times 2\,,\;3 = 1 \times 3\,,\;\cdots\,.
                    \end{equation*}
                ベクトルの世界では,単位元は1という単なる数値ではなく,
                大きさが1で方向
                    \footnote{
                        「向き」と書いていないことに注意.
                        ここで言っているのは,正の向きと負の向きを同時に
                        考えたいので,「方向」と書いた.
                    }
                をもつ,単位ベクトルという概念が定義される
                    \footnote{
                        定義とは,後の議論の発展のためになされるのであり,
                        導かれるものではない.初学の数学に不慣れである時期には,
                        このことは特に忘れがちである.
                    }.
                    \begin{figure}[hbt]
                        \begin{center}
                            \includegraphicsdefault{TaniVector00.pdf}
                            \caption{単位ベクトル}
                            \label{fig:TaniVector00}
                        \end{center}
                    \end{figure}

                単位ベクトルがあれば,同じ向きを持つベクトルは,その単位ベクトル
                にその大きさをかけることで,ベクトルを表せるのである.
                単位ベクトルと1との違いは,方向をもつか,否かである
                    \footnote{
                        単位ベクトルには方向があり,1には方向がない.
                    }.

                言葉で「単位ベクトル」といったところで,数学的に扱う
                事はできない.なので,単位ベクトルを文字で表そう.

                まだ,単位ベクトルが定義可能か否かは分からないが,
                単位ベクトルなるものがあると仮定して,それを $\bn$ と書くことと
                する.この単位ベクトルを用いると,$\bn$ と同じ方向をもつベクトル
                ならば,$\bn$ の実数倍で表現できる.例えば,ベクトル $\ba$ の向きが
                単位ベクトル $\bn$ と同じ方向であったなら,
                    \begin{equation*}
                        \ba = | \ba | \bn
                    \end{equation*}
                と書ける.これは $\bn$ と同じ向きを持つ任意のベクトルに対して
                成り立つ
                    \footnote{
                        当然,$\bn$ と向きが異るベクトルはこのように記述はできない.
                        その場合には新たに,その方向をもつ別の単位ベクトルを作らないと
                        いけない.つまり,単位ベクトルはその方向によって無数に存在する.
                        これも,自然数の単位元1との違うところだ.
                    }.
                上式を $\bn$ について解くと,
                    \begin{align}
                        \bn = \frac{\ba}{|\ba|}
                    \end{align}
                なる.だから,単位ベクトルとは,ベクトルを自身の大きさで割ったものである
                といえよう.

                念のため,成分表示もしておこう.
                    \begin{align*}
                        \bn &= \frac{\ba}{|\ba|} \\
                            &= \left(
                                    \,\frac{a_{1}}{|\ba|}\,,\,
                                    \,\frac{a_{2}}{|\ba|}\,,\,
                                    \,\frac{a_{3}}{|\ba|}\,\,
                               \right) \\
                            &= \left(
                                    \,\frac{a_{1}}{\sqrt{{a_{1}}^{2}+{a_{2}}^{2}+{a_{3}}^{2}}}\,,\,
                                    \,\frac{a_{2}}{\sqrt{{a_{1}}^{2}+{a_{2}}^{2}+{a_{3}}^{2}}}\,,\,
                                    \,\frac{a_{3}}{\sqrt{{a_{1}}^{2}+{a_{2}}^{2}+{a_{3}}^{2}}}\,\,
                               \right).
                    \end{align*}

                これは大きさが1であり,$\ba$ と同じ方向のベクトルである.なぜなら,ベクトルの
                成分を全てを同じ実数で割ったベクトルと,元のベクトルの方向は同一であることは,
                ベクトルの加減乗除の定義によって簡単に示せる.また,大きさも次の計算から,
                簡単に1であることも確かめられる.
                    \begin{align*}
                        |\bn| &= \left(\frac{a_{1}}{\sqrt{{a_{1}}^2+{a_{2}}^2+{a_{3}}^2}}\right)^{2} +
                                 \left(\frac{a_{2}}{\sqrt{{a_{1}}^2+{a_{2}}^2+{a_{3}}^2}}\right)^{2} \\
                              &\quad + \left(\frac{a_{3}}{\sqrt{{a_{1}}^2+{a_{2}}^2+{a_{3}}^2}}\right)^{2} \\
                              &= \frac{{a_{1}}^{2}+{a_{2}}^{2}+{a_{3}}^{2}}{\left(\sqrt{{a_{1}}^2+{a_{2}}^2+{a_{3}}^2}\right)^{2}} \\
                              &= \frac{{a_{1}}^{2}+{a_{2}}^{2}+{a_{3}}^{2}}{{a_{1}}^2+{a_{2}}^2+{a_{3}}^{2}} \\
                              &= 1 \\
                        \therefore\quad
                        |\bn| &= 1.
                    \end{align*}

                \begin{memo}{具体例}
                    ベクトル $\ba=\left[\,1,\,2,\,4\,\right]$ を,大きさと
                    単位ベクトルに分解してみよう.

                    まず大きさは,定義に従って
                        \begin{equation*}
                            |\ba| = \sqrt{1^{2}+2^{2}+4^{2}} = \sqrt{1+4+16} = \sqrt{21}
                        \end{equation*}
                    である.よって,求めたい単位ベクトルを $\bn$ と書くことにすれば,
                        \begin{equation*}
                            \bn = \frac{\ba}{|\ba|} = \frac{1}{\sqrt{21}}\left[\,1,\,2,\,4\,\right]
                                = \left[
                                    \,\frac{1}{\sqrt{21}},\,\frac{2}{\sqrt{21}},\frac{4}{\sqrt{21}}\,
                                  \right].
                        \end{equation*}

                    以上から,ベクトル $\ba$ は大きさと単位ベクトルに分解すると,
                        \begin{equation*}
                            \ba = |\ba|\bn = \sqrt{21}
                                  \left[
                                    \,\frac{1}{\sqrt{21}},\,\frac{2}{\sqrt{21}},\frac{4}{\sqrt{21}}\,
                                  \right]
                        \end{equation*}
                    と表現される.
                \end{memo}

        %==================================================================
        %  SubSection
        %==================================================================
            \subsection{基底ベクトル}
                先に単位ベクトルという概念を説明した.そして,ここでは
                単位ベクトルを使って,さらに新し考え方を導入する.

                いきなりだけど,具体例から考える.
                ひとつの任意のベクトル $\ba$ を
                もってきて,$\ba$ が次のようなベクトルだったとしよう
                    \footnote{
                        まずは,馴染み深い3次元ベクトルで考える.
                    }.
                    \begin{equation*}
                        \ba
                        =
                        \left[
                            \begin{array}{c}
                                1 \\
                                3 \\
                                5 \\
                            \end{array}
                        \right].
                    \end{equation*}
                この式は次のように見ることもできる.
                    \begin{equation*}
                        \ba
                        =
                        1\left[
                            \begin{array}{c}
                                1 \\
                                0 \\
                                0 \\
                            \end{array}
                        \right]
                        +
                        3\left[
                            \begin{array}{c}
                                0 \\
                                1 \\
                                0 \\
                            \end{array}
                        \right]
                        +
                        5\left[
                            \begin{array}{c}
                                0 \\
                                0 \\
                                1 \\
                            \end{array}
                        \right].
                    \end{equation*}
                3つの方向に対し,ひつの方向に1の大きさをもち,かつ,
                他の方向には0であるような単位ベクトルを作り,
                それに係数をかけて,足し合わせるのである.

                任意のベクトルに対して,この表し方が可能であることは,
                明白である.任意の $n$ 次元ベクトル $\bx$ が
                    \begin{equation*}
                        \bx
                        =
                        \left[
                            \begin{array}{c}
                                x_{1} \\
                                x_{2} \\
                                \vdots \\
                                x_{n} \\
                            \end{array}
                        \right].
                    \end{equation*}
                と成分表示されるとしよう.このとき,$\bx$ は
                次のようにも表現することが可能である.
                    \begin{equation*}
                        \bx
                        =
                        x_{1}\left[
                            \begin{array}{c}
                                1 \\
                                0 \\
                                0 \\
                                0 \\
                                \vdots \\
                                \vdots \\
                                0 \\
                            \end{array}
                        \right]
                        +
                        x_{2}\left[
                            \begin{array}{c}
                                0 \\
                                1 \\
                                0 \\
                                0 \\
                                \vdots \\
                                \vdots \\
                                0 \\
                            \end{array}
                        \right]
                        +
                        \cdots
                        +
                        x_{i}\left[
                            \begin{array}{c}
                                0 \\
                                \vdots \\
                                0 \\
                                1 \\
                                0 \\
                                \vdots \\
                                0 \\
                            \end{array}
                        \right]
                        +
                        \cdots
                        +
                        x_{n}\left[
                            \begin{array}{c}
                                0 \\
                                0 \\
                                0 \\
                                \vdots \\
                                \vdots \\
                                0 \\
                                1 \\
                            \end{array}
                        \right]
                        .
                    \end{equation*}

                    このような書き方だと,記述が少々面倒なので,
                    次のような記号を導入し,表現を簡略化させよう.
                    \begin{align}
                        \be_{1}
                        =
                        \left[
                            \begin{array}{c}
                                1 \\
                                0 \\
                                0 \\
                                0 \\
                                \vdots \\
                                \vdots \\
                                0 \\
                            \end{array}
                        \right]
                        \; , \;
                        \be_{2}
                        =
                        \left[
                            \begin{array}{c}
                                0 \\
                                1 \\
                                0 \\
                                0 \\
                                \vdots \\
                                \vdots \\
                                0 \\
                            \end{array}
                        \right]
                        \; , \cdots , \;
                        \be_{i}
                        =
                        \left[
                            \begin{array}{c}
                                0 \\
                                \vdots \\
                                0 \\
                                1 \\
                                0 \\
                                \vdots \\
                                0 \\
                            \end{array}
                        \right]
                        \; , \cdots , \;
                        \be_{n}
                        =
                        \left[
                            \begin{array}{c}
                                0 \\
                                0 \\
                                0 \\
                                \vdots \\
                                \vdots \\
                                0 \\
                                1 \\
                            \end{array}
                        \right].
                    \end{align}

                    このように,一成分が1であり,他の成分がすべて0であるようなベクトルを,
                    \textbf{基底ベクトル} という.もちろん,基底ベクトルとは,
                    任意のベクトルを表現する場合に用いるベクトルである.しかし,たまたま任意の
                    ベクトルが基底ベクトルの成分と一致してしまうこともある.この場合のベクトルは
                    基底ベクトルとは認識されない.

                    基底ベクトルという概念を使うと,$\bx$ は以下のように書き直せる.
                        \begin{align}
                            \bx = x_{1}\be_{1} + x_{2}\be_{2} + \cdots + x_{i}\be_{i} + \cdots + x_{n}\be_{n}.
                        \end{align}

                    和の記号 $\sum_{i=1}^{n}$ を使うともっと簡単になる.
                        \begin{align}
                            \bx = \sum_{i=1}^{n} x_{i}\be_{i}.
                        \end{align}

                    \begin{memo}{注意}
                        基底ベクトルを先に「一成分が1であり,他の成分がすべて0であるようなベクトル」と
                        定義してしまった.しかし,注意してほしい.より細かく言うと,“直線直交座標系”における
                        基底ベクトルと表現すべきであった.

                        基底ベクトルは,座標系によってことなる.座標系には,直線直交座標意外にも,
                        極座標系や円筒座標系
                            \footnote{
                                後で説明する.中学から使ってきた,$x$--$y$ 座標のことを,直線直交座標というが,
                                これ以外にも,上に書いたように,極座標系や円筒座標系など,様々な座標系を設定
                                できる.ここでは,座標系は直線直交座標系だけではないということを
                                分かってもらえれば,それでいい.
                            }
                        があり,各座標系の基底ベクトルは全く異なるベクトルである
                            \footnote{
                                ただし,ある基底ベクトルが存在するとき,この基底ベクトルを別の基底ベクトル
                                に変換する方法は存在する.
                            }.
                    \end{memo}
