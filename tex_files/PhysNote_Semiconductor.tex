%===================================================================================================
%  Chapter : 半導体
%  説明    : 半導体の基本的な性質を説明する.
%===================================================================================================
%   %==========================================================================
%   %  Section
%   %==========================================================================
    \section{半導体とは}
        この世界に存在する全ての物体は,電流を流すことができる.
        しかし,この電流の生じやすさは,物体ごとに異なり,電流
        が流れやすい物体と,流れにくい物体が存在する.慣習的に,
        電流の生じやすい物体を \textbf{導体} といい,電流が流れ
        にくい物体を \textbf{不導体},または \textbf{絶縁体} と
        いう.では,全ての物体が,導体か絶縁体のどちらか一方に
        分類されるかといえば,実はそうとは言い切れない.“どっち
        つかず”の性質をもつ物体もあるのだ.導体にしては電流を流し
        にくく,かといって,絶縁体というほど電流を流しにくいわけ
        でもない物体が存在するのである.このように,導体と絶縁体
        の中間のような性質をもつ物体のことを,\textbf{半導体} と
        いう.どうにも曖昧な説明だが,イメージは浮かぶだろう.
        以降では,半導体のもつ性質について考えることで,半導体を
        イメージできるよう,学習をしよう.

%   %==========================================================================
%   %  Section
%   %==========================================================================
    \section{半導体の種類}
        一口に半導体といっても,いろんな種類がある.図\ref{fig:handoutai_bunrui} に
        これを示す.
                        \begin{figure}[htbp]
                            \begin{center}
                                \includegraphicslarge{handoutai_bunrui.pdf}
                                \caption{半導体の分類}
                                \label{fig:handoutai_bunrui}
                            \end{center}
                        \end{figure}

        図からわかるように,半導体の種類を大きく分けるなら,
        2種類に分けることができる.真性半導体といわれるものと,
        外因性半導体といわれるものである.\textbf{真性半導体} とは,
        一種類の元素からなる半導体である.例えば,Si(シリコン
        ,または,珪素)や,
        Ge(ゲルマニウム)がそうである.\textbf{$i$ 型半導体} といわれ
        ることもある.
        \textbf{外因性半導体} とは
            \footnote{
                外因性半導体:\textbf{不純物半導体} ともいわれる.
            },
        複数の元素により構成される半導体である.
        有名な \textbf{$n$ 型半導体},\textbf{$p$ 型半導体} は,
        外因性半導体の一種である.

%   %==========================================================================
%   %  Section
%   %==========================================================================
    \section{キャリア(電荷担体)}
        電気伝導の発生源は,電荷の移動である.ただし,多くの場合,物質を構成する
        原子のもつ全ての電子が,電気伝導を生じさせている電荷であるというではない.
        物質を構成する原子の,周囲の電子の一部が,移動して電流が生じる.この移動
        する一部の電子のことを,\textbf{キャリア(電荷担体)} という.ところで,
        電荷には正と負の2種類あることから,キャリアにはもう一種類あることが容易に
        推察されよう.このもうひとつの正のキャリアのことを,\textbf{正孔(hole)} と
        いう.

        下に,キャリアである電子と正孔について,簡単にイメージしておこう.

%       %======================================================================
%       %  SubSection
%       %======================================================================
        \subsection{電子(electron)}
            言わずと知れた,小学生でも知っている,電子である.ただし,キャリアとし
            ての電子とは,上にも書いたように,原子を構成するすべの電子というわけで
            はない.主に,原子を構成する最外殻にある,価電子がキャリアになる場合が
            多い.

%       %======================================================================
%       %  SubSection
%       %======================================================================
        \subsection{正孔(hole)}

%       %======================================================================
%       %  SubSection
%       %======================================================================
        \subsection{電子の移動速度 と 正孔の移動速度}

%   %==========================================================================
%   %  Section
%   %==========================================================================
    \section{真性半導体の電導機構}
%       %======================================================================
%       %  SubSection
%       %======================================================================
        \subsection{Si原子の古典的イメージ}
            半導体は導体と絶縁体の中間のような性質をもつ.ここでは,半導体が導体
            として機能するときに,その電気伝導はどのように起こっているかを考える.
            それにはまず,シリコン(以降,元素記号の Si を用いて表現する)原子の構
            造を知っていることが必要である.Si原子の構造を,古典的なイメージで描
            いたのが,図\ref{fig:SiGenshi}である.
                            \begin{figure}[htbp]
                                \begin{center}
                                    \includegraphicslarge{SiGenshi.pdf}
                                    \caption{Si原子の古典的イメージ}
                                    \label{fig:SiGenshi}
                                \end{center}
                            \end{figure}
            Si原子の周囲には,14個の電子が存在している.その構成は,図のように,内
            側に2個,さらにその外側に8個,そして外周に4個である.この電子配置は,量
            子力学的に説明されるのであるが,半導体の電導機構にはあまり関係がない.
            大事なのは,一番外側に存在している4つの電子である.化学的に言えば,価電
            子と呼ばれる電子のことである.この4つの電子が結合の手になり,Si原子の共
            有結合を実現している.Siの電導機構は,この結合の手である4つの価電子の移
            動によって,説明される.以降では,最外殻電子(価電子)の個数のみが,導電
            機構に関与することから,価電子以外の電子は無視する.

%       %======================================================================
%       %  SubSection
%       %======================================================================
        \subsection{電導機構の説明}
            実際の物質は,原子ひとつではなく,複数個の原子が集まってできている.Si原子も
            同様で,多くのSi原子が共有結合している.それを模式的に表したのが,図\ref{fig:SiDendou}
            である.ただし,この図には,左右に電位を与えている.右側に正の電位を与え,左
            側に負の電位を与えている.このようにSi原子に電位を与えると,電気伝導が生じる.
            つまり,両端に電圧を加えると,電流が生じるのである.
                            \begin{figure}[htbp]
                                \begin{center}
                                    \includegraphicslarge{SiDendou.pdf}
                                    \caption{Siの電導機構イメージ}
                                    \label{fig:SiDendou}
                                \end{center}
                            \end{figure}

            この電流の電導機構は次のように説明できる.まず,電位を加えると,Si原子間の共
            有結合を担っている価電子の一部が,両側から与えられた電位によるエネルギーを受
            けとる.ネルギーを受け取った電子は,ある程度の運動量をもつようになり,Si原子
            を離れる.Si原子を離れた電子は,金属で言うところの自由電子と同じ振る舞いをす
            るようになる.この電子が,Siの電気伝導を担う電子である.ところで当然,Si原子
            から電子が離れてしまったので,そこには穴ができる.電子は負の値を持っているの
            で,開いた穴には電気的に正の性質がある.周囲の電子はこの穴を埋めるべく,移動
            してくる.そしてその穴がふさがる.そうするとまた,穴をふさいだ電子の元にいた
            場所が穴になる.さらにこの穴をふさぐべく,周囲の電子がやってくる.Siの電気伝
            導はこのように説明できる.

%   %==========================================================================
%   %  Section
%   %==========================================================================
    \section{外因性半導体の電導機構}
%       %======================================================================
%       %  SubSection
%       %======================================================================
        \subsection{外因性半導体の構成1(donor 注入)}
            外因性半導体とは,複数の元素から構成される半導体である.例えば,
            Si原子から構成される真性半導体の一部のSi原子を,価電子を5つもつ
            原子As(砒素)に置き換えると
                \footnote{
                    As(砒素)は周期表における第5族元素である.
                    第5族元素は,価電子を5つもっている.
                },
            これはn型半導体とよばれるものになる.
            このn型半導体は外因性半導体の一種である.
            図\ref{fig:SiAs}にそのイメージを示す.
                \begin{figure}[htbp]
                    \begin{center}
                        \includegraphicslarge{SiAs.pdf}
                        \caption{Si原子配列にAs(砒素)を少量埋め込む}
                        \label{fig:SiAs}
                    \end{center}
                \end{figure}

            元々,ひとつのSi原子のもつ価電子数は4つであった.そして,このSi原子
            の一部が5つの電子をもつAs原子に置き換わったのだから,単純に考えて,
                \begin{equation*}
                    (\mathrm{As}\mbox{原子の価電子数}) - (\mathrm{Si}\mbox{原子の価電子数}) = 5 -4 = 1
                \end{equation*}
            で,電子が1つ余ることになる.実は,この余った電子が外因性半導体
            の電気伝導を担う電荷なのである.

            Asの添加等により
            真性半導体中のに電子を増やすことで電気伝導性を
            高めた半導体のことを,\textbf{n型半導体} という
                \footnote{
                    nはnegative(英単語)の頭文字をとったものである.
                    英単語のnegativeには「否定的」という意味が
                    ある.電子のもつ電気量がマイナスであることから,
                    negativeが連想され,n型といわれる.
                }.

            「真性半導体にAsを注入する」ということは,「電子を注入する」
            という意味にとることができる
                \footnote{
                    実際に添加するのはAs元素なのだが,このAs添加
                    の目的は,実は,電子を注入することである.
                }.
            電子を注入するという意味で,真性半導体にAs元素を添加するとき,
            Asのこをドナー(donor)という.一般に,電子を注入するために
            添加される元素のことを \textbf{ドナー(donor)} という.

%       %======================================================================
%       %  SubSection
%       %======================================================================
        \subsection{電導機構の説明1(donor 注入)}

%       %======================================================================
%       %  SubSection
%       %======================================================================
        \subsection{外因性半導体の構成2(acceptor 注入)}

%       %======================================================================
%       %  SubSection
%       %======================================================================
        \subsection{電導機構の説明2(acceptor 注入)}

%   %==========================================================================
%   %  Section
%   %==========================================================================
    \section{ホール効果}
%       %======================================================================
%       %  SubSection
%       %======================================================================
        \subsection{ホール効果の概要}
            ホール効果(Hall effect) はHall
                \footnote{
                    Edwin Herber Hall(1855-1938, アメリカ);物理学者
                }
            が発見した電流と磁束密度に関連した現象のことである.
            1879年に発見されたといわれている.

            電流 $\bI$ が生じている導体に,この $\bI$ と直交するような
            方向に,磁束密度 $\bB$ をかける.このとき,電流 $\bI$ を担う
            キャリアは,$\bI$ と $\bB$ の両方に直交する方向にローレン
            ツ力を受ける.そうすると,導体内のキャリア密度に偏りが
            生じる
                \footnote{
                    キャリアがローレンツ力に引っ張られて,
                    導体の端のほうへ移動してしまう.
                    そうすると,導体内部にキャリアの多い部分と,
                    少ない部分が生じる.ただし,導体内キャリア
                    の,正味の個数に変化はない.
                }.
            偏ったキャリア密度は,導体の表面に電位差を発生させる.
            図\ref{fig:HallEfect1}は現象の全体像を表したものである.
            この現象を \textbf{ホール効果} という.また,ホール効果
            により生じる導体の両端の電位差を \textbf{ホール電圧} という.

                        \begin{figure}[htbp]
                            \begin{center}
                                \includegraphicslarge{HallEfect1.pdf}
                                \caption{ホール効果(イメージ)}
                                \label{fig:HallEfect1}
                            \end{center}
                        \end{figure}

%       %======================================================================
%       %  SubSection
%       %======================================================================
        \subsection{ホール効果発生の機構}
            先に,ホール効果発生のおおよその道筋を追ったが,
            ここでは,もう少し詳細に,ホール効果の発生について
            考えていこう.

            ホール効果の発生とは,ホール電圧の発生と同じことである.
            ホール電圧の発生機構を順を追って調べていこう.

            まず,発生機構を箇条書きにすれば,以下の通りになる.

                \begin{enumerate}
                    \item 導体に電流 $\bI$ が生じている
                    \item 電流 $\bI$ が生じている導体の近くに,磁束密度 $\bB$ を発生させる
                    \item 電流 $\bI$ を担うキャリアは,磁束密度 $\bB$ よりローレンツ力をうける
                    \item キャリアはローレンツ力を受け,導体内の一方向に偏る(キャリア密度に偏りが生じる)
                    \item キャリア密度の偏りにより,導体の端に電位差(ホール電圧)が生じる.
                \end{enumerate}

            また,各段階のイメージを図\ref{fig:HallEfec_Mech}に描く.

                    \begin{figure}[htbp]
                        \begin{tabular}{cc}
                            \begin{minipage}{0.5\hsize}
                                  \begin{center}
                                      \includegraphicsdouble{HallEfect2.pdf}

                                      (1)
                                  \end{center}
                            \end{minipage}
                            \begin{minipage}{0.5\hsize}
                                \begin{center}
                                    \includegraphicsdouble{HallEfect3.pdf}

                                      (2)
                                \end{center}
                            \end{minipage}
                        \end{tabular}
                    \end{figure}

                    \begin{figure}[htbp]
                        \begin{tabular}{cc}
                            \begin{minipage}{0.5\hsize}
                                  \begin{center}
                                      \includegraphicsdouble{HallEfect4.pdf}

                                      (3)
                                  \end{center}
                            \end{minipage}
                            \begin{minipage}{0.5\hsize}
                                \begin{center}
                                    \includegraphicsdouble{HallEfect5.pdf}

                                      (4)
                                \end{center}
                            \end{minipage}
                        \end{tabular}
                        \caption{ホール電圧の発生}
                        \label{fig:HallEfec_Mech} % ホール効果発生メカニズム
                    \end{figure}

%       %======================================================================
%       %  SubSection
%       %======================================================================
        \subsection{ホール効果におけるキャリアの運動}
             ホール効果における,キャリアの動きを考える.
             簡単のために,キャリア1個の運動を考える.
             ここでは,キャリアは電子(負の電荷をもつ粒子)とする.

             ホール効果は導体中に生じている電流 $\bI$ と,その
             周囲に生じている磁束密度 $\bB$ に起因する現象である.
             そこで,導体に電流 $\bI$ が生じていることと,
             その周囲に磁束密度 $\bB$ が生じていることの2点を
             仮定しよう.

%   %==========================================================================
%   %  Section
%   %==========================================================================
    \section{Schottkyダイオード}
%       %======================================================================
%       %  SubSection
%       %======================================================================
        \subsection{Schottky障壁ダイオード ダイオードの $I-V$ 特性}
        図\ref{fig:Schottky}は,通常のp-n接合とSchottky障壁ダイオードとの $I-V$ 特性の
        比較したものである.
                        \begin{figure}[htbp]
                            \begin{center}
                                \includegraphicslarge{Schottky.pdf}
                                \caption{Schottky障壁ダイオードの $I-V$ 特性}
                                \label{fig:Schottky}
                            \end{center}
                        \end{figure}

        Schottky接触ダイオードの $I-V$ 特性を考える.
        上図\ref{fig:Schottky}の $V>0$ の部分は,順バイアス
            \footnote{
                p側を正極,n側を負極に接続.
            }
        をかけたときのものである.
        このときの電流 $I$ と電圧 $V$ の関係は,
        金属から半導体へ流れる電流密度を $J_{\mbox{金属} \rightarrow \mbox{半導体}}$ として,
        \begin{align}\label{RD_1}
        J_{\mbox{金属} \rightarrow \mbox{半導体}}
        =\frac{4\pi qmk_{B}^{2}}{h^{3}}T^{2}
        \mathrm{exp}\left(-\frac{q\phi_{Bn}}{k_{B}T}\right)
        \mathrm{exp}\left(\frac{qV}{k_{B}T}\right)
        \end{align}
        が成り立つ.
        ここに,$q$ はキャリアの電気量,$\phi_{Bn}$はSchottky障壁,$m$ は有効質量,$k_{B}$ は
        ボルツマン定数,$h$ はプランク定数,$T$ は温度である.

        次に,逆バイアス
            \footnote{
                n側を正極,p側を負極に接続.
            }
        をかけたときを考える.理想的には電流は $0$ であるが,
        非常に微小な実際は電流が生じる.これは,p型半導体部分のminority carrierである
        電子がダイオードを通過するためである.図\ref{fig:Schottky}は分かりやすさのために,かなり誇張されて
        描かれている.

        図には描かれていないが,逆バイアスの電圧を徐々に大きくしていくと,
        p型半導体部分の電子が {\bf 電子なだれ} を引き起こす.この説明はタウンゼント
            \footnote{
                John Sealy Edward Townsend (1868-1957,アイルランド)
            }
        によってなされている.
        高電圧をかけるとp側の少数キャリアである電子 $e_{1}$ が大きな運動エネルギーをもち,
        半導体を構成する原子に衝突する.この衝突でその原子内部の電子 $e_{2}$ が
        飛び出す.この現象を {\bf 衝突電離} という.衝突を引き起こした $e_{1}$ は高電圧によって生ずる電場によって
        なおも運動し続ける.同様に $e_{2}$ も運動する.従って,今度は $e_{1}$ と $e_{2}$ が
        他の原子にぶつかっていって衝突電離を引き起こし,次第に大量の電子が半導体を流れていく.
        これが電子なだれと呼ばれる現象である.電子なだれが発生すると,それ以上
        大きい電圧をかけることができなくなる.

%       %======================================================================
%       %  SubSection
%       %======================================================================
        \subsection{p-n接合ダイオードのエネルギーバンド図}
        p型半導体とn型半導体を
        接触させると,
        熱平衡状態において各半導体内の電子のフェルミエネルギー
        が等しくなる.もしフェルミエネルギーが等しくならなければ
        半導体内に電流が生じるのであるが,これは外部からの
        仕事なしに電流が生じていることになり,
        すなわちエネルギー保存則に反するからである.
        従って,p-n 接合のエネルギーバンド図は
        図\ref{fig:semi_con_ketugou}のようになる.
        参考のために,接合前のn型半導体,p型半導体の
        それぞれのエネルギーバンド図を
        図\ref{fig:semi_con_bunri}に描いておく.

                    \begin{figure}[htbp]
                        \begin{tabular}{cc}
                            \begin{minipage}{0.5\hsize}
                                \begin{center}
                                    \includegraphicsdouble{semi_con_bunri.pdf}
                                    \caption{接合前}
                                    \label{fig:semi_con_bunri}
                                \end{center}
                            \end{minipage}
                            \begin{minipage}{0.5\hsize}
                                \begin{center}
                                    \includegraphicsdouble{semi_con_ketugou.pdf}
                                    \caption{平衡状態}
                                    \label{fig:semi_con_ketugou}
                                \end{center}
                            \end{minipage}
                        \end{tabular}
                    \end{figure}

        $E_{c}$ はConduction band (伝導帯) のエネルギー準位であり,
        $E_{c}$ はValence band (荷電子帯) のエネルギー隼位である.
        以降の図でもこれらの記号を用いる.


    \newpage

%       %======================================================================
%       %  SubSection
%       %======================================================================
        \subsection{Schottky障壁ダイオードのエネルギーバンド図}
        金属の仕事関数を $\phi_{m}$ とし,
        半導体の仕事関数を $\phi_{s}$ とする.
        n型半導体の場合において,
        $\phi_{m}>\phi_{s}$ のとき,金属と半導体の接触は
        Schottky接触となる.また,p型半導体の場合においてはその逆で,
        $\phi_{m}<\phi_{s}$ のときに,Schottky接触となる.

                        \begin{figure}[htbp]
                            \begin{tabular}{cc}
                                \begin{minipage}{0.5\hsize}
                                    \begin{center}
                                        \includegraphicsdouble{Shottky_p.pdf}
                                        \caption{平衡状態;p側}
                                        \label{fig:Shottky_p_bi}
                                    \end{center}
                                \end{minipage}
                                \begin{minipage}{0.5\hsize}
                                    \begin{center}
                                        \includegraphicsdouble{Shottky_p_bi.pdf}
                                        \caption{平衡状態;n側}
                                        \label{fig:Shottky_p_bi.}
                                    \end{center}
                                \end{minipage}
                            \end{tabular}
                        \end{figure}

%       %======================================================================
%       %  SubSection
%       %======================================================================
        \subsection{オーミック接触のダイオードのエネルギーバンド図}
        金属の仕事関数を $\phi_{m}$ とし,
        半導体の仕事関数を $\phi_{s}$ とする.
        n型半導体の場合において,
        $\phi_{m}>\phi_{s}$ でないとき,金属と半導体の接触は
        オーミック接触となる.また,p型半導体の場合においてはその逆で,
        $\phi_{m}<\phi_{s}$ でないときに,オーミック接触となる.

                \begin{figure}[htbp]
                    \begin{tabular}{cc}
                        \begin{minipage}{0.5\hsize}
                    \begin{center}
                        \includegraphicsdouble{ohmic.pdf}
                        \caption{平衡状態;n側}
                        \label{fig:ohmic}
                    \end{center}
                        \end{minipage}
                        \begin{minipage}{0.5\hsize}
                    \begin{center}
                        \includegraphicsdouble{ohmic2.pdf}
                        \caption{平衡状態;p側}
                        \label{fig:ohmic2}
                    \end{center}
                        \end{minipage}
                    \end{tabular}
                \end{figure}

%   %==========================================================================
%   %  Section
%   %==========================================================================
    \section{電子親和力}
    1つの原子に電子を受け入れたときに
    放出されるエネルギーのことをいう.


        \begin{table}[htbp]
            \caption{金属の電子親和力と仕事関数} \label{densin_sgotokansu}
            \begin{center}
                \begin{tabular}{|c|r|l|} \hline
                    金属 & \multicolumn{1}{c|}{電子親和力[eV]} & \multicolumn{1}{c|}{仕事関数} \\ \hline
                    Mg & 0\phantom{.}\phantom{0000} &  \\ \hline
                    Al & 0.43\phantom{00} &  \\ \hline
                    NI & 1.156\phantom{0} &  \\ \hline
                    Cu & 1.235\phantom{0} &  \\ \hline
                    Zn & 0\phantom{.}\phantom{0000} &  \\ \hline
                    Ga & 0.43\phantom{00} &  \\ \hline
                    Se & 2.0206 &  \\ \hline
                    Pd & 0.562\phantom{0} &  \\ \hline
                    Ag & 1.302\phantom{0} &  \\ \hline
                    In & 0.3\phantom{000} &  \\ \hline
                    Au & 2.3086 &  \\ \hline
                \end{tabular}
            \end{center}
        \end{table} %

    ※ 「『理科年表』,国立天文台編,丸善,2007」を参照しました.



%   %==========================================================================
%   %  Section
%   %==========================================================================
    \section{MOSFET}
%       %======================================================================
%       %  SubSection
%       %======================================================================
        \subsection{ピンチオフ効果}
        ピンチオフ効果について説明する.MOSFETは通常,図\ref{fig:MOSFET_1}に描いたような
        動作をする.この図の場合,ゲート(Gate)に正のバイアスをかけて,p型半導体に反転層
        を生成し,電子のチャネルを生成している.これにより,電子がソース側からドレイン側
        に流れることができる.
                    \begin{figure}[htbp]
                        \begin{center}
                            \includegraphicslarge{HotElectorn1.pdf}
                            \caption{通常のMOSFETの動作}
                            \label{fig:MOSFET_1}
                        \end{center}
                    \end{figure}
