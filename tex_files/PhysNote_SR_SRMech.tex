\begin{mycomment}
    以上で,主要な特殊相対論的現象を考えてきた.光速不変の原理を仮定すると,
    不思議で納得しかねる現象をが現れる.こうなると,物体の運動法則である,
    ニュートン力学も少しばかり変更する必要がある.ニュートン力学には光
    速不変の原理を採用し,特殊相対性理論と矛盾しないように,書き換えてい
    こう
        \footnote{
            ここから,だんだんと特殊相対性理論の核心に迫っていく.
            目標を,特殊相対性理論を共変形式(ここではそういう形式
            があると思ってもらえればそれでよい)
            で表すことにしよう.この章はそのための,前段階的準
            備を行い,次章で共変形式について考えていこう.
        }.
\end{mycomment}

 \section{目標}
    これからの目標を,簡単に記述しておこう.特殊相対性理論は,
    アインシュタインの最初の論文でほとんど完成している.これから特殊
    相対性理論を数学的にスマートになるように記述するが,これ
    を考え出したのはアインシュタインではなく,ミンコフスキー
        \footnote{
            Hermannn Minkowski(1864--1909,ロシア) 数学者.
            チューリッヒ工科大学(スイス連邦工科大学)の教授.
            アインシュタインはこの教授の講義を受けていたらしい.詳
            しくは,アインシュタインの伝記を読もう.
        }
    である.アインシュタインの説明では,時間と $t$ と位置座標 $\br$ は
    互いに密接な関係があることを示しながらも,その数学的表現は
    区別されていた.今日のいわゆる“時空”という考え方はあった
    ものの,数学的表現はなされていなかった.時空という概念を数
    学的に表現したのがミンコフスキーである.ミンコフスキーは時間 $t$ と
    位置座標 $\br$ を同じ次元として考えることで,特殊相対性理
    論が数学的に綺麗な形で表現できる事を示した.そのような経緯
    から,“時空”のことを,しばしば,ミンコフスキー空間ということ
    がある.このノートでは,ミンコフスキー空間という言葉を用いるこ
    とにする.

    すこし先走って,ミンコフスキー空間について記述しよう.ミンコフスキー
    空間とは,空間の3つの次元と時間の1次元を一括して考え,3$+$1
    次元として創られた新しい座標系である
        \footnote{
            3$+$1次元と表現したのは,実は,時間と空間の次元
            が全く同等に扱われているわけではないからである.
            とはいっても,この表現は煩わしいので4次元と表現
            することにしよう
        }.
    ただし,時間はそのままでは空間の次元と一致しないので,
    時間 $t$ に光速 $c$ をかけて空間と同じ次元にし
        \footnote{
            $c$[m/s]$\times t$[s]$=ct$[m]
        },
    ($ct$\,,$x$\,,$y$\,,$z$)とされる.
    ミンコフスキー空間は4次元空間なので,図に描くことはできない
        \footnote{
            3次元空間も,図で表現することは無理であった.
            紙は2次元だからだ.しかし,目の錯覚を利用し
            て,2次元空間に3次元空間を描く事はできた.
            3次元は,実際に目で見て,肌で触れて感じてい
            る事なので,このことが可能であった.もしか
            したら,4次元空間も3次元空間に描く事ができ
            るのも知れなが,生憎(あいにく),2次元の紙表
            面に描く事はできない.
        }.
    描くことができるのは,2次元のみである
        \footnote{
            1次元すらも描く事はできない.ペンの線には
            “太さ”があるからだ.
        }.
    しかし,空間はどれも同等であり,そのうちの2次元を取り
    上げて描けばよい.そこで,図で表現したいときには,時
    間 $ct$ と空間 $x$ を取り上げて描く事が多い.縦軸に
    時間 $ct$,横軸に $x$ をとして描く.このミンコフスキー空
    間を表現する図と,ニュートン力学の $x$ - $t$ は別物
    なので,気をつけよう.

\section{ローレンツ変換式の行列表示}\label{sec:Lorentz_trans_by_matrix}
    イメージを簡単にするため,$x$方向のローレンツ変換を考える.
        \footnote{
            もちろん,3方向の変換を考えてもよいが,イメージと式が複雑になるだけで,メリットは少ない.
            1方向に慣れてきたら,3方向の場合の式の導出をしてみればよいだろう.
        }.
    $x$方向のローレンツ変換は以下のようであった.
    \begin{align*}
        ct' &= \frac{1}{\sqrt{ 1-(v/c)^{2} }} ct -\frac{v/c}{\sqrt{ 1-(v/c)^{2} }} x \\
        x'  &= -\frac{v/c}{\sqrt{ 1-(v/c)^{2} }}ct+\frac{1}{\sqrt{ 1-(v/c)^{2} }} x \\
        y'  &= y \\
        z'  &= z
    \end{align*}

    これは,ローレンツ因子 $\gamma := 1/\sqrt{1-(v/c)^{2}}$ を使うと次のようになる.
    \begin{align*}
        ct' &= \gamma  ct - \displaystyle \gamma \frac{v}{c}  x \\
        x'  &= - \displaystyle \gamma \frac{v}{c} ct + \gamma  x \\
        y'  &= y \\
        z'  &= z
    \end{align*}

    さらに,$\beta := v/c$ を使うと次のようになる.
    \begin{align*}
        ct' &=   \gamma       ct - \gamma \beta x \\
        x'  &= - \gamma \beta ct + \gamma       x \\
        y'  &= y \\
        z'  &= z
    \end{align*}

    行列の形に書き換えると,次のようになる.
    \begin{align*}
        \begin{bmatrix}
                ct'\\
                x' \\
                y' \\
                z'
        \end{bmatrix}
        =
        \begin{bmatrix}
              \gamma       & - \gamma \beta & 0 & 0 \\
            - \gamma \beta &   \gamma       & 0 & 0 \\
            0              & 0              & 1 & 0 \\
            0              & 0              & 0 & 1
        \end{bmatrix}
        \begin{bmatrix}
            ct\\
            x \\
            y \\
            z
        \end{bmatrix}.
    \end{align*}

    これで十分なんだが,
    \begin{align}\label{eq:Lorentz_trans_by_matrix}
        \Lambda :=
        \begin{bmatrix}
              \gamma       & - \gamma \beta & 0 & 0 \\
            - \gamma \beta &   \gamma       & 0 & 0 \\
            0              & 0              & 1 & 0 \\
            0              & 0              & 0 & 1
        \end{bmatrix},
    \end{align}
    \begin{align}
        X :=
        \begin{bmatrix}
            ct\\
            x \\
            y \\
            z
        \end{bmatrix}\quad,\quad
        X' :=
        \begin{bmatrix}
                ct'\\
                x' \\
                y' \\
                z'
        \end{bmatrix},
    \end{align}
    \begin{align}
        \beta := v/c \quad,\quad \gamma := 1/\sqrt{1-(v/c)^{2}}
    \end{align}
    としてみると,
        \[
            X' = \Lambda X
        \]
    という単純な形式になっていることが見て取りやすくなる.
    1つの式になったという感じがする
        \footnote{
            実際,感じがするだけだ.その中身は4つの連立方程式である.
            悪く言えば,行列を用いて4つの式を無理やり1つに詰め込んだ,といえよう.
            よく言うならば,行列の論理で整理することによって,4つのバラバラの式を
            1つの式にまとめ上げた,ともいえる.主観次第だ.
        }.
    実は,行列の成分で式を書くことが一般的である.節を改めて,成分で表示してみよう.
    その前に,世界間隔と固有時間について触れておきたい.ローレンツ変換と密接に関係
    のある概念だからだ.

\section{時空座標の導入}
    \subsection{時空座標の次元($ct$ の導入)}\label{cap:ct_jikuu}
        ローレンツ変換は時間と空間が絡み合った変換であることが,先の計算でわかった.
        また,行列表示によって,1つの式としてまとめられることもわかっている.そうであれば,
        時間と空間をまとめて,1つの時空座標として扱うべきだ.ということで,時間座標と空間座標を
        1つの4次元のベクトルとして,書き表そう.
        時間と空間を区別する必要はあるのだが,1つの数式にまとめたい.また,
        時間と空間は同等である表現できるとよい.そこで,時間座標と空間座標で同じ文字で表現することにして,
        $x$ という文字を割り当てよう.いまここで使う $x$ は,空間座標の $x$ 座標ではない.
        全く新しい概念としての座標記号である
            \footnote{
                だったら,$x$ なんて紛らわしいことしないで,別の文字を使えばいいではないかと思う.
                でも,相対性理論の教科書はことごとく,$x$ が使われている.意味は違えど,$x$ とかいたら,
                空間をイメージしてしまうことに由来するのだろうか.
            }.
        1つのベクトルに組み上げるのだが,その要素の次元(単位)は揃えておきたい.そうしないと,計算時に
        時間成分だけ特別扱いが必要になるからだ.時間成分 $t$([s]) を空間成分の距離 $x$([m]) になるように,
        調整してやると良い.それには光速 $c$([m/s]) という都合の良い次元をもった,しかも定数がある.
        時間 $t$ に光速 $c$ をかけて,$ct$ とすることで,次元を[m]にできる.
        \[
            c\mbox{[m/s]} * t\mbox{[s]} = ct\mbox{[m]}.
        \]

    \subsection{時空座標の組み上げ}
        時間座標 $ct$ の位置は空間座標の前に置かれる.空間座標 $x,\, y,\, z$ も改めて,時空座標 $x$ で
        表すことになる.後々,式をまとめ上げる際に役に立つ以下のように,上付きの添数字を導入して,各々に
        割り当てる.具体的には,次のとおりだ.
        \[
            x^{0} := ct, \quad x^{1} := x, \quad x^{2} := y, \quad x^{3} := z.
        \]
        これらを要素として,時空ベクトル $x^{\mu}$ を作ろう.
        \[
            x^{\mu} := \left( x^{0}, \, x^{1}, \, x^{2}, \, x^{3} \right)
        \]
        これまで,ベクトルを表現するのに,太文字を使ったり文字の上に矢印を書いたりしていたが,また新しい
        書き方に出くわした.添字にギリシャ文字を使うパタン.これも式をまとめるための小細工の1つである
            \footnote{
                なれるまで使い続けるしかない.一度なれてしまえば,かなり便利な記法であることが実感できよう.
            }.

        添字の $\mu$ は実際は右辺で示したような具体的な添字が入るものであり,この添字自体に意味はない.
        式をまとめるに便利なだけの,無意味なダミー変数である.したがって,$\nu$ でも $\rho$ でもなんでも
        よい.ただ,4次元ベクトルであることを容易に想起させるため,ギリシア文字を使うことにする.つまり,
        以下の $x^{\mu}$ も $x^{\nu}$ も $x^{\rho}$ も同じことである.
        \[
            x^{\nu}  := \left( x^{0}, \, x^{1}, \, x^{2}, \, x^{3} \right). \\
            x^{\rho} := \left( x^{0}, \, x^{1}, \, x^{2}, \, x^{3} \right).
        \]
        慣性座標を複数使って考察する場合には,このようにいろいろなギリシア文字の添字を使うことになる.
        さて,上で導入した時間座標と空間座標を合わせた位置に関する時空ベクトルの他に,
        位置・速度・角速度・運動量・角運動量・力・エネルギーに関するものがある.順次確認していこう.

\section{四元位置ベクトル}
    いまさっき確認した,位置ベクトルを4次元化したものを,\textbf{四元位置ベクトル} という
        \footnote{
            四元:読み方の流儀がいくつかありそう.四を「よ」というか,「よん」というか,「し」というか.
            よげん/よんげん/しげん.どれも間違いではないだろう.僕の今の感覚を書いておくと,
            四元ベクトルは「よんげんベクトル」,四次元は「よじげん」,四元数は「しげんすう」とよむ.
        }.
    \begin{align}
        x^{\mu} := \left( x^{0}, \, x^{1}, \, x^{2}, \, x^{3} \right)
    \end{align}

\section{世界間隔}
    ニュートン力学において,ある2点の距離$d$は,$(\,\,x,\,y,\,z\,)$ 座標において,三平方の定理によって,
        \begin{align*}
            d &= \sqrt{\sum_{i=1}^{3} \left({x_{i}}\right)^{2}} \\
              &= \sqrt{{x_{1}}^{2} + {x_{2}}^{2} + {x_{3}}^{2}} \\
              &= \sqrt{{x}^{2} + {y}^{2} + {z}^{2}}
        \end{align*}
    である.
    特殊相対性理論では,空間座標と時間座標は互いに影響しあっているとわかっているので,上の距離の式に時間も加えて,
    ニュートン力学の3次元空間を例にして,これに時間を加えて時空の4次元に拡張したい.
    いま,観測者はある1つの慣性系に静止しているとしよう.その座標系を $(\,ct,\,x,\,y,\,z\,)$ とする.
    光速不変の原理によると,運動している物体から光が生じようが,
    静止している物体から光が生じようが,両者の光速は $c$ という一定値で観測される.

    光が原点から一瞬だけ生じたとしよう.このとき,光が進む距離は $ct$ であり,
    三平方の定理から,
        \begin{equation*}
            ct = \sqrt{{x}^{2} + {y}^{2} + {z}^{2}}
        \end{equation*}
    であり,両辺を自乗すれば,
        \begin{equation*}
            ct^{2} = {x}^{2} + {y}^{2} + {z}^{2}
        \end{equation*}
    と書ける.
    これは,特殊相対性原理によって,
    別の慣性系 $(\,ct',\,x',\,y',\,z'\,)$ の観測者が観測しても,同じ
    ことである.その場合は,
        \begin{equation*}
            {ct'}^{2} = {x'}^{2} + {y'}^{2} + {z'}^{2}
        \end{equation*}
    である.上の2つの式を次のように左辺の${ct}^{2}$を右辺に移項すると,どの慣性系でも成立する式を得る.
        \begin{align}
            0 = -(ct)^{2} + {x}^{2} + {y}^{2} + {z}^{2}.
        \end{align}
    3次元での距離 $d$ を,4次元に拡張したような形になっている.この量には名前が
    ついていて,それは \textbf{世界間隔} とよばれている
        \footnote{
            教科書により,世界長・世界長さ・世界距離など,表現が異なることがある.
        }.
    その記号として,$s$ が用いられる.つまり.
        \begin{align}
            s^{2} := -(ct)^{2} + {x}^{2} + {y}^{2} + {z}^{2}
        \end{align}
    である.この世界間隔 $s$ は,4次元世界での量であり,
    今まで考えてきた3次元の距離 $d$ と似てはいるものの,
    いろいろと異なった性質ももつ.
    また,\textbf{無限小線素}(または単に,\textbf{線素})を考えるときには,
        \begin{align}
            \df s^{2} = (\df s)^{2} := -(\df (ct))^{2} + (\df x)^{2} + (\df y)^{2} + (\df z)^{2}
        \end{align}
    である.
    以下,上式の表現を簡略化して(例えば,$\df a^{2}:= (\df a)^{2}$ とう略記を採用する),
    \begin{align}
        \df s^{2} = -\df (ct)^{2} + \df x^{2} + \df y^{2} + \df z^{2}
    \end{align}
    と書くことも多い.\textbf{ミンコフスキー空間} とは,Euclid空間における距離の定義を
    上に説明した世界間隔に置き換えたものに他ならない.

    時空座標$x^{\mu}$を使うと,次のようになる.添字を使うと,足し算を現すのに$\sum$が利用できるようなり,
    記述が短くなる.
    \begin{align}
        \df s^{2} &= -{\df {x}^{0}}^{2} + {\df x_{1}}^{2} + {\df x_{2}}^{2} + {\df x_{3}}^{2} \notag \\
                  &= \sum_{\mu=0}^{3} {{x}^{\mu}}^{2}.
    \end{align}

    \begin{memo}{$-(ct)^{2} + {x}^{2} + {y}^{2} + {z}^{2} = 0$だから常に$s=0$?}
        $-(ct)^{2} + {x}^{2} + {y}^{2} + {z}^{2} = 0$だから常に$s=0$であるはずがない.
        光速度動く物体など存在しない.光以外の物体は光速以下で運動するため,常に${\df s}^{2}<0$である.
        光の場合の特別な状況で,
        \begin{align}
            (ct)^{2} = {x}^{2} + {y}^{2} + {z}^{2}
        \end{align}
        が成り立つ.
        物体の速度$v$は光速以下のなので,
        \begin{align}
            (ct)^{2} > (vt)^{2} (= {x}^{2} + {y}^{2} + {z}^{2})
        \end{align}
        である.だから,
        \begin{align*}
                            &(ct)^{2} >  {x}^{2} + {y}^{2} + {z}^{2} \\
            \Leftrightarrow &(ct)^{2} - ({x}^{2} + {y}^{2} + {z}^{2}) > 0 \\
            \Leftrightarrow &-(ct)^{2} + {x}^{2} + {y}^{2} + {z}^{2} < 0 \\
            \Leftrightarrow &{\df s}^{2} < 0.
        \end{align*}

        光円錐の内側が光速以下の我々の世界(${\df s}^{2}<0$)で,時間的といわれ,因果関係が成立する領域.
        光円錐の外側が光速を超えた物体があるかもしれない未知の領域で,因果関係が崩れている(${\df s}^{2}>0$).
        光円錐のちょうど表面が光速の世界(${\df s}^{2}=0$)で,ヌルと言われ,時間領域と空間領域の境界.
        \begin{figure}[hbt]
            \begin{center}
                \includegraphicsdefault{minkowski_light_cone.pdf}
                \caption{光円錐と${\df s}^{2}$}
                \label{fig:koyuuji1}
            \end{center}
        \end{figure}


        左辺が0であるのは,原点から出発した光を仮定しているからではない.
            \[
                0 = -{c}^{2}(t-{t}_{0})^{2} + {(x-{x}_{0})}^{2} + {(y-{y}_{0})}^{2} + {(z-{z}_{0})}^{2}.
            \]
        展開して適当に移項してみると,
            \begin{align*}
                &-{(ct)}^{2} + {x}^{2} + {y}^{2} + {z}^{2} \\
                &= 2(-{c}^{2}t{t}_{0}+x{x}_{0}+y{y}_{0}+y{y}_{0})
                -(-{(c{t}_{0})}^{2} + {{x}_{0}}^{2} + {{y}_{0}}^{2} + {{z}_{0}}^{2}).
            \end{align*}
        左辺が$s$の形になった.
            \begin{align*}
                s &= -{(ct)}^{2} + {x}^{2} + {y}^{2} + {z}^{2} \\
                  &= 2(-{c}^{2}t{t}_{0}+x{x}_{0}+y{y}_{0}+y{y}_{0})
                  - (-{(c{t}_{0})}^{2} + {{{x}_{0}}^{2}} + {{{y}_{0}}^{2}} + {{{z}_{0}}^{2}}).
            \end{align*}
        確かに,0ではないが,こういうことではない$\cdots$.
    \end{memo}

\section{固有時間}
    既に確認したように,時間の進み具合は座標系のとり方に依存する.
    つまり,座標系のとり方しだいで,時間の進み具合が変わってしまうのである.けれども,
    これで議論はがしにくい.
    しかも,座標系に依存するということは,基準が存在しないことを意味する.
    そこで,座標系のとり方にかかわらずに,しかも
    時間と同じ意味をもつ量を新たに導入する.それは \textbf{固有時間} と
    呼ばれる量であり,記号 $\tau$ によって表される
        \footnote{
            固有時(こゆうじ)と言う場合も多い.世界間隔にしても固有時間にしても,
            単語が安定ていない.言葉が微妙に違っても,混乱することがないため,
            今のところは厳密な統一はされてない.
            ちなみに,物性物理学で使われる電子緩和時間 $\tau$ とはまったく別の概念である.念のため.
        }.
    固有時間 $\tau$ を具体的に表現することを考える.固有時間に要請される性質として,
    どの座標系をとっても不変であることである.ところで,このような性質をもつ量と
    して,先ほど,世界間隔について考え,無限小線素を導出した.それは以下のように
    書き表せる.
     \begin{align*}
         \df s^{2} &= \sum_{\mu=0}^{3} \left({{x}^{\mu}}\right)^{2} \\
                   &= -\df {x^{0}}^{2} + {\df x_{1}}^{2} + {\df x_{2}}^{2} + {\df x_{3}}^{2} \\
                   &= -{\df (ct)}^{2} + {\df x}^{2} + {\df y}^{2} + {\df z}^{2}.
     \end{align*}
    この式を足がかりとして,固有時間について考えていこう.
    今,観測者は系Sに対して静止しているとする.さらに系S$'$が,系Sに対して速度 $\bv '$ で運動しているとしよう.このとき,系S$'$に対して静止している時計の時刻を,観測者が観測している状況を考える.
            \begin{figure}[hbt]
                \begin{center}
                    \includegraphicslarge{koyuuji1.pdf}
                    \caption{固有時間1}
                    \label{fig:koyuuji1}
                \end{center}
            \end{figure}
    系Sにおいて,時刻 $t=0$ で時計が原点にあったとする.そして,$\df t'$ の後,時計は位置 $\df \br ' = (\,\df x'\,,\,\,\df y'\,,\,\,\df z'\,)$ へ移動したとき,速度 $\bv '$ は
        \begin{align}
            \bv '   =  \frac{\df \br '}{\df t'}
        \end{align}
    となる.すると,速さは
        \begin{equation*}
            v' = \frac{\left|\df \br '\right|}{\df t'}
               = \frac{\sqrt{{\df x'}^{2} + {\df y'}^{2} + {\df z'}^{2}}}{\df t'}
        \end{equation*}
    と表現できる.次のように式変形をしよう
        \footnote{
        この変形は数学的に正しいことを確認(証明)すべきだろう.
        しかし,ここでは数式を感覚的に扱う.
        正しさを確認したい場合は,微分の教科書を参照のこと.
        }.
        \begin{equation*}
            {v'}^{2} {\df t'}^{2}={\df x'}^{2} + {\df y'}^{2} + {\df z'}^{2}.
        \end{equation*}

        ところで,系S から観測する系 S$'$ の無限小線素を $\df s'$ とすると以下が成り立つ.
         \begin{align}\label{eq:bmugenshou_sennso}
             \df s^{2} = -{\df (ct')}^{2} + {\df x'}^{2} + {\df y'}^{2} + {\df z'}^{2}.
         \end{align}
    先ほど計算した $v'$ を,式(\ref{eq:bmugenshou_sennso})に置き換えてみる.右辺の空間座標部分を
    次のように見てみよう.
        \begin{align*}
            &{\df x'}^{2} + {\df y'}^{2} + {\df z'}^{2} \\
            &\quad=\frac{{\df x'}^{2} + {\df y'}^{2} + {\df z'}^{2}}{{\df t'}^{2}} {\df t'}^{2} \\
            &\quad={\left(
                \frac{\sqrt{{\df x'}^{2} + {\df y'}^{2} + {\df z'}^{2}}}{{\df t'}^{2}}
                   \right)}^{2} {\df t'}^{2}\\
            &\quad={v'}^{2} {\df t'}^{2}.
        \end{align*}
    これを使って置き換えると,次になる.
        \begin{align*}
            \df s^{2} &= -{\df (ct')}^{2} + {\df x'}^{2} + {\df y'}^{2} + {\df z'}^{2} \\
                      &= -{\df (ct')}^{2} + {v'}^{2} {\df t'}^{2} \\
                      &= {\df t'}^{2}\left( -c^{2} + {v'}^{2} \right) \\
                      &= -c^{2}{\df t'}^{2}\left( 1 - \frac{{v'}^{2}}{c^{2}} \right).
        \end{align*}
    ここでちょっと計算ととめて,式をじっくり眺めてみよう.すると,次のことに気づく
        \footnote{
            気づかなくても,先を読んでもらえばいいんだけどね.
        }.
    左辺の $\df s^{2}$ は微小線素であり,どのような座標系でも不変な量である.
    つまり,等式が成立するのであれば,右辺も座標によらないはずである.括弧の外の $c$ は定数であるから
        \footnote{
            $c = 3.0 \times 10^{10}$[m/s]
        },
    不変なのは明らか.残る $\df {t'}^{2}\left(1-{v'}^{2}/c^{2}\right)$ を見てみると,$\gamma$ がある
        \footnote{
            ローレンツ因子とは次のように表されるものである.
                \begin{equation*}
                    \gamma = \frac{1}{\sqrt{1-\left(v^{2}/c^{2}\right)}}.
                \end{equation*}
        }.
    ローレンツ因子 $\gamma$ を用いることで,式は次のようになる.
        \begin{align*}
            \df s^{2} &= -c^{2}{\df t'}^{2}\left( 1 - \frac{{v'}^{2}}{c^{2}} \right). \\
                      &= -c^{2}\frac{{\df t'}^{2}}{{\gamma}^{2}}
        \end{align*}
    ここで,\textbf{固有時間} $\df \tau$ 次のように定義する.すなわち,
        \begin{align}
            \df \tau := \frac{\df t'}{\gamma}
                 =      \sqrt{1-\left(v^{2}/c^{2}\right)}\df t'.
        \end{align}
    固有時間 $\tau$ を使って,
        \begin{align*}
            \df s^{2} &= -c^{2}\frac{{\df t'}^{2}}{{\gamma}^{2}} \\
                      &= -c^{2} \df \tau^{2}
        \end{align*}
    とかける.
    固有時間 $\tau$ は系S$'$に固有な時間である
        \footnote{
            次のような仮定をしていることを,思い出そう.すなわち,観測者は系Sに対して静止している.系S$'$は,系Sに対して速度 $\bv '$ で等速直線運動をしている.
        }.
    つまり,この時間は系S$'$においてのみ計測し得る値である.観測者は系Sに対して静止していることから,当然 $\tau$ とは異なった時間を感じている.慣性系に対して静止している時計で測った時間が $\tau$ なのである.だから慣性系に「固有」なのである.
    固有時間に対して,座標に依存する時間を \textbf{座標時} という.座標時は静止系から見た,運動系の時間のことである.
    \begin{memo}{光の固有時間}
        光は速度 $v=c$ なので,その固有時間は0である.よって,光速で進む座標系では時間はすすまない.
        固有時間の定義式の $v$ に $c$ を代入すると,以下のように0になるからだ.
            \[
                \sqrt{1-\left(v^{2}/c^{2}\right)}\df t' = \sqrt{1-\left(c^{2}/c^{2}\right)}\df t' = 0.
            \]
    \end{memo}

\section{四元速度ベクトル}
    四元位置ベクトル${x}^{\mu}$を固有時間$\tau$で微分して,\textbf{四元速度ベクトル} ${u}^{\mu}$を構成する.
    \begin{align}
        {u}^{\mu} &= \left( u^{0}, \, u^{1}, \, u^{2}, \, u^{3} \right) \\
                &=\left(
                    \frac{{\df x^{0}}}{\df \tau}, \,
                    \frac{{\df x^{1}}}{\df \tau}, \,
                    \frac{{\df x^{2}}}{\df }, \,
                    \frac{{\df x^{3}}}{\df \tau}
                  \right)
    \end{align}

\section{四元加速度ベクトル}
    四元速度ベクトル${x}^{\mu}$を固有時間$\tau$で微分して,\textbf{四元加速度ベクトル} ${u}^{\mu}$を構成する.
    \begin{align}
        {u}^{\mu} &= \left( u^{0}, \, u^{1}, \, u^{2}, \, u^{3} \right) \\
            &=\left(
                \frac{{\df u^{0}}}{\df \tau}, \,
                \frac{{\df u^{1}}}{\df \tau}, \,
                \frac{{\df u^{2}}}{\df \tau}, \,
                \frac{{\df u^{3}}}{\df \tau}
              \right)
    \end{align}

\section{$\eta_{\mu\nu}$の導入}
    \begin{mycomment}
        世界間隔 $\df s$ を,時空座標$x^{\mu}$を使って,
            \[
                {\df s}^{2} = \sum_{\mu=0}^{3} \left({{x}^{\mu}}\right)^{2}
            \]
        とあらせた.さらに新しい記号$\eta_{\mu\nu}$を導入して,式表現を変更する.
        これは,テンソル表記へに向けての一歩になる.
    \end{mycomment}

    微小な世界間隔は以下である.
    \begin{align*}
        {\df s}^{2} &= \sum_{\mu=0}^{3} \left({{x}^{\mu}}\right)^{2} \\
                    &= -\df (ct)^{2} + \df x^{2} + \df y^{2} + \df z^{2}.
    \end{align*}
    これを時空座標 $x^{\mu}$ を使った形に書き改めると,次のようになる.
    \begin{align}
        {\df s}^{2} = - {(\df x^{0})}^{2} + {(\df x^{1})}^{2} + {(\df x^{2})}^{2} + {(\df x^{3})}^{2}.
    \end{align}
    唐突だが,ここで,以下のような行列 $\eta_{\mu\nu}$ を導入する.
    \begin{align}
        \eta_{\mu\nu} =
            \begin{bmatrix}
                -1 & 0 & 0 & 0 \\
                 0 & 1 & 0 & 0 \\
                 0 & 0 & 1 & 0 \\
                 0 & 0 & 0 & 1
            \end{bmatrix}.
    \end{align}
    この $\eta_{\mu\nu}$ を使うと,世界間隔が以下のように書ける.
    \begin{align}
        {\df s}^{2} = \sum_{\nu=0}^{3} \sum_{\mu=0}^{3} \eta_{\mu\nu} \df x^{\mu} \df x^{\nu}.
    \end{align}

    本当かどうか,展開して,確かめてみよう.
    \begin{align*}
        \df &{s}^{2} \\
        =\;&\sum_{\nu=0}^{3} \sum_{\mu=0}^{3} \eta_{\mu\nu} \df x^{\mu} \df x^{\nu} \\
        =\;&\sum_{\nu=0}^{3} \left( \sum_{\mu=0}^{3} \eta_{\mu\nu} \df x^{\mu} \df x^{\nu} \right) \\
        =\;&\sum_{\nu=0}^{3} \left( \eta_{0\nu} \eta_{0\nu} \df x^{0} \df x^{\nu} \right) \\
        \;+&\sum_{\nu=0}^{3} \left( \eta_{0\nu} \eta_{1\nu} \df x^{1} \df x^{\nu} \right) \\
        \;+&\sum_{\nu=0}^{3} \left( \eta_{0\nu} \eta_{2\nu} \df x^{2} \df x^{\nu} \right) \\
        \;+&\sum_{\nu=0}^{3} \left( \eta_{0\nu} \eta_{3\nu} \df x^{3} \df x^{\nu} \right) \\
        =\;&\eta_{00} \df x^{0} \df x^{0} + \eta_{01} \df x^{0} \df x^{1} + \eta_{02} \df x^{0} \df x^{2} + \eta_{03} \df x^{0} \df x^{3} \\
        \;+&\eta_{10} \df x^{1} \df x^{0} + \eta_{11} \df x^{1} \df x^{1} + \eta_{12} \df x^{1} \df x^{2} + \eta_{13} \df x^{1} \df x^{3} \\
        \;+&\eta_{20} \df x^{2} \df x^{0} + \eta_{21} \df x^{2} \df x^{1} + \eta_{22} \df x^{2} \df x^{2} + \eta_{23} \df x^{2} \df x^{3} \\
        \;+&\eta_{30} \df x^{3} \df x^{0} + \eta_{31} \df x^{3} \df x^{1} + \eta_{32} \df x^{3} \df x^{2} + \eta_{33} \df x^{3} \df x^{3} \\
        =\;&-{(\df x^{0})}^{2} + {(\df x^{1})}^{2} + {(\df x^{2})}^{2} + {(\df x^{3})}^{2}.
    \end{align*}
    添字がしんどい.

    または,直接,行列表示での計算は以下の通り.
    \begin{align*}
        {\df s}^{2} &= \begin{bmatrix}
                           -1 & 0 & 0 & 0 \\
                            0 & 1 & 0 & 0 \\
                            0 & 0 & 1 & 0 \\
                            0 & 0 & 0 & 1
                       \end{bmatrix}
                       \begin{bmatrix}
                           \df x^{0} \\
                           \df x^{1} \\
                           \df x^{2} \\
                           \df x^{3}
                       \end{bmatrix}
                       \begin{bmatrix}
                           \df x^{0} \\
                           \df x^{1} \\
                           \df x^{2} \\
                           \df x^{3}
                       \end{bmatrix} \\
                    &=  \begin{bmatrix}
                            - \df x^{0} \\
                              \df x^{1} \\
                              \df x^{2} \\
                              \df x^{3}
                        \end{bmatrix}
                        \begin{bmatrix}
                              \df x^{0} \\
                              \df x^{1} \\
                              \df x^{2} \\
                              \df x^{3}
                        \end{bmatrix} \\
                    &= - {(\df x^{0})}^{2} + {(\df x^{1})}^{2} + {(\df x^{2})}^{2} + {(\df x^{3})}^{2}.
    \end{align*}
    行列 $\eta_{\mu\nu}$ と ベクトル $x^{\mu}$ の積は行列積の計算で,その結果はベクトルである.
    このベクトルと $x^{\nu}$ の内積をとる(各同じ成分同士掛け合わせた後,それらを加算する).
    すると,世界間隔になる.世界間隔がコンパクトに表現されるようになった(と感じませんか?).
    さらに,\textbf{同じギリシア文字2つが添字の上下対になって現れた場合,その文字について常に0から3までの和を取る} と約束する
    ならば
        \footnote{
            この約束のことを,いくつかバリエーションがあるが,
            \textbf{アインシュタインの規約},
            \textbf{アインシュタインの縮約記法},
            \textbf{アインシュタインの総和規約} と
            言うらしい.どれも同じことである.
            アインシュタインが導入した規則だと伝えられている.
            アインシュタイン自身も,この規約が気に入っていたらしい.
            根拠はわからんが,いくつかの教科書に記載がある.
            まぁ,名前はどうでもいいや.大事なことは規則を覚えることだ.
        },
    $\sum$ 記号の記載を省略できて式の見た目がよくなる.省略した形を書いてみよう.
    \begin{align}
        {\df s}^{2} = \eta_{\mu\nu} \df x^{\mu} \df x^{\nu}.
    \end{align}
    とても短くなった.
    また,世界間隔と固有時間の関係 ${\df s}^{2} = -c^{2} \df \tau^{2}$ から,
    \[
        -c^{2} \df \tau^{2} = \eta_{\mu\nu} \df x^{\mu} \df x^{\nu}
    \]
    も成り立つ.

    \begin{memo}{行列表示とその成分表示?}
        行列が,$\eta_{\mu\nu}$ ではない場合,素直に行列を書いて計算した場合と成分表示した場合で,
        計算結果が変わる.つまり,一般の行列には成り立たない.計算してみよう.
        \begin{align*}
            \sum_{\nu=0}^{3} \sum_{\mu=0}^{3} \eta_{\mu\nu} \df x^{\mu} \df x^{\nu}
                    &= \begin{bmatrix}
                           -1 & 0 & 0 & 0 \\
                            0 & 1 & 0 & 0 \\
                            0 & 0 & 1 & 0 \\
                            0 & 0 & 0 & 1
                       \end{bmatrix}
                       \begin{bmatrix}
                           \df x^{0} \\
                           \df x^{1} \\
                           \df x^{2} \\
                           \df x^{3}
                       \end{bmatrix}
                       \begin{bmatrix}
                           \df x^{0} \\
                           \df x^{1} \\
                           \df x^{2} \\
                           \df x^{3}
                       \end{bmatrix}
        \end{align*}
        この等式は先に計算したとおり,成立している.おかしくなるのは,行列を一般化した以下のような式のときだ.
        \begin{align*}
            \sum_{\nu=0}^{3} \sum_{\mu=0}^{3} {u}_{\mu\nu} {x}^{\mu} {y}^{\nu}
                    \neq \mqty[ \xmat*{u}{4}{4} ]
                         \mqty[ \xmat*{x}{4}{1} ]
                         \mqty[ \xmat*{y}{4}{1} ]
        \end{align*}

        例として,別の$2 \times 2$行列で考えよう.2つの $x$ があるが,文字が2つだと
        見えにくいので,区別して,$x$,$y$ とする.$\df$ も例では不要だ.段階を踏んで考える.
        数学形式で書くので,ベクトルと行列成分の添字の開始番号が1とする.
        \begin{align*}
            \sum_{\nu=1}^{2} \sum_{\mu=1}^{2} {u}_{\mu\nu} {x}^{\mu} {y}^{\nu}
                    \neq \mqty[ \xmat*{u}{2}{2} ]
                         \mqty[ \xmat*{x}{2}{1} ]
                         \mqty[ \xmat*{y}{2}{1} ]
        \end{align*}

        まずはベクトル$x^{i}$とベクトル$y_{j}$の内積を成分表示したもの.以下の式を理解してほしい.
        これは$\sum$の規則の復習で,簡単なので説明不要だろう.内積はスカラーになるので,結果を$a$と
        しておこう.
        \begin{align*}
            a &= \sum_{j=1}^{2} \sum_{i=1}^{2} {x}_{i} {y}_{j} \\
              &= \sum_{j=1}^{2} \left( \sum_{i=1}^{2} {x}_{i} \right) {y}_{j} \\
              &= \sum_{j=1}^{2} \left( {x}_{1} + {x}_{2} \right) {y}_{j}    \\
              &= \left( {x}_{1} + {x}_{2} \right) {y}_{1} +\left( {x}_{2} + {x}_{2} \right) {y}_{2}
        \end{align*}

        次は行列2$\times$2表列$u_{ij}$とベクトル$v_{j}$の積を成分表示した中級編.
        結果はベクトルとなるから$b_{i}$としておこう.
        \begin{align*}
            b_{i} &= \sum_{j=1}^{2} {u}_{ij} {v}_{j} \\
                  &=
                  \begin{bmatrix}
                        {u}_{1i} {v}_{1} + {u}_{2i} {v}_{2}
                  \end{bmatrix}.
        \end{align*}
        念の為,$i=1,\,2$だから,全部書くと以下.
        \begin{align*}
            \mqty[ \xmat*{b}{2}{1} ] =
                  \begin{bmatrix}
                        {u}_{11} {v}_{1} + {u}_{21} {v}_{2} \\
                        {u}_{12} {v}_{1} + {u}_{22} {v}_{2}
                  \end{bmatrix}.
        \end{align*}

        さて最終段階だ.$b_{i}$ に対して,更にベクトル$w_{j}$をかける形で,式を発展させる.
        結果はスカラーになるので$c$としておこう.成分を列挙すると添字地獄に陥る.
        \begin{align*}
            c   &= \sum_{i=1}^{2} \sum_{j=1}^{2} {u}_{ij} {v}_{i} {w}_{j} \\
                &= \sum_{i=1}^{2} \left( \sum_{j=1}^{2} {u}_{ij} {v}_{i} {w}_{j} \right) \\
                &= \sum_{i=1}^{2}
                    \begin{bmatrix}
                        {u}_{i1} {v}_{i} {w}_{1} + {u}_{i2} {v}_{i} {w}_{2}
                    \end{bmatrix} \\
                &= {u}_{11} {v}_{1} {w}_{1} + {u}_{12} {v}_{1} {w}_{2}
                  +{u}_{21} {v}_{2} {w}_{1} + {u}_{22} {v}_{2} {w}_{2}.
        \end{align*}

        他方の成分表示で書かれた場合,アインシュタインの規約による記述を展開して計算すると,
        \begin{align*}
            &\mqty[ \xmat*{u}{2}{2} ]
             \mqty[ \xmat*{v}{2}{1} ]
             \mqty[ \xmat*{w}{2}{1} ] \\
            &=
            \begin{bmatrix}
                {u}_{11} {v}_{1} + {u}_{12} {v}_{2} \\
                {u}_{21} {v}_{1} + {u}_{22} {v}_{2}
            \end{bmatrix}
            \mqty[ \xmat*{w}{2}{1} ] \\
            &=
            \left(
              {u}_{11} {v}_{1} + {u}_{12} {v}_{2}
            \right) {w}_{1}
            +
            \left(
              {u}_{21} {v}_{1} + {u}_{22} {v}_{2}
            \right) {w}_{2} \\
            &= \left(
              {u}_{11} {v}_{1} {w}_{1} + {u}_{12} {v}_{2} {w}_{1}
            \right)
             +
            \left(
              {u}_{21} {v}_{1} {w}_{2} + {u}_{22} {v}_{2} {w}_{2}
            \right)
        \end{align*}

        $u_{ji}$の$i \ne j$の成分に関する後が,行列そのまま表示と成分表示で入れ替わっている.
        $\eta_{\mu\nu}$ の場合,$i \ne j$の部分は0になるため,この問題は生じない.
    \end{memo}

\section{4元運動量の定義}
    観測者に対して,光速に近い速さで運動している
    物体の,運動量を考える.ニュートン力学における運動量 $\bp$ は,
    質量 $m$ と速度 $\bv$ の積で表現された.
        \begin{equation*}
            \bp = m \bv.
        \end{equation*}
    この運動量 $\bp$ を成分表示すると.
        \begin{align*}
            \begin{bmatrix}
                p_{x} \\
                p_{y} \\
                p_{z}
            \end{bmatrix}
            =
            \begin{bmatrix}
            mv_{x} \\
            mv_{y} \\
            mv_{z}
            \end{bmatrix}
            =
            \begin{bmatrix}
                m(\df x/\df t) \\
                m(\df y/\df t) \\
                m(\df z/\df t)
            \end{bmatrix}
        \end{align*}
    である.

    このままでは,空間の概念しかなく,
    特殊相対性理論を考える上で,不都合がある.
    特殊相対性理論では,空間と時間は同等であるから,
    上の運動量に,時間も組み込んでおきたい.そこで,
    \textbf{4元運動量} として,今までの運動量を,拡張する.
        \begin{align}
            p^{\mu} =
            \begin{bmatrix}
                p_{0} \\
                p_{1} \\
                p_{2} \\
                p_{3}
            \end{bmatrix}
            :=
            \begin{bmatrix}
                -E/c   \\
                mv_{x} \\
                mv_{y} \\
                mv_{z}
            \end{bmatrix}
            =
            \begin{bmatrix}
                -{E/c} \\
                m(\df x/\df t) \\
                m(\df y/\df t) \\
                m(\df z/\df t)
            \end{bmatrix}
        \end{align}
    である.

    ここで,心配になるのは,時間成分にある $p^{0}=-E/c$ だ.
    運動方程式や,エネルギー保存の法則に矛盾しないのか.次節以降で確認していこう.

\section{相対論的運動方程式}
    \begin{mycomment}
        ニュートン力学において,物体はニュートンの運動方程式
        に従っていることを学んだ.しかし,ニュートンの運動方
        程式は,空間と時間が全く別のものとして扱われてい
        る.それに加えて,空間や時間は,度の観測者から見
        ても,同一であることを仮定している.この仮定は,
        特殊相対性理論と矛盾してしまう.

        ここでは,ニュートンの運動方程式を,特殊相対性理論と
        適合するように,書きかえることが,目的である.そ
        れには,今まで3次元的に考えてきた運動量,エネル
        ギーを,4次元に拡張する作業が必要になってくる.
    \end{mycomment}

    ニュートンの運動方程式を4元運動量を使って書き換えよう.
    といっても,単純に $\bp$ を $p^{\mu}$ に機械的に置き換えるだけ.
    結論は簡単なので,先にその式を書いておこう.
    \begin{align}
        f^{\mu} = \frac{\df p^{\mu}}{\df \tau}.
    \end{align}

    この式の妥当性を考える.
    相対性理論は,物体の速度が光速に近いときに有効な理論であるが,
    ニュートン力学と矛盾なく理論を構築すべきである.そのためには,
    物体の速度が光速に比べてとても遅い場合
        \footnote{
            光速 $c$ を無限大と扱える場合.というか,そのまま数値として計算したとしても,
            光速が大きすぎて,結果にほとんど反映されない場合.$\beta=v/c$ が限りなく0に近い場合.
        },
    ニュートンの運動方程式に帰着するべきだ.つまり,
    \textbf{速度が0の場合はニュートンの運動方程式に完全に一致する}ことが条件の1つにある.
    物体の静止系 S$'$ を考える
        \footnote{
            $'$ をつける理由:$'$ をつけたのは,1つの慣性座標系に固定して考察をしたいからである.
            1つの座標系のみで考察するため,特殊な状況であることを表現しておきたいので,$'$をつけた.
            考察の最後で,ーレンツ変換を導入して,一般的な慣性座標系 S での表現に改める($'$はなくなる).
        }.
    S$'$での,時間を $t'$,位置ベクトルを$\br' := (x',\, y',\, z')$,
    物体に加えられれている力を $\bldf'$ とするとき,
    ニュートンの運動方程式は,
        \begin{equation}\label{eq:newton_eq_base}
            m\frac{\df^{2} \br'}{\df {t'}^{2}} = \bldf'.
        \end{equation}
    とである.

    特殊相対性原理によれば,運動方程式はローレンツ変換に対して不変である
        \footnote{
            これは理論構築の要請である.原理であり,根拠はない.どのような慣性系で物体の
            運動を観測しようと,運動方程式は同じ形になるはずだという,信念のもと,特殊
            相対性理論を構築する.

            工学的立場から考えれば,極論を言うと,原理が正しさはさほど重要なことではない
            (理論家からすれば,最も大事な論理的基礎なので重要視されるべきだが).
            原理から導かれた結論(今の場合は,相対性理論的な運動方程式)が現実を説明できるか否か,
            これが一番の注目点である.説明できるとなれば,原理の設定はさほど見当違いなこと
            ではなかったと考えられよう.
        }.
    とすると,$t'$ のままでは変換(の表現)が複雑になる
        \footnote{
            速度の変換公式を思い浮かべると,その複雑さを実感できる(式(\ref{eq:velocity_conversion})参照).
                \[
                    u=\frac{v+u'}{1+u'v/{c}^{2}}
                \]
        }.
    この場合,S$'$系の固有時間$\df \tau$を導入するとよい.立場をS$'$系に移すのだ.
    このS$'$系では,$x'=0$,$y'=0$,$z'=0$ だから,$\df \tau$ と $\df t'$ は等しい.
        \begin{align*}
            \df \tau &= \sqrt{1-\left(v^{2}/c^{2}\right)}\df t' \\
                 &= \sqrt{1-({x'}^{2}+{y'}^{2}+{z'}^{2})/c^{2}}\df t'\\
                 &= \sqrt{1-0}\df t'\\
                 &= \df t'.
        \end{align*}

    よって,式(\ref{eq:newton_eq_base})の $\df t'$ は $\df \tau$ で置き換えることができて,
    \begin{equation}\label{eq:newton_eq_tau_version}
        m\frac{\df^{2} \br'}{\df {\tau}^{2}} = \bldf'.
    \end{equation}

    今,$\br'$ は3次元を想定して記述したが,4次元に拡張するため,座標の時間成分を考察してみよう.
    時間成分 ${x'}^{0}$ は $ct'$ と書ける
        \footnote{
            この時点では,${x'}^{0}$ は議論のために一時的に導入する仮の変数である.しかし,この時間成分
            は相対性理論に重要になる.そのため,${x'}^{0}$ などと特殊な表現をしている.この意味は後に明
            らかになるので,ここでは単なる記号として捉えてもらえれば,それでよい.
        }.
    ${x'}^{0}=ct'$ を強引にニュートンの運動方程式に当てはめてみよう.
        \begin{align*}
            m \frac{\df^{2} {x'}^{0}}{\df \tau^{2}} &= m  \frac{\df^{2} ct'}{\df {t'}^{2}} \\
                                                    &= mc \frac{\df^{2}  t'}{\df {t'}^{2}} \\
                                                    &= mc \frac{\df}{\df t'} \left( \frac{\df t'}{\df t'} \right)\\
                                                    &= mc \frac{\df}{\df t'} \left( 1 \right) \\
                                                    &= 0.
        \end{align*}
    定数は微分すると0になる(図\ref{fig:const_graph_grad}参照)
        \footnote{
            定数は微分する変数によらず,0になる.変数をもってないから.定数のグラフの傾きはゼロだから.
        }.
    ここでは定数1を微分を計算して,0になった.
        \begin{figure}[hbt]
            \begin{center}
                \includegraphicsdefault{const_graph_grad.pdf}
                \caption{定数の微分は0である}
                \label{fig:const_graph_grad}
            \end{center}
        \end{figure}

    一つの例しか確認しておらず粗雑であるが,この時間成分 ${x'}^{0}=ct'$ はニュートン方程式と矛盾なく,整合が取れそうである.
    となれば,この時間成分と空間成分の $\br'$ をまとめてしまい,以下のように,時空座標として拡張したい.時間と空間を1つのベクトルに
    集約するのだ.ということで,座標の記号も統一感を持たせるべく,
        \[
            {x'}^{1} := {x'}, \quad {x'}^{2} := {y'}, \quad {x'}^{3} := {z'}
        \]
    と表現を改めよう.${x'}^{0}$ は時間成分で,3つの空間成分よりも前の成分として記述する.すると,
    拡張した座標(時空座標)を ${x'}^{\mu}$ と表すことにすれば,
        \begin{align*}
            {x'}^{\mu} &:= ({x'}^{0},\,\br) \\
                       &=  ({x'}^{0},\,{x'},\,{y'},\,{z'}) \\
                       &=  ({x'}^{0},\,{x'}^{1},\,{x'}^{2},\,{x'}^{3}).
        \end{align*}
    注意してほしいのは,前に使っていた空間座標の $x'$ と今導入した ${x'}^{\mu}$ は全く別物ということである
        \footnote{
            同じアルファベットを使っているせいで紛らわしいが,落ち着いて読めば誤解はしないはずだ.
            相対性理論の初頭的などの教科書はどれも,このような説明になっている.
        }.
    先に導入した時空座標が,運動方程式にも問題なく当てはめられることを確認できた.

    これに対する力 $\bldf'$ であるが,これをどのように4次元に拡張できるのかが疑問だが,
    形式的に,同じように ${f'}^{\mu}$ と書くことにしよう.${f'}^{\mu}$は時間成分1つと空間成分3つをもつ,合計4成分
    の4次元ベクトルである.
        \[
            {f'}^{\mu} = ({f'}_{t},\,{f'}_{x},\,{f'}_{y},\,{f'}_{z}).
        \]
    さらに,上付き添字を導入して,
        \[
            {f'}^{0} := {f'}_{t}, \quad {f'}^{1} := {f'}_{x}, \quad {f'}^{2} := {f'}_{y}, \quad {f'}^{3}:= {f'}_{z}
        \]
    と書き表すことにする.整理すると,
        \[
            {f'}^{\mu} := ({f'}^{0},\,{f'}^{1},\,{f'}^{2},\,{f'}^{3}).
        \]
    である.4次元空間の力を現実世界の現象としてイメージすることはできないが,理論と割り切って,抽象的に捉えておこう.

    さて,長くなったが,運動方程式を4次元に拡張して(${x'}^{\mu}$ と ${f'}^{\mu}$ を使って),
        \[
            {f'}^{\mu} = m \frac{\df^{2} {x'}^{\mu}}{{\df \tau}^{2}}
        \]
    と書けることが確認できた.4次元に拡張できたところで,やっとローレンツ変換を施すことができる形になった.
        \begin{align}
            {x}^{\nu} = \Lambda {x'}^{\mu} \\
            {f}^{\nu} = \Lambda {f'}^{\mu}
        \end{align}
    とすれば,任意の慣性系Sでの式になる.ちなみに,$\Lambda$ は以下のとおりであった(式(\ref{eq:Lorentz_trans_by_matrix})).
        \[
            \Lambda =
            \begin{bmatrix}
                \gamma       & - \gamma \beta & 0 & 0 \\
                - \gamma \beta &   \gamma       & 0 & 0 \\
                0              & 0              & 1 & 0 \\
                0              & 0              & 0 & 1
            \end{bmatrix}
        \]

    ローレンツ変換しても,式の形は変わらない.
    \begin{align*}
        {f}^{\nu} = m \frac{\df^{2} {x}^{\nu}}{{\df \tau}^{2}}
    \end{align*}

    でも添字の $\mu,\,\nu$ が気になる.今まで $\mu$ で書いてきたので$\cdots$.
    添字の記号には特に意味はないので,$\mu$ で統一しておこう.
    \begin{align*}
        {f}^{\mu} = m \frac{\df^{2} {x}^{\mu}}{{\df \tau}^{2}}.
    \end{align*}

    また,$m \displaystyle \frac{\df {x'}^{\mu}}{\df \tau}$ は4次元化された運動量 $p^{\mu}$ に他ならない.
        \[
            p^{\mu} = m \frac{\df {x}^{\mu}}{\df \tau}
        \]

    だから,運動方程式は以下のようになる.
        \begin{align}
            {f}^{\mu} &= m \frac{\df^{2} {x}^{\mu}}{{\df \tau}^{2}} \notag \\
                      &= m \frac{\df }{{\df \tau}} \frac{\df {x}^{\mu}}{\df \tau} \notag \\
                      &=   \frac{\df }{{\df \tau}} \left( m \frac{\df {x}^{\mu}}{\df \tau} \right) \notag \\
                      &=   \frac{\df p^{\mu}}{{\df \tau}}. \notag \\
            \therefore {f}^{\mu} &= \frac{\df p^{\mu}}{{\df \tau}}.
        \end{align}

\section{4元運動量の時間成分}

\section{運動する物体の質量}
        観測者に対して,光速に近い速さで運動している物体の,質量について,考える.

