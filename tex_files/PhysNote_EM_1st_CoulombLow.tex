%===================================================================================================
%  Chapter : クーロンの法則
%  説明    : 電磁気学の要である,電気力=クーロン力について説明する
%===================================================================================================
%======================================================================
%  Section
%======================================================================
    \section{クーロン力}\label{sec:CoulomnbForce}
%   %==================================================================
%   %  Subsection
%   %==================================================================
    \subsection{法則}
        電荷を帯びた物体が受ける,電気的な力というものが,世の中に
        存在する.これは万人が知っている事実だから,ここで改めて明
        記することは,バカバカしく感じられる.しかし,この電気的な
        力の存在は,大変重要なものである.この電気的な力のことを,
        \textbf{クーロン力} とよぶ.電気量の単位であるクーロンと同
        じ名前を持っているが,お察しのとおり,同一人物に由来するも
        のである.

        クーロン(Coulomb)は,電気的な力の性質を実験的に知ることに
        成功した
            \footnote{
                電気量の定義は,クーロン力を基にしてなされるもの
                である.
            }.
        そして,クーロンは,電気的な力が,次のような性質を持っていることを
        明らかにした.
        \\
        \begin{itembox}[l]{\textbf{クーロン力(クーロンの法則)}}
            ここに,電荷が2つあるとしよう.この2つの電荷は区別する
            ことができて,$q_{1}$,$q_{2}$ という電気量を持っている
            とする.電荷 $q_{1}$ と $q_{2}$ との距離を $r$ としたと
            き,この2つの電荷が受ける力は,以下のような規則がある.
            \begin{itemize}
                \item 2つの電荷の電気量が互いに異なる符号をもってい
                      るならば,両電荷は互いに引き合う向きに力を受
                      ける
                \item 2つの電荷の電気量が同じ符号を持っているならば,
                      両電荷は互いに反発しあう向きに力を受ける.
                \item 2つの電荷の受ける力の大きさは等しく,
                      向きは互いに逆向きである
                \item 2つの電荷が受ける力の大きさは,
                      2つの電荷の電気量の積($q_{1}q_{2}$)に比例する.
                \item 2つの電荷が受ける力の大きさは,
                      2つの電荷間の距離の2乗($r^{2}$)に反比例する.
            \end{itemize}
        \end{itembox}
        \\
        \begin{figure}[hbt]
            \begin{tabular}{cc}
                \begin{minipage}{0.5\hsize}
                    \begin{center}
                        \includegraphicsdouble{coulombs_low1.pdf}

                        (A)
                    \end{center}
                \end{minipage}
                \begin{minipage}{0.5\hsize}
                    \begin{center}
                        \includegraphicsdouble{coulombs_low2.pdf}

                        (B)
                    \end{center}
                \end{minipage}
            \end{tabular}
                        \caption{クーロン力}
                        \label{fig:coulombs_low}
        \end{figure}

        上に書いたような,クーロン力が示す性質のことを,\textbf{クーロンの法則} と
        いう.これは電磁気学の最も基本的な法則であり,大変重要な法則である.
        あとに説明する \textbf{電場} という重要な概念の導入も,
        このクーロンの法則を基にしている.

        言葉で書いてしまうと,ちょっとややこしいかもしれない.
        しかし,いきなり数式を出してしまうと,それはそれで
        尻込みしてしまうので,とりあえず言葉で説明してみた.

%   %==================================================================
%   %  Subsection
%   %==================================================================
    \subsection{定量化}
        では,次の段階に進み,クーロンの法則を数式で表現してみよう.
        数式で表現すると,とても簡潔になることを感じ取ることができる
        だろう.

        図\ref{fig:Coulombs_Force}のような状態であるとしよう.
        \begin{figure}[hbt]
            \begin{center}
                \includegraphicslarge{Coulombs_Force.pdf}
                \caption{クーロンの法則}
                \label{fig:Coulombs_Force}
            \end{center}
        \end{figure}

        2つの区別可能な電荷が存在し,それぞれの電気量が,$q_{1}$,$q_{2}$ で
        あるとする.また今後,これらの電荷自体を表現する場合にも,
        「電荷 $q_{1}$」などのように記述する
            \footnote{
                同じ記号に二つの意味を込めるのはよくないが,
                そうかと言って,無嫌味に記号を増やして読みづらくしたくも
                ない.ここでは,誤解を生むことがないと判断し,同じ記号で
                “電荷それ自体”と“その電気量”の2つを同じ記号で表すこと
                とする.
            }.
        この2つの電荷がある位置を,それぞれ $\br_{1}$,$\br_{2}$ と
        する.このとき,電荷間の距離 $r$ は,
            \begin{equation*}
                r = | \br_{2} - \br_{1} |.
            \end{equation*}
        また,電荷 $q_{2}$ から見た,電荷 $q_{1}$ の位置 $\br_{12}$は,
            \begin{equation*}
                \br_{12} = \br_{1} - \br_{2}.
            \end{equation*}
        同様に,電荷 $q_{1}$ から見た,電荷 $q_{2}$ の位置 $\br_{21}$は,
            \begin{equation*}
                \br_{21} = \br_{2} - \br_{1}.
            \end{equation*}

        「2つの電荷が受ける力は,大きさが同じで,向きが逆である.」これを
        数式で表すには,まず,大きさと向きを文字で表現すべきだ.
        同時に考えるのは難しいので,まずはクーロン力の大きさだけを考える.
        電荷 $q_{1}$ が受けるクーロン力 $F_{12}$ は,
        2点電荷の電気量の積 $q_{1}q_{2}$ に比例するので,数式的には,
            \begin{equation*}
                F_{12} = \alpha q_{1}q_{2}
            \end{equation*}
        とかける.ここに,$\alpha$ 比例定数である
            \footnote{
                この比例定数 $\alpha$ には全く意味が無い.単に
                比例を表すのに,便宜的に使ったに過ぎない.
                同じことがすぐ後に使う,$\beta$ についても言える.
                しかし,最後に現れる比例定数 $k$ については,
                重要であるので注意すべきだ.
            }.
         また同時に,
        「2つの電荷が受ける力の大きさは,2つの電荷間の距離の
        2乗($r^{2}$)に反比例する」から,
            \begin{equation*}
                F_{12} = \beta \frac{1}{r^{2}}
            \end{equation*}
        ともかける.$\beta$ も比例定数である.
        この2つの $F_{12}$ の式は矛盾なく両立する.
        この2つの式をまとめて,
            \begin{equation*}
                F_{12} = \alpha \beta \frac{q_{1}q_{2}}{r^{2}}
            \end{equation*}
        となる.ここで,式の見易さのため,比例定数 $\alpha \beta$ を
        改めて $k$ とおいて($k=\alpha \beta$),
            \begin{align}\label{eq:coulomb_force_f1_ookisa}
                F_{12} = k \frac{q_{1}q_{2}}{r^{2}}
            \end{align}
        とすれば,この式(\ref{eq:coulomb_force_f1_ookisa})によって,
        電荷 $q_{1}$ が受けるクーロン力の大きさを
        記述できたことになる.

        残りはその方向であるが,これは簡単だ.単位ベクトルを考えれば
        よい.一般のベクトル $\bA$ に対する単位ベクトルとは,大きさが $1$ で,
        その方向が $\bA$ と同じ向きのようなものである.このような単位ベクトルが
        存在したとして,$\bn$ と表そう.この時,$\bA$ は,$\bA=|\bA|\bn$ と書き
        表せる.つまり,単位ベクトル $\bn$ について解けば,
            \begin{equation*}
                \bn = \frac{\bA}{|\bA|}
            \end{equation*}
        である.

        今の場合に当てはめて考えれば,$\bA=\br_{12}$ であるから,
        単位ベクトルは
            \begin{equation*}
                \bn = \frac{\br_{12}}{|\br_{12}|}
                    = \frac{\br_{1} - \br_{2}}{|\br_{1} - \br_{2}|}
            \end{equation*}
        である.これが,電荷 $q_{1}$ が受けるクーロン力の向きを
        表している.

        これで,電荷 $q_{1}$ の受けるクーロン力の大きさと向きの
        数式的表現を,別々ではあるが,表現できた.あとは
        この2つを一緒に表せれば,完了となる.

        ここで改めて,電荷 $q_{1}$ の受ける
        クーロン力を向きも考慮して $\bF_{12}$ と表すこととすると,
        $\bF_{12}$ は,その大きさ $F_{12}$ と単位ベクトル $\bn$ を
        用いて,
            \begin{equation*}
                \bF_{12} = F_{1}\bn
            \end{equation*}
        とかける.
        これに,上で得た結果を代入すればよい.すると,
            \begin{align}\label{eq:coulomb_force_f1}
                \bF_{12} = k \frac{q_{1}q_{2}}{r^{2}} \frac{\br_{1} - \br_{2}}{|\br_{1} - \br_{2}|}
            \end{align}
        となる.この式(\ref{eq:coulomb_force_f1})が目標としていた,
        電荷 $q_{1}$ が受けるクーロン力 $\bF_{12}$を,
        式で表したものである.

        これ同様に,電荷 $q_{2}$ が受けるクーロン力 $\bF_{2}$を考えること
        ができるが,「2つの電荷の受ける力の大きさは等しく,向きは互いに逆
        向きである」ということを考慮すれば,直ちに,次式を得る.
            \begin{align*}
                \bF_{21} &= - \bF_{12} \\
                        &= - k \frac{q_{1}q_{2}}{r^{2}} \frac{\br_{1} - \br_{2}}{|\br_{1} - \br_{2}|}.
            \end{align*}
        ここで,
            \begin{align*}
                -(\br_{1} - \br_{2}) &= \br_{2} - \br_{1} \\
                |\br_{1} - \br_{2}|  &= |\br_{2} - \br_{1}| \\
                q_{1}q_{2}           &= q_{2}q_{1}
            \end{align*}
        であることに気付けば
            \footnote{
                数式を見れば当たり前のように感じるかもしれないが,重要な式である.というのも,
                この関係式は作用反作用の法則を表す数式にほかならないからである.
            },
            \begin{align}\label{eq:coulomb_force_f2}
                \bF_{21} = k \frac{q_{2}q_{1}}{r^{2}} \frac{\br_{2} - \br_{1}}{|\br_{2} - \br_{1}|}
            \end{align}
        となる.$\bF_{12}$ の式(\ref{eq:coulomb_force_f1}) と比較すると,
        添字の1と2が逆になっているだけであることに気づくだろう.

        最後に,比例定数 $k$ について記述しよう.
        この比例定数は,基準とする単位系によって値は
        変化するが,今日一般的に使用されているSI単位系を
        採用するならば,
            \begin{align}
                k = \frac{1}{4\pi\varepsilon_{0}} = 8.989 \times 10^{9}
            \end{align}
        である.$\varepsilon_{0}$ は真空の \textbf{誘電率} と言われる
        物理定数であるが,これについての解説は後回しにする.
        また,$\pi$ は円周率である.

%   %==================================================================
%   %  Subsection
%   %==================================================================
    \subsection{まとめ}
        以上の計算より得た結果をまとめよう.
        \begin{myshadebox}{クーロン力}
            ある空間に2つの電荷 $q_{1}$,$q_{2}$ が,それぞれ
            位置 $\br_{1}$,$\br_{2}$ に
            存在するとき,この2つの電荷には,次式で
            表されるような力が作用する.この力のこと
            を \textbf{クーロン力} という.

            電荷 $q_{1}$ に対して働く力は以下.
               \begin{align}
                   \bF_{12} =
                       \frac{1}{4\pi\varepsilon_{0}} \frac{q_{1}q_{2}}{r^{2}}
                           \frac{\br_{1} - \br_{2}}{|\br_{1} - \br_{2}|}.
               \end{align}

        ここに,$\varepsilon_{0}$ は真空の誘電率
           \footnote{
               詳細は後述.
           }
        である.
        \end{myshadebox}


        電荷 $q_{2}$ に対しては,以下の力が働く.
           \begin{align}
               \bF_{12} = -\bF_{21} =
                   \frac{1}{4\pi\varepsilon_{0}} \frac{q_{2}q_{1}}{r^{2}}
                       \frac{\br_{2} - \br_{1}}{|\br_{2} - \br_{1}|}.
           \end{align}


    \begin{memo}{(例)2つの点電荷同士のクーロン力}
    クーロンの法則を,より感覚的に分かるように,ここで,
    最も簡単な,2つの点電荷間に働く,クーロン力を考てみよう.

    存在する電荷が点電荷の場合,クーロンの法則は次式で表せる.
            \begin{align}
                \bF_{12}=\frac{1}{4\pi\varepsilon_{0}}
                \frac{q_{1}q_{2}}{|\br_{1}-\br_{2}|^{2}}
                \frac{\br_{1}-\br_{2}}
                     {|\br_{1}-\br_{2}|}.
            \end{align}
    より考えやすくするために,2次元で考えてみよう.座標系は,直交座標系とする.
    この場合,
        \begin{equation*}
            |\br_{1}-\br_{2}| = \sqrt{ {\left(x_{2} - x_{1}\right)}^{2}
            + {\left(y_{2} - y_{1}\right)}^{2} }
        \end{equation*}
    である.
        \begin{figure}[hbt]
            \begin{center}
                \includegraphicslarge{2point_distance.pdf}
                \caption{一般の2つの点の間の距離}
                \label{fig:2point_distance}
            \end{center}
        \end{figure}


    点電荷の配置を,$x$ 軸上にし,各電荷が $x=-1/2$,$x=1/2$ に存在しているとする.
    そうすると,2つの点電荷のそれぞれの位置ベクトルは,
    $\br_{1}  =  ( \,-1/2\,,\,0\, )$,$\br_{2}  =  ( \,1/2\,,\,0\, )$ となる.
    そうすると,
            \begin{align*}
                |\br_{1}-\br_{2}|
                &= \sqrt{ {\left(x_{2} - x_{1}\right)}^{2} + {\left(y_{2} - y_{1}\right)}^{2} } \\
                &= \sqrt{ {\left(\frac{1}{2} - \left(-\frac{1}{2}\right)\right)}^{2} + {\left(0-0\right)}^{2} } \\
                &= 1.
            \end{align*}
    電荷量の
    大きさは,両電荷ともに等しく,$1$[C] として考える.
        \begin{figure}[hbt]
            \begin{center}
                \includegraphicslarge{example_Coulombs_low1.pdf}
                \caption{例:2つの点電荷間のクーロン力}
                \label{fig:example_Coulombs_low1}
            \end{center}
        \end{figure}

    すると,クーロンの法則は,次のようになる.電荷 $q_{1}$ が,電荷 $q_{2}$ から
    受けるクーロン力 $\bF_{12}$ は
            \begin{align*}
                \bF_{12}
                &= \frac{1}{4\pi\varepsilon_{0}}
                \frac{q_{1}q_{2}}{|\br_{1}-\br_{2}|^{2}}
                \frac{\br_{1}-\br_{2}}
                     {|\br_{1}-\br_{2}|} \\
                &= \frac{1}{4\pi\varepsilon_{0}}
                \frac{1}{ 1 }
                \frac{\left( \,-1\,,\,0\, \right)}
                     { 1 } \\
                &= \frac{1}{4\pi\varepsilon_{0}} \left( \,-1\,,\,0\, \right)
            \end{align*}
    と書ける.

    クーロン力の向きは,$\left( \,-1\,,0\,\right)$ であることが
    分かった.

    以下では,クーロン力の大きさのみ($| \bF_{12} | := F_{12}$)を考えていこう.
        \begin{align*}
            |\bF_{12}|  &= F_{12} \\
                        &= \frac{1}{4\pi\varepsilon_{0}} \sqrt{(-1)^{2}+0^{2}} \\
                        &= \frac{1}{4\pi\varepsilon_{0}} \cdot 1               \\
                        &= \frac{1}{4\pi\varepsilon_{0}}
        \end{align*}

    最後に,$\varepsilon_{0}$,$\pi$ に具体的な数値を代入する.
    ここではとりあえず,
        \begin{equation*}
            \varepsilon_{0} = 8.854 \times 10^{-12}
        \end{equation*}
    であることが知られているので,この数値を使うことにする.
    しかし,どのようにして,このような数値が分かるかについては,
    後ほど,電磁気学をさらに学んでから,考えなおすことにしたい.
    $\pi$ は周知のように,
        \begin{equation*}
            \pi = 3.141
        \end{equation*}
    である.

    以上から,
            \begin{align*}
                F_{12}
                &= \frac{1}{4\pi\varepsilon_{0}} \\
                &= \frac{1}{4 \times 3.1415 \times 8.854 \times 10^{-12}} \\
                &= \frac{10^{12}}{ 222.483 }  \\
                &= 0.008989 \times 10^{12}  \\
                \therefore\quad
                F_{12}
                &= 8.989 \times 10^{9}
            \end{align*}
    を得る.

    実は,今までの計算は,単位電荷1[C]をもつ2つの点電荷が,1[m]離れて位置する
    場合のクーロン力を計算していた.つまり,
        \begin{equation*}
            \frac{q_{1}q_{2}}{r^{2}} = 1
        \end{equation*}
    となるのは,あたり前のことであった.しかし,
    あえて,座標から丁寧に計算したのは,点電荷がどのような位置に存在しても,
    同じように計算できることを,示したかったからである
        \footnote{
            この例はとても簡単だが,一般性が高い理論であることを認識
            することはできるはず.
        }.

    上の計算から,
        \begin{align}
            \frac{1}{4\varepsilon_{0} \pi} \simeq 9.0 \times 10^{9}
        \end{align}
    が分かる.この数値を用いて,クーロン力を表すと,
        \begin{align}
            F = 9.0 \times 10^{9} \times \frac{q_{1}q_{2}}{r^{2}}
        \end{align}
    となる.

    高校物理では,$k=9.0 \times 10^{9}$ と置いて,
        \begin{equation*}
            F = k \frac{q_{1}q_{2}}{r^{2}}
        \end{equation*}
    と書かれることも多い.
\end{memo}

%======================================================================
%  Section
%======================================================================
    \section{力の重ねあわせの原理}
        クーロン力は,力学的な力と同様に,重ねあわせの原理が成立
        していることが,実験的に確認されている.

        具体例で示したほうが,分かりやすい.
        3つの点電荷が存在する場合を考える.
        \begin{figure}[hbt]
            \begin{center}
                \includegraphicslarge{EM_Coulomb_KasaneAwase01.pdf}
                \caption{クーロン力(3つの点電荷)}
                \label{fig:EM_Coulomb_KasaneAwase00}
            \end{center}
        \end{figure}

        電気量 $q_{1}$,$q_{2}$,$q_{3}$ をもつ3つの点電荷
        の内,任意に2つを選ぶ.ここでは $q_{1}$ と $q_{2}$ をえらぼう.
        ここでは,例として,点電荷 $q_{1}$ が,他の点電荷 $q_{2}$ と $q_{3}$ から受ける
        クーロン力 $\bF_{1}$ を計算する.
        計算方法は,最初に点電荷 $q_{2}$ から受けるクーロン力 $\bldf_{12}$ を
        計算する.この時,$q_{3}$ はとりあえず存在しないとして考える
        (図\ref{fig:EM_Coulomb_KasaneAwase02}(A)).
            \begin{equation*}
                \bldf_{12} = \frac{1}{4\pi\varepsilon_{0}}
                           \frac{q_{2}q_{1}}{{|\br_{1} - \br_{2}|}^{2}}
                           \frac{\br_{1} - \br_{2}}{|\br_{1} - \br_{2}|}.
            \end{equation*}
        その次に,点電荷 $q_{3}$ から受けるクーロン力 $\bldf_{13}$ を
        計算する.この時,$q_{2}$ はとりあえず存在しないとして考える
        (図\ref{fig:EM_Coulomb_KasaneAwase02}(B)).
            \begin{equation*}
                \bldf_{13} = \frac{1}{4\pi\varepsilon_{0}}
                           \frac{q_{3}q_{1}}{{|\br_{1} - \br_{3}|}^{2}}
                           \frac{\br_{1} - \br_{3}}{|\br_{1} - \br_{3}|}.
            \end{equation*}
        \begin{figure}[hbt]
            \begin{tabular}{cc}
                \begin{minipage}{0.5\hsize}
                    \begin{center}
                        \includegraphicsdouble{EM_Coulomb_KasaneAwase02a.pdf}

                        (A)
                    \end{center}
                \end{minipage}
                \begin{minipage}{0.5\hsize}
                    \begin{center}
                        \includegraphicsdouble{EM_Coulomb_KasaneAwase02b.pdf}

                        (B)
                    \end{center}
                \end{minipage}
            \end{tabular}
            \caption{重ねあわせの原理(クーロン力)}
            \label{fig:EM_Coulomb_KasaneAwase02}
        \end{figure}

        最後に,今得た $\bldf_{12}$ と $\bldf_{13}$ を足し合わせれば,
        $\bF_{1}$ を得る(図\ref{fig:EM_Coulomb_KasaneAwase03}).
        \begin{align}
            \bF_{1} &=  \bldf_{12} + \bldf_{13}
        \end{align}
        \begin{figure}[hbt]
            \begin{center}
                \includegraphicslarge{EM_Coulomb_KasaneAwase03.pdf}
                \caption{クーロン力の重ねあわせの結果(3つの点電荷)}
                \label{fig:EM_Coulomb_KasaneAwase03}
            \end{center}
        \end{figure}

        同様に,
        \begin{align*}
            \bF_{2} &= \bldf_{21} + \bldf_{23}. \\
            \bF_{3} &= \bldf_{31} + \bldf_{32}.
        \end{align*}


%======================================================================
%  Section
%======================================================================
    \section{クーロンの法則($N$個の点電荷)}
    まず,一般的に表現する方法についての,説明する.

    いま,3つの電荷 $q_{1}$,$q_{2}$,$q_{3}$ について考えたが,
    一般的に表現したい場合には,点電荷の個数を具体的な自然数で
    表現することはできない.そこで,任意の自然数を表す記号 $N$ を
    用意する.

    さて,$N$ 個ある点電荷のうちで着目したい1つの点電荷を指したい
    場合を考える.この場合,あらかじめ $N$ 個の点電荷に番号付けを
    しておく.その上で,例えば「番号1の点電荷に着目して$\cdots$」な
    どといえば,特定の点電荷に着目できる.しかし,全ての電荷につい
    て一度に当てはまる一般的な性質を議論するときには,具体的な番号
    を指定して一つずつ議論するのは,とても効率が悪い
        \footnote{
            $N$ 個の電荷について,すべて同じ議論を繰り返すことになる.
        }.
    そこで,任意の番号を表す記号 $i$ を導入する
        \footnote{
            電流 $i$ と同じ記号だが,意味はぜんぜん違う.
            ここで用いられる記号 $i$ は添字である.
            しかし,文脈で誤解なく判断できるので,
            特に断りなく使われる.ちなみに,数学の虚数単位 $i$ も
            同じ記号だけど,コレとも全く違う意味である.
        }.
    これはよく
        \begin{equation*}
            i = 1,\,2,\,3,\,4,\,\cdots,\,N-1,\,N
        \end{equation*}
    と書かれる.「$i$ は1から $N$ までの自然数のうちのどれか」といった
    意味で用いられる書かれ方である.

    こうすると $N$ 個存在する,番号付けされた点電荷について,一般的に
    表記できる.つまり,$q_{i}$ と書くだけで,$q_{1}$ から $q_{N}$ の
    任意の一つを表現できるのである.これは実質的に,番号付けされた
    全ての点電荷を表していると解釈できる.
        \begin{figure}[hbt]
            \begin{center}
                \includegraphicslarge{EM_GenKasaneAwase.pdf}
                \caption{$N$ 個の点電荷の番号付け}
                \label{fig:EM_GenKasaneAwase}
            \end{center}
        \end{figure}

    ようやく本題に入れる.
    $N$ 個の点電荷が存在するときは,クーロン力についても,
    力の重ね合わせの原理が成立している.
    すなわち,位置 $\br_{i}$ に存在する電気量 $q_{i}$ を持った点電荷が,
    各点電荷から受ける力 $\bldf_{ij}$ の合力 $\bF_{i}(\br_{i})$ は次式
    で表現される.

    しかし,ここで注意が必要である.気が付いているだろうか.
    $j=i$ の場合に,どうなるかを考えてみただろうか.$j=i$,
    つまり,クーロン力が $\bldi{f}_{ii}$ と表されることになり,
    これは電荷自分自信から受ける
    クーロン力を表す.クーロンの法則は,あくまでも2つの点電荷から
    なる系についての法則である.そこには,1つの電荷がそれ自身に与える
    クーロン力というものは,説明されていない.
    なので,ここでは,$j=i$ の場合を除おくことにしよう
        \footnote{
            しかし,クーロンの法則に1つの電荷が自身に与える
            クーロン力について,何も説明されていないからとい
            って,それが生じないと結論されるわけではない.
            実際,これは「自己力」として,よく取り上げられる
            問題である.古典的な電磁気学(量子力学的でない電磁気学)
            では,この自己力は $\bld{0}$ となることが説明できが,
            これについては,後ほど考えることにしたい.
        }.
        \begin{myshadebox}{クーロンの法則($N$個の点電荷)}
            点電荷が $N$ 個存在するとき,この点電荷に適当に番号付けをする.
            この時,番号 $i$ の点電荷にかかるクーロン力は,次式で表せる.
            \begin{align}
                \bF_{i}(\br_{i})&=\sum_{j=1}^{N}\bldf_{ij} \notag \\
                &=\sum_{j=1,j\neq i}^{N}\frac{1}{4\pi\varepsilon_{0}}
                \frac{q_{i}q_{j}}{|\br_{i}-\br_{j}|^{2}}
                \frac{\br_{i}-\br_{j}}
                     {|\br_{i}-\br_{j}|}
            \end{align}
            ここで,$ \displaystyle \sum_{j=1,j\neq i}$ という表現は,
            $j=i$ の場合のみを除いた,$j=1$ から $N$ までの総和を意味する
        \end{myshadebox}

        \begin{figure}[hbt]
            \begin{center}
                \includegraphicslarge{EM_CoulombN.pdf}
                \caption{クーロンの法則($N$個の点電荷)}
                \label{fig:EM_CoulombN}
            \end{center}
        \end{figure}

        \begin{memo}{和の記号: $\sum_{j=1,j\neq i}^{N}$ の注意}
            例えば,$i=3$ 番目を考えると,
                \begin{equation*}
                    \sum_{j=1,j\neq i}^{N} 2j
                    = 2 \cdot 1 +  2 \cdot 2 + 2 \cdot 4 + \cdots + 2 \cdots N
                \end{equation*}
            と展開される.3番目の項が,無いことに注目してもらいたい.

            さらに,計算開始の番号が $j=1$ であることが,明らかな場合,
            省略して,
                \begin{equation*}
                    \sum_{j\neq i}^{N} 2j
                    = 2 \cdot 1 +  2 \cdot 2 + 2 \cdot 4 + \cdots + 2 \cdots N
                \end{equation*}
            のように書かれることもある.
        \end{memo}


%======================================================================
%  Section
%======================================================================
    \section{クーロンの法則(電荷の連続分布)}
    電荷が連続分布しているならば,電荷密度 $\rho(\br^{*})$ で考えるほうがよい.
    このとき,和の記号は積分記号に変わる.また,点電荷が連続
    分布していることから,その位置 $\br_{i}$ を示すのではなく,
    任意の位置を示す必要がある.そこで,添字 $i$ を取り払って,
    $\br$ と表すことにする.以下の式は,位置 $\br$ でのクーロン力を
    表す.さらに,積分するときの変数記号を,$\br^{*}$ で表す.

    すると,電荷が連続分布する場合のクーロンの法則は,以下のように
    表現できる.
    \begin{myshadebox}{クーロン力(電荷の連続分布)}
        \begin{align}\label{coulomb'slow2}
            \bF(\br)
            &=\int_\Omega \frac{1}{4\pi\varepsilon_{0}}
            \frac{q\rho(\br^{*})}{|\br-\br^{*}|^{2}}
            \frac{\br-\br^{*}}
                 {|\br-\br^{*}|}\df V^{*}.
        \end{align}
    \end{myshadebox}

    ここで,積分はアスタリスク記号「$^{*}$」ついたものについて行う
        \footnote{
            アスタリスク(Asterisk): 記号の名前.「アステリスク」とも言われる.
            「アステリスク」という呼び方は,コンピュータ関係の技術者によく使われる
            (そのままローマ字読みすると,そうなる).本ノートでは,「アスタリスク」と
            記述していこう.
        }.
    また,式の $\Omega$ は任意の領域である.これが,電荷が連続分布し
    ている場所における,電気量 $q$ の電荷が受ける力である.
        \begin{figure}[hbt]
            \begin{tabular}{cc}
                \begin{minipage}{0.5\hsize}
                    \begin{center}
                        \includegraphicsdouble{EM_CoulombRho.pdf}

                        (A)
                    \end{center}
                \end{minipage}
                \begin{minipage}{0.5\hsize}
                    \begin{center}
                        \includegraphicsdouble{EM_DenkamdV.pdf}

                        (B) [再揚(図\ref{fig:EM_DenkamdV})]
                    \end{center}
                \end{minipage}
            \end{tabular}
            \caption{クーロンの法則(電荷の連続分布)}
            \label{fig:EM_CoulombRho}
        \end{figure}


%======================================================================
%  Section
%======================================================================
\section{クーロン力(電気的な力)と力学的な力}
    クーロンの法則:
        \begin{align*}
            \bF(\br)
            &=\int_\Omega \frac{1}{4\pi\varepsilon_{0}}
            \frac{q\rho(\br^{*})}{|\br-\br^{*}|^{2}}
            \frac{\br-\br^{*}}
                 {|\br-\br^{*}|}\df V^{*}
        \end{align*}
    の左辺の力はニュートン力学
    で導入した力学的な力のことであり
        \footnote{
            力学的な力とは,運動方程式で導入される力を指している.
        },
    電気的な力ではない.それに対して,右辺は,クーロンの法則によって示される電気的な力である.
    要するに,右辺の式で示される力と左辺で示される力は,
    定義がことなるものである.この式の等号は,右辺と左辺の
    種類の異なる力が物理学的に等価に扱えることを示すものである.
    右辺のクーロン力の原因は電荷だから,電気的な力であるが,
    電気的な力を直接的に測定することはできない
        \footnote{
            ニュートン力学で導入た力は,例えば,天秤やバネ秤を使って測定できる.
            しかし,クーロン力は直接測定する方法がない.
        }
    .だから,
    測定のできる力学的に力に換算して,電気的な力を表現するのであ
    る.

    クーロンの法則の前提条件として,「固定されてる点電荷にはたら
    く力」というものがある.この条件は,クーロン力によって点電荷
    が運動しないように設定した条件である.力を受けている物体は加
    速度運動するということが,ニュートン運動方程式の意味するとこであ
    った.クーロン力を受けている電荷は固定されていなければ加速度
    運動してしまうのである.だから,固定されているという条件をつ
    けたのである.固定されているということは,クーロン力を受けな
    がら静止しているということである.従って,クーロン力と釣り合
    う外力が働いていることになる.クーロンの法則を確認するには,電
    荷の電気量や,電荷間の距離をいろいろ変えてみて,そのときに電荷
    を固定するのに必要な外力を測定すればよい.
