%===================================================================================================
%  Chapter : 電磁気学が対象とする物理現象
%  説明    : 電磁気学の対象となる,電気的現象・磁気的現象の実在についての確認をする
%===================================================================================================

%======================================================================
%  Section
%======================================================================
    \section{はじめに}\label{sec:EM_ObjPreMsg}
        \begin{mycomment}
            本節では,以降の電磁気学への導入と,これからの電磁気学の
            学習の段取りを説明する.
        \end{mycomment}

    %======================================================================
    %  SubSection
    %======================================================================
    \subsection{電気と磁気}\label{subsec:ElcAndMgn}
        これから学習する電磁気学は,\textbf{電気} と \textbf{磁気} が起こす
        様々な現象を説明する物理学の分野の1つである.
        電気とか磁気とかというと,日常的に用いられている使い慣れた言葉であり,
        そのイメージも多くの人々が共有している.「何を今更」と思われるかもしれないが,
        今まで日常的に使われてきた「電気」とか「磁気」とか
        という言葉と,そのイメージを一度整理しておきたい.
        これから学習する電磁気学は,電気とは何か,あるいは,磁気とは何かを
        探求するものであり,その疑問となる根本的な現象についてを確認せずに
        話を進めるには,甚だ滑稽なことだろう.

    %======================================================================
    %  SubSection
    %======================================================================
    \subsection{電気と磁気の伝わり方}%\label{subsec:ElcAndMgn}
        私達の感じてる電気や磁気は,実際には,静電気の引力
        (あるいは斥力)というように,力として現れている.つまり,力の伝達の
        表現方法が必要になってくる.
        “力の伝達”と聞いて,なんのことだ? と思ったかもしれない.
        というのも,ニュートン力学では,力の伝播については無視していたからである.
        そこでは,暗黙の了解として,「力は瞬間的に,つまり,時間0で伝わる」
        ということが仮定されていたのだ.しかし,このような考え方では,無限に遠くに存在する
        物体にも,近隣に存在する物体にも,力が "同時" に伝わる事になってしまう.
        これは直感に反するのではないだろうか.近くにある物体には,すぐに力が
        伝わり,遠くにあるものほど力の伝わり方が遅いというように,
        力の伝播に時間がかかるとしたほうが,自然な考え方では
        なかろうか.まあ,何れにしても,実験的に確かめないといけないところ
        だが,現在では,一般相対性理論で説明されるとおり,力の伝播には,時間
        がかかることがわかっている.つまり,力が一瞬で伝わると仮定されている
        ニュートン力学は,この部分において,間違っている.その間違いの修正は
        追々やっていくとして,ここでは,力の伝播をどうやって式で表現できるか
        を考えないといけない.
        そこで,力の伝播を表現するための概念である,\textbf{場} という考え方が
        導入される
            \footnote{
                小言を言っておこう.

                「場」という考え方は自然な考え方であるが,概念が抽象的すぎて,
                なかなか初学者にとって受け入れ難いことだろう.
                しかし,我慢して欲しい.「場」という考え方は,これからいっそう重要に
                なってくる.現代の物理学は,「場」という考え方が理論構築の基礎になっているからである.
                はじめに「場」ありき,という考えが一般的なのだ.
                理由はわからないが,そういう考え方のもとで,理論構築をし,成功を収めている.

                この部分は,"自然だけども特殊な考え方"なので,
                最初学習する上でつっかえる所だろう.理解するのに,
                時間がかかってしまうが,悩まずに,学習を進めていってもらいたい.
                数式とそのイメージをリンクさせようともがきながら
                (いろいろ考えたり,ヒントとなる本,Webサイトを探したりしよう)
                学習を進めれば,いつの間にか,「場」という考え方に慣れて
                (毒されて?)しまうものである.

                こんなことをここで言うな,と言われるかもしれない.実際そのとおりで,
                あとでまた同じ事を言うことだろう.ここでは,“この先に困難がありますよ”
                という案内として記述したまでである.
                (RPG風に言うなら,「(王様の台詞) 勇者よ,お前はこれから多くの困難に直面
                することだろう.しかし,それに屈してはならない.どんな困難だろうとも,
                それに立ち向かい,解決せねばならない.いかなる困難も克服し,壮大な目標に
                向かって,前進するのだ.行け,勇者よ,そして,いつの日か目的を果たし,
                帰還するのだ.」といった感じだ.)
            }.

    %======================================================================
    %  SubSection
    %======================================================================
    \subsection{電気・磁気の研究の歴史(ダイジェスト)}
        電磁気学で着目する力は,\textbf{電磁気力} と言われる.これは,
        \textbf{電気的な力} と \textbf{磁気的な力} の両方を指す言い方
        である.電気的な力や磁気的な力の存在自体は,摩擦時に起こる静電
        気や磁石の存在から,古くから知られていたことだろう.
        しかし,その力の持つ性質を科学的に扱うことができるのは,16世紀
        になってからである.ギルバート
            \footnote{
                William Gilbert(1544--1603, イギリス):Gilberd と
                表現されることもある.電磁気現象を近代的な実験方法で
                研究した,最初の人物のひとりとして有名である.検電器
                を発明している.医者としての仕事の傍ら,電磁気の研究
                をしていたらしい.
            }
        による電磁気現象の研究が,電磁気学の幕開けとするのが通説のようである.
        しかし,より正確に電磁気現象が扱えるようになるのは,
        キャベンディッシュ
            \footnote{
                Henry Cavendish(1731--1810, イギリス):化学と物理学
                の研究で有名.人間嫌いであったり,研究した結果を秘密
                にしておいたりと,特異な性格を強調されることが紹介さ
                れる言が多い(あ,ここにも書いてしまった).地球の
                比重を測定したことでも有名.これにより,万有引力の存
                在の裏付け,並びに,地球の重力定数の測定がなされた.
                また,電磁気学に関して言えば,クーロンの法則をクーロ
                ンよりも前に発見したことが,キャベンディッシュ死後に,
                マクスウェル(※1)より明らかにされている.

                (※1)James Clerk Maxwell(1831--1879, イギリス).
                \ref{subseq:4fundlaw_Hajimeni}節の脚注を参照.
            }
        やクーロンが電気的な力のもつ性質を実験的に解析する18世紀ごろである.
        アンペール
            \footnote{
                Andre-Marie Amp\'{e}re(1775--1836, フランス):電流に関
                する実験で有名.電流の単位「アンペア[A]」は彼の名にちなん
                だものである.
            }
        による電流と力の関係
            \footnote{
                この関係は,アンペールの法則と言われる.詳しいことは,後述する.
            }
        の発見や,ビオ
            \footnote{
                Jean-Baptiste Biot(1774--1862, フランス):物理学者であり,
                数学者,天文学者でもある.後に述べる,ビオ$=$サバールの法則
                の発見者のひとりとして有名.大気圧の測定も行っていたらしい.
            }
        とサバール
            \footnote{
                F\'{e}lix Savart(1791--1841, フランス):ビオと共同で,
                ビオ$=$サバールの法則を提唱したことで有名.カタカナ表記では,
                「サヴァール」と書いたほうが,正確なのかもしれない(だけど,
                このノートでは「サバール」と書くことにしたい.こっちのほうが
                見慣れたカタカナので,つっかえることなく読めると思う).
                外科医でもあったらしい.また,現在の音程の単位はセント(1
                オクターブ$=$1200セント)であるが,それ以前の単位として,
                サバール(savart)が使われていた.ちなみに,音程1サバールは
                だいたい4セントくらいである(なので,だいたい1オクターブは
                300サバールということになるのか).
            }
        の磁気と電流の関係の研究もだいたい19世紀初期に行われていている.最終的な
        電磁気学の確立がマクスウェルによってなされるのが19世紀中頃(1864年)である.

        電磁気現象は古くから知られていたのに対し,その現象を科学的に
        扱えるようになるのは,19世紀中頃になってからであった.そもそも科学という
        考え方自体が,ルネッサンス期に芽生えたものとされているので,仕方がないの
        かもしれないが,それにしても,電磁気現象を人間が把握するのに,これだけの時間が
        かかっているのには驚きである
            \footnote{
                話がそれるが,今日ある私たちの生活環境は,
                パソコンや携帯電話など,電磁気学を応用して作り
                出されている.そう,私たちの掌の中には,それだけの
                研究の重さがのしかかっているのである.ただ持っているだけでは
                なんにも感じないけど,少し学習すると,それらの機器を見たとき,
                先人の研究努力に対し,感謝の気持ちを懐くことだろう.
            }

%======================================================================
%  Section
%======================================================================
    \section{電気的現象}
        世の中には,接触していないにもかかわらず,力を受けることがある.
        この非接触で感じる力の中に,\textbf{電気的な力} がそのひとつとして
        存在する.

        例えば,髪の毛を下敷きでこすり,その後すぐに下敷きを頭の上の方へゆっ
        くりと持ち上げてみると,髪の毛は下敷きに吸いつけられるかのように持ち
        上がる.この現象を,一度は,小学生のころに実験や遊びで経験したことと思う.
        \textbf{電気的現象} の一例として,頻繁に頻繁にあげれる現象だ.
        この現象を物理学的には電磁気学で説明される.
        特に,静電気学として語られることが多い.
        静電気学は,電磁気学でも最も基本となる考え方である.
        になる.

%======================================================================
%  Section
%======================================================================
    \section{磁気的現象}
        非接触的に受ける力の別の例として,\textbf{磁気的な現象} も考えられる.
        鉄などの特定の金属をひきつける
            \footnote{
                ひきつける:相対的に考えれば,「引き寄せられる」といっても同
                じこと.
            }
        石がこの世界には存在し,日本では \textbf{磁石} と呼ばれている.これ
        は電気的な現象とは異なる原因から生じる.この磁気的な現象についても,
        後ほど詳しく考えることになる.

%======================================================================
%  Section
%======================================================================
    \section{電磁気的現象}
        電気的現象と磁気的現象は,その発生原因は異なるのだが,それらの振
        る舞いはとても似ている部分が多い.このことから,電気的現象と磁気的現
        象は密接な関係があるが容易に想像され,実際にあとで示す通り,
        この予想は正しい.
        両者を共に扱う場合,これらの現象をひっくるめて,\textbf{電磁気的現象} と
        よぶ.

        電磁気的現象の例として,\textbf{電磁波} という,物理現象がある
            \footnote{
                そして,これの例に尽きる.
            }.
        携帯電話や無線LANに代用される無線通信機器は,この電磁波を利用している.

        第一段階の電磁気学の学習目標は,電磁気的現象を数式で表現することである.


    \begin{memo}{非接触的な力}
        物体に触れることなしに与える力を,非接触的な力という.
        非接触的な力は磁石などでも馴染みがあり,馴染みのある現象
        だ.しかし,よく考えてみると,不思議な現象だ.
        触っていないのに力が伝わるのである.このような現象を見て,
        どのようにこの力の伝達を説明できるだろうか.
        私達の直感では,物体に触れていないのに力が伝わるということを理解し難い.
        しかし,現実に磁石は存在して,非接触的な力が目の前で起こっている.
        どうしたことだろうか.物理学者はこの不可解な非接触的な力を説明すべく,
        \textbf{場} という概念を発案した
            \footnote{
                英語で言うと,Fieldである.
            }.
        物体が力を受けるということの原因はその周囲の場の歪みであると解釈せよ,
        というのである.

        いきなり場という考え方を提示され,わけわからん状態に陥ってるかもしれない.
        しかし,安心してほしい.場という概念は,誰にとっても,言葉で説明されただけでは理解し難いものだ
            \footnote{
                物理の教科書を書いている偉い先生も,場という概念を理解するのに
                苦労した経験があるそうだ.
            }.
        これからの物理学の学習(演習)を続けることで,言葉だけでなく,感覚的にも
        理解できるようになるだろう
            \footnote{
                場という概念は非常に抽象的(数学的)であり,実際にその存在を示すことはできない.
                だから理解し難いし,初めのうちは胡散臭く感じるのだが,学習を進めることでそれなしでは
                物理学を構築に欠かせない概念であることを悟るだろう.
            }.
    \end{memo}



