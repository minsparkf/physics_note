%===================================================================================================
%  Chapter : その他の電磁気学的現象
%  説明    : その他の電磁気現象を学習する.
%===================================================================================================
    %=======================================================================
    %  Section
    %=======================================================================
    \section{電気双極子}

    %=======================================================================
    %  SubSection
    %=======================================================================
    \section{熱電効果}
        %===================================================================
        %  SubSection
        %===================================================================
        \subsection{はじめに}
            ここでは,熱と電磁気とに関係する現象を考える.\textbf{熱電効果} と
            は,熱によって電流を生じさせる現象のことをいう.例えば,一様な物体
            があったとして,この物体の一部に熱を与えて,周囲よりも高温にしてみ
            る.この時,物体には温度勾配が生じる.熱い部分のエネルギーは冷たい
            う部分よりも高いので,当然,熱した部分の原子の持つ電子も,冷たい部
            分の電子よりも活発に運動していることだろう.活発な電子はその周囲に
            拡散することが容易に想像できる.この電子の拡散こそが,電流であり,
            熱電効果とよばれる理由である.

        %===================================================================
        %  SubSection
        %===================================================================
        \subsection{トムソン効果}\label{subsec:ThomsonEffect}

        %===================================================================
        %  SubSection
        %===================================================================
        \subsection{ペルチェ効果}

        %===================================================================
        %  SubSection
        %===================================================================
        \subsection{ゼーベック効果}

    %=======================================================================
    %  Section
    %=======================================================================
    \section{ゼーマン効果}\label{subsec:ZeemanEffect}
        原子にある程度エネルギーを与えると,その原子内の電子がエネルギー
        を吸収する.そして,電子のエネルギーがある一定値を超えると,エネル
        ギーを抱えきれなくなり,電磁波としてそれを放出する.このとき放出さ
        れる電磁波の波長は原子ごとに決まっている.一般に,放出される電磁波
        の波長は複数である.つまり,原子はエネルギーを外部から与えると,決
        まったいくつかの波長の電磁波を放出する.この原子が出す複数の波長の
        電磁波のことを,\textbf{原子スペクトル} という.また,そのひとつひ
        とつを \textbf{原子スペクトル線} という.

        単一の波長しか放出しない原子
            \footnote{
                スペクトル線をひとつしか持たない原子のこと.
            }
        も存在する.しかし,磁束密度中でエネルギー
        を与えた場合,それが複数種類の電磁波を放出するようになる
            \footnote{
                スペクトル線が2つ以上に分裂するということ.
            }.
        この現象を \textbf{ゼーマン効果} という.

    %=======================================================================
    %  Section
    %=======================================================================
    \section{電子の実験的発見}
        電子は,どのようにしてその存在が認められたのだろうか.
            \begin{figure}[hbt]
                \begin{center}
                    \includegraphicslarge{ThomomExpElec001.pdf}
                    \caption{トムソンの陰極線の実験}
                    \label{fig:ThomomExpElec001}
                \end{center}
            \end{figure}

