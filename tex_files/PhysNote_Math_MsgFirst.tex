%===================================================================================================
%  Chapter : はじめに
%  説明    : 数学の学習マップを記述する.動機付けも含めて.
%===================================================================================================

%   %==========================================================================
%   %  Section
%   %==========================================================================
        \section{物理学と数学}
            物理学の勉強を始めようとしているのに,なぜその前に数学を勉強しなければ
            ならないのか.
            あるいは,なぜ,物理学の学習と,数学の学習を同時に進めないといけないのか.
            それは,物理学では数学を道具として扱うからである.その道具の使い方を
            知らずして,物理学は学習できないのである.

            もちろん,近代物理学の創始者ニュートン
                \footnote{
                    Sir Isaac Newton(1643--1727, イギリス):古典力学の創始者.
                    微分法という就学的手法を発明し,物体の運動を記述することがで
                    きるようになった.彼の名をとって,「ニュートン力学」とよばれ
                    ることも多い.また後で,ニュートン力学の部分において,紹介す
                    ることになろう.
                }
            の時代には,物理学のための数学なんてものはなく,物理学の研究に伴って,
            その数学的手法である微分積分学を作る必要があった.数学的手法の発明時
            には記号の統一ができていなかったり,不明確な問題も多いものである.そ
            してこれらの問題は,後の時代の多くの数学者によって解決され,その表現
            方法もよりわかりやすい形に書き改められ,きれいな体系に整えられていく.
            そのため,ニュートンの理論の数学的表現は現在とは全く異なる.現在の物
            理学で用いられる数学は,代数学や微分積分学がある
                \footnote{
                    特に,微分積分学はニュートンが物理学を構築するために発明され
                    た数学的手法である.(ライプニッツも同時期に微分積分学を発明
                    している.)微分積分学の発明当時には,極限の定義が曖昧であっ
                    たり,使用される記号も分かりにくいものであった.
                }.
            現在の物理学は,これらが当たり前のように使われる.いわば,物理学を記
            述するために欠くことのできない言語なのである
                \footnote{
                    物理学は数学の助けを借りて成立している.
                }.
            だから,物理学を学習する前に,まずは最低限の数学を学習する必要がある.

            物理学を学んでいく途中で,必要になったらその数学を学ぶという方法もあ
            るが,ある程度の数学的知識を予め身に着けておいた方が,学習するのに効
            率的である.「ある程度」を見極めるのは難しいが,ここではさしあたり,
            高校レベルの数学がわかるくらいを目標に,数学を学ぶ.


%   %==========================================================================
%   %  Section
%   %==========================================================================
        \section{「数学的準備」の学習マップ}
            このPertで学習する数学について,簡単に触れておこう.まず,内容を列挙
            してみよう.
                \begin{enumerate}
                    \item 関数
                    \item 微分積分学
                    \item 微分方程式
                    \item ベクトル
                    \item ベクトル解析(ベクトルの微分積分)
                    \item 行列
                \end{enumerate}

            まず,「関数」を学習する.中学校では,1次関数や2次関数に触れたこと
            と思う.ここでは,これらをもう少し一般的に考える.物理学ではこの関数
            がメインとなる.関数により,自然法則を記述するからである.

            次に,「微分積分学」を学習する.ただし,深入りはしない.図形的直感を
            第一に考える.物理学は物体の運動や挙動を数式で表現することがその目的
            のひとつである.つまり,物体の位置の時間的な変化を記述できなければい
            けない.この時間変化を表現する最善の手段として,ニュートンは微分積分
            法という新しい数学分野を開拓した.

            その次に,「微分方程式」について学習する.微分方程式とは,微分積分法で
            定義される微分を含む数式のことである.通常の方程式では,未知変数 $x$ や $y$ が
            方程式を満たす値を求める.これに対して,微分方程式で,未知変数に対応する
            のが,未知関数である.微分方程式を解くということは,その微分方程式を
            満たす関数の具体的な形を求めることにほかならない.

            その次に,「ベクトル」について学習する.ベクトル
            は,物体の位置を数学的に表現するものである.これも図形的イメージ習得
            を第一にする.

            その次に,「ベクトル解析」を学習する.ベクトル解析は,ベクトルに微分積
            分学を組み合わせたものであるとも言えよう.ベクトルは物体の位置を表現
            するものであり,微分積分学は時間的変化を記述するものである.つまり,
            ベクトルに対して,微分積分を適用すると,物体の位置の時間変化を数学的
            に扱うことができるようになるのだ.

            その次に,「行列」を学習する.行列とは,何個かの数を縦横に並べてひとつの
            組として扱われるものである.行列の概念は,言葉で説明するよりも,具体例を
            示したほうが分かりやすいだろう.以下は,行列の具体的な例である.
            \begin{equation*}
                \begin{bmatrix}
                    a_{00} & a_{01} \\
                    a_{10} & a_{11}
                \end{bmatrix}
                \;\;\;,\,\;
                \begin{bmatrix}
                    a_{00} & a_{01} & a_{02} \\
                    a_{10} & a_{11} & a_{12}
                \end{bmatrix}
                \;\;\;,\,\;
                \begin{bmatrix}
                    a_{00} & a_{01} & a_{02} \\
                    a_{10} & a_{11} & a_{12} \\
                    a_{20} & a_{21} & a_{22}
                \end{bmatrix}
            \end{equation*}
            これらについての,具体的な性質について考えることは,行列という数学分野である.
            しかしここでは,行列というものの具体的な形を紹介しただけにとどめておこう.
            具体的な定義や性質は,後で考えることにする.
