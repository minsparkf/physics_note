%   %==========================================================================
%   %  Section
%   %==========================================================================
    \section{量子力学の教科書}
        \subsection{説明方法}
            多くの量子力学の教科書は次の2つの種類のうちのどちらかである.
            \begin{itemize}
                \item 歴史的なこと(量子力学の由来)から説明していくもの
                \item 量子力学の本質を,体系立てて説明するもの
            \end{itemize}
            以下で,その長所と短所を説明した後に,このノートの量子力学の
            学習方針を示す.

        \subsection{歴史的記述から入る教科書}
            \begin{mysmallsec}{長所}
                歴史的なことから記述されている教科書は,学習の動機とか,量子力学の必要
                性とかをつかむことができる.
            \end{mysmallsec}

            \begin{mysmallsec}{短所}
                本論に入るまでには歴史的記述が長く続くことになるので,
                量子力学の本論に到達するまでに時間が掛かってしまう.
                志が低いと,本論に入る前に挫折してしまうことがある
                \footnote{
                    単位取得のみを目的とする大学生や,
                    趣味で勉強しようとする人にとっては,つらいのだ.
                    (私は,昔は前者であり,今は後者である)
                }
            \end{mysmallsec}

        \subsection{いきなり体系から入る教科書}
            \begin{mysmallsec}{長所}
                量子力学の本質から入る教科書は,はじめから本質から入るので時間は掛からない.
            \end{mysmallsec}

            \begin{mysmallsec}{短所}
                量子力学の必要性がよくつかみにくくなってしまう.
                また,量子力学特有の抽象的な概念が容赦なく押し付けられるため,
                お手上げ状態になる可能性が高い.
            \end{mysmallsec}

        \subsection{このノートの学習方針}
            このノートは,量子力学を以下の手順で学習する.
            \begin{mysmallsec}{step1. 数式の少ない入門書を読む}
                \begin{itemize}
                \item ブルーバックスなど一般向けの量子力学の入門書が多く出版されている.
                      まず,これにより量子力学の概要をつかむ.
                \item この段階ではWebを参照しないほうが良い.
                      なんの学習をするにしても,Webからの情報を得ることは不可欠だが,
                      誤っている記述も散見されるため,注意が必要.
                      気になって見たとしても,信じこまないように.
                \end{itemize}
            \end{mysmallsec}

            \begin{mysmallsec}{step2. 易しい入門書を読む}
                \begin{itemize}
                \item 易しく記述されている量子力学の教科書を読む.
                      数式を用いた説明がなされてはいるが,
                      本当の初学者向けに書かれている独習書を選ぶ.
                      "高校数学だけでわかる"的なものがいい.
                \end{itemize}
            \end{mysmallsec}

            \begin{mysmallsec}{step3. 標準的な教科書を読む}
                \begin{itemize}
                \item 最後に,原島や砂川,小出,シッフなどの,標準的な教科書を読む.
                      教科書を2,3冊つ選び学習したいところだ.
                      ただし,学習の軸とする教科書を1つに定めてこれを基本に学習を進め,
                      他は副読的に読むこと.2,3冊を同時に読み込もうとしないこと.
                      読み込むのは1冊で十分.
                      基本とする教科書で,内容が足りなかったり,わかりにくい記述が
                      あった場合に,副読用の教科書を参照するという方法がいい.
                \item 物理学の解説Webサイトの参照も有用である.
                      ただし,その記述を読む際には,批判的であること.
                      Webの情報の多くは,他人の検閲が行われないため,
                      執筆者の個人的なの考えや感想である.
                      大学の教授によって,その大学の名(研究室名)の下に書かれていれば,
                      その内容の信頼性は高いが,個人サイトの閲覧は要注意である.
                \item さらに,この段階で,量子力学の学習と並行して,\textbf{解釈問題} など
                      の哲学的問題にも触れておきたい.解釈問題は理解しなくともよいが,
                      理論の基本思想が複数あるということを知っておくべきである.
                      この解釈問題を追っていくと,\textbf{多世界解釈} といった話題に
                      つながっていく.
                \item また,\textbf{因果律の捉え直しが必要} であったり,観測行為が
                      実験結果に強く影響を与えてしまうと問題(\textbf{観測問題})が
                      あることを認識しておきたい.
                      観測問題は量子力学の基本原理の1つである \textbf{不確定性原理} に
                      関連した問題だ.
                \end{itemize}
            \end{mysmallsec}

            \begin{mysmallsec}{step4. もっと難しい教科書を読む}
                \begin{itemize}
                \item ディラックやノイマン,ランダウ$=$リフシッツなどの,
                      難しい教科書に挑戦する.あるいは,本屋や図書館で,
                      その内容をチラ見する.
                \item 量子力学の難しさを痛感することが目的だ.
                      教科書の内容を理解することが目的ではない.
                \item 万が一,更に学習を進めたいという向学心に目覚めた場合に,
                      挑戦すべき教科書を予め定めておくのもよいことだと思う.
                      標準的な教科書を読んで,量子力学をわかった気になってはいけない.
                \end{itemize}
            \end{mysmallsec}

        \subsection{使用する教科書}
            このノートにおける量子力学の教科書として,以下を使用する.
                \begin{itemize}
                    \item (step1) J.C.ポーキングホーン 著,宮崎 忠 訳,『量子力学の考え方』,講談社ブルーバックス
                    \item (step1) 山田 克哉 著,『量子力学のからくり --- 「幽霊波」の正体』,講談社ブルーバックス
                    \item (step2) 竹内 淳 著,『高校数学でわかるシュレーディンガー方程式』, 講談社ブルーバックス
                    \item (step2) 都筑 卓司 著,『なっとくする量子力学 』,講談社
                    \item (step3) 中嶋 貞雄 著,『量子力学 \I / \II 』(物理入門コース5/6),岩波書店
                    \item (step3) 前野 晶弘 著,『よくわかる 量子力学』,東京図書
                    \item (step3) 佐川 弘幸,清水 克多郎 著,『量子力学(第2版)』,丸善
                    \item (step3) コリン・ブルース 著,和田 純夫 訳,『量子力学の解釈問題』,講談社ブルーバックス
                \end{itemize}

%   %==========================================================================
%   %  Section
%   %==========================================================================
    \section{量子力学での運動方程式}
        \subsection{はじめに}
            上に書いたように,原子のような非常に微小な世界では,ニュートン力学が成り立たなくなってしまう.つま
            り,原子や電子,光子の運動はニュートンの運動方程式では表現できない.これは実験事実であって,認める
            よりしかたがない.しかし,ニュートンの運動方程式に従わないからといって,デタラメに運動しているわけ
            ではない.何か“別の運動法則”に従って運動しているのである.

        \subsection{量子力学での運動方程式}
            \subsubsection{運動方程式は2種類の表現方法がある}
            では,どのような運動法則に従っているのだろうか.この疑問に答えを見つけたのが,ハイゼンベルクである.
            シュレディンガーもハイゼンベルクと独立に,一年遅れだが答えを見つけている.
            ハイゼンベルクの見つけた運動方程式は,行列という数学の形式を用いて表現されるものだった.
            発見当時は,科学者は行列という形式に不慣れだったため,ハイゼンベルクの
            発見はあまり目立ったものではなかった.ハイゼンベルクが行列形式の運動方程式を発見した一年後に,
            シュレディンガーは偏微分方程式を用いた形式の運動方程式を発見した.当時の科学者にとって,偏微分方程式は
            研究に欠かせいない道具であったので,シュレディンガーの発見した方程式はすぐに有名になった.
            もちろんハイゼンベルクの発見した運動方程式が間違っていたのではない.
            ハイゼンベルクとシュレディンガーが発見した2つの運動法則
            は全く同等であることは,シュレーディンガーによって証明されている.違いは,単に,数学形式だけである.
            なので,量子力学には2種類の記述方式があるということになる.
            一つは,ハイゼンベルクの行列を用いた表現方法で,\textbf{行列力学} とよばれ,
            その方程式は \textbf{ハイゼンベルク方程式} といわれる.
            もう一つは,シュレーディンガーの偏微分方程式による表現方法で,\textbf{波動力学} とよばれ,
            その方程式は \textbf{シュレーディンガー方程式} といわれる.

            行列力学は,波動力学に比べると,抽象的である.量子力学の体系を構築する場合には,
            行列力学が有用である.しかし,実際に生じる個々の物理現象を解析するには,波動力学が
            便利である.個人的に,行列よりも偏微分方程式のほうが扱いなれているので,
            シュレディンガー方程式を先に学習する.数学的テクニックや学習のしやすさを考えても,
            まずはシュレーディンガー方程式から量子力学へ入門する方がよいと思う.実際に,
            そういう教科書のほうが圧倒的に多く出版されている.

            \subsubsection{シュレーディンガー方程式の導入}
            \begin{mysmallsec}{方程式は論理的に導けない}
            光子や電子などの量子は,全てシュレディンガー方程式に従っている
                \footnote{
                    相対論的効果が無視できない場合は,シュレディンガー方程式を拡張した,ディラックの運動方程式で
                    記述される.しかし,ここでは,相対論的効果が無視できるような状況を仮定する.
                }.

            シュレディンガー方程式は,ニュートン方程式と同様に,論理的考察によって導き出すことはできない.だか
            らといって,シュレディンガー方程式はこうだ!!と強要されても,納得できない.
            しかし,シュレディンガーの推論を辿ることは可能である.
            \end{mysmallsec}

            \begin{mysmallsec}{自由粒子の一次元の運動}
            以下では,非常に微小で電荷等をもたない,純粋な自由粒子の一次元の運動を考える.自由粒子は,ポテ
            ンシャルエネルギーをもっておらず,自身のもつ速度による運動エネルギーのみをもっていると仮定する
            .自由粒子にはド$\cdot$ブロイの \textbf{物質波} という波の性質がある.
            この物質波を $\psi(x,\,t) $ と表す.$\psi(x,\,t) $ は以下のような波動関数で記述されると仮定する.
                \begin{align}\label{e_bussituha}
                    \psi(x,\,t) = A\e^{i\left( \frac{p}{\hbar}x-\frac{E}{\hbar} t\right)}.
                \end{align}
            ここに,$A$ は定数で,波動関数の振幅である.$x$ 軸は自由粒子の運動方向にとる.この $\e$ は指数
            関数である.なぜこの関数が自由粒子の物質波を表現するかについてはよく分からないが,こう仮定する
            と,実験結果とよく整合が取れるのである.とりあえず,ここでは,波動関数の意味を考えず,これを仮
            定した上で,話を進めていこう.
            \end{mysmallsec}

            \begin{mysmallsec}{量子力学における,運動量とエネルギー}
            波動関数(\ref{e_bussituha})の波数 $k$ と,角周波数 $\omega$ はそれぞれ,ド$\cdot$ブロイの関係から,
            運動量 $p$ とエネルギー $E$ で表現できる.つまり,$k=p/\hbar$,$\omega=E/\hbar$ の関
            係によって波動関数(\ref{e_bussituha})は
                \begin{align}\label{e_bussituha_2}
                    \psi(x,\,t) = A\e^{i\left( kx -  \omega t\right)}
                \end{align}
            と書き表せる.波動関数の変数は位置 $x$ と時間 $t$ である.この波動関数 $\psi(x,\,t)$ が満たし
            ている方程式を考えるため
                \footnote{
                    運動量に対応する演算子と,エネルギーに対応する演算子を知りたいから.
                },
            その独立変数 $x,\,t$ でそれぞれ偏微分する.

            まず,波動関数 $\psi(x,\,t)$ を位置 $x$ で偏微分して,
                \begin{align}\label{hadoukansu_p}
                    \frac{\rd \psi(x,\,t)}{\rd x}
                        &= i\frac{p}{\hbar}A\e^{i\left( kx -  \omega t\right)} \notag \\
                        &=i\frac{p}{\hbar}\psi(x,\,t) \notag \\
                    \therefore \quad
                    \frac{\rd \psi(x,\,t)}{\rd x}
                        &= i\frac{p}{\hbar}\psi(x,\,t).
                \end{align}

            次に $\psi(x,\,t)$ を時間 $t$ で偏微分して
                \begin{align}\label{hadoukansu_e}
                    \frac{\rd \psi(x,\,t)}{\rd t}
                        &= -i\frac{E}{\hbar}A\e^{i\left( kx -  \omega t\right)} \notag \\
                        &=-i\frac{E}{\hbar}\psi(x,\,t) \notag \\
                    \therefore \quad
                    \frac{\rd \psi(x,\,t)}{\rd t}
                        &= -i\frac{E}{\hbar}\psi(x,\,t).
                \end{align}

            そして,式(\ref{hadoukansu_p}),(\ref{hadoukansu_e})をそれぞれ $p$,$E$ について解いて,
                \begin{align*}
                    p\psi(x,\,t) &= -i\hbar\frac{\rd }{\rd x}\psi(x,\,t) \\
                    E\psi(x,\,t) &=  i\hbar\frac{\rd }{\rd t}\psi(x,\,t)
                \end{align*}
            を得る.ここで,計算中に分母に現れる虚数単位 $i$ は,式の見やすさを考えて有理化した,
            両辺に $\psi(x,\,t)$ がかかるように記述したのは,運動量に対応する演算子と,エネルギーに対応す
            る演算子を知りたかったからである.この式を見ると,運動量 $p$ とエネルギー $E$ は,次のように対
            応していることがわかる.
                \begin{align}
                    p &\Leftrightarrow  -i\hbar\frac{\rd }{\rd x}. \\
                    E &\Leftrightarrow   i\hbar\frac{\rd }{\rd t}.
                \end{align}
            \end{mysmallsec}

            \begin{mysmallsec}{エネルギー保存則が成立すると仮定する}
            運動量 $p$ とエネルギー $E$ の関係を記述する式として,エネルギー保存の法則がある.
            ここで,何らかのポテンシャル(電位でも重力ポテンシャルでもなんでもいい)を加えて,
            これを $V(x,\,t)$ として,
                \begin{equation*}
                    E = \frac{p^{2}}{2m} + V(x,\,t).
                \end{equation*}
            ここに,$m$ は質量である.

            量子力学でもエネルギー保存の法則を満たしているはずである.
                \footnote{
                    古典力学のエネルギー保存の法則は,反証的な実験結果が報告されていないという意味で,
                    今までに破られたことはない.これは,量子力学を適用すべき微視領域の実験でも同じである.
                    実際,時間の対称性から,ネーターの定理によって,エネルギー保存則が導かれる.

                    しかし,時間とエネルギーの不確定性原理によると,エネルギー保存則が破られる
                    瞬間がある.微小時間内ではエネルギーの値が1つの値に定まらず,エネルギーが
                    突然増えたり減ったりする時間があるのだ.このエネルギー保存則の敗れは,
                    量子力学的な微視的範囲でのみ発生し,マクロには現れない現象である.

                    ついでに,もっと言うと,時間も不確定になるために,量子力学的領域では,
                    因果関係が曖昧になっている.
                }
            として,運動量 $p$ とエネル
            ギー $E$ を上で得た対応する演算子に置き換えて,
                \begin{equation*}
                    i\hbar\frac{\rd }{\rd t}
                        = \frac{1}{2m}\left( i\hbar\frac{\rd }{\rd x} \right)^{2} + V(x,\,t).
                \end{equation*}
            整理しよう.
            \begin{equation*}
                i\hbar\frac{\rd }{\rd t}
                    = -\frac{\hbar^{2}}{2m}\frac{\rd^{2}}{\rd x^{2}} + V(x,\,t).
            \end{equation*}
            \end{mysmallsec}

            \begin{mysmallsec}{演算子の導入}
            形式的に \textbf{ハミルトニアン演算子} というものを導入してみよう.その記号として,
            $\hat{H}_{x}$ を用いることにする
                \footnote{
                    ただ,注意したいのは,量子力学でのハミルトニアン演算子は,
                    解析力学でのハミルトニアン $H$ とは全く別物であるということである.しかし,概念的に量
                    子力学のハミルトニアン演算子は,解析力学のハミルトニアン $H$ をヒントに導入されている
                    ので,似たような記号を使いたいというもある.そのため,$\hat{H}$ の$\hat{}$ (「ハット」
                    と読む)はそれらを区別するためにつけてある.

                    また,下添え字の $x$ は, いまは $x$ 軸方向だけしか扱っていないことを明示するためにつけた.
                    空間の3次元を考えるときは添え字を書くことはせず,単に $\hat{H}$ と記述する.
                }.

            解析力学で学んだハミルトニアンは,系の全エネルギーと等しかった.
            今の場合,ハミルトニアンそのものではなく,演算子的性質を持つハミルトニアン演算子である.
            これらを区別するために,演算子のほうには,文字の頭に $\hat{}$ をつけることにしよう.
            つまり,
                \begin{align}
                    \hat{H}_{x} = -\frac{\hbar^{2}}{2m}\frac{\rd^{2}}{\rd x^{2}} + V(x,\,t).
                \end{align}

            さらに,\textbf{エネルギー演算子} も同様に導入しよう.エネルギー演算子の記号は $\hat{E}$ とい
            う記号をつかう.
                \begin{align}
                    \hat{E} = i\hbar\frac{\rd }{\rd t}.
                \end{align}
            すると,形式的にと $\hat{H}_{x} = \hat{E}$ と書ける
                \footnote{
                    右辺と左辺を入れ替えてしまった.量子力学の教科書を見ると,
                    左辺にハミルトニアン演算子 $\hat{H}$,右辺にエネルギー演算子 $\hat{E}$ を書いている.
                    ここではそれに習うことにした.
                }.
            \end{mysmallsec}

            \begin{mysmallsec}{シュレーディンガー方程式の完成}
            しかしこの演算子の関係式は,物理的に何も意味しない.この演算子は,波動関数 $\psi(x,\,t)$ に作
            用させて始めて運動方程式の意味をなす.従って,
                \begin{align}\label{eq:Schrding_eq}
                    \hat{H}_{x} \psi(x,\,t) = \hat{E} \psi(x,\,t)
                \end{align}
            を得る.この式(\ref{eq:Schrding_eq})が \textbf{シュレディンガー方程式} である.
            \end{mysmallsec}

            \begin{mysmallsec}{3次元へ拡張}
            もちろん,$x$ 軸の一方向だけでなく,これらは簡単に3次元空間に拡張できる.$y$,$z$ 方向にも同様
            に考えられる.その場合,シュレディンガー方程式は次のように記述される.
                \begin{align}\label{eq:Schrding_eq_3d}
                    \hat{H} \psi(\br,\,t) = \hat{E} \psi(\br,\,t).
                \end{align}
            \end{mysmallsec}

            \begin{mysmallsec}{一言メモ}
            ここで得たシュレディンガー方程式(\ref{eq:Schrding_eq_3d})は時間を含むものであり,
            最も一般的なものである.ここではあたかも方程式を導いたように見えてしまうが,
            実はそうではない.物質が従うとする波動関数 $\psi(x,\,t)$ が天下り的に与えられていて,
            その信憑性に関する言及は全くない.

            同じ方程式ではあるが,ハミルトニアン演算子とエネルギー演算子を明示したシュレディンガー方程式を
            以下に書いておこう.先ほどの $\hat{H}$,$\hat{E}$ を具体的に記述しただけである.
                \begin{align}\label{eq:Schrding_eq2}
                    -\frac{\hbar^{2}}{2m}\biggl\{
                    \frac{\rd^{2}}{\rd x^{2}} + \frac{\rd^{2}}{\rd y^{2}} + \frac{\rd^{2}}{\rd z^{2}}
                    + V(\br,\,t) \biggr\} \psi(\br,\,t) = i\hbar\frac{\rd }{\rd t} \psi(\br,\,t).
                \end{align}
            \end{mysmallsec}

%       %======================================================================
%       %  Subsection
%       %======================================================================
            \section{対応原理}
                シュレディンガー方程式は
                解析力学における,ハミルトニアンとエネルギーの関係
                から推察される方程式である.\textbf{
                シュレディンガー方程式は論理的に
                導かれる方程式ではなく,
                経験によって得られる方程式である}.
                しかし,この方程式で多くの物理現象を
                説明することができ,それに加えて,この方程式を満たさない現象が
                確認されていないことから,シュレディンガー方程式
                は物理的に正しい方程式であるといってよい.

                古典力学(解析力学)では,
                ハミルトニアン $H$ は座標変数 $(x,\,y,\,z)$ と正準運動量 $(p_{x},\,p_{y},\,p_{z})$ と時間 $t$ の
                関数であり,
                \begin{align}
                  H=H(x,\,y,\,z,\,p_{x},\,p_{y},\,p_{z},\,t)
                \end{align}
                と書かれる.ハミルトニアンは,エネルギー $E$ と等しく,
                \begin{align}
                  H(x,\,y,\,z,\,p_{x},\,p_{y},\,p_{z},\,t)=E
                \end{align}
                である.

                量子力学における
                シュレディンガー方程式は
                ハミルトニアン $H$ の正準運動量 $(p_{x},\,p_{y},\,p_{z})$ とエネルギー $E$ を
                演算子に置き換える(この機械的な置換えのことを \textbf{対応原理} とよぶ).置き換えの際の
                対応は以下の通り.
                    \begin{myshadebox}{対応原理}
                      運動量演算子 $\hat{p}_{x},\,\hat{p}_{y},\,\hat{p}_{z}$ と
                      エネルギー演算子 $\hat{E}$ は,
                      運動量 ${p}_{x},\,{p}_{y},\,{p}_{z}$ とエネルギー $E$ に対して,
                      以下のうように対応させる.
                        \begin{align}
                          p_{x}\, &\rightarrow  \hat{p}_{x}:=-i\hbar\frac{\rd}{\rd x}\,, \\
                          p_{y}\, &\rightarrow  \hat{p}_{y}:=-i\hbar\frac{\rd}{\rd y}\,, \\
                          p_{z}\, &\rightarrow  \hat{p}_{z}:=-i\hbar\frac{\rd}{\rd z}\,, \\
                          E    \, &\rightarrow  \hat{E}:=i\hbar\frac{\rd}{\rd t}
                        \end{align}
                    \end{myshadebox}

                矢印の左側が運動量 $\bp$ と エネルギー $E$ である.左側の
                アルファベットの上に $\hat{}$ が上についた文字は,演算子を意味すると約束しよう
                  \footnote{
                    $\bp$ と $E$ に対して深い関係がありそうなので,アルファベットはその関連する
                    ものを使用しておきたいということと,同時に,演算子であることを示す必要があるので,
                    このような表記になっている.$\hat{}$ の有無で意味が全く異なるので,注意すること.
                  }.

                そして,置き換えた演算子をそれぞれ波動関数 $\psi(x,\,t)$ に作用させて,次式を得る.
                        \begin{myshadebox}{時間を含むシュレディンガー方程式}
                          シュレディンガーの式(\ref{eq:shrdngr_tin})を,\textbf{時間を含むシュレディンガー方程式} という.

                            \begin{align}\label{eq:shrdngr_tin}
                              \hat{H}\left(x,\,y,\,z,\,\hat{p}_{x},\,\hat{p}_{y},\,\hat{p}_{z},\,t\right)\psi(x,\,t)=\hat{E}\psi(x,\,t)
                            \end{align}
                        \end{myshadebox}

%   %==========================================================================
%   %  Section
%   %==========================================================================
    \section{電子の速度}

%       %======================================================================
%       %  Subsection
%       %======================================================================
            \subsection{間違った考察}
                電子の速度を,量子力学の関係式から,
                どう表されるかを,考えてみよう.
                量子力学では,次の関係式が成り立つことが
                要求される.すなわち,
                    \begin{align}
                        p   =   \frac{h}{\lambda }.  \\
                        E   =   h\nu.
                    \end{align}

                さて,波動の速度の式を思い出してみると,
                    \begin{align}
                        v  =  \nu \lambda.
                    \end{align}
                量子力学が要請する2つの式より,$\lambda = h/p$,
                $\nu = E/h$ を上式に代入すると,
                    \begin{equation*}
                        v = \frac{E}{h} \frac{h}{p} = \frac{E}{p}
                    \end{equation*}
                となる.これを $E$ について解いてみる.
                    \begin{equation*}
                        E = pv.
                    \end{equation*}
                ところで,シュレディンガー方程式の考察でも用いたように,
                量子力学でも,運動エネルギーの式 $E = p^{2}/2m$ は有効である.
                これを上式 $E$ に代入すると,
                    \begin{equation*}
                        \frac{p^{2}}{2m} = pv.
                    \end{equation*}
                整理すると,
                    \begin{equation*}
                        p = 2mv.
                    \end{equation*}
                この式は,今まで考えてきた,運動量の式 $p=mv$ と
                異なっている.量子力学的には,$p=mv$ が間違っていて,
                $p=2mv$ が成立しているのだろうか.
                次の項目で,このことについて考えてみよう.

%       %======================================================================
%       %  Subsection
%       %======================================================================
            \subsection{電子の群速度}
                実は,波動の速度を考えるときには,その速度が2種類
                あることを忘れてはいけない.波動のもつ速度には,
                \textbf{位相速度} と \textbf{群速度} の2つがある.
                ここで考えていたのは,位相速度の方であった.
                しかし,波動の運動量を考えるときには,
                群速度を考えないといけない.


%       %======================================================================
%       %  Subsection
%       %======================================================================
            \subsection{電子の有効質量}
                電子を量子力学的にみた場合,粒子の性質だけでなく,
                波動現象も考慮する必要がある.量子力学では,
                電子のもつ運動量 $\bp$ とエネルギー $E$ は,
                それぞれ,波数 $\bk$ と各周波数 $\omega$ と
                次の関係がある.
                    \begin{align}
                        \bp  &=  \hbar \bk    \\
                        E     &=  \hbar \omega
                    \end{align}
                この関係から,電子の運動の速度を考えてみよう.この場合,
                電子の速度とは,位相速度ではなく,群速度で考える必要がある.

                ポテンシャルが存在しない場合は,
                電子は何とも相互作用しないから,質点と同じように扱うことができるが,
                固体中の電子は,固体原子と相互作用をしてしまうので,質点と同じように考えることができない.
                電子が波の性質の有することを考慮しているためである.
                電子のもつエネルギー $E$ は,
                エネルギーに関するアインシュタインの関係式により,
                   \begin{align}
                        E=\hbar \omega
                   \end{align}
                である.ここに,$\hbar$ はプランク定数であり,$\omega$ は
                角周波数である.電子の波動の郡速度 $v_{g}$ を求めるために,$\omega$ について解き,波数 $k$ で
                微分すると,
                    \begin{align}
                        v_{g}=\frac{\mathrm{d} \omega}{\mathrm{d} k} = \frac{\mathrm{d} }{\mathrm{d} k}\frac{E}{\hbar}
                        =\frac{1}{\hbar}\frac{\mathrm{d} E}{\mathrm{d} k}
                    \end{align}
                さらに,$v_{g}$ を時間 $t$ で微分すると,
                   \begin{align}
                    \frac{\mathrm{d} v_{g}}{\mathrm{d} t}=\frac{1}{\hbar}\frac{\mathrm{d}}{\mathrm{d} t}
                    \left(\frac{\mathrm{d} E}{\mathrm{d} k}\right)
                    =\frac{1}{\hbar}
                    \left(\frac{\mathrm{d}^{2} E}{\mathrm{d} k^{2}}\right)
                    \frac{\mathrm{d} k}{\mathrm{d} t}
                   \end{align}
                両辺に $\hbar^{2}$を掛けて,
                    \begin{align}
                     \hbar^{2}\frac{\mathrm{d} v_{g}}{\mathrm{d} t}
                     =\hbar\left(\frac{\mathrm{d}^{2} E}{\mathrm{d} k^{2}}\right)\frac{\mathrm{d} k}{\mathrm{d} t}
                    \end{align}
                さらに以下のように書き換える.
                    \begin{align}
                        \frac{\hbar^{2}}{
                        \displaystyle\left(\frac{\mathrm{d}^{2} E}{\mathrm{d} k^{2}}\right)}\frac{\mathrm{d} v_{g}}{\mathrm{d} t}
                        =\frac{\mathrm{d} (\hbar k)}{\mathrm{d} t}
                        \end{align}
                ここで,運動量に関するアインシュタインの関係式 $p=\hbar k$ を
                右辺に考慮すると,
                   \begin{align}
                    \frac{\hbar^{2}}{
                    \displaystyle\left(\frac{\mathrm{d}^{2} E}{\mathrm{d} k^{2}}\right)}\frac{\mathrm{d} v_{g}}{\mathrm{d} t}
                    =\frac{\mathrm{d} p}{\mathrm{d} t}
                   \end{align}
                ここで,{\bf 有効質量} $m^{\ast}$ を以下のように定義することで,
                電子に対する運動方程式を得る.
                   \begin{align}
                     m^{\ast}:=\frac{\hbar^{2}}{
                     \displaystyle\left(\frac{\mathrm{d}^{2} E}{\mathrm{d} k^{2}}\right)}
                   \end{align}
                運動方程式は,
                   \begin{align}
                     m^{\ast}\frac{\mathrm{d} v_{g}}{\mathrm{d} t}
                     =\frac{\mathrm{d} p}{\mathrm{d} t}
                   \end{align}

                現実の電子の質量はどのようなものかを
                具体的に考えることはできないが,質量を
                有効質量として繰り込む
                   \footnote{
                    繰り込み;無限に小さな微小領域で観測するといった
                    現実では不可能な視点にたったときに,
                    値が無限大に発散してしまう等といった計算が不都合になる場合がある.
                    しかし,ある程度巨視的な視点にたって考えるならば,
                    その値を発散しないようにできる等の解決策がある.
                    その解決策の一つとして繰り込みという概念がある.
                   }
                ことで
                近似して考えることはできる.
                これがこの式の意味するところである.

%   %==========================================================================
%   %  Section
%   %==========================================================================
    \section{トンネル効果}

                        \begin{figure}[hbt]
                            \begin{center}
                                \includegraphicsdefault{tonnel_ef.pdf}
                                \caption{トンネル効果:ポテンシャル障壁}
                                \label{fig:tonnel_ef}
                            \end{center}
                        \end{figure}

%   %==========================================================================
%   %  Section
%   %==========================================================================
    \section{Fermi-Dirac 分布関数}
    電子はfermion(Fermiオン)である
        \footnote{
            素粒子は,boson(ボソン)またはfermionのどちらか一方の性質を
            もつ.phononやphoton,meson(中間子)はbosonであり,electoronやproton,neutronはfermionである.
            fermionは1つの位置には1つの粒子しか存在することができないという特徴をもっている.細かくいえば,
            量子力学における粒子のスピンという概念を考慮すれば,1つの位置に2つの互いに異なったスピンをもつ
            粒子が存在し得る.それに対して,bosonは1つの位置に対して複数の粒子が存在できる性質をもっている.
        }.
    従って,Fermi-Dirac統計に従う粒子である.この理論によれば,fermionは以下の式を満足する.
        \begin{align}
        f(E,T)=\frac{1}{1+\exp\left(\frac{E-E_{f}}{k_{B}T}\right)}
        \end{align}
    この式は {\bf Fermi分布関数} とよばれる.この式の $E_{F}$ は
    ある特定の温度におけるFermiエネルギーを表している.また,$k_{B}$ はボルツマン定数である.

    この式をグラフで表すことを考える.グラフの横軸を $f(E)$,縦軸を $E$ としたい.
    そこで,上式を $E$ について解く.すると,
        \begin{align*}
        E=E_{f}+k_{B}T\log\left(\frac{1}{f(E)}-1\right)
        \end{align*}
    を得る.最後に,フェルミ準位を基準とするために,$E_{f}=0$ とする.
        \begin{align*}
        E=k_{B}T\log\left(\frac{1}{f(E)}-1\right)
        \end{align*}
    これを,グラフに書いてみよう.


    \begin{itembox}[l]{{\bf 問題}}
        Fermi-Dirac 分布関数
        $f(E)$ は点($E_{f}$, $\displaystyle\frac{1}{2}$)に関して
        点対称であることを示せ.
    \end{itembox}
            \subsubsection*{問題の解答}
            \paragraph*{方針}
            Fermi-Dirac 分布関数上の任意の点を $(E_{1},\,f(E_{1}))$ にとる.
            また,このようにとった $E_{1}$ に対応して,
            \begin{align*}
            \frac{E_{1}+E_{2}}{2}=E_{f}
            \end{align*}
            となるような $E_{2}$ をとる.
            以上の $E_{1}$ ,$E_{2}$ に対して,もし,
            \begin{align*}
            \frac{f(E_{1})+f(E_{2})}{2}=\frac{1}{2}
            \end{align*}
            という等式が成立していれば,
            Fermi-Dirac 分布関数
            $f(E)$ は点($E_{f}$, $\displaystyle\frac{1}{2}$)に関して
            点対称であると言える.

            \subsubsection*{解答}
                実際に計算する.
                まず,$f(E_{1})$ については,Fermi-Dirac 分布関数に $E_{1}$ を代入して,
                \begin{align}\label{1}
                f(E_{1})=\frac{1}{1+\exp\left(\frac{E_{1}-E_{f}}{k_{B}T}\right)}
                \end{align}
                であることがわる.これは,仮定により,成立している.
                次に,$f(E_{2})$ について考えるが,
                ここで, $E_{2}$ は $E_{1}$ と
                \begin{align*}
                E_{2}=2E_{f}-E_{1}
                \end{align*}
                の関係があるとしているので,
                \begin{align}\label{2}
                f(E_{2})=f(2E_{f}-E_{1})=\frac{1}{1+\exp\left(\frac{-(E_{f}-E_{1})}{k_{B}T}\right)}
                \end{align}
                となる.

                目的は,以上の二つの式(\ref{1}),(\ref{2})から,$\displaystyle\frac{f(E_{1})+f(E_{2})}{2}=\frac{1}{2}$ の
                関係を導くことでる.導出の過程で,式変形が複雑にならないように,
                以下のような文字の置き換えを定義する.
                \begin{align}\label{x_def}
                x:=\exp\left(\frac{E_{1}-E_{f}}{k_{B}T}\right)
                \end{align}
                この $x$ を用いると,式(\ref{1})と(\ref{2})はそれぞれ
                \begin{align}\label{3}
                f(E_{1})=\frac{1}{1+x}
                \end{align}
                \begin{align}\label{4}
                f(E_{2})=\frac{1}{1+\frac{1}{x}}
                \end{align}
                となる.そして,
                式(\ref{3})と式(\ref{4})の辺々を加え合わせると,
                \begin{align}\label{5}
                f(E_{1})+f(E_{2})=\frac{1}{1+x}+\frac{1}{1+\frac{1}{x}}=\frac{1}{1+x}+\frac{x}{1+x}=1
                \end{align}
                となる.

                最後に,$x$ を定義式(\ref{x_def})により書き換え,
                式(\ref{5})の両辺に $\displaystyle\frac{1}{2}$ を
                掛けることで,
                \begin{align*}
                \frac{f(E_{1})+f(E_{2})}{2}=\frac{1}{2}
                \end{align*}
                を得る.これは,最初に期待していた式と一致し,
                これによって,問題の求める解答を得た.



%           %==================================================================
%           %  Subsubsection
%           %==================================================================
%            \subsubsection{}
