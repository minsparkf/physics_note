\section{特殊相対論的な現象}
\begin{mycomment}
この節では,特殊相対性原理と光速度不変の原理
によって起こる例を考える.最初に,
同時刻について考える.
その次に,Lorentz収縮と
時間の遅れを考える.
そして,
マイケルソンとモーリーの実験
矛盾を解決するローレンツ変換式
を,上の2つの基本原理だけによって
導出する.
\end{mycomment}


\subsection{同時刻}
\begin{mycomment}
光速度不変の原理によって,同時刻という従来のニュートン力学的な概念は,成り立たなくなって
しまうのである.そこで次に,同時刻の概念の考え直しをしたいと思う.
\end{mycomment}


同時刻については,
具体的に考えるとわかりやすい.例として,「2つの異なる場所より発せられる光が
“同時刻”に届く」とはどういうことかを考える.2つの光源の名前をそれぞれA,Bとする.
当たり前のことではあるが,\textbf{
光源Aと光源Bの光が観測点Oに“同時刻”に届くとは,
それぞれ光源から発せられた光が
時間差なく届くということである.}
さて,より詳しく考えるために,
以下のような状況を設定し,考察していこう.

まず理想的な
\footnote{
ここでいう「理想的」とは,
車輪の摩擦や空気抵抗,
燃料の増減による
質量の増加等の
現実性を全て仮想的に
排除したということである.
}
電車を想定し,この電車両端(前と後ろ)に光源を
取り付ける.前の光を
青にし,
後ろの光を赤にする.
そして,電車が
前の方向に等速度速度 $V$ をもって運動しているとする.
但し,この電車の速度は
光速 $c$ に比べて十分に小さいものとする(すなわち,$V\ll c$).
このとき,
光が電車の真中に到達する様子を
以下の2つの立場で考える.
まず第1に,
観測者がこの車両の真中で
この光の到達を観測するという立場で考える.そして
第2に,
これを電車の外で静止している別の観測者の
立場で考える.
この考察で,この両者の同時刻は異なってしまうことを
みることになる.

早速考える.第1の立場の状況は図\ref{fig:doujikoku1}のように描ける.
                \begin{figure}[hbt]
                    \begin{center}
                        \includegraphicsdefault{doujikoku1.pdf}
                        \caption{同時刻1(電車の中の観測者からの視点)}
                        \label{fig:doujikoku1}
                    \end{center}
                \end{figure}

この立場の観測者は,光源と同時に運動している.従って,
前から来る青い光 と 後ろから来る赤い光 は同時刻に
観測者の目に入ってくる.

では次に第2の立場,すなわち,電車の外で静止している
観測者からの視点で考える.この場合の状況は
図\ref{fig:doujikoku2}のように描ける.
\textbf{
光は,電車内の観測者が「“同時刻”に前後の
光源が発光した」と主張するように
光源を発光させているものとする.}
すなわち,第1の立場と同じ
状況を,電車外の静止した場所から
観測してみようということである.
実はこの仮定は同時刻を
考え直すのに大事になるので,
この仮定を覚えておくこと.
                \begin{figure}[hbt]
                    \begin{center}
                        \includegraphicsdefault{doujikoku2.pdf}
                        \caption{同時刻2(電車外の静止した観測者からの視点)}
                        \label{fig:doujikoku2}
                    \end{center}
                \end{figure}

この図で注意したいのは
“電車が動いても,光速は常に一定値 $c$ をとる”
ということである.
これは,光速度不変の原理による要請である.

この図を見ると,後ろから
発光した赤い光は電車の速度の方向に
進むだけ,
また他方では,前から発光した青い光は
電車の進行方向と逆向きに
進むので,
従って,
青い光が赤い光よりも先に
電車の真中に
到達することになる.
だから,
電車外で静止している
観測者には,赤と青の
光が同時に
電車の真中に
到達しない
と主張することになる.
これは,
光速が一定であるから,
電車が動けば,その電車の動いた分だけ
光が進むからである.


では,青い光と赤い光との
到達時刻の差は
どの程度だろうか.その時間差を考えることにする.
この電車の長さを $l$ とする.
このとき,光源Aまたは光源Bからの
電車の真中までの距離は,それぞれ $l/2$ である.
青い光についてまず考えると,青い光が電車の
真中までに要する時間を $t_{\mbox{青}}$ として,
青い光の進む距離 $X_{1}$ を考えると,
電車の速度が前の方向に $V$ であることを考慮し,
\begin{align}\label{eq:douji_1}
X_{1}=\frac{l}{2}-Vt_{\mbox{青}}
\end{align}
である.ところで,光速度不変の原理によって,
光の速度は光源のもつ速度に関係ないので,
\begin{align}\label{eq:douji_2}
X_{1}=ct_{\mbox{青}}
\end{align}
も成立しているはずである.従って,式(\ref{eq:douji_1}),式(\ref{eq:douji_2})から,
\begin{align}\label{eq:douji_3}
ct_{\mbox{青}}&=\frac{l}{2}-Vt_{\mbox{青}} \notag \\
\Leftrightarrow \quad
ct_{\mbox{青}}+Vt_{\mbox{青}}&=\frac{l}{2}\notag \\
\Leftrightarrow \quad
(c+V)t_{\mbox{青}}&=\frac{l}{2} \notag \\
\Leftrightarrow \quad
t_{\mbox{青}}&=\frac{l}{2(c+V)}
\end{align}
である.一方,赤い光について
も同様に考えると,光の進行方向が
お会い光と逆方向なだけなので,
\begin{align}\label{eq:douji_4}
t_{\mbox{赤}}=\frac{l}{2(c-V)}
\end{align}
計算される.
式(\ref{eq:douji_3})と式(\ref{eq:douji_4})を比較すると,
$l$,$c$,$V$ は全て定数であるから,
$t_{\mbox{赤}}$ が $t_{\mbox{青}}$ よりも大きいことがわかる.
すなわち,赤い光が電車の真中に到達する時刻は
青い光よりも遅れるということになる.どのくらい遅れるか
といえば,
\begin{align}\label{eq:douji_5}
t_{\mbox{赤}}-t_{\mbox{青}}&=\frac{l}{2(c-V)}-\frac{l}{2(c+V)}\notag \\ \notag \\
&=\frac{2l(c+V)-2l(c-V)}{2(c-V)\times 2(c+V)}\notag \\ \notag \\
&=\frac{2lc+2lV-2lc+2lV}{4(c-V)(c+V)}\notag \\ \notag \\
&=\frac{lV}{c^{2}-V^{2}}
\end{align}
である.

ここで最初の光の点灯時刻に関する
仮定を思い起こすと,
“光源の発光は電車内の観測者が
同時刻に発光したと主張する
ように光源を発光させている”
ということであった.
これはつまり,
\newline \newline
\textbf{主張1};「電車内の観測者にとって,
赤い光と青い光が
同時刻に到達したと
観測される」
\newline \newline
ということである.
しかし,
電車外の人間から見れば,
青い光が赤い光
よりも先に電車の真中に
到達していて,
つまり,電車外の人間は,
\newline \newline
\textbf{主張2};「
電車内の人間は同時刻に
赤い光と青い光を観測していない」
\newline \newline
と主張するのである.

主張1と主張2は明らかに矛盾している.
これは,「電車内の人間の同時刻の発光」と
「電車外の人間の同時刻の発光」が
全く別の見方であるということを示唆する.
そこで,同時刻の概念の
考え直しが必要になるのである.
この矛盾を解決するには次のように考えるとよい.

電車内の観測者が青い光と
赤い光が同時に発光したと主張しているということは,
電車内の人間にとって,
青い光と赤い光が時間差なく
届いたということである.
この仮定を前提としたここでの
考察で矛盾を生まないためには,
\textbf{電車外の静止した観測者にとって,
青い光と赤い光の発光時刻は
異なっている} と考えるのである.
すなわち,電車外の人にとっては,
赤い光が先に点灯し,
赤い光の点灯時刻より $lV/({c^{2}-V^{2}})$ の
時間間隔の後に青い光が点灯している
ように見えるのである.

電車内の観測者は
青い光と赤い光は
同時に点灯したと
主張し,
電車外の観測者は
青い光と赤い光は
異なった時刻に点灯
していると
主張する.
観測者によって,
その主張の内容が
異なっているが,
矛盾はしていない.
この両者の
主張の違いは,
それぞれの属する
慣性系が異なることに
依るものである.
今までの同時刻の概念は,
慣性系に依らないものであった.
しかし,光速度不変の原理を仮定ている今,
もはや従来の同時刻の概念は通用しない
ものであるということは,
以上の確認ではっきりと分かった.
つまり,\textbf{
同時刻の概念は
慣性系によって異なるものである} ということである.
ある慣性系では同時刻に起こる現象が,
別の慣性系では異なる時刻に起こっている現象である
ということを認めねばならない.






\subsection{ローレンツ変換式}
\begin{mycomment}
さて,以上で特殊相対性理論の基本原理を
確認し終わった.特殊相対性理論の基本原理とは,
「特殊相対性原理」と「光速度不変の原理」の2つである.
この2つの原理を用いて,\textbf{ローレンツ変換式} を導くことが
ここでの目的である.但し,ここで確認する
ローレンツ変換式は特殊な
状況下に置かれた場合に限られている.
一般の場合への拡張は後の項目で
考える.
ここでの目標は,
ローレンツ変換式を
直感的に捉えることである.
\end{mycomment}



アインシュタインは,特殊相対性原理と
光速度不変の原理から,
ローレンツ変換式を求めることができることを示した.
それにより,ローレンツ変換のときに考えた
局所時や物体の収縮等の,
物理的に奇妙な現象を
用いなくても,
ごく自然にローレンツ変換式を
受け入れることが可能になる.
つまり,物理的な矛盾を生むエーテルの存在を仮定しなくてもよいのである.
従って,アインシュタインは,事実上,
エーテルの存在を否定したものといってよいだろう.
しかし,アインシュタインはエーテルの存在を
直接否定するのではなく,
あくまでも,“エーテルが存在しなくともよい”
と主張するのである.
エーテルの存在を
完全に否定しているのではないことに注意したい
\footnote{
実は,
エーテルとは“光子”という形で,
再び,物理理論上に現れることになる.
これについては,量子力学(場の量子論)の
部分で確認したいと思う.
}.

では,特殊相対性理論の2つの基本原理から,
ローレンツ変換式を求めていくとしよう
\footnote{
但し,ここでのローレンツ変換式は,
直感的イメージを第1に考えたいので,
$x$ 軸方向だけに動く物体について
の記述をする.従って,以下の議論は
かなり特殊なローレンツ変換式だが,
この一般論は,章を改めて,
確認していくことにしたい.
}.
まず,基準慣性系 $S$ を
導入し,この基準慣性系 $S$ における
立場で,現象を考える.
系 $S$ に対して,$x$ 方向の正の向きへ速度 $v$ を
もち,$y$,$z$ の2方向には速度をもたないような
座標系 $S'$ を考える.
すなわち基準系 $S$ に対して,
系 $S'$ は
相対速度 $\bv
=(v,\,0,\,0)$ をもっているとする(図\ref{fig:L_rt1}).
さらに簡単のために,
$t=t'=0$ のときに,
基準系 $S$ と運動系 $S'$ の両方の原点O,O$'$が一致するようにする.
                \begin{figure}[hbt]
                    \begin{center}
                        \includegraphicsdefault{L_TR1.pdf}
                        \caption{基準座標系 $S$ と $S$ に対して速度をもつ座標系 $S'$ のイメージ}
                        \label{fig:L_rt1}
                    \end{center}
                \end{figure}


これから基準系 $S$ から見た系 $S'$ の座標を
考えるわけだが,ここで1つの重要な
仮定を設けたい.それは,
\textbf{座標変換は1次変換である} ということである
\footnote{
このように仮定する理由は,基準系 $S$ から系 $S'$ 見る場合と,
その逆変換の
系 $S'$ から $S$ をみる場合で,その運動の法則が同じでなければ
ならないからである.
つまり,特殊相対性原理を満たすような座標変換でないといけないのである.
もし,2次変換であったならば,この特殊相対性原理を満たさないことは明らか
である.(例えば, $x'=x^{2}$ の逆変換は $x=\sqrt{x'}$ であり,形が異なっている.)
3次以上についても同様である.
}
.
ある点Pを
基準系 $S$ から見た座標を $(ct,\,x,\,y,\,z)$ とし,
この点Pを
運動系 $S'$ から見た座標を $(ct',\,x',\,y',\,z')$ とする
\footnote{
$ct$ について\,;\;
同時刻の項目で確認したように,
時間 $t$ も座標系により
違った値となる.従って,
時間 $t$ も空間座標と同等に扱う必要がある.
しかし,$t$ のままでは次元が合わないので,
光速 $c$ をかけて $ct$ とし,
これを4つめの座標とするのである.光速 $c$ は
座標系に関係なく一定の値をとるので,
実質的に $t$ だけを考えているのである.
}
.
このとき,$S$ での座標と $S'$ の座標は1次変換で
結ばれているという仮定から,以下のような座標
変換式を得る.
\begin{align}\label{eq:L_tr1}
ct'=A_{00}ct+A_{01}x+A_{02}y+A_{03}z
\end{align}
\begin{align}\label{eq:L_tr2}
x'=A_{10}ct+A_{11}x+A_{12}y+A_{13}z
\end{align}
\begin{align}
y'=A_{20}ct+A_{21}x+A_{22}y+A_{23}z
\end{align}
\begin{align}
z'=A_{30}ct+A_{31}x+A_{32}y+A_{33}z
\end{align}
ここで,$A_{00}$,$A_{01}$ …などは,
まだ決定されていない定数である.これから,$A_{00}$,$A_{01}$ …を
決定していこう.

\vspace{6mm}

まず,簡単に求められるのは,
運動系 $S'$ は 静止系 $S$ に対する $y$,$z$ 方向には
運動していない という仮定より, $y=y'$,$z=z'$ であるから,
この2式の対応する係数を比較すれば,
\begin{align}
A_{20}=A_{21}=A_{23}=A_{30}=A_{31}=A_{32}=0
\end{align}
であり,また,
\begin{align}
A_{22}=A_{33}=1
\end{align}
である.さらに,
\begin{align}
A_{02}=A_{03}=A_{12}=A_{13}=1
\end{align}
であることもわかる.
これは,特殊相対性原理
によって要請されるものである.
というのも,$A_{02}$,$A_{03}$,$A_{12}$,$A_{13}$ が全て
0でないと,逆変換
\footnote{
系 $S'$ から見た 基準系$S$ の座標のこと.
}したときに,$x$ や $t$ が $y$ と $z$ に
依存してしまい,
特殊相対性原理に反してしまうのである.

これらによって,随分と変換式が
すっきりとした形になった.
これを書いておこう.
\begin{align}\label{eq:L_trr1}
ct'=A_{00}ct+A_{01}x
\end{align}
\begin{align}\label{eq:L_trr2}
x'=A_{10}ct+A_{11}x
\end{align}
\begin{align}\label{eq:L_trr3}
y'=y
\end{align}
\begin{align}\label{eq:L_trr4}
z'=z
\end{align}










次に,残りの4つの係数($A_{00}$,$A_{01}$,$A_{10}$,$A_{20}$)
を求めていこう.運動系 $S'$ は,基準系 $S$ に対して,
$x$ 方向性の向きに速度 $v$ で運動している.
従って,運動系 $S'$ から基準系 $S$ の原点 O を
見ると,その位置は
\begin{align}\label{eq:trrrr}
x'=-vt'
\end{align}
である.
この式は,2つの式(\ref{eq:L_trr1}),式(\ref{eq:L_trr2})で,
$x=0$ をそれぞれに代入した式
\begin{align}\label{eq:L_trr11}
ct'=A_{00}ct
\end{align}
\begin{align}\label{eq:L_trr21}
x'=A_{10}ct
\end{align}
と一致しているはずである
\footnote{
原点の $x$ 座標は $x=0$ である.
}
.
式(\ref{eq:trrrr})と式(\ref{eq:L_trr21})から,
\begin{equation*}
-vt'=A_{10}ct
\quad\Leftrightarrow\quad
ct=-\frac{vt'}{A_{10}}
\end{equation*}
であり,これを式(\ref{eq:L_trr11})に代入すると,
\begin{align}\label{eq:L_trr111}
ct'&=-A_{00}\frac{vt'}{A_{10}} \notag \\
\quad\Leftrightarrow\quad  A_{10}c&=-A_{00}v \notag \\
\quad\Leftrightarrow\quad \frac{A_{10}}{A_{00}}&=-\frac{v}{c}
\end{align}
である.

また,基準系 $S$ から運動系 $S'$ の原点 O$'$ を
見ると,その位置は
\begin{align}\label{eq:L_trS}
x=vt
\end{align}
と書ける.この式(\ref{eq:L_trS})は,
式(\ref{eq:L_trr2})で,
$x'=0$ とした場合に一致するはずある.つまり,
まず式(\ref{eq:L_trr2})から,
\begin{align}
0=A_{10}ct+A_{11}x
\quad\Leftrightarrow\quad
x=-\frac{A_{10}}{A_{11}}ct
\end{align}
であり,$x=vt$ から,
\begin{align}\label{eq:L_trr222}
vt=-\frac{A_{10}}{A_{11}}ct
\quad\Leftrightarrow\quad
\frac{A_{10}}{A_{11}}=-\frac{v}{c}
\end{align}
となる.

式(\ref{eq:L_trr111})と式(\ref{eq:L_trr222})から,
両者の値は $-(v/c)$ と等しく,
\begin{equation*}
\frac{A_{10}}{A_{00}}=\frac{A_{10}}{A_{11}}
\quad\Leftrightarrow\quad
\frac{1}{A_{00}}=\frac{1}{A_{11}}
\end{equation*}
すなわち,$A_{00}=A_{11}$ である.これを $\gamma$ とおく.
\begin{align}
\gamma:= A_{00}=A_{11}
\end{align}

これらの計算により,
\begin{align}\label{eq:L_trr2_2}
ct'=\gamma ct +A_{01}x
\end{align}
\begin{align}\label{eq:L_trr2_1}
x'=A_{10}ct+\gamma x
\end{align}
である.$y=y'$,$z=z'$ は
以後省略する.

残った $A_{01}$,$A_{10}$ を求めよう.それには,
光速度不変の原理を利用する.光速度不変の原理とは,
光速は,どのような速度で等速直線運動する座標から
光の速度を測定しても,光の速さは座標の動く速度に
関わらず,一定の値 $c$ を示すという原理である.つまり,
\begin{align}\label{eq:x_ct1}
x=ct
\end{align}
\begin{align}\label{eq:x_ct2}
x'=ct'
\end{align}
の2式が成立していることになる
    \footnote{
        どちらの座標系でも,光速は同じ値 $c$ を示す.
    }.
                \begin{figure}[hbt]
                    \begin{center}
                        \includegraphicsdefault{L_TR2.pdf}
                        \caption{光の伝播}
                        \label{fig:L_rt2}
                    \end{center}
                \end{figure}

式(\ref{eq:x_ct1})を式(\ref{eq:L_trr2_2})と式(\ref{eq:L_trr2_1})に代入して,
\begin{align}
ct'=\gamma ct +A_{01}ct = (\gamma + A_{01})ct
\end{align}
\begin{align}
x'=A_{10}ct+\gamma ct =  (A_{10}+\gamma )ct
\end{align}
また,式(\ref{eq:x_ct2})の $x'=ct'$ をこの2式に考慮すれば,
\begin{align}
(\gamma + A_{01})ct &= (A_{10}+\gamma )ct \notag \\
\Leftrightarrow \quad
\gamma + A_{01} &= A_{10}+\gamma \notag \\
\therefore \quad
\gamma\,' := A_{01} &= A_{10}
\end{align}
これまでの計算で,
\begin{align}
ct'=\gamma ct +\gamma\,' x
\end{align}
\begin{align}
x'=\gamma\,' ct+\gamma x
\end{align}
を得た.次に,$\gamma$ と $\gamma\,'$ の関係を調べる.


ところで,先ほど計算した式(\ref{eq:L_trr222})の $A_{10}$,$A_{11}$ は,
それぞれ,$\gamma$,$\gamma\,'$ であるので,これは
\begin{align}
\frac{\gamma\,'}{\gamma} = -\frac{v}{c}\notag \\
\therefore \quad
\gamma\,' = -\gamma \frac{v}{c}
\end{align}
である.これによって,運動系 $S'$ への変換式は次のような形になる.
\begin{align}\label{eq:ctp}
ct'=\gamma ct -\gamma \frac{v}{c} x.
\end{align}
\begin{align}\label{eq:xp}
x'=-\gamma \frac{v}{c} ct+\gamma x.
\end{align}
後の式変形の関係で,式を簡単にすることはしなかった.

さて,変換式は $S$ 系に逆変換したときにも
成り立たないといけない.二つの変換式が矛盾してはならない.
そこで,$ct'$,$x'$ の式をそれぞれ $ct$,$x$ について解き,
逆変換しても式の形が変わらないような $\gamma$ の形を求めていこう.
\begin{equation*}
ct'= \gamma ct - \gamma \frac{v}{c}x\,,\qquad x' = -\gamma \frac{v}{c}ct + \gamma x
\end{equation*}
\begin{equation*}
\gamma ct = ct' + \gamma \frac{v}{c}x\,,\qquad \gamma x = x' + \gamma \frac{v}{c}ct
\end{equation*}
$x$,$ct$ について解くと,
\begin{align}\label{eq:ct}
ct=\frac{1}{\gamma}ct' + \frac{v}{c}x
\end{align}
\begin{align}\label{eq:x}
x=\frac{1}{\gamma}x'+\frac{v}{c}ct
\end{align}
式(\ref{eq:ct})の $x$ に式(\ref{eq:x})を代入して,
\begin{align*}
                     ct&=\frac{1}{\gamma}ct' + \frac{v}{c}\left(\frac{1}{\gamma}x'+\frac{v}{c}ct\right)     \\
\Leftrightarrow ct&=\frac{1}{\gamma}ct' + \frac{1}{\gamma}\frac{v}{c}x'+\left(\frac{v}{c}\right)^{2}ct \\
\Leftrightarrow ct-\left(\frac{v}{c}\right)^{2}ct&=\frac{1}{\gamma}ct' + \frac{1}{\gamma}\frac{v}{c}x' \\
\Leftrightarrow \biggl\{1-\left(\frac{v}{c}\right)^{2}\biggr\}ct&=\frac{1}{\gamma}ct' + \frac{1}{\gamma}\frac{v}{c}x'
\end{align*}

\begin{align}\label{eq:ctct}
\therefore\quad ct = \frac{1}{\gamma\{1-(v/c)^{2}\}}ct'+\frac{v/c}{\gamma\{1-(v/c)^{2}\}}x'.
\end{align}
同様に式(\ref{eq:x})の $ct$ に式(\ref{eq:ct})を代入して,
\begin{align}\label{eq:xx}
x = \frac{v/c}{\gamma \{ 1- (v/c)^{2}\}}ct'+\frac{1}{\gamma \{ 1- (v/c)^{2}\}}x'.
\end{align}
先ほども書いたように,この式は,変換前の式と式の形が一致していなければならない.
ただし,座標変換した場合,速度の方向は逆方向に見えるはずであり,$S$ 座標系
から見た相対速度 $v$ は $S'$ 座標系から見れば,$-v$ とならないとおかしい.
従って,上の $ct$ の式と $x$ の式の2式と比較すべき式は
変換前の $ct'$ と $x'$ の式の $v$ を,$-v$ で置き換えた
\begin{align}\label{eq:ct2}
ct=\gamma ct' +\gamma \frac{v}{c} x'.
\end{align}
\begin{align}\label{eq:x2}
x=\gamma \frac{v}{c} ct'+\gamma x'.
\end{align}
である.式(\ref{eq:ctct})と式(\ref{eq:ct2})が対応していて,
式(\ref{eq:xx})と式(\ref{eq:x2})が対応する.この対応から,
$\gamma$ を求めていこう.
両対応ともに,$\gamma$ は以下の式を満たしていればよい.
\begin{align}
\gamma = \frac{1}{\gamma \{ 1-(v/c)^{2} \} }.
\end{align}
計算して $\gamma$ について解けば,
\begin{align*}
\gamma = \frac{1}{\gamma \{ 1-(v/c)^{2} \} }
\quad\Leftrightarrow\quad
\gamma^{2} = \frac{1}{1-(v/c)^{2}}
\end{align*}
\begin{align}
\therefore\quad \gamma = \frac{1}{\sqrt{ 1-(v/c)^{2} }}.
\end{align}
この $\gamma$ を式(\ref{eq:ctp}),式(\ref{eq:xp})に
代入すれば,求めたかった変換式を得ることができる.すなわち,\\
\begin{itembox}[l]{\textbf{$x$軸方向に運動する系のローレンツ変換式}}
    \begin{align}
    ct'=\frac{1}{\sqrt{ 1-(v/c)^{2} }} ct -\frac{v/c}{\sqrt{ 1-(v/c)^{2} }} x
    \end{align}
    \begin{align}
    x'=-\frac{v/c}{\sqrt{ 1-(v/c)^{2} }}ct+\frac{1}{\sqrt{ 1-(v/c)^{2} }} x
    \end{align}
    \begin{align}
    y'=y
    \end{align}
    \begin{align}
    z'=z
    \end{align}
\end{itembox}\\
である.この逆変換式
    \footnote{
        逆変換式とは,座標系が入れ替わった場合の式である.
        逆変換式の導き方は,変数を機械的に入れ替えて,
        速度を $-v$ に置き換えることである.
    }
は\\
\begin{itembox}[l]{\textbf{逆変換した式}}
    \begin{align}
    ct=\frac{1}{\sqrt{ 1-(v/c)^{2} }} ct' +\frac{v/c}{\sqrt{ 1-(v/c)^{2} }} x'
    \end{align}
    \begin{align}
    x=\frac{v/c}{\sqrt{ 1-(v/c)^{2} }}ct'+\frac{1}{\sqrt{ 1-(v/c)^{2} }} x'
    \end{align}
    \begin{align}
    y=y'
    \end{align}
    \begin{align}
    z=z'
    \end{align}
\end{itembox}\\
である.

また,式の表現を簡単にするために,$\beta = v/c$ とおくことで,
\begin{align}
\beta = \frac{v}{c}
\end{align}
\begin{align}
ct'=\frac{1}{\sqrt{ 1-\beta^{2} }} ct -\frac{v/c}{\sqrt{ 1-\beta^{2} }} x
\end{align}
\begin{align}
x'=-\frac{v/c}{\sqrt{ 1-\beta^{2} }}ct+\frac{1}{\sqrt{ 1-\beta^{2} }} x
\end{align}
\begin{align}
y'=y
\end{align}
\begin{align}
z'=z
\end{align}
と書かれることもある.

\subsection{ローレンツ因子}
上で得たローレンツ変換式の
\begin{equation*}
     \gamma = \frac{1}{\sqrt{1-\left(v^{2}/c^{2}\right)}}.
\end{equation*}
を \textbf{ローレンツ因子} とよぶ.教科書によっては,このように特別に名前をつけていないものもある.しかし,このノートでは,今後の説明のために,ローレンツ因子とよぶことにしよう
    \footnote{
        ローレンツ因子は一般的に使われる用語である.私の造語ではない.
    }.

ローレンツ因子 $\gamma$ を用いると,ローレンツ変換式は次のようになる.
    \begin{itembox}[l]{\textbf{ローレンツ変換式($x$軸方向, $\gamma$で記述を圧縮)}}
        \begin{align}
        ct' &= \gamma  ct - \gamma \frac{v}{c}  x \\
        x'  &= -\gamma \frac{v}{c} ct + \gamma  x \\
        y'  &= y \\
        z'  &= z
        \end{align}
    \end{itembox}\\
記号の対象性を残しておくため,$x'$ の第一項の $c$ は残しておいたほうがいい.

更に,名も無き因子 $\beta$(言うなら,光速に対する物体の速度比)をつかうと,
\begin{itembox}[l]{\textbf{ローレンツ変換式($x$軸方向, $\gamma$と$\beta$で記述を圧縮)}}
    \begin{align*}
    ct' &= \gamma  ct - \gamma \beta  x \\
    x'  &= -\gamma \beta ct + \gamma  x \\
    y'  &= y \\
    z'  &= z
    \end{align*}
\end{itembox}\\
となる.

\subsection{方程式の変数}
上では,ローレンツ変換式の特別な場合($x$ 軸方向に運動する系のみ)における
式を導出した.この式の変数について少し考えておこう.
これら4つの式の変数は $t$,$x$,$y$,$z$ である.
しかし,このうちの $t$ は時間を表わし,その他の空間を表す $x$,$y$,$z$ と
は次元が異なっていて,同時に扱いづらい.そこで,時間 $t$ に光速 $c$ をかけて $ct$ と
して,空間の次元をもたせることで,その他の空間的な変数と同時に扱うことにする.
すなわち,
\begin{equation*}
\mbox{式の変数は}\;\; ct,\,x,\,y,\,z
\end{equation*}
である.光速 $c$ の次元は[m/s],時間の次元は書くまでもないが[s]であるので,
$ct$ の次元は[(m/s)$\times$s]$=$[m]であり,空間的な変数 $x$,$y$,$z$ と
同じ次元で扱うことが可能になる.

\subsection{ローレンツ変換とガリレイ変換}
当然のことながら,ローレンツ変換式はどのような速度で運動する系に対しても
成立していなければならない.従って,運動座標系の速度が,光速に対して,十分に
小さい場合にも成り立っていないといけない.つまり,光速に対して十分に小さい場合には,
ガリレイ変換が成立していることが必要なのである.
ローレンツ変換式が,ガリレイ変換式を特別な場合(速度が十分に小さい)において含んでいるかどうかを
確認する必要がある.運動座標系の速度が,光速に対して十分に小さいという条件を表す式は,
$v\ll c$ である.

\begin{mycomment}
以下の確認に必要な近似式を書き下しておこう.
$\alpha \ll 1$ の場合,
    \begin{equation*}
        (1+\alpha )^{n}\,\simeq\, 1+n\alpha
    \end{equation*}
が成立する.
\end{mycomment}

まず,時間 $ct$ について考える.時間に関するローレンツ変換式は
    \begin{align*}
    ct'=\frac{1}{\sqrt{ 1-(v/c)^{2} }} ct -\frac{v/c}{\sqrt{ 1-(v/c)^{2} }} x
    \end{align*}
であった.この式を近似式が適用しやすい形に書き換える.
    \begin{equation*}
    ct'={\biggl\{{ 1-\left(\frac{v}{c}\right)^{2} }\biggr\}}^{-\frac{1}{2}} ct -\frac{v}{c}\biggl\{{{ 1-\left(\frac{v}{c}\right)^{2} }}\biggr\}^{-\frac{1}{2}} x.
    \end{equation*}
この式の括弧中の式にたいして,近似式を適用する.
    \begin{align*}
        ct'&\simeq \frac{1}{2}\left(\frac{v}{c}\right)^{2}ct-\frac{v}{c}\biggl\{1+\frac{1}{2}\left(\frac{v}{c}\right)^{2}\biggr\}x \\
           &= ct + \frac{1}{2}\left(\frac{v}{c}\right)^{2}ct-\frac{v}{c}x-\frac{1}{2}\left(\frac{v}{c}\right)^{3}x
    \end{align*}
運動座標系の速度 $v$ が,光速 $c$ に対して十分に小さいとき,
$v\ll c$ であるので,
    \begin{equation*}
    \frac{v}{c}\quad \longrightarrow \quad 0
    \end{equation*}
として,
    \begin{equation*}
    ct'=ct \quad \therefore\; t'=t
    \end{equation*}
を得る.これはガリレイ変換式と一致する.
逆変換に関しても同様に成り立つことを示せる.
よって,時間 $t$ に関して,ローレンツ変換は
ガリレイ変換と矛盾しないことが分かった.

次に,$x$ について考える.$x$ も $ct$ の場合と同様に考えられる.
$x$ に関するローレンツ変換式は
    \begin{equation*}
    x'=-\frac{v/c}{\sqrt{ 1-(v/c)^{2} }}ct+\frac{1}{\sqrt{ 1-(v/c)^{2} }} x
    \end{equation*}
であった.近似式を適用しやすいように書き換えると,
    \begin{equation*}
    x'=-\frac{v}{c}\biggl\{1-\left(\frac{v}{c}\right)^{2}\biggr\}^{-\frac{1}{2}}ct
    +\biggl\{1-\left(\frac{v}{c}\right)^{2}\biggr\}^{-\frac{1}{2}}x
    \end{equation*}
近似式を適用して,
    \begin{align*}
    x'&\simeq -\frac{v}{c}\biggl\{1+\frac{1}{2}\left(\frac{v}{c}\right)^{2}\biggr\}ct
    +\biggl\{1+\frac{1}{2}\left(\frac{v}{c}\right)^{2}\biggr\}x \\
    &=-vt-\frac{1}{2}\left(\frac{v}{c}\right)^{3}ct+x+\frac{1}{2}\left(\frac{v}{c}\right)^{2}x
    \end{align*}
運動座標系の速度 $v$ が,光速 $c$ に対して十分に小さいとき,
$v\ll c$ であるので,
    \begin{equation*}
    \frac{v}{c}\quad \longrightarrow \quad 0
    \end{equation*}
として,
    \begin{equation*}
    x'=-vt+x
    \end{equation*}
を得る.これはガリレイ変換に一致する.
逆変換に関しても,同様に一致することを示せる.
よって,空間座標 $x$ に対して,
ローレンツ変換はガリレイ変換に矛盾しないことが確認できた.



以上のように,ガリレイ変換はローレンツ変換の特殊な場合に含まれる.
今まで,ガリレイ変換しか実感できなかったのは,対象の物体の速度が
,光速に比べて非常に小さかったからである.
光速不変の原理から,ガリレイ変換は成り立たないと思われたが,
実際はむしろその逆で,ガリレイ変換を包括する形で,ローレンツ変換
に拡張された.従って,ガリレイ変換という考え方は間違っているわけではない.
ただ,光速に比べて非常にゆっくりとした物体の速度という,
特殊な場合を見ていただけだったのである.




\subsection{棒の長さの収縮}

    異なる速度で運動する2つの慣性系$S$系,$S'$ 系を考える.
    ニュートン力学で物体の大きさを考えるとき,どのような慣性系
    からその物体の大きさを測定しても,全く同じ大きさを
    得ることができる.しかし,ローレンツ変換に従う特殊相対性理論に
    よれば,物体の大きさは慣性系によって違った値をとることが
    結論される.

    例えば,$x$ 軸上に,軸に平行に置かれた長さ $l$ の棒を考えてみよう.
    この棒は,$S$系に対して静止しているものとする.従って,この棒の長さを
    $S$系で測定したとき,$l$ を得る.しかし,この棒を$S$系に対して運動する
    $S'$系で,その長さを測定したとき,どのような長さになるだろうか.
    これを調べるには,ローレンツ変換式を用いるとよい.ニュートン力学における
    ガリレイ変換に従うのであれば,棒の長さは,どのような速度で運動する
    系から見ても,同じ大きさを示す.しかし,光速度不変の原理によれば,
    光速に近い速度ではもはやガリレイ変換は成り立たず,
    それはローレンツ変換式を考慮しなければならない.

    考察を簡単にするために,$S$系と$S'$系が同軸($x$,$x'$軸上)で
    等速運動している状態を考える.さらに,$S$系と$S'$系は,時刻 $t=t'=0$ で
    それぞれの原点 O,O$'$ が重なるとする.
    そして,棒の長さの測定を $t=t'=0$ の時刻に行うことにする.
    測定する棒は $x$ 軸方向に伸びているから,その両端も $x$ 軸上にある.
    この棒の両端の座標を $x_{1}$,$x_{2}$ としよう.ただし,
    $x_{2}>x_{1}$ であるとする.このとき棒の長さ $l$ は,$l=x_{2}-x_{1}$ と
    表現できる.
    \textbf{棒の長さを測定するということは,棒の両端の座標 $x_{1}$ と $x_{2}$ を
    “同時に”測定するということである}.
    棒に対して静止している系から,この棒の長さを測定するともちろん,$l$ である.
    しかし,棒に対して光速に近い速度で等速直線運動している系から測定すると,
    これは $l$ よりも小さい値を示すことになる.
                \begin{figure}[hbt]
                    \begin{center}
                        \includegraphicsdefault{Length.pdf}
                        \caption{異なる2つの系から,棒の長さを測定する}
                        \label{fig:Length}
                    \end{center}
                \end{figure}


    今回は,静止している棒の長さを,運動している座標系から測定するので,
    上で導出したローレンツ変換式の逆変換の式を適用することになる.
    $x$ に関するローレンツ変換式に今回の仮定($t=t'=0$)を代入して,
        \begin{align}
        x=\frac{1}{\sqrt{ 1-(v/c)^{2} }} x'
        \end{align}
    とする.この式から,$S'$系から見た $x_{1}$,$x_{2}$ の座標を
    計算できて,
        \begin{equation*}
        x_{1}=\frac{1}{\sqrt{ 1-(v/c)^{2} }} x_{1}'
        \;,\quad
        x_{2}=\frac{1}{\sqrt{ 1-(v/c)^{2} }} x_{2}'
        \end{equation*}
    となる.$S'$系から見た棒の長さ $l'$ は $l'=x'_{2}-x'_{1}$ であるので,
    両辺の差をとって
        \begin{align*}
        x_{2}-x_{1} &=\frac{1}{\sqrt{ 1-(v/c)^{2} }} x'_{2}-\frac{1}{\sqrt{ 1-(v/c)^{2} }} x'_{1} \\
                    &=\frac{1}{\sqrt{ 1-(v/c)^{2} }}(x_{2}'-x_{1}')
        \end{align*}
    である.そして,$l=x_{2}-x_{1}$,$l'=x'_{2}-x'_{1}$ から
        \begin{equation*}
        l\;=\;\frac{1}{\sqrt{ 1-(v/c)^{2} }}l'.
        \end{equation*}
    よって,\\
    \begin{itembox}[l]{\textbf{棒の長さの収縮}}
        \begin{align}
        l'=l\sqrt{ 1-(v/c)^{2} }
        \end{align}
    \end{itembox}\\
    を得る.この式から,物体の長さを,
    光速に近い速度で運動している系から測定すると,
    静止している系から測定する場合に比べて,
    $\sqrt{ 1-(v/c)^{2} }$ の長さに変化することがわかる.$v$ の大きさは
    光速よりも小さいと考えたとき
    \footnote{
        運動座標系の速度 $v$ が光速 $c$ を越えられないことについては後で確認することである.
    },
    $\sqrt{ 1-(v/c)^{2} }$ は常に $0<\sqrt{ 1-(v/c)^{2} }<1$ の
    範囲の値をとる.従って,運動系から見ると,棒の長さは短くなることが
    結論できる.

    \begin{memo}{座標の収縮}
    棒の長さが収縮するということを確認したときに,棒の長さ $l$ とは,
    その両端の $x$ 座標である $x_{1}$,$x_{2}$ の差 $x_{2}-x_{1}$ であるとした.
    つまり,
    \begin{equation*}
        l=x_{2}-x_{1}
    \end{equation*}
    である.そして,静止している座標系から,運動する棒の長さを測定すると,その棒の長さは,
    静止している座標系で測るよりも $\sqrt{ 1-(v/c)^{2} }$ 倍だけ短く見えることを
    確認した.確かに,棒は収縮して見えるが,棒そのものが収縮しているわけではない.
    棒の長さは,その両端の座標の差なので,つまり,座標の間隔が収縮しているということになる.
    棒が収縮したように見える原因は,\textbf{座標が収縮したように観測される} からである.
        \begin{figure}[hbt]
            \begin{center}
                \includegraphicsdefault{shushuku1.pdf}
                \caption{運動方向($x$ 軸方向)の座標間隔が収縮して見える}
                \label{fig:shushuku1}
            \end{center}
        \end{figure}

    物体の収縮が観測される原因は,この空間の収縮によるものである.
    実際に観測者が収縮している慣性系にいたとしても,その収縮を感じ取ることはできない.
    観測者自身も座標の収縮により同じように縮んでしまうからだ.自身の収縮を測ることは
    原理的に不可能である.別の慣性系にいる観測者が,自分に対し「お前は収縮している」と
    言っていても,自分はその収縮を観測することは絶対に不可能である
        \footnote{
            更に言うなら,自分はその別の慣性系の観測者が収縮していることを観測する.
        }.
    \end{memo}

    \begin{memo}{例}
        例えば,$S'$系が光速の0.8倍の速度で運動している場合を考えてみると,
        $v=0.8c$ を代入して,
            \begin{align*}
            l'&=\sqrt{ 1-(0.8c/c)^{2} }\, l \\
              &=\sqrt{ 1-0.8^{2} }\, l \\
              &=\sqrt{ 0.36 }\, l \\
              &=0.6\, l.
            \end{align*}
        この場合,棒の長さは,0.6倍に短くなって見えることになる.
    \end{memo}

\subsection{運動する時計の,時間の遅れ}
異なる速度で運動している2つの系 $S$,$S'$ に,それぞれ
時計を置く.このとき,一方の座標系から見た,他方の時計の進み具合を,
自身の時計の進み具合とと比較することを考える.
ここでは,$S$ 系の時計から見た,$S'$ 系の時計の進み具合を
考える.同時刻の概念が成り立たなくなったことを確認したことから,
これら2つの時計の進み具合が一致しないことは想像がつくだろう.
ここでそれを確認しておこう.

考察を簡単にするために,いくつかの仮定をおこう.
まず,座標系の運動は一方向で,これを $x$ 軸方向にとる.
また,両座標系の $x$ 軸,$x'$ 軸は,同じ直線上にあるとする.
さらに,2つの座標系は時刻 $t=t'=0$ に原点が一致するものとする.
各時計は,それぞれの座標原点に置かれているものとする.
時刻の進み具合を比較するために,時刻 $t=t'=0$ には2つの時計は
同じ時刻(例えば12時など)を指し示しているとしよう.

                \begin{figure}[hbt]
                    \begin{center}
                        \includegraphicsdefault{RT_jikan_okure.pdf}
                        \caption{時間の遅れ}
                        \label{fig:RT_jikan_okure}
                    \end{center}
                \end{figure}

さて,今この瞬間に両座標系の原点が一致し,
異なる速度で $x$ 方向を運動し始めた.$S$ 系ある時計を基準として,
$S'$ 系の時計の進み具合を観測しよう.$S$ 系から見て,時間 $t$ が経過したとき,
$S'$ 系にある時計の指す時刻を測ればよい.$S$ の時計が時刻 $t$ を示したとき,
$S'$ 系の時計の位置は,$x=vt$ である.これをローレンツ変換式に代入すると,
    \begin{align*}
    ct'&=\frac{1}{\sqrt{ 1-(v/c)^{2} }} ct -\frac{v/c}{\sqrt{ 1-(v/c)^{2} }} x \\
    ct'&=\frac{1}{\sqrt{ 1-(v/c)^{2} }} ct -\frac{v/{c}^{2}}{\sqrt{ 1-(v/c)^{2} }} vt \\
    \Leftrightarrow \quad
    t'&=\left(\frac{1}{\sqrt{ 1-(v/c)^{2} }} -\frac{(v/c)^{2}}{\sqrt{ 1-(v/c)^{2} }}\right) t \\
    &=\frac{1-(v/c)^{2}}{\sqrt{ 1-(v/c)^{2} }}t \\
    \end{align*}
よって,\\
    \begin{itembox}[l]{\textbf{運動する系の時間の遅れ}}
    \begin{align}
    \therefore \quad
    t'=t\sqrt{ 1-(v/c)^{2} }
    \end{align}
    \end{itembox}\\

この式から,静止している系から,運動している系の時間を測定しようとすると,
静止している系の時間の進む速さよりも,運動している系の時間の進みの方が
ゆっくりであるように観測されることがわかる.

    \subsection{複数の時計の,時刻の合わせ方}
    同じ座標系において,2つの時計があるとしよう.この2つの時計の性能は全く同じであると
    する.従って,時計の進み具合も全く同じである.同一の座標系であっても,座標の各点に
    おける時刻が同一でないと,議論ができない.従って,同一座標系の各店の時刻は,全て同じ時刻
    を指していなければならない.全ての時刻を合わせる最初の手順として,2つの点の時刻を合わせて,
    この2つの点の時刻と,また別の時刻の点を合わせて,というような方法をとろう.では,
    2つの時計の間の距離が大きいとき,
    この二つの時計の時刻を合わせるにはどうしたらよいだろうか.
    一方の時計を,他方に
    近づけてしまうと,動かした時計は時間の進みが遅くなり,時間を合わせられたとしても,元の
    位置に戻した時には時間はあっていなくなる.どうにかして,時計の位置を変化させることなく,
    離れた2つの時計の時刻を合わせたい.アインシュタインは次のように時刻を合わせることを提案した.

    時刻を合わせるために,光を利用する.一方の時計のある点をA,他方をBとしよう.
    まず,点Aの時刻($t_{0}$ としよう)を測ると同時に,点Aから点Bに向けて光を発信する.
    光の速さは,光速度不変の原理から,常に一定の値 $c$ をとる.すると,光は点Bに届くはずである.
    点Bに光が届いた瞬間($t_{1}$ としよう)を記録すると同時に,点Bからもう一度点Aに向けて光を返信する.
    そして,点Aに光が戻ってきたときの時刻($t_{2}$ としよう)を記録する.
    点Aと点Bの時計の指す時刻が同じであるとき,
    \begin{align}\label{eq:jikokuawase}
    t_{1}-t_{0}=t_{2}-t_{1}
    \end{align}
    が成立しているはずである.また逆に,この式(\ref{eq:jikokuawase})が成り立っていれば,
    2点のそれぞれの時計は同じ時刻を指しているとする.

    さて,点Aと点Bの間の距離を $\|AB\|$ と表わせば,
    以下の式が成立している.
    \begin{align}
    \frac{2\| AB \|}{t_{2}-t_{0}}=c.
    \end{align}
                \begin{figure}[hbt]
                    \begin{center}
                        \includegraphicsdefault{jikokuawase1.pdf}
                        \caption{時刻合わせ}
                        \label{fig:jikokuawase}
                    \end{center}
                \end{figure}


\subsection{速度の合成}

    光速 $c$ で動く光源から光を発したら,その速度静止系に対して $2c$ を示すだろうか.
    もし $2c$ が成り立ってしまったら,光速度不変の原理を満たさなくなってしまう.
    実際計算してみると,値はどのような速度で運動している座標系で見ても光速は $c$ である.
    これは光速度不変の原理を仮定しているからあたりまえのことであると考えられるが,
    ここではローレンツ変換公式から,光速度不変の原理と矛盾していないかを確認する.


    考察を簡単にするために以下のような条件の下で考える.
    まず,運動の方向を $x$ 軸方向とする.つまり,座標系・物体は $x$ 軸に平行に運動するのものする.
    速度 $v$ で運動する座標系から観測して,速度 $u'$ で動く物体を考える.
                \begin{figure}[hbt]
                    \begin{center}
                        \includegraphicsdefault{RT_sokudo_gousei.pdf}
                        \caption{速度の合成}
                        \label{fig:RT_sokudo_gousei}
                    \end{center}
                \end{figure}


    目標は,静止している座標系から見た物体の速度 $u$ を知ることである.
    これは一気に求めることができないので,順を追って考える.
    まず,速度 $v$ 運動している座標系からみた物体の速度 $u'$ は,
        \begin{align}
            u' = \frac{\df x'}{\df t'}
        \end{align}
    である.ここに,$x'$,$t'$ は運動座標系の座標変数を表す.
    さて,求めたいのは,静止系から見た物体の速度 $u$ であり,これは
        \begin{equation*}
            u= \frac{\df x}{\df t}
        \end{equation*}
    である.これは,次のように計算できる.
    $x$ はローレンツ変換式で $t'$ の関数と見ることができる($x=x(t')$).さらに,
    $t'$ は,同様に,ローレンツ変換式から $t$ の関数と見ることができる($t'=t'(t)$).
    以上から $x$ は,合成関数 $x=x\left( t'(t) \right)$ とみなせる.
    従って,合成関数の微分法を用いることで,$u$ を計算できる.
        \begin{align}
        u=\frac{\df x}{\df t'}\frac{\df t'}{\df t}=\frac{\df x}{\df t'}\bigg/ \frac{\df t}{\df t'}.
        \end{align}
    さて,$\df x/\df t'$ から計算していこう.
        \begin{equation*}
            x=\frac{v/c}{\sqrt{ 1-(v/c)^{2} }}ct'+\frac{1}{\sqrt{ 1-(v/c)^{2} }} x'
        \end{equation*}
        \begin{align*}
        \frac{\df x}{\df t'}&=\frac{v/c}{\sqrt{ 1-(v/c)^{2} }}c+\frac{1}{\sqrt{ 1-(v/c)^{2} }} \frac{\df x'}{\df t'} \\
            &=\frac{v}{\sqrt{ 1-(v/c)^{2} }}+\frac{u'}{\sqrt{ 1-(v/c)^{2} }}
        \end{align*}
    \begin{align}
        \therefore\quad\frac{\df x}{\df t'}=\frac{v+u'}{\sqrt{ 1-(v/c)^{2} }}
    \end{align}
    次に,$\df t'/ \df t$ を計算する.
        \begin{align*}
            ct&=\frac{1}{\sqrt{ 1-(v/c)^{2} }} ct' +\frac{v/c}{\sqrt{ 1-(v/c)^{2} }} x' \frac{\df t'}{\df t} \\
            &=\frac{1}{\sqrt{ 1-(v/c)^{2} }} +\frac{v/{c}^{2}}{\sqrt{ 1-(v/c)^{2} }} \frac{\df x'}{\df t'} \\
            &=\frac{1}{\sqrt{ 1-(v/c)^{2} }}+\frac{u'v/{c}^{2}}{\sqrt{ 1-(v/c)^{2} }}
        \end{align*}
    \begin{align}
    \therefore\quad\frac{\df t'}{\df t}=\frac{1+u'v/{c}^{2}}{\sqrt{ 1-(v/c)^{2} }}
    \end{align}
    以上から,
        \begin{equation*}
            u=\frac{\df x}{\df t'}\bigg/ \frac{\df t}{\df t'}
            =\frac{v+u'}{\sqrt{ 1-(v/c)^{2} }}\bigg/\frac{1+u'v/{c}^{2}}{\sqrt{ 1-(v/c)^{2} }}
        \end{equation*}
    従って,\\
    \begin{itembox}[l]{\textbf{速度の合成}}
        \begin{align}\label{eq:velocity_conversion}
            u=\frac{v+u'}{1+u'v/{c}^{2}}
        \end{align}
    $u$ は静止した座標系から観測した物体の速度であり,$v$ は運動する座標系の速度であり,
    $u'$ は速度 $v$ で運動する座標系から観測した物体の速度である.
    \end{itembox}\\
    を得る.


    \begin{memo}{例1}
        運動座標系の速度が光速の0.8倍であるとき,この運動座標系に対して光速の0.4倍の速度で運動する
        物体を考える.この物体を静止している座標系から眺めたとき,物体の速度はどの程度かを計算する.
        これは上式に直接代入すればよい.運動座標系の速度 $v$ は光速の0.8倍であるから,$v=0.8c$ である.
        また,物体の運動座標系に対する速度 $u'$ は光速の0.4倍だから $0.4c$ である.
        従って,
        \begin{align*}
         u&=\frac{v+u'}{1+u'v/{c}^{2}}=\frac{0.8c+0.4c}{1+0.4c\times 0.8c/{c}^{2}} \\
          &=\frac{1.2c}{1.32}=0.909c.
        \end{align*}
        \begin{equation*}
        \therefore \quad u=0.91c.
        \end{equation*}
        従って,静止している座標系から物体の速度を測定すると,光速の0.91倍の速度である.

        この速度の合成式には,「物体の速さは光速 $c$ を越えることは無い」ということを暗示する.
    \end{memo}

    \begin{memo}{例2}
        光速で運動している座標系から光を発すると,静止座標系から見て
        光の速度は $c+c=2c$ となってしまうのではないか.
        もしそうなれば,光速度不変の原理は成り立たなくなってしまう.どうなるかを確認しておこう.
        \begin{equation*}
        u=\frac{v+u'}{1+u'v/{c}^{2}}=\frac{c+c}{1+c\times c/{c}^{2}}
        =\frac{2c}{2}=c.
        \end{equation*}
        \begin{equation*}
        \therefore \quad u=c.
        \end{equation*}
        従って,静止している座標系において,運動する座標系から発した光の速度を
        測定しても,その値は $c$ であることがわかる.
        これは,光速度不変の原理に矛盾しない.
    \end{memo}

\subsection{相対性}
        相対性理論を学習するとき,「長さの収縮」,「時間の遅れ」など誤解を招きやすい表現が多用される.

        例えば,二人の観測者A,Bが互いに相対速度をもっていて,
        両者共に相手の進行方向の長さを比較するとしよう.
        観測者Aが,観測者Bの進行方向の長さを測ったとき,相対速度が0のときに比べてその長さは小さくなる.
        このときに他方の観測者Bの立場から考えれば,観測者Aが収縮していることになる.
        両者共に,自分自身との相対速度は0だから,自分の収縮は観測されない.
        また,時間についても同じことが言える.両者の立場で共に,相手の時間の
        進み具合は,自分の時間の進み具合に比べて,ゆっくりであると観測される.

        たしかに,「収縮・遅延」という語を,普段の生活で使う言葉の意味と
        して考えてしまうと,観測者AとBの主張は,互いに矛盾しいる.
        しかし,相対性理論を考える場合には,「収縮・遅延」という語は,
        このような意味で使っているのではない.本当に収縮あるいは遅延している
        訳ではない.
        観測者とは別の速度で運動している慣性系を観測すると,
        収縮あるいは遅延しているように見えるのである.

        もう少し詳しく考えてみよう.
        ここで問題としているのは,観測者とは異なる速度で運動している物体の長さや時間である.
        物体の「長さ」は,観測者がその長さを測ることで知ることができる.
        ”長さを測る”とはどういうことか.アインシュタインは長さの測定方法として,2つの提案をしている.
            \begin{description}
                \item{方法1) } 自分と同じ慣性系にある定規を観測対象に直接くっつけて,長さを特定する方法
                \item{方法2) } もう一つは,空間に座標を設けて,物体の両端の座標を“同時に”特定する方法
            \end{description}
        物体の両端の位置を同時に把握するという意味で,方法1も方法2も本質的には同じ行為である.
        ここでは多くの教科書で説明のある,方法2に着目してみる.
        方法2は座標を設けて,物体の両端の座標を“同時に”捕える方法である.
        これまで見てきたように,“同時”という事実は,観測者ごとに異なる.
        観測者Aが物体の両端を“同時”に測定したとしても,それを別の観測者Bが見たときに“同時”に測定していないことになる.
        物体の両端の座標を“同時”に測定していないとすれば,長さを測れていないことになる.
        観測者Aは間違いなく同時に両端の位置を測定しているのにもかかわらず,その観測者Aの行動を別の慣性系にいる観測者Bが見ると
        両端の位置をそれぞれ別の時刻に観測しているのである.

        どう考えても矛盾だと考えがちだが,これはよく考えれば矛盾ではない.
        物体の両端を同時に測定するということは,両端の位置を同時に見るということだ.
        ものを見るということは,ものから発光あるいは反射される光をとらえる必要がある.
        物体の長さを測るには,物体の両端からくる光を同時に見ることと同じだ.
        しかし,光の速度は一定
            \footnote{
                どんな速度で運動していようと,光の進む速さ(光速)は常に一定である.
                これは特殊相対性理論の根本原理であり,光速不変の法則と呼ばれる.
                光速普遍のはマクスウェル方程式から暗示されるが,別の原理から導かれる
                ものではなく,その原因はわからない.
                しかし,マイケルソンとモーリーの実験に代表されるように,
                この光速不変の原理は実験的に実証されている現象である.
            }
        だから,観測者との距離により,光の届く時刻はずれる
            \footnote{
                光源との距離が大きいほど,光の到達時間は大きくなる.
            }.
        観測者AとBは異なる慣性系にいるので,物体との距離も異なる.
        要するに,物体から発せられた光が観測者AとBに届く時刻は同時ではない.
        だから,観測者Aにとって同時でも,観測者Bにはずれて見える,ということが起こる.

        観測は観測者が主観的に行うものである.
        仮に,この観測者Aと観測者Bの両者を観測する観測者Cが存在しても,矛盾は起きない.
        観測者Cに対して同時に起きる現象は,観測者Aとも観測者Bとも異なるのだ.
        複数の異なる慣性系で同時に測定することは原理的に不可能なのである.

        ニュートン物理学では,物体の長さや時間の進み具合は,どの観測者でも全く同じであると考えてきた.
        そういう感覚があると,相対性理論で提示される棒の収縮や時間の遅れという現象に,矛盾があるように感じてしまう.
        しかし,これはあくまでも矛盾しているような気がするだけであって,実際に矛盾しているのではない.
        物体の収縮や時間の遅れという現象は,特殊相対性原理と光速不変の原理から,論理とまた数学により,必然的に導かれる結果である.
        特殊相対性理論は,物体の長さや時間の進み具合が,観測者と物体の相対速度に依存することを示したのである.
        考え方を改めないとならない.自分はあくまでも一人の観測者である.
        別の慣性系にいる観測者の観測結果と自分の観測結果が異なっていても,矛盾ではない.
        なぜなら,別の観測者と観測対象の相対速度と,自分と観測対象の相対速度が違うから,異なる結果になるのだから.

\subsection{光のドップラー効果}

\subsection{アインシュタインの理論と,他の物理学者の理論}
Poincar\'{e}等の多くの物理学者が,光速度不変の問題を解決しようと,数学的に
特殊相対性理論と同等な理論を,発表しているそうである.しかし,これらの理論
は,なぜエーテルが実験的に見つからないかを説明しようとするものであったり,
エーテルの存在を仮定しているものである等,根本的な解決にはならなかったよう
である.これに対して,アインシュタインは,大胆にも,絶対静止系を捨てる,つまり,エ
ーテルなんてものは,はじめからなかったのだ,と宣言し,理論を組み立てた.
物理的に最も納得のできる方法で説明したのが,アインシュタインである.
しかも,前提となる仮定は特殊相対性原理と光速度不変の原理の2つだけで.
