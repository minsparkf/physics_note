\section*{このノートで使用する記号}
    このノートで使用する記号について,
    以下に示しておく.
    ノートを読んでいて分からくなった記号
    があったら,これを参考にしてもらいたい.
    もちろん,用語については本文を参照にすること.

    複数の意味で使われる記号がいくつもあるが,
    それらについてはスラッシュ / で区切りをいれた.

    % [約束] 一行に収まる程度(30文字程度)で説明すること(体裁のため)

    %===========================================================
    % * アルファベット(斜太字:大文字)
    %===========================================================
    \subsubsection*{アルファベット(太字:大文字)}
    \begin{tabular}{lll}
        $\bE$                   &:  & 電場                                                                  \\
        $\bA$                   &:  & ベクトルポテンシャル                                                  \\
        $\bB$                   &:  & 磁束密度                                                              \\
        $\bI$                   &:  & 電流                                                                  \\
        $\bD$                   &:  & 電束密度                                                              \\
        $\bH$                   &:  & 磁場                                                                  \\
        $\bP$                   &:  & 誘電分極                                                              \\
        $\bL$                   &:  & 角運動量                                                              \\
        $\bN$                   &:  & 回転力(トルク)                                                      \\
        $\bF$                   &:  & 力(合成力)                                                          \\
        $\bM$                   &:  & 磁化                                                                  %\\
    \end{tabular}

    %===========================================================
    % * アルファベット(斜太字:小文字)
    %===========================================================
    \subsubsection*{アルファベット(太字:小文字)}
    \begin{tabular}{lll}
        $\br$                   &:  & 位置,$(\,x,\,y,\,z\,)$ と同じ.                                      \\
        $\bv$                   &:  & 速度,$(\,v_{x},\,v_{y},\,v_{z}\,)$ と同じ                            \\
        $\ba$                   &:  & 加速度,$(\,a_{x},\,a_{y},\,a_{z}\,)$ と同じ                          \\
        $\bp$                   &:  & 運動量,$(\,p_{x},\,p_{y},\,p_{z}\,)$ と同じ                          \\
        $\bi$                   &:  & 電流密度,$(\,i_{x},\,i_{y},\,i_{z}\,)$ と同じ                        \\
        $\bldf$                 &:  & 分解された力                                                          %\\
    \end{tabular}

    %===========================================================
    % * アルファベット(斜細字:大文字)
    %===========================================================
    \subsubsection*{アルファベット(細字:大文字)}
    \begin{tabular}{lll}
        $G$                     &:  & 万有引力定数,または,重力定数                                        \\
        $U$                     &:  & ポテンシャルエネルギー                                                \\
        $T$                     &:  & 運動エネルギー/張力                                                   \\
        $E$                     &:  & 全エネルギー/電場の大きさ                                      \\
        $U$                     &:  & ポテンシャル/内部エネルギー                                 \\
        $L$                     &:  & 角運動量/ラグランジアン                                               \\
        $I$                     &:  & 力積/電流の大きさ                                              \\
        $S$                     &:  & 面積/任意の閉曲面                                                     \\
        $V$                     &:  & 体積                                                                  \\
        $O$                     &:  & 原点                                                                  \\
        $H$                     &:  & ハミルトニアン/磁場の大きさ                            %\\
    \end{tabular}

    %===========================================================
    % * アルファベット(斜細字:小文字)
    %===========================================================
    \subsubsection*{アルファベット(細字:小文字)}
    \begin{tabular}{lll}
        $l$                     &:  & 長さ/任意の閉曲線                                                     \\
        $m$                     &:  & 質量(慣性質量と重力質量)                                                   %\\
    \end{tabular}

    %===========================================================
    % * アルファベット(添字付き)
    %===========================================================
    \subsubsection*{アルファベット(添字付き)}
    \begin{tabular}{lll}
        $m_{\mathrm{i}}$        &:  & 慣性質量                                                              \\
        $m_{\mathrm{g}}$        &:  & 重力質量                                                              \\
        $S_{l}$                 &:  & 任意の閉曲線 $l$ を縁とする曲面                   \\
        $k_{B}$                 &:  & ボルツマン定数                                                        %\\
    \end{tabular}

    %===========================================================
    % * アルファベット(立太字:大文字)
    %===========================================================

    %===========================================================
    % * アルファベット(立太字:小文字)
    %===========================================================

    %===========================================================
    % * アルファベット(立細字:大文字)
    %===========================================================
    \subsubsection*{アルファベット(立細字:大文字)}
    \begin{tabular}{lll}
        K                       &:  & 任意の慣性系                                                          \\
        K$'$                    &:  & Kとは別の任意の慣性系                                                 %\\
    \end{tabular}

    %===========================================================
    % * アルファベット(立細字:小文字)
    %===========================================================

    %===========================================================
    % * 花文字
    %===========================================================
    \subsubsection*{花文字}
    \begin{tabular}{lll}
        $\qL$                   &:  & ラグランジアン密度                                                    \\
        $\qH$                   &:  & 量子論ハミルトニアン演算子                                            \\
        $\flwF$                 &:  & 一般化された力                                                        \\
        $\flwE$                 &:  & 起電力                                                                %\\
    \end{tabular}

    %===========================================================
    % * ギリシア文字(大文字)
    %===========================================================
    \subsubsection*{ギリシア文字(大文字)}
    \begin{tabular}{lll}
        $\Sigma$                &:  & 総和の記号                                                            \\
        $\Omega$                &:  & 任意の領域の内部                                 %\\
    \end{tabular}

    %===========================================================
    % * ギリシア文字(小文字)
    %===========================================================
    \subsubsection*{ギリシア文字(小文字)}
    \begin{tabular}{lll}
        $\gamma$                &:  & ローレンツ因子                                                        \\
        $\nu$                   &:  & 周波数                                                                \\
        $\rho$                  &:  & 電荷密度/電気抵抗率                                                   \\
        $\sigma$                &:  & 電気伝導率                                                            \\
        $\phi$                  &:  & 電気的なポテンシャルエネルギー                                        %\\
    \end{tabular}

    %===========================================================
    % * ギリシア文字(添字付き)
    %===========================================================
    \subsubsection*{ギリシア文字(添字付き)}
    \begin{tabular}{lll}
        $\Omega_{S}$            &:  & 任意の閉曲面の内側の領域                         %\\
    \end{tabular}

    %===========================================================
    % * 数式記号
    %===========================================================
    \subsubsection*{数式記号の意味}
    $\bA$,$\bB$ は一般の(特に意味を与えない)ベクトルであるとする.
    $\br$ は位置ベクトル,$\bt$ は線素ベクトル,$\bS$ は面素ベクトルとする.
    また,$f(x)$ は $x$ を独立変数とする一次関数であり,
    $f(x,\,y,\,\cdots)$ は $(x,\,y,\,\cdots)$ を独立変数とするとする多変数関数である.\\

    \begin{tabular}{lll}
            $\bA \perp \bB$                                         &:  & $\bA$ は $\bB$ に \textbf{垂直}                          \\
            $\bA \parallel \bB$                                     &:  & $\bA$ は $\bB$ に \textbf{平行}                          \\
            $\bA_{\perp \bB}$                                       &:  & $\bA$ の $\bB$ に \textbf{垂直な成分}                    \\
            $\bA_{\parallel \bB}$                                   &:  & $\bA$ の $\bB$ に \textbf{平行な成分}                    \\
            $\displaystyle \int f(x) \df x$                         &:  & 積分         \\
            $\displaystyle \oint f(\br) \cdot \df \bt$              &:  & 線積分        \\
            $\displaystyle \sint f(\br) \cdot \df \bS$              &:  & 面積分        \\
            $\displaystyle \vint f(\br) \df V$                      &:  & 体積積分      \\
            $\df f$                                                 &:  & 微分/全微分     \\
            $\displaystyle \frac{\df}{\df x} f(x)$                  &:  & 導関数          \\
            $\displaystyle \frac{\rd}{\rd x} f(x,\, y,\, \cdots)$   &:  & 偏微分      \\
    \end{tabular}

    %===========================================================
    % * 行内に書く場合の,演算子について
    %===========================================================
    \subsubsection*{行内に書く場合の演算子の形}
        \begin{tabular}{lll}
            $ab/cd$                 &:  & $\displaystyle \frac{ab}{cd}$ を文中で記す場合.            \\
            $(a/b)(c/d)$            &:  & $\displaystyle \frac{a}{b}\frac{c}{d}$ を文中で記す場合.   \\
            $\sum_{n=1}^{N}$        &:  & $\displaystyle \sum_{n=1}^{N}$ を文中で記す場合.           \\
            $\bigcap_{n=1}^{N}$     &:  & $\displaystyle \bigcap_{n=1}^{N}$ を文中で記す場合.        \\
            $\bigcup_{n=1}^{N}$     &:  & $\displaystyle \bigcup_{n=1}^{N}$ を文中で記す場合.        \\
            $\int$                  &:  & $\displaystyle \int$ を文中で記す場合.                     \\
            $\sint$                 &:  & $\displaystyle \sint$ を文中で記す場合.                    \\
            $\vint$                 &:  & $\displaystyle \vint$ を文中で記す場合.                    \\
    \end{tabular}


