%   %==============================================================================================
%   %  Section
%   %    Section Name : \section{量子力学と相対性理論との矛盾の原因}
%   %==============================================================================================
    \section{量子力学と相対性理論との矛盾の原因}
        量子力学の基本方程式は,エネルギーの関係式
            \begin{equation*}
                E = \frac{{p_{x}}^{2}}{2m} + U(x)
            \end{equation*}
        から
            \footnote{
                簡単のために,一次元で考える.$x$ 軸方向のみの運動であるとする.
            },
        対応原理により,エネルギーと運動量は演算子(作用素)によって書き換えられる.すなわち,
            \begin{equation*}
                E     \longrightarrow  i \hbar \frac{\rd}{\rd t}
                \quad,\quad
                p_{x} \longrightarrow -i \hbar \frac{\rd}{\rd x}
            \end{equation*}
        という変更がなされる.この演算子を,波動関数 $\psi$ に左側から作用させることにより,量子力
        学の基本方程式が得られる.すなわち,
            \begin{align}
                i \hbar \frac{\rd}{\rd t} \psi  =
                    \left( -\frac{\hbar^{2}}{2m} \frac{\rd^{2}}{\rd x^{2}} + U(x) \right) \psi
            \end{align}
        である.これはシュレーディンガー方程式とよばれる.

        しかし,このシュレーディンガー方程式は,相対性理論と相性が悪い.なぜかと言うと,
        シュレーディンガー方程式を作る上で基礎にしていたエネルギーの関係式が,相対性理論のエネルギー
        の関係式と異なるからである.

        シュレーディンガー方程式は,水素原子の電子軌道の計算などで,十分な
        精度の解を与える.けれども,ニュートン力学的なエネルギー関係式を採用していることにより,特殊相
        対性理論によってニュートン物理学が修正を余儀なくされるのと同じように,光速で運動する対象を記述
        しようとすると,事実と式とが一致しないのである.シュレーディンガー方程式は,ローレンツ変換に対して,不変ではないからである.

        そこで,以降では,相対性理論と矛盾なく成立するように,シュレーディンガー方程式を書き換えてい
        くことにしよう.
%   % End \section{量子力学と相対性理論との矛盾の原因}


%   %==============================================================================================
%   %  Section
%   %    Section Name : \section{クライン$=$ゴルドンの方程式}
%   %==============================================================================================
    \section{クライン$=$ゴルドンの方程式}
    エネルギーの関係式が相対性理論と合わないのであれば,この関係式を相対性理論に従った形のものを採
    用すればよい.

    相対性理論によれば,エネルギーの関係式は,次のように修正される.一次元で考える.\\
        \begin{itembox}[l]{{\bf 相対性理論によるエネルギー}}
            特殊相対性理論から要請されるエネルギーは,次式で表される.
            \begin{equation*}
                E^{2} = \left( mc^{2} \right)^{2} + \left( p_{x}c \right)^{2}.
            \end{equation*}
        \end{itembox}\\

    この式に,対応原理;
        \begin{equation*}
            E     \longrightarrow  i \hbar \frac{\rd}{\rd t}
            \quad,\quad
            p_{x} \longrightarrow -i \hbar \frac{\rd}{\rd x}
        \end{equation*}
    を適用しよう.その際,演算子の右側に波動関数 $\psi$ を記述しておく.
        \begin{align}
            E^{2}
                &= c^{2}{p_{x}}^{2} + m^{2}c^{4}  \notag \\  \notag \\
            - \hbar^{2} \frac{\rd^{2}}{\rd t^{2}} \psi
                &= \left( - c^{2} \hbar^{2} \frac{\rd^{2}}{\rd x^{2}} + m^{2}c^{4} \right) \psi
        \end{align}
    このままの式の形でも十分だが,もう少し整理してきれいにできる.両辺に $-1/c^{2}\hbar^{2}$ をか
    けるだけなのだが.
        \begin{equation*}
            \frac{1}{c^{2}}\frac{\rd^{2}}{\rd t^{2}} \psi
                = \left( \frac{\rd^{2}}{\rd x^{2}} - \frac{m^{2}c^{2}}{\hbar^{2}} \right) \psi
        \end{equation*}

    この式を \textbf{クライン$=$ゴルドンの方程式} とよぶ.\\
            \begin{center}
                \begin{shadebox}
                \textbf{クライン$=$ゴルドンの方程式}
                    \begin{align}\label{eq:KGEquation}
                        \frac{1}{c^{2}}\frac{\rd^{2}}{\rd t^{2}} \psi
                             = \left(
                                    \frac{\rd^{2}}{\rd x^{2}} - \frac{m^{2}c^{2}}{\hbar^{2}}
                                \right) \psi
                    \end{align}
                \end{shadebox}
            \end{center}

    クライン$=$ゴルドンの方程式は,相対性理論で要求されるローレンツ変換に対して,不変である.それは簡単に確
    かめることができる.上のクライン$=$ゴルドンの方程式を,次のようにみればよいだけである.
        \begin{equation*}
            \left(
                \frac{\rd^{2}}{\rd x^{2}} - \frac{1}{c^{2}}\frac{\rd^{2}}{\rd t^{2}} \psi
            \right)
                =   - \frac{m^{2}c^{2}}{\hbar^{2}} \psi
        \end{equation*}
    元の形空間微分 $\rd^{2}/{\rd x}^{2}$ を移項しただけだ.これを3次元に拡張すれば,
    よりはっきりするだろう.
        \begin{equation*}
            \left(
                \Delta - \frac{1}{c^{2}}\frac{\rd^{2}}{\rd t^{2}}
            \right)  \psi
                =   - \frac{m^{2}c^{2}}{\hbar^{2}} \psi.
        \end{equation*}
    ダランベルシアン $\Dal$ が見てとれる.すなわち,
        \begin{equation*}
                \Dal \psi
                =   - \frac{m^{2}c^{2}}{\hbar^{2}} \psi
        \end{equation*}
    である.ダランベルシアンと定数のみの式であるから,このKlien--Gordonの方程式はローレンツ変換に対し
    て不変であることが,一目瞭然である.
%   % End \section{クライン$=$ゴルドンの方程式}


%   %==============================================================================================
%   %  Section
%   %    Section Name : \section{クライン$=$ゴルドンの方程式の欠点}
%   %==============================================================================================
    \section{クライン$=$ゴルドンの方程式の欠点}
        クライン$=$ゴルドンの方程式により,シュレーディンガー方程式が相対性理論と矛盾しない形に近づいたのだ
        が,実はこのクライン$=$ゴルドンの方程式には欠点がある.それは,時間微分が2階になっていることに起
        因する欠点である.

        クライン$=$ゴルドンの方程式とは,次式のことであった.
            \begin{equation*}
                \frac{1}{c^{2}}\frac{\rd^{2}}{\rd t^{2}} \psi
                    = \left(
                        \frac{\rd^{2}}{\rd x^{2}} - \frac{m^{2}c^{2}}{\hbar^{2}}
                      \right) \psi.
            \end{equation*}
        この方程式を元にして粒子の確率密度を計算すると,その値が負の数であることも可能なのである.
%   % End \section{クライン$=$ゴルドンの方程式の欠点}

