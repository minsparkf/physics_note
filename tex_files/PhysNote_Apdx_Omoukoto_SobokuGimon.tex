%===================================================================================================
%  Chapter : はじめに
%  説明    : 「考えるとはどういうことか」とかを考えてみる.
%===================================================================================================
%   %==========================================================================
%   %  Section
%   %==========================================================================
        \section{最も基本的なこと}
            学問を学ぶにあたって,“最も基本的で信頼できること”を
            基礎にして,その上で学習を進めることは,当たり前のこと
            である.では,その“最も基本的で信頼できること”とは何
            だろうか.

            これは多分に哲学的な問題提起であるが,ここでは,今後,
            物理学の学習を進めていくにあたり,思想の最も基本的なよ
            りどころを確認するためのもので,哲学に深く介入すること
            はしない
                \footnote{
                    正直に書こう.哲学に介入することは,私のような
                    低レベルの頭では,不可能である.開き直って,言
                    うならば,そこまで深く考えてもしょうがない,思
                    うところもある.
                }.

            私は,最も基本的で信頼できることとして,「考えること」
            をあげたい.これは独我論てきな思想である.独我論とは,
            極端に言えば,この世界に存在を確信できるのは私の思考の
            みであり,私が今感じている温度や光などは,私の思考によ
            って感じていると錯覚しているのであって,実在しているの
            ではない,という考え方である.こう考えると,自分以外の
            人間とは,私の思考が作り出した幻想であり,実際にそこに
            いるわけではないということになる.そう,信じられるのは,
            今考えている私がここにあるということである.


%   %==========================================================================
%   %  Section
%   %==========================================================================
        \section{私の思想の根本}
            では,もう一段階突っ込んでみよう.「考える」ということ
            とは,どうすることなのだろうか.「考える」という動詞の
            使い方は,おおよそ次の様だろう.
                今晩の献立を考える.
                人の気持ちを考える.
                将来の進路を考える…
            などなど.
            考えるという作業を行っているとき,「言葉」を道具として
            使う.また,時には「図」を使って考えることもあるだろう
            が,これは単に言葉で考えるよりも図を用いた方が考えやす
            いからであり,言葉では思考不可能であるということではな
            い.思考はすべて言葉で表現できる(と信じる).

            私の,最も信頼できる唯一の基本的なことである,私の思考
            は言葉を用いて実行される.では,その次の疑問として,
            「言葉」とは何か,ということが生まれてこよう.

            この章では,「考えるとは何か」について,私が考えること,
            というか,思っていることを記述する.


%   %==========================================================================
%   %  Section
%   %==========================================================================
        \section{思考の道具}
            考えるという動作は,言葉を用いて行っていることを確認し
            た.言葉を用いて考えているので,この言葉の使用限界が思
            考の限界であるということになる
                \footnote{
                    Wittgensteinは「論理哲学論考」という著
                    作で,このことを詳しく論じている.後に,彼自身
                    によってこの著作は間違いであるとされてしまうの
                    であるが…このノートでは,そこまで深く入ら
                    ない.だって,とっても難しいから.
                }.

            単に「言葉」といっても,それは様々な形で存在する.英語
            やドイツ語,フランス語,イタリア語などたくさんだ.各国
            の人々は,自国のあるいは使い慣れた言葉で考えていること
            だろう.ここでは,どのような国の言葉も,その適用限界
            は変わらないと仮定して,話を進めて生きたい.多少,言葉
            の適用限界があったとしても,それは話にならないくらい,
            細かいことに過ぎないと信じる
                \footnote{
                    実際,各国の人々が同じように「考えて」いるとい
                    う現状からこのような仮定を設けてもよいと考えて
                    いる.ただし,ここでは,「考える」ということに
                    関して,文化や伝統,生活習慣などの影響は無視す
                    る.
                }.

             もちろん,言葉で説明できないこともある.ある種の“ひらめき”
             とか,もろもろの感情とかを言葉で表現することは難しいこ
             とである.いや,不可能といってもよい.しかし,考えると
             いうことに関しては,言葉のもつ機能は十分である考える.私
             は,「どんな思考も言葉にできる」と信じて,こ
             のノートを作成する.


%   %==========================================================================
%   %  Section
%   %==========================================================================
        \section{言語の曖昧さ}
            思考を言葉で記述できたとしよう.その次に問題となるのは,どれ
            だけ正確に思考できるか,ということだろう.いや,視点を変えて
            言い換えよう.私たちが行う多くの思考の中で,正しい思考とはど
            ういうものなのかを,整理しなければいけない.わけのわからない
            思考や,意味を成さない思考などを排除したいのである.


%   %==========================================================================
%   %  Section
%   %==========================================================================
        \section{日常言語}
            普段の生活で使っている言語は,曖昧な表現をすることが多い.曖昧
            表現というのは,人によって解釈が異なってしまう表現のことである.
            「美しい景色」だの,「大きな木」だのと言って,すべての人が同じ
            情景を浮かべることはまずない.これでは正しい思考が,十分ではな
            い他の人間に伝わることはない.

            しかし,このような例から,普段使っている言語は正しい思考に適し
            ていないと,判断してはいけない.事実,過去の多くの頭のいい学者
            さん達は,言語を用いて正しい思考をし,様々な学問を作り上げてい
            る.大切なのは,曖昧な表現を避けることである.ただ,言語には使
            い方によって,曖昧に表現できてしまうだけなのである.

            では,言語の曖昧な表現を使わないようにするには,どうしたらよい
            だろうか.まず考えつくのは,日常言語から万人が認める最も基本的
            な部分を抽出し,それを元に思考をすればよいことである.言語の最
            も基本的な部分とは,「論理」である.次に,論理について簡単に触
            れよう.

