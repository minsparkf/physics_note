%===================================================================================================
%  Chapter : 他愛のない,思ったこと
%  説明    : 
%===================================================================================================

%   %==========================================================================
%   %  Section
%   %==========================================================================
        \section{生まれ変わる?}
            「生まれ変われるとしたら,次は何になっていたい?」と聞かれることが
            ある.この質問には,何も考えずに答えることが,コミュニケーションを
            円滑にする為に良いのだが,やはり引っかかる部分がある.

            引っかかることとは,「生まれ変わる」ということの定義である.生まれ
            変われるか否かということも当然疑問なのだが,それよりももっと疑問な
            ことがある.生まれ変われることが可能かどうかという疑問には,おそら
            く答えることは不可能だろう.生まれ変わることが不可能であるとした
            ら,話はそれで終わりであるので,ここでは,生まれ変われることができ
            るものとして,話を進めたいと思う.

            疑問というのは,“今の記憶が忘れ去られていたとしても,生まれ変わったと言えるか”と
            いうものである.以前までの記憶がない以上,当然,自分自身には生まれ変わ
            ったという意識は生まれない.第三者的な立場にたって,他人の生まれ変
            わる瞬間を見たとしよう.その場合,生まれ変わることが可能だと,納得
            することだろう.しかし,その他人には以前までの記憶がなく,やはり,
            その人にとって,生まれ変わったという意識はないはずである.たとえ,その瞬間を見てい
            たと教えてやったとしても,その人は生まれ変わったのだと明確に認識することは不可能である.

            生まれ変わって以前までの記憶がない以上,たとえ本当に生まれ変わったのだとしても,
            自分自身にとっては,別人であると意識せざるを得ないと考えるのが,自然なのではなかろうか.
            実際,今の私には,こう考えることが一番妥当だと思っている.
            たとえ,生まれれ変わっていたとしても,記憶が残っていない以上,
            それはその人にとって別ものなのだと考えたい.

            まとめよう.求める答えは,生まれ変われるか否か,ということだったが,これには,
            答えることはできない.
            生まれ変わることができないのであれば,話はここで終了になる.もし,
            生まれ変われることがわかったとしても(他人が生まれ変わったことを見るなどして),
            自分自身では認識できないのであれば,それは生まれ変わったと考えるべきではない,
            と思う.

%   %==========================================================================
%   %  Section
%   %==========================================================================
        \section{教科書に書かれていること}
            物理学や化学$\cdot$生物学$\cdot$天文学などの自然現象を説明すべく,それを
            文字として記述し,本という形で記録できる.
            世の中には多くの専門書,教科書,解説書がある.しかし,
            どれをとっても自然現象をすべて説明するものはない.
            つまり,本を読んだところで,世界を理解できるわけではない.
            本を読んでわかることは,先人たちが苦心して築きあげてきた
            壮大な理論体系ではあるものの,自然現象についてのほんの僅かな
            ことでしかない.

            物理学を学ぶということは,物理学の論文や専門書,教科書を
            読むということではなく,実際の自然現象に触れるということ
            である.そして,なぜだろうと疑問に感じることであり,
            さらに,それを解き明かしたいと思うことである.

            物理学を学びたいから物理学の教科書を読む,なんてことは,
            甚だ見当違いである.物理学は自然現象を説明する理論であり,
            つまり,実際に起こっている現象を説明しようとするものである.
            重要なのは,現象に触れること.そして,その現象について,
            その特徴をできるかぎり詳しくしらべること.そうしてやっと,
            現象の特徴はどのような法則に従っているかといった,理論的
            研究に入るのである.

            物理学の本を読むということは,今知られている理論を把握
            するということであり,物理を追求するという行為ではない.
            あくまでも,先人の得た知恵を吸収するということである.
            しかし,それは,探求の第一歩ではない.
            物理学の本を読んで,物理をわかった気になっているとしたら,
            とても残念なことである.


%   %==========================================================================
%   %  Section
%   %==========================================================================
        \section{心配レベル}
            心配という心情には,4つの段階があると思う.それは,次の通りだ.
                \begin{enumerate}
                    \item 心配
                    \item 不安
                    \item 恐怖
                    \item 絶望
                \end{enumerate}
            下に行くほど(数字が大きくなるほど),心配レベルが上がる.
            
            具体例を示してみよう.
            いつもそばにいる大切な人が,一週間の間,自分の前からはなれることに
            なることを考えてみる.海外旅行にでも出かけることにでもしておこう.
            
            その時,あなたは大切な人が,目の前から離れることで,心配になる(はずだ).
            交通事故に遭わないだろうか,悪い人に騙されたりしないだろうか,等々.
            これが,心配という心情だ.この段階では,心配するだけで,何も行動を
            起こさないことだろう.

            一週間たっても,1日,2日,大切な人が帰って来なかったとしよう.
            あなたは不安に陥ることだろう.何があったのか気になって仕方がない.
            こうなると,あなたは,どうにかして,連絡を取ろうと必死になるはずである.
            これが不安という心情.

            どうしても連絡がつかなかったら,その不安は恐怖になる.
            事件に巻き込まれたとか,事故にあったのではないかとかと,考え始める.
            この時,大切な人が傷ついているかもしれないという,恐怖を覚える.

            その恐怖が,最悪の形で現実出会ったとしよう.このとき,あなたは為す術がなく,
            絶望に至る.その後,自分のできる限りの行動を,世の中に対して必死に働きかける
            ことになる.



%   %==========================================================================
%   %  Section
%   %==========================================================================
        \section{"分からないこと" と "知らないこと"}
            突然,新しい環境に放り込まれたとしよう.
            このとき,大変幸福なことに,近くにその環境に詳しい人がいるとする.しかし,
            その人は,私に対して,その環境のことを説明することを
            あまりしない.そのひとは,
            「わからないことがあったら,何でも質問してください」と言う.
            新参者の私にとっては,その環境に詳しい人は,唯一頼りにできる
            大変ありがたい存在である.
            しかし,私がその人に質問するには,時間がかかる.
            自分が分からないことを把握しなければならないからである.
            わからないことを把握するには,知らないことをリストアップしていく
            必要がある.知らないことは,当然,質問できないからだ.例えば,
            「不確定性原理」という言葉を知らない人は,それについて詳しい人が
            そばに居たとしても,質問することはない.
            質問できないのである.

            何が言いたいかというと,教わる側の人間にが取るべき行動は,
            その周囲の環境を,できる限り,見て聞いて把握することである.
            そして,教える側の人間が取るべき行動は,その新参者が知らないことを
            示してあげる事である.

            わからないことが質問できないと言って,嘆く必要はない.
            そんな場合は,周囲の環境を執拗に見たり聞いたりして,
            できる限り早く把握する,という目標があるのだから,
            それを行えば良い.それができなければ,諦めて,
            別の場所に行くより他はない.
